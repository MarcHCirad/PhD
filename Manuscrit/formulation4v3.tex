\documentclass{article}
\usepackage{graphicx,ulem} 
\usepackage{color}
\usepackage{amsfonts,amsmath}
\usepackage{amsthm}
\usepackage{empheq}
\usepackage{mathtools}
\usepackage{multirow}
%\usepackage{tikz}
\usepackage{titlesec}
\usepackage{caption}
%\usepackage{lscape}
\usepackage{graphicx}
\captionsetup{justification=justified}
\usepackage[toc,page]{appendix}
\usepackage{hyperref}
\usepackage{subcaption}
\usepackage{pdftricks}
\usepackage{xcolor}
\begin{psinputs}
\usepackage{amsfonts,amsmath}
	\usepackage{pstricks-add}
   \usepackage{pstricks, pst-node}
   \usepackage{multido}
   \newcommand{\lfw}{\lambda_{F}}
\end{psinputs}

\textheight240mm \voffset-23mm \textwidth160mm \hoffset-20mm

% \graphicspath{{./Images/}{../Schema}}
\graphicspath{{Figures}}
\setcounter{secnumdepth}{4}
\titleformat{\paragraph}
{\normalfont\normalsize\bfseries}{\theparagraph}{1em}{}
\titlespacing*{\paragraph}
{0pt}{3.25ex plus 1ex minus .2ex}{1.5ex plus .2ex}

\newcommand{\lfd}{\lambda_{F, D}}
\newcommand{\lfw}{\lambda_{F}}
\newcommand{\Kfa}{K_{F,\alpha}}
\newcommand{\cI}{\mathcal{I}}
\newcommand{\N}{\mathcal{N}}

\newcommand{\marc}[1]{\textcolor{teal}{#1}}
\newcommand{\YD}[1]{\textcolor{magenta}{#1}}
\newcommand{\VY}[1]{\textcolor{blue}{#1}}

\DeclareMathOperator{\Tr}{Tr}
\newtheorem{theorem}{Theorem}
\newtheorem{prop}{Proposition}
\newtheorem{definition}{Definition}
\newtheorem{remark}{Remark}
\newtheorem{cor}{Corollary}
\newcommand*\phantomrel[1]{\mathrel{\phantom{#1}}}

\title{Modèle Chasseur}
\author{Marc Hétier, Yves Dumont  and Valaire Yatat-Djeumen}

\begin{document}

\maketitle
%{\hypersetup{hidelinks}
%\tableofcontents}
%\newpage

\section{Introduction}

\begin{enumerate}
\item Contexte général : forêt tropical, population, environnement affecté : chasse et industrialisation. Perte de biodiv. Important car ..
\item  Intérêt de la modèlisation math
\item Nouveauté de l'étude : contrairement aux autres modèles, on considère une pop de chasseur : l'interaction se fait directement entre pop humaine et la faune sauvage
\item Zone d'étude considérée : sud cameroun, population principalement de chasseur, peu de ressources (nourriture) proviennent de la végétation
\item But de l'article : analyser l'impact de la (sur)-chasse et de l'anthropisation du milieu sur les populations (animales et humaines)
\item Plan de l'article

\begin{enumerate}
\item Présentation du modèle
\item Analyse théorique : équilibre, stabilité locale et globale
\item Effet de variation de paramètres : diagramme de bifurcation + interprétation
\item Présentation d'orbites
\item 
\item Conclusion
\end{enumerate}
\end{enumerate}
\YD{Marc, ton niveau d'anglais.... si tu as des doutes utilise un "translator", ou https://www.deepl.com/fr/translator... mais on ne peut pas continuellement corriger et recorriger des formulations ou des expressions qui n'existent pas en anglais! Je doute que "defaunation" existe, même en français.... Il vaut mieux faire des phrases simples et compréhensibles. On n'a pas besoin de beaucoup de blabla....} \marc{Defaunation existe bien en anglais, ou est du moins utilisé par les auteurs qui parlent du sujet ; voir par exemple \cite{benitez-lopez_intact_2019} dont le titre est "Intact but empty forests? Patterns of hunting induced mammal defaunation in the tropics"...} 

Tropical forests are \sout{specially}\YD{particularly} rich ecosystems, in term of plants and animals diversity. They also provide resources for forest-based people, and other human populations living nearby \marc{donner des exemples?} \VY{donner quelques types de ressources et des références qui vont avec}. However, tropical forests are more and more degraded and fragmented by human \YD{settlement and }activities, as the development of infrastructures, agriculture, and industrial complex, etc \marc{donner des exemples ?} \VY{oui et enrichir avec des refs..}. Today, only 20\% of the remaining area are considered as intact, see \cite{benitez-lopez_intact_2019}. It is not only the vegetation which is endangered, but also the wildlife it shelters. Of course, the destruction of their habitat has indirect consequences on animal population, but they are also directly threaten by human activities, and specially by over-hunting \cite{wilkie_empty_2011, benitez-lopez_intact_2019}. In 2019, authors of \cite{benitez-lopez_intact_2019} estimated that the abundance of tropical mammal species had declined of 13\% in average, and predicted a decline superior at \VY{of more than} 70\% for mammals in West Africa. 

Defaunation \YD{???} is not harmless. \YD{In \cite{ripple_bushmeat_2016}, the }authors recall that even partial defaunation \YD{(ce n'est pas de l'anglais! "loss of wild fauna"?)} \marc{Je n'ai pas inventé ce terme ! Il est utilisé tel quel dans les articles qui traitent du sujet}have consequences on the environment and on the human population. Indeed, extinction of mammals may affect the forest regeneration, since they play a role, among others, of seed disperser \YD{donner exemple}. Moreover, bushmeat is still an important source of food and income for some population, \cite{jones_incentives_2019}. This both means that hunt can not just be prohibited, but also that defaunation threaten the food security and livelihood conditions of those population. Therefore, it is necessary to tackle the subject of sustainable hunt.

It can be hard to determine if hunt, as it is practiced in a specific place, is sustainable or not. Firstly because getting knowledge about wildlife and hunt practice can not be done using remote-sensing \cite{peres_detecting_2006}, but also because specific field studies are hardly \YD{generalized} to other places. However, we believe that mathematical modeling can help to overcome those problems. Indeed, mathematical models allow not only to synthesis and generalizes complex reality, but they also permit to determine parameters and thresholds of interest. These knowledge may in return facilitate future field studies, and help decision makers to take justified decisions.

Some mathematical models describing the human-environment interactions have already been studied, see for example the review \cite{fanuel_modelling_2023} and the references therein. \YD{A détailler.... Il faut donner les hypothèses et résultats principaux. Il ne suffit pas juste de balancer une phrase....} \marc{Oui bien sûr !! Je n'ai pas encore fini ce paragraphe !}

For this study, we consider a situation typical of South-Cameroon, where human population mainly live from hunt and agricultural resources. These populations do not threaten the vegetation (consummation of plant for food or medical uses are low compared \YD{to???}, see \cite{koppert_consommation_1996}). However, the forest has been \YD{(is or will be?)} impacted by industrial development: a deep sea harbor has been built near Kribi, industrial palm groves founded, roads created.

\section{Hunter Model}

We model the interactions between a hunter community and wilderness by a consumer-resource model. We consider two areas, one corresponding to a domestic area (typically a village, etc), the second to a wild area (Forest, etc).

Humans in the domestic area, $H_D$, are modeled by a consumer equation, based on \YD{available resource}. The dynamic followed by $H_D$ can be separated in different terms. 
First there is a constant growth term, $\cI$, which represents immigration from other inhabited area. For example, the development of industrial complex bring new workers to the site.

Villagers are also able to produce a certain amount of food at the rate $f_D$. Note that $f_D$ can be understand as $f_D = e \lfd F_D$ where $F_D$ is a constant amount of cattle or agricultural resources. 
Moreover, we assume that $H_D$ has a natural death-rate, $\mu_D$.

Villagers come back and forth \YD{???????? come and go?} \marc{come back and force était une coquille, l'expression voulue était come back and forth} \YD{donc "come and go". Merci d'utiliser des outils de traduction appropriés; demander à Yannnick pour de l'aide...} between the domestic and wild area, mainly for hunt activities. This migration is modeled by linear functions: $-m_D H_D + m_W H_W$, where $H_W$ is the human population present in the wild area, and $m_D$ (respectively $m_W$) is the migration rate from the domestic (resp. wild) area to the wild (resp. domestic) area. Note that we assume that the migration rates are such that $m = \dfrac{m_D}{m_W} < 1$. This assumption makes sense, because humans stay a shorter amount of time in the Wild area than in the Domestic area.
The dynamic of $H_W$ corresponds simply to the migration terms. 

\YD{Wild animals, $F_W$,} hunted in the wild area are partially used to feed the villagers. Therefore, we consider a growth term $e \lfw H_W F_W$ \YD{(que représente ce terme? pourquoi sous cette forme?)}, where $e$ is a conversion rate \YD{(non, ce n'est plus un taux, c'est plutôt une proportion!)} between these hunted prey and the number of human it can nourish. This parameter encompasses both direct consumption of the meat by villagers, but also the fact that a large amount of bush-meat is not consumed by the village inhabitants, but is sold on city market and allow to buy other supply \cite{wilkie_bushmeat_1998}.


The dynamic of $F_W$ follows a logistic equation, with a carrying capacity, $K_F$, dependent on the surrounding vegetation. To take into account the level of anthropization of the habitat, we introduce the non-negative parameter $\alpha \in [0, 1)$. When $\alpha > 0$, the carrying capacity of the habitat is reduced of $\alpha \%$ from its original value. Anthropization may also have a negative impact on the animal's growth rate $r_F$. For the sake of simplicity, we model this impact in the same way, by multiplying the growth rate by $(1-\alpha)$.

We consider two different interactions between wild fauna and the population located in the wild area. On the one hand, wild fauna is hunted by humans present in the wild area. This is take into account by the functional response $\lfw H_W$, where $\lfw$ is the hunting rate and $H_W$ the number of hunter. The functional response is unbounded, to take into account the possibility of over-hunt. On the other hand, it is known that housing, culture and food supply may attract some mammals, specially rodents (\marc{donner des noms?}) and favor their reproduction \YD{(see for example \cite{dounias_foraging_2011, dobigny_zoonotic_2022})}. We take this effect into account by multiplying the wild animal's growth rate, $r_F$, by the functional $(1 +  \beta H_W)$, where $\beta$ is the influence rate that human activities \YD{may} have on wild animal's growth.

\begin{table}[ht]
\center
\begin{tabular}{|c|c|c|}
\hline 
Parameter & Description & Unit \\ 
\hline \hline
$t$ & Time & Year \\
\hline
$H_D$ & Humans in the domestic area & Ind \\
$H_W$ & Humans in the wild area & Ind \\
$F_W$ & Wild fauna & Ind \\
\hline
$e$ & Prey-food conversion \YD{(non, si $F_W$ et $H_W$ ont la même unité)} & - \\
$f_D$ & Food produced by human population & Year$^{-1}$ \\
$\mu_D$ & Human mortality rate  & Year$^{-1}$ \\
$m_D$ & Migration from domestic area to wild area & Year$^{-1}$ \\
$m_W$ & Migration from wild area to domestic area & Year$^{-1}$ \\
$r_F$ & Wild animal growth rate & Year$^{-1}$ \\
$K_F$ & Carrying capacity for wild fauna, fixed by the environment& Ind \\
$\alpha$ & Proportion of anthropized environment & - \\
$\beta$ & Positive impact from human activities to animal growth rate & Ind$^{-1}$  \\
$\lfw$ & Hunting rate & Ind/Year\\
$\mathcal{I}$ & Immigration rate &Ind/Year\\
\hline
\end{tabular}
\caption{Parameters and variables of the model}
\end{table}

Considering the characteristic times of humans and wildlife, the time scale considered all along the study will be the year.
\medskip

Finally, the model is given by the following equations:
\begin{equation}
\def\arraystretch{2}
\left\{ 
\begin{array}{l}
\dfrac{dH_D}{dt}= \cI + e\dfrac{\lfw F_W}{1 + \lfw \theta F_W}H_W + (f_D - \mu_D) H_D - m_D H_D + m_W H_W. \\
\dfrac{dF_W}{dt} = r_F(1- \alpha) (1+ \beta H_W) \left(1 - \dfrac{F_W}{K_F(1-\alpha)} \right) F_W - \dfrac{\lfw F_W}{1 + \lfw \theta F_W} H_W \\
\dfrac{dH_W}{dt}= m_D H_D - m_W H_W 
\end{array} \right.
\label{equationsHDFWHW}
\end{equation}




\section{Existence and uniqueness of global solutions}
In this section, we state general results on system \eqref{equationsHDFWHW}:  existence of an invariant region, existence and uniqueness of global solutions.

We begin by proving the local existence and uniqueness of solutions of system \eqref{equationsHDFWHW}. The right hand side of equations \eqref{equationsHDFWHW} defines a function $f(y)$ (with $y = (H_D, F_W, H_W)$) which is of class $\mathcal{C}^1$ on $\mathbf{R}^3$. The theorem of Cauchy-Lipschitz ensures that model \eqref{equationsHDFWHW} admits a unique solution, at least locally, for any given initial condition, see \cite{walter_ordinary_1998}.

We need to add a constraint on the sign of $f_D - \mu_D$, to avoid infinite growth of human population, and ensure that the system is well defined. On the following, we will assume that $f_D - \mu_D < 0$. This means that the food produced by the human population living in the domestic area is not sufficient to ensure the permanence of the population. Hunt, or immigration, is necessary. 

The following proposition indicates a compact and invariant subset of $\mathbf{R}_+^3$, on which the solutions are bounded.

\begin{prop}\label{invariantRegion} 
Assume 
\begin{equation*}
\beta < \dfrac{4(\mu_D - f_D)}{m e r_F (1-\alpha)^2 K_F} := \beta^*
\end{equation*}
Then, the region
$$\Omega = \Big\{\Big(H_D, F_W, H_W \Big) \in \mathbb{R}_+^3  \Big|H_D + H_W + eF_W \leq S^{max}, F_W \leq F_W^{max}, H_W \leq H_W^{max} \Big\},$$
is a compact and invariant set for system \eqref{equationsHDFWHW}, 
where
$$
S^{max} = \Big(1 + \dfrac{m_D}{m_W} \Big) \dfrac{\cI + \left(\mu_D - f_D + \dfrac{r_F}{4}(1-\alpha) \right) e (1-\alpha)K_F }{\dfrac{\mu_D - f_D}{m} - er_F (1-\alpha)^2 \beta \dfrac{K_F}{4}},
\quad
F_W^{max} = (1-\alpha)K_F,
\quad
H_W^{max} = \dfrac{m_D}{m_D + m_W} S^{max}
$$
In particular, this means that any solutions of equations \eqref{equationsHDFWHW} with initial condition in $\Omega$ remains in $\Omega$.
\end{prop}
%
\begin{proof}
To prove this proposition we will use the notion of invariant region, see \cite{smoller_shock_1994}. Before, we introduce the variable $S = H_D + H_W + e F_W$. We have:

\begin{equation}
\dfrac{dS}{dt} = \cI + (f_D - \mu_D) \Big(S - H_W - eF_W \Big) + e (1-\alpha)(1+\beta H_W)r_F  \left(1 - \dfrac{F_W}{(1-\alpha)K_F} \right) F_W.
\end{equation}

With this new variable, the model writes:

\begin{equation}
\left\{ \begin{array}{l}
\dfrac{dS}{dt} = \cI + (f_D - \mu_D) \Big(S - H_W - e F_W \Big) + e(1-\alpha)(1+\beta H_W) r_F \left(1 - \dfrac{F_W}{(1-\alpha)K_F} \right) F_W, \\
\dfrac{dF_W}{dt} = (1-\alpha)(1+\beta H_W) r_F \left(1 - \dfrac{F_W}{K_F(1-\alpha)} \right) F_W - \dfrac{\lfw F_W}{1 + \lfw \theta F_W} H_W \\
\dfrac{dH_W}{dt}= m_D \left(S - eF_W\right) - (m_W + m_D) H_W 
\end{array} \right.
\label{equationsSFWHW}
\end{equation}


We define the function $g(z)$, with $z=\Big(S, F_W, H_W \Big)$ as the right hand side of this equations. We also introduce the following functions:
$$
G_1(z) = S - S^{max},
\quad
G_2(z) = F_W - F_W^{max},
\quad
G_3(z) = H_W - H_W^{max}
$$

Following \cite{smoller_shock_1994}, we will show that quantities $(\nabla G_1 \cdot g)|_{S = S^{max}}$, $(\nabla G_2 \cdot g)|_{F_W = F_W^{max}}$ and $(\nabla G_3 \cdot g)|_{H_W = H_W^{max}}$ are non-positive for $z \in \Omega_S = \Big\{ \Big(S, F_W, H_W \Big) \in (\mathbb{R})^3  \Big|S \leq S^{max}, F_W \leq F_W^{max}, H_W \leq H_W^{max} \Big\}$.

Using the fact that $\mu_D - f_D >0$ and $z\in \Omega_S$, we have:

\begin{align*}
(\nabla G_1 \cdot g)|_{S = S^{max}} &= \cI + (f_D - \mu_D) \Big(S^{max} - H_W - eF_W \Big) + e r_F(1-\alpha)(1+\beta H_W)  \left(1 - \dfrac{F_W}{(1-\alpha)K_F} \right) F_W, \\
&\leq \cI + (f_D - \mu_D) S^{max} + (\mu_D - f_D) H_W + \Big(\mu_D - f_D\Big) eF_W +er_F (1-\alpha)(1+\beta H_W) \dfrac{(1-\alpha)K_F}{4} \\
&  \leq \cI + (f_D - \mu_D) S^{max} + (\mu_D - f_D) \dfrac{m_D}{m_D + m_W} S^{max} + \Big(\mu_D - f_D\Big) e (1-\alpha)K_F + \\&er_F (1-\alpha)(1+\beta \dfrac{m_D}{m_D + m_W} S^{max}) \dfrac{(1-\alpha)K_F}{4}, \\
&  \leq \cI + \left( -(\mu_D - f_D) +  \dfrac{(\mu_D - f_D)m_D}{m_D + m_W} +  \dfrac{er_F (1-\alpha)^2\beta m_D}{4(m_D + m_W)}K_F \right)S^{max} +\\&  \Big(\mu_D - f_D + \dfrac{r_F}{4} (1-\alpha)\Big) e (1-\alpha)K_F,  \\
&  \leq \cI + \dfrac{m_D}{m_D + m_W}\left( -\dfrac{(\mu_D - f_D)}{m}  +  e\dfrac{r_F}{4} (1-\alpha)^2\beta K_F \right) S^{max}+\\&  \Big(\mu_D - f_D + \dfrac{r_F}{4} (1-\alpha)\Big) e (1-\alpha)K_F,  \\
&  \leq \cI - \dfrac{m_D}{m_D + m_W}\Big(1 + \dfrac{m_D}{m_W} \Big)\left( \cI + \left(\mu_D - f_D + \dfrac{r_F}{4}(1-\alpha) \right) e (1-\alpha)K_F \right)+\\&  \Big(\mu_D - f_D + r_F (1-\alpha)\Big) e (1-\alpha)K_F,  \\
&  \leq \cI - \left( \cI + \Big(\mu_D - f_D + \dfrac{r_F}{4}(1-\alpha) \Big) e (1-\alpha)K_F \right)+\\&  \Big(\mu_D - f_D + r_F (1-\alpha)\Big) e (1-\alpha)K_F,  \\
&\leq 0
\end{align*}

that is $(\nabla G_1 \cdot g)|_{S = S^{max}} \leq 0$. The two others inequalities are straightforward to obtain. We have:
\begin{align*}
(\nabla G_2 \cdot g)|_{F_W = F_W^{max}} &= r_F  \left(1 - \dfrac{K_F (1-\alpha)}{K_F (1-\alpha)}\right)K_F (1-\alpha)  - \lfw H_W K_F (1-\alpha), \\
(\nabla G_2 \cdot g)|_{F_W = F_W^{max}} & = - \lfw H_W K_F (1-\alpha), \\
(\nabla G_2 \cdot g)|_{F_W = F_W^{max}} & \leq 0.
\end{align*}

The computations for $(\nabla G_3 \cdot g)|_{H_W = H_W^{max}}$ give:

\begin{align*}
(\nabla G_3 \cdot g)|_{H_W = H_W^{max}} &= m_D (S - eF_W) - (m_W + m_D) H_W^{max}, \\
(\nabla G_3 \cdot g)|_{H_W = H_W^{max}} &= m_D (S - eF_W) - m_D S^{max}, \\
(\nabla G_3 \cdot g)|_{H_W = H_W^{max}} & \leq m_D (S - S^{max} -  eF_W), \\
(\nabla G_3 \cdot g)|_{H_W = H_W^{max}} & \leq 0.
\end{align*}

We have shown that $(\nabla G_1 \cdot g)|_{S = S^{max}} \leq 0$, $(\nabla G_2 \cdot g)|_{F_W = F_W^{max}} \leq 0$ and $(\nabla G_3 \cdot g)|_{H_W = H_W^{max}} \leq 0$ in  $\Omega_S$.  According to \cite{smoller_shock_1994}, this prove that $\Omega_S$ is an invariant region for system \eqref{equationsSFWHW}.

This also shows that the set  $\Big\{\Big(H_D, F_W, H_W \Big) \in \mathbb{R}^3  \Big|H_D + H_W + eF_W \leq S^{max}, F_W \leq F_W^{max}, H_W \leq H_W^{max} \Big\}$ is invariant for system \eqref{equationsHDFWHW}. 


Moreover, for any point $y \in \partial (\mathbb{R}_+)^3$, the vector field defined by $f(y)$ is either tangent or directed inward. Then, $\Omega$ is an invariant region for equations \eqref{equationsHDFWHW}. 

\end{proof}

The next proposition shown that equations \eqref{equationsHDFWHW} define a dynamical system on $\Omega$.

\begin{prop}
Equations \eqref{equationsHDFWHW} define a dynamical system on $\Omega$, that is, for any initial condition $(t_0, y)$ with $t_0 \in \mathbf{R}$ and $y \in \Omega$, it exists a unique solution of equations \eqref{equationsHDFWHW}, and this solution is defined for all $t \geq t_0$.
\end{prop}

\begin{proof}
We already prove that equations \eqref{equationsHDFWHW} admit, at least locally, a unique solution for every initial condition. Moreover, since $\Omega$ is an invariant region, the solutions with initial condition on $\Omega$ are bounded. Based on uniform boundedness, we deduce that solutions of system \eqref{equationsHDFWHW} with initial condition on $\Omega$ exists globally, for all $t\geq t_0$. Therefore, $\eqref{equationsHDFWHW}$ defines a dynamical system on $\Omega$.
\end{proof}

\section{Preliminary Results}

We start this section by providing some computational but useful result.

\begin{prop} \label{propBeta}
Assuming $\beta < \beta^*$, the following inequality holds for all $D > 0$:
$$
\beta D \Big(1 - \dfrac{\mu_D - f_D}{me r_F (1-\alpha)^2K_F } D\Big) < 1
$$
\end{prop}

\begin{proof}
If $\Big(1 - \dfrac{\mu_D - f_D}{me r_F (1-\alpha)^2K_F } D\Big) \leq 0$, the proposition is obviously true. When $\Big(1 - \dfrac{\mu_D - f_D}{me r_F (1-\alpha)^2K_F } D\Big) > 0$, we have:

\begin{equation*}
\beta D \Big(1 - \dfrac{\mu_D - f_D}{me r_F (1-\alpha)^2K_F } D\Big) < 4 \dfrac{\mu_D - f_D}{me r_F (1-\alpha)^2K_F } D \left(1 -\dfrac{\mu_D - f_D}{me r_F (1-\alpha)^2K_F } D \right)
\end{equation*}

It is straightforward to show that $x(1 - x) \leq \dfrac{1}{4}$ for $x \in \mathbb{R}$. Therefore,
$$
\beta D \Big(1 - \dfrac{\mu_D - f_D}{me r_F (1-\alpha)^2K_F } D\Big) < 1
$$
\end{proof}

%\begin{prop} \label{propPF}
%We define the following polynomial, which is used all along section \ref{section:with immigration}:
%\begin{multline}
%P_F(X) := X^2 \left(\dfrac{er_F}{K_F} \right) - X \left(e(1-\alpha)r_F + \dfrac{(\mu_D - f_D) r_F}{\lfw m K_F} + \dfrac{\cI \beta r_F}{\lfw K_F} \right) + \\ \left(\dfrac{(\mu_D - f_D)(1-\alpha) r_F}{\lfw m} - \cI\Big(1 - \dfrac{(1-\alpha)\beta r_F}{\lfw} \Big) \right).
%\label{polynome-Feq}
%\end{multline} When $\beta < \beta^*$ and $\cI > 0$, the following results are true:
%
%\begin{itemize}
%\item $P_F\Big((1-\alpha)K_F\Big) = -\cI < 0$
%\item $P_F\Big(\dfrac{\mu_D - f_D}{\lfw m e}\Big) < 0$
%\item $P_F\Big((1-\alpha)K_F - \dfrac{K_F \lfw}{\beta r_F}\Big) > 0$
%\item $P_F$ admits two real roots, $F_1^* \leq F_2^*$. 
%\item If $\dfrac{(\mu_D - f_D) r_F}{\lfw m } \leq \cI\Big(1 - \dfrac{(1-\alpha)\beta r_F}{\lfw} \Big)$,  $F_2^*$ is positive and $F_1^*$ is non positive. If $\dfrac{(\mu_D - f_D) r_F}{\lfw m } > \cI\Big(1 - \dfrac{(1-\alpha)\beta r_F}{\lfw} \Big)$, $F^*_1$ and $F^*_2$ are positive. Moreover, $P_F$ is positive on $(-\infty, F_1^*)$, negative on $(F^*_1, F^*_2)$ and positive on $(F^*_2, +\infty)$.
%\item From the precedent points, it follows that $$(1-\alpha)K_F - \dfrac{K_F \lfw}{\beta r_F} \leq F^*_1 \leq (1-\alpha)K_F, \dfrac{\mu_D - f_D}{\lfw m e} \leq F_2^* $$
%\end{itemize}
%
%\end{prop}
%
%\begin{proof}
%We have:
%\begin{align*}
%P_F((1-\alpha) K_F) &= \Big((1-\alpha) K_F \Big)^2 \left(\dfrac{er_F}{K_F} \right) - (1-\alpha) K_F \left(e(1-\alpha)r_F + \dfrac{(\mu_D - f_D) r_F}{\lfw m K_F} + \dfrac{\cI \beta r_F}{\lfw K_F} \right) + \\ &\left(\dfrac{(\mu_D - f_D)(1-\alpha) r_F}{\lfw m} - \cI\Big(1 - \dfrac{(1-\alpha)\beta r_F}{\lfw} \Big) \right), \\
%&=(1-\alpha)^2 e K_F r_F - e(1-\alpha)^2 K_F r_F - \dfrac{(\mu_D - f_D) (1-\alpha) r_F}{\lfw m} - \dfrac{\cI \beta (1-\alpha)r_F}{\lfw}  + \\ &\dfrac{(\mu_D - f_D)(1-\alpha) r_F}{\lfw m} - \cI +\cI \dfrac{(1-\alpha)\beta r_F}{\lfw}, \\
%&= -\cI< 0.
%\end{align*}
%Then, 
%
%\begin{align*}
%P_F\Big(\dfrac{\mu_D - f_D}{\lfw m e}\Big) &= \left(\dfrac{\mu_D - f_D}{\lfw m e}\right)^2 \left(\dfrac{er_F}{K_F} \right) - \dfrac{\mu_D - f_D}{\lfw m e} \left(e(1-\alpha)r_F + \dfrac{(\mu_D - f_D) r_F}{\lfw m K_F} + \dfrac{\cI \beta r_F}{\lfw K_F} \right) + \\ & \left(\dfrac{(\mu_D - f_D)(1-\alpha) r_F}{\lfw m} - \cI\Big(1 - \dfrac{(1-\alpha)\beta r_F}{\lfw} \Big) \right), \\
%&= - \dfrac{(\mu_D - f_D) \cI \beta r_F}{e \lfw ^2 K_F} - \cI + \dfrac{\cI (1-\alpha)r_F \beta}{\lfw}, \\
%&= -\cI \left( 1 - \dfrac{(1-\alpha)r_F \beta }{\lfw} + \dfrac{(\mu_D - f_D)  \beta r_F}{e \lfw ^2 K_F} \right), \\
%&= -\cI \left( 1 - \dfrac{\beta(1-\alpha)r_F  }{\lfw}\Big(1 - \dfrac{(\mu_D - f_D) }{ m e \lfw (1-\alpha) K_F}\Big) \right), \\
%& < 0,
%\end{align*}
%thanks to proposition \ref{propBeta}. We have:
%
%\begin{align*}
%P_F\Big((1-\alpha)K_F - \dfrac{K_F \lfw}{\beta r_F}\Big) &= P_F\Big((1-\alpha)K_F\Big) + \Big(\dfrac{K_F \lfw}{\beta r_F}\Big)^2 \dfrac{er_F}{K_F} - 2(1-\alpha)K_F \dfrac{K_F \lfw}{\beta r_F}\dfrac{er_F}{K_F} + \\ &\left(e(1-\alpha)r_F + \dfrac{(\mu_D - f_D) r_F}{\lfw m K_F} + \dfrac{\cI \beta r_F}{\lfw K_F} \right) \dfrac{K_F \lfw}{\beta r_F}, \\
%&= -\cI + \dfrac{K_F \lfw^2}{\beta^2 r_F} - 2 \dfrac{(1-\alpha)K_F \lfw e}{\beta} +\dfrac{(1-\alpha)K_F \lfw e}{\beta} + \dfrac{\mu_D - f_D}{\beta m} + \cI, \\
%&= \dfrac{K_F \lfw^2}{\beta^2 r_F} -  \dfrac{(1-\alpha)K_F \lfw e}{\beta} + \dfrac{\mu_D - f_D}{\beta m}, \\
%&= \dfrac{K_F \lfw^2}{\beta^2 r_F} \left(1 - \dfrac{\beta (1-\alpha) r_F}{\lfw} \Big(1 - \dfrac{\mu_D - f_D}{m e \lfw K_F(1-\alpha)} \Big) \right), \\
%&> 0,
%\end{align*}
%using proposition \ref{propBeta}.
%
%To show the last point of the proposition, we start by computing the discriminant of $P_F$, $\Delta_F$. We have:
%\begin{align*}
%\Delta_F &= \left(e(1-\alpha)r_F + \dfrac{(\mu_D - f_D) r_F}{\lfw m K_F} + \dfrac{\cI \beta r_F}{\lfw K_F} \right)^2 - 4\dfrac{er_F}{K_F}  \left(\dfrac{(\mu_D - f_D)(1-\alpha) r_F}{\lfw m} - \cI\Big(1 - \dfrac{(1-\alpha)\beta r_F}{\lfw} \Big) \right), \\
%\Delta_F &= \left(e(1-\alpha)r_F - \dfrac{(\mu_D - f_D) r_F}{\lfw m K_F}\right)^2 + \dfrac{\cI \beta r_F}{\lfw K_F} \left(\dfrac{\cI \beta r_F}{\lfw K_F} + 2\dfrac{(\mu_D - f_D) r_F}{\lfw m K_F} + 2e(1-\alpha)r_F \right) + 4\dfrac{er_F}{K_F}  \cI\Big(1 - \dfrac{(1-\alpha)\beta r_F}{\lfw} \Big), \\
%\Delta_F & > 0.
%\end{align*}
%
%Therefore, $P_F$ admits two real roots. Their sign depends on the sign of the constant coefficient. $P_F$ admits:
%\begin{itemize}
%\item One non positive root $F^*_1$ and one positive root $F^*_2$ if $$\dfrac{(\mu_D - f_D)(1-\alpha) r_F}{\lfw m} - \cI\Big(1 - \dfrac{(1-\alpha)\beta r_F}{\lfw} \Big) \leq 0 \Leftrightarrow \dfrac{(\mu_D - f_D) r_F}{\lfw m } \leq \cI\Big(1 - \dfrac{(1-\alpha)\beta r_F}{\lfw} \Big).$$
%\item Two positive roots $F^*_1\leq  F^*_2$ if $\dfrac{(\mu_D - f_D) r_F}{\lfw m } > \cI\Big(1 - \dfrac{(1-\alpha)\beta r_F}{\lfw} \Big)$.
%\end{itemize}
%They are given by:
%
%\begin{equation*}
%F_i^* = \dfrac{K_F(1-\alpha)}{2}\left(1 \pm \dfrac{\sqrt{\Delta_F}}{e(1-\alpha)r_F}\right) + \dfrac{\mu_D - f_D}{2\lfw m e} + \dfrac{\cI \beta}{2\lfw e}, \quad i=1,2.
%\end{equation*}
%\end{proof}

%The following definitions will be used all along the theoretical study.
%
%
%\begin{definition}
%An open set $\mathcal{D} \in \mathbf{R}^n$ is said to be p-convex provided that for every $x, y \in \mathcal{D}$, with $x \leq y$, the line segment joining $x$ and $y$ belongs to $\mathcal{D}$.
%\end{definition}
%
%\begin{definition}\cite{kaszkurewicz_matrix_2012}
%A square matrix $A \in Mn (\mathbf{R})$ is said reducible if there exists a permutation matrix $P$ such that $P^ T AP$ is block triangular. Otherwise, $A$ is called irreducible.
%
%\marc{OR a square matrix $A \in Mn (\mathbf{R})$ is said irreducible if for each nonempty proper subset $I$ of $N = \{1, ..., n\}$, there exists $i \in I$ and $j \in N\backslash I$ such that $A_{i,j} \neq 0$}
%\end{definition}
%
%
%\begin{definition}\cite{smith_monotone_1995}
%A system of differential equations
%$$ \dfrac{d x}{dt} = g(x), \quad x \in \mathcal{D}$$
%where $\mathcal{D}$ is an open subset of $\mathbf{R}^n$ and $g$ is continuously differentiable in $\mathcal{D}$ is said 
%
%\begin{itemize}
%\item irreducible if the Jacobian matrix of $g$ at $x$, $\mathcal{J}_g(x)$ is irreducible.
%\item competitive if $\mathcal{J}_g(x)$ has non positive off-diagonal elements:
%$$ \dfrac{\partial g_i}{\partial x_j}(x) \leq 0, \quad i \neq j.
%$$
%\end{itemize}
%
%\end{definition}
%
%
%
%\begin{theorem}\cite{zhu_stable_1994}\label{theorem: periodicASOrbit}
%We consider the system of differential equations
%$$
%\dfrac{dx}{dt} = g(x), \quad x \in \mathcal{D}.
%$$
%If
%\begin{itemize}
%\item $\mathcal{D}$ is an open, $p$-convex subset of $\mathbf{R}^3$,
%\item $\mathcal{D}$ contains a unique equilibrium point $x^*$ and $\det(\mathcal{J}_g(x^*)) < 0$,
%\item $g$ is analytic in $\mathcal{D}$,
%\item the system is competitive and irreducible in $\mathcal{D}$,
%\item the system is dissipative: For each $x_0 \in \mathcal{D}$, the positive semi-orbit through $x_0$, $\phi^+(x_0)$ has a compact closure in $\mathcal{D}$ . Moreover, there exists a compact subset $\mathcal{B}$ of $\mathcal{D}$ with the property that for each $x_0 \in \mathcal{D}$, there exists $T(x_0) > 0$ such that $x(t, x_0) \in \mathcal{B}$ for $t \geq T(x_0)$.
%\end{itemize}
%
%then either $x^*$ is stable, or there exists at least one non-trivial orbitally asymptotically stable  periodic orbit in $\mathcal{D}$.
%\end{theorem}
%
%One of the assumptions of theorem \eqref{theorem: periodicASOrbit} is that the system of equations is competitive, and that is not the case of system \eqref{equationsHDFWHW}. However, we will see that this system is equivalent to a competitive system, on which we can apply the theorem, and thus get back the conclusions.
%
%\begin{prop} \label{equivalentSystem}
%System \eqref{equationsHDFWHW} is equivalent to an irreducible and dissipative system. Moreover, if we note $z = (h_D, f_W, h_W)$ the variables of the equivalent system, that is competitive on
%\begin{itemize}
%\item $\Big\{z = (h_D, f_W, h_W) | 0 \leq h_D, f_W  \leq 0, h_W \leq 0 \Big\}$ if $\lfw - (1-\alpha)\beta r_F \geq 0$
%\item $\Big\{z = (h_D, f_W, h_W) | 0 \leq h_D, f_W \leq -K_F(1-\alpha) + \dfrac{K_F \lfw}{\beta r_F} , h_W \leq 0 \Big\}$ if $\lfw - (1-\alpha)\beta r_F < 0$
%\end{itemize}
%
%\end{prop}
%
%\begin{proof}
%Following \cite{wang_predator-prey_1997}, we do the following change of variable: $h_D =  H_D$, $f_W = -F_W$ and $h_W = -H_W$.  The system \eqref{equationsHDFWHW} is transformed into:
%
%\begin{equation}
%\def\arraystretch{2}
%\left\{ \begin{array}{l}
%\dfrac{dh_D}{dt}= \cI + e\dfrac{\lfw f_W}{1 - \lfw \theta f_W} h_W + (f_D - \mu_D) h_D - m_D h_D - m_W h_W, \\
%\dfrac{df_W}{dt} = (1-\alpha)(1 - \beta h_W) r_F \left(1 + \dfrac{f_W}{K_F(1-\alpha)} \right) f_W + \dfrac{\lfw f_W}{1 - \lfw \theta f_W} h_W, \\
%\dfrac{dh_W}{dt}= -m_D h_D - m_W h_W 
%\end{array} \right.
%\label{equationshDfWhW}
%\end{equation}
%
%
%We note $\mathcal{D} = \Big\{z = (h_D, f_W, h_W) | 0 < h_D, f_W < 0, h_W < 0 \Big\}$, and $g(z)$ the right hand side of the system. It is clear that $\mathcal{D}$ is a $p$-convex set, in which $g$ is analytic. According to proposition \ref{invariantRegion}, it exists an invariant region $\tilde{\Omega}$ for system \eqref{equationshDfWhW}, which is a compact subset of $\mathcal{D}$. This show that the system is dissipative for initial condition in  $\tilde{\Omega}$.
%
%The Jacobian of $g$ is given by:
%
%\begin{multline*}
%\mathcal{J}_g(z) = \\ \begin{bmatrix}
%f_D -\mu_D - m_D & e \dfrac{\lfw}{(1 - \lfw \theta f_W)^2} h_W & e \dfrac{\lfw f_W}{(1 - \lfw \theta f_W)} - m_W \\
%0 & r_F (1-\alpha)(1-\beta h_W) \Big(1 + \dfrac{2 f_W}{K_F(1-\alpha)}\Big) + \dfrac{\lfw}{(1 - \lfw \theta f_W)}  h_W & \dfrac{\lfw f_W}{(1 - \lfw \theta f_W)} - (1-\alpha)\beta r_F f_W - \beta r_F \dfrac{f_W^2}{K_F}\\
%-m_D & 0 & -m_W
%\end{bmatrix}.
%\end{multline*}
%Therefore, it is clear that system \eqref{equationshDfWhW} is irreducible. The system is competitive if the non-diagonal term $\lfw f_W - (1-\alpha)\beta r_F f_W - \beta r_F \dfrac{f_W^2}{K_F}$ is non-positive. For $f_w \in (-\infty, 0]$, we have:
%
%\begin{align*}
%&\dfrac{\lfw f_W}{(1 - \lfw \theta f_W)} - (1-\alpha)\beta r_F f_W - \beta r_F \dfrac{f_W^2}{K_F} \leq 0 \\
%&\Leftrightarrow \dfrac{\lfw}{(1 - \lfw \theta f_W)} - (1-\alpha)\beta r_F - \beta r_F \dfrac{f_W}{K_F} \geq 0 \\
%&\Leftrightarrow \dfrac{\lfw}{(1 - \lfw \theta f_W)} - (1-\alpha)\beta r_F \geq \beta r_F \dfrac{f_W}{K_F}.
%\end{align*}

% Therefore, the system is competitive on $(-\infty, 0]$ if $\lfw - (1-\alpha)\beta r_F \geq 0$. 

% When $\lfw - (1-\alpha)\beta r_F<0$, previous computations show that the system is competitive only on $\Big(-\infty, -K_F(1-\alpha) + \dfrac{K_F \lfw}{\beta r_F}\Big]$. 
% \end{proof}

%\begin{prop}
%When $\dfrac{r_F(1-\alpha) \beta}{\lfw} > 1$, we subdivide $\Omega$ into
%$$
%\Omega = \Omega_1 \cup \Omega_2
%$$
%where
%$$
%\Omega_1 = \Big\{\Big(H_D, F_W, H_W \Big) \in \mathbb{R}_+^3  \Big|H_D + H_W + eF_W \leq S^{max}, 0 \leq F_W < F_W^{compet}, H_W \leq H_W^{max} \Big\},
%$$
%
%$$
%\Omega_2 = \Big\{\Big(H_D, F_W, H_W \Big) \in \mathbb{R}_+^3  \Big|H_D + H_W + eF_W \leq S^{max}, F_W^{compet} \leq F_W \leq F_W^{max}, H_W \leq H_W^{max} \Big\},
%$$
%and
%$F_W^{compet} = K_F(1-\alpha) - \dfrac{K_F \lfw}{\beta r_F} > 0
%$
%
%Any solution starting in $\Omega_1$ will enter in $\Omega_2$, which is an invariant region, on which the equivalent system \eqref{equationshDfWhW} is competitive.
%\end{prop}
%
%\begin{proof}
%We start by showing that $\Omega_2$ is an invariant region. In fact, since we already prove that $\Omega$ is an invariant region, we only need to show that 
%$\nabla G \cdot g _{|F_W = F_W^{compet}}(y) > 0$, for $y \in \Omega_2$ where $G = F_W - F_W^{min}$ and $g$ is the right hand side of system \eqref{equationsSFWHW}. We have:
%
%\begin{align*}
%\nabla G \cdot g _{|F_W = F_W^{compet}} &= r_F(1-\alpha)(1+\beta H_W) \left(1 - \dfrac{F_W^{compet}}{(1-\alpha) K_F} \right)F_W^{compet} - \lfw H_W F^{compet}_W \\
%&= \left(r_F(1-\alpha)(1+\beta H_W) \left(1 - \dfrac{(1-\alpha) K_F - \dfrac{K_F \lfw}{r_F \beta}}{(1-\alpha) K_F}\right) - \lfw H_W \right) F^{compet}_W \\
%&= \left((1+\beta H_W) \left( \dfrac{\lfw}{\beta}\right) - \lfw H_W \right) F^{compet}_W \\
%&= \dfrac{\lfw}{\beta} F_W^{compet} \\
%&> 0
%\end{align*}
%
%Therefore, $\Omega_2$ is an invariant region. Now, we show that any solution with positive initial condition in $ \Omega_1$ enter in $\Omega_2$. We consider $F_W \in (0, F_W^{compet}]$, and using the same computations than before, we obtain:
%
%\begin{align*}
%\dfrac{dF_W}{dt} = &r_F(1-\alpha)(1+\beta H_W) \left(1 - \dfrac{F_W}{(1-\alpha) K_F}\right)F_W - \lfw H_W  F_W, \\
%& \geq \left(r_F(1-\alpha)(1+\beta H_W) \left(1 - \dfrac{F_W^{compet}}{(1-\alpha) K_F}\right) - \lfw H_W  \right) F_W, \\
%& \geq \dfrac{\lfw}{\beta} F_W,\\
%&> 0
%\end{align*}
%This means that any solution with positive initial condition in $\Omega_1$ will enter in $\Omega_2$.
%\end{proof}


\begin{theorem}\label{theoremVidyasagar} \cite{vidyasagar_decomposition_1980, dumont_mathematical_2012}
Consider the following $\mathcal{C}^1$ system
\begin{equation}
\def\arraystretch{2}
\left\{ \begin{array}{l}
\dfrac{dx}{dt} = f(x), \\
\dfrac{dy}{dt} = g(x, y) 
\end{array} \right.
\label{equationVidyasagar}
\end{equation}

with $(x, y) \in \mathbf{R}^n \times\mathbf{R}^m$. Let $(x^*, y^*)$ be an equilibrium point.
If $x^*$ is GAS in $\mathbf{R}^n$ for the system $\dfrac{dx}{dt} = f(x)$, and if $y^*$ is GAS in $\mathbf{R}^m$ for the system $\dfrac{dy}{dt} = g(x^*, y)$, then $(x^*, y^*)$ is (locally) asymptotically stable for system \eqref{equationVidyasagar}. Moreover, if all trajectories of \eqref{equationVidyasagar} are forward bounded, then $(x^*, y^*)$ is GAS for \eqref{equationVidyasagar}.
\end{theorem}


\section{Model analysis under the quasi steady state assumption}
In this section, we consider that the characteristic time spend by humans in the wild area is much shorter than the one spend by human in the domestic area. Therefore, we introduce a new time variable $\tau = \epsilon t$ such that the equation for $H_W$ becomes:
$$
\epsilon \dfrac{dH_W}{dt} = m_DH_D - m_W H_W
$$

Under the quasi steady state approximation, \textit{ie} when $\epsilon$ tends to $0$, the model becomes:

\begin{equation}
\def\arraystretch{2}
\left\lbrace \begin{array}{l}
\dfrac{dH_D}{dt} = \cI + (f_D - \mu_D) H_D + e m \dfrac{\lfw F_W}{1 + \lfw \theta F_W}H_D \\
\dfrac{dF_W}{dt} = (1-\alpha) (1+\beta m H_D) r_F \left(1 - \dfrac{F_W}{(1-\alpha)K_F} \right) F_W - m \dfrac{\lfw F_W}{1 + \lfw \theta F_W} H_D \\
H_W(t) = \dfrac{m_D}{m_W} H_D(t) = m H_D(t)
\end{array} \right.
\label{QSSA}
\end{equation}

On the following, we study the long term behavior of this system. The first part is dedicated to the case where no immigration occurs ($\cI = 0$) and the second part to the case with immigration.

\subsection{Model analysis in the case without immigration}
In this case, the system rewrites:

\begin{equation}
\def\arraystretch{2}
\left\lbrace \begin{array}{l}
\dfrac{dH_D}{dt} = (f_D - \mu_D) H_D + e m \dfrac{\lfw F_W H_D}{1 + \lfw \theta F_W} \\
\dfrac{dF_W}{dt} = (1-\alpha) (1+\beta m H_D) r_F \left(1 - \dfrac{F_W}{(1-\alpha)K_F} \right) F_W - m \dfrac{\lfw F_W}{1 + \lfw \theta F_W} H_D \\
H_W(t) = \dfrac{m_D}{m_W} H_D(t) = m H_D(t)
\end{array} \right.
\label{QSSA, I=0}
\end{equation}

\begin{prop}
\label{theoremEquilibre, I=0}
The following results hold:
\begin{itemize}
\item System \eqref{QSSA, I=0} admits a trivial equilibrium $TE = \Big(0,0\Big)$ and a fauna-only equilibrium $EE^{F_W}=\Big(0, (1-\alpha)K_F \Big)$ that always exist.

\item When
$$
%\mathcal{N}_{\theta} := \dfrac{me}{\theta(\mu_D - f_D)} > 1 \quad \text{and} \quad 
\mathcal{N}_{\cI = 0} := \dfrac{\lfw (1-\alpha)K_F\big(me - \theta (\mu_D - f_D) \big)}{\mu_D - f_D} >1,
$$ 
then system \eqref{equationsHDFWHW, cI=0} admits a unique coexistence equilibrium $EE^{HF_W} = \Big(H^*_{D, \cI = 0}, F^*_{W, \cI = 0}\Big)$ \\ 
where 

$$
F^*_{W, \cI = 0} = \dfrac{\mu_D - f_D}{\lfw \big(me - \theta (\mu_D - f_D) \big)  },
\quad 
H^*_{D, \cI = 0} = \dfrac{(1-\alpha)r_F\Big(1 - \dfrac{F^*_{W}}{K_F(1-\alpha)} \Big)}{m\left(\dfrac{\lfw}{1 + \lfw \theta F_W^*} - \beta (1-\alpha) r_F + \beta r_F  \dfrac{F^*_{W}}{K_F}\right)}
.$$
\end{itemize}
\end{prop}

\begin{proof}
To derive the equilibrium we solve \eqref{QSSA, I=0} with $\dfrac{d y}{dt} = 0$. Therefore, an equilibrium satisfies the system of equations:
\begin{equation}\label{systemEquilibre, I=0}
\def\arraystretch{2}
\left\lbrace \begin{array}{cll}
 e m \dfrac{\lfw F_W^*}{1 + \lfw \theta F_W^*} + f_D - \mu_D = 0& \mbox{or} & H_D^* = 0,\\
m H_D^*\Big(\dfrac{\lfw}{1 + \lfw \theta F_W^*} - (1-\alpha)r_F \beta + \dfrac{r_F \beta}{K_F}F_W^* \Big) - r_F(1-\alpha) \left(1- \dfrac{F_W^*}{(1-\alpha)K_F}\right)= 0& \mbox{or} & F^*_W = 0,\\
H_W^* = \dfrac{m_D}{m_W} H_D^* = m H_D^*.&&
\end{array} \right.
\end{equation}
When $H_D^*=0$ and $F_W^*=0$, we recover the trivial equilibrium $TE = \Big(0,0\Big)$. When $H_D^*=0$ and $F_W^*\neq0$, we obtain the fauna-only equilibrium $EE^{F_W} = \Big(0, K_F(1-\alpha) \Big)$. Finally, when $H_D^*\neq0$ and $F_W^*\neq0$, direct computations lead to a unique set of values given by:
$$
F^*_{W} = \dfrac{\mu_D - f_D}{\lfw \big(me - \theta (\mu_D - f_D) \big)  },
\quad 
H^*_{D} = \dfrac{(1-\alpha)r_F\Big(1 - \dfrac{F^*_{W}}{K_F(1-\alpha)} \Big)}{m\left(\dfrac{\lfw}{1 + \lfw \theta F_W^*} - \beta (1-\alpha) r_F + \beta r_F  \dfrac{F^*_{W}}{K_F}\right)}
.$$

Those values are biologically meaningful when $0 < F_W^* \leq (1-\alpha) K_F$ and when $H_D^*$ is positive. 

The positivity of $F_W^*$ implies $$\dfrac{me}{\mu_D - f_D} > \theta.$$
The inequality $F_W^* \leq (1-\alpha) K_F $ and $H_D^* \neq 0$ implies that
$$
\lfw > \dfrac{\mu_D - f_D}{ K_F(1-\alpha) \big(me - \theta (\mu_D - f_D) \big)}.
$$

$H_D^*$ is positive if its denominator is positive. We have:

\begin{equation*}
\dfrac{\lfw}{1 + \lfw \theta F_W^*} - \beta (1-\alpha) r_F + \beta r_F  \dfrac{F^*_{W}}{K_F} = \dfrac{\lfw}{1 + \lfw \theta F_W^*}\left(1 - \dfrac{1 +\theta \lfw F_W^*}{\lfw}\beta (1-\alpha) r_F\Big(1 - \dfrac{F_W^*}{(1-\alpha)K_F} \Big)\right) 
\end{equation*}

Using $F^*_{W} = \dfrac{\mu_D - f_D}{\lfw \big(me - \theta (\mu_D - f_D) \big)  }$ and $\dfrac{1 + \lfw \theta F_W^*}{\lfw} = \dfrac{me}{\lfw \big(m e - \theta (\mu_D - f_D)\big)}$, the denominator is equal at:
\begin{multline*}
\dfrac{\lfw}{1 + \lfw \theta F_W^*} - \beta (1-\alpha) r_F + \beta r_F  \dfrac{F^*_{W}}{K_F} = \\ \dfrac{\lfw}{1 + \lfw \theta F_W^*}\left(1 - \dfrac{me \beta(1-\alpha)}{\lfw \big(m e - \theta (\mu_D - f_D)\big)}\left(1 - \dfrac{\mu_D - f_D}{\lfw \big(me - \theta (\mu_D - f_D) \big)(1-\alpha)K_F } \right)\right).
\end{multline*}

Thanks to proposition \ref{propBeta}, we know that this quantity is positive. Therefore, the equilibrium of coexistence is biologically meaningful if:

\begin{equation*}
\lfw \dfrac{ K_F(1-\alpha) \big(me - \theta (\mu_D - f_D) \big)}{\mu_D - f_D}> 1,
\end{equation*}

since this condition implies $\dfrac{me}{\mu_D - f_D} > \theta$.

\end{proof}


Now, we look for the local asymptotic stability of the equilibrium.

\begin{prop}\label{propLAS, I=0} The following results are valid.
\begin{itemize}
\item The trivial equilibrium $TE$ is unstable.
\item When $\mathcal{N}_{\cI = 0} < 1$, the fauna equilibrium, $EE^{F_W}$, is Locally Asymptotically Stable (LAS).
\item When $\mathcal{N}_{\cI = 0} > 1$, the coexistence equilibrium, $EE^{HF_W}_{\cI =0}$, exists. It is LAS if 
$$\theta K_{F}(1-\alpha)\lambda_{F}<\dfrac{me+\theta(\mu_{D}-f_{D})}{me-\theta(\mu_{D}-f_{D})}$$
and unstable otherwise.
\end{itemize}
\end{prop}

\begin{proof}
To prove this theorem, we look at the Jacobian of system \eqref{QSSA, I=0}. It is given by:

\begin{multline*}
\mathcal{J}(H_D, F_W) = \\
\begin{bmatrix}
- (\mu_D-f_D) + e m \dfrac{\lfw F_W}{1 + \lfw \theta F_W}&  \dfrac{e m \lfw H_D}{(1 + \lfw \theta F_W)^2} \\
\left(- \dfrac{m \lfw}{1 + \lfw \theta F_W} + m\beta (1-\alpha) r_F \left(1 -\dfrac{F_W}{(1-\alpha) K_F} \right) \right) F_W  & r_F(1-\alpha)(1+\beta m H_D) \left( 1 - \dfrac{2F_W}{K_F(1-\alpha)} \right) -  \dfrac{ m \lfw H_D}{(1 + \lfw \theta F_W)^2}
\end{bmatrix}.
\end{multline*}

\begin{itemize}
\item At equilibrium $TE$, we have:
\begin{equation*}
\mathcal{J}(TE) = \begin{bmatrix}
- (\mu_D-f_D) & 0  \\
0 & r_F(1-\alpha) 
\end{bmatrix}.
\end{equation*}
and $r_F(1-\alpha) > 0$ is an eigenvalue of $\mathcal{J}(TE)$. So, $TE$ is unstable.

\item At equilibrium $EE^{F_W}$, we have
\begin{equation*}
\mathcal{J}(EE^{F_W}) = \begin{bmatrix}
- (\mu_D-f_D) + \dfrac{em \lfw (1-\alpha)K_F}{1+ \lfw \theta (1-\alpha)K_F} & 0 \\
- & -(1-\alpha)r_F 
\end{bmatrix}.
\end{equation*}

Eigenvalues of $\mathcal{J}(EE^{F_W})$ are $-(1-\alpha)  r_F < 0$ and $- (\mu_D-f_D) + \dfrac{em \lfw (1-\alpha)K_F}{1+ \lfw \theta (1-\alpha)K_F}$. $EE^{F_W}$ is LAS if they are both negative, \textit{ie} if $- (\mu_D-f_D) + \dfrac{em \lfw (1-\alpha)K_F}{1+ \lfw \theta (1-\alpha)K_F} < 0 \Leftrightarrow \mathcal{N}_{\cI = 0} < 1$.

\item At equilibrium $EE^{HF_W}_{\cI = 0}$, the Jacobian is given by:

\begin{multline*}
\mathcal{J}_(EE^{HF_W}_{\cI = 0}) = \\ \begin{bmatrix}
0 & \dfrac{e m \lfw H_D}{(1 + \lfw \theta F_W)^2}  \\
\left(- \dfrac{m \lfw}{1 + \lfw \theta F_W} + m\beta (1-\alpha) r_F \left(1 -\dfrac{F_W}{(1-\alpha) K_F} \right) \right) F_W &
- r_F(1+\beta m H_D) \dfrac{F_W}{K_F} +  \dfrac{ m \lfw^2 \theta H_D F_W}{(1 + \lfw \theta F_W)^2}
\end{bmatrix}
\end{multline*}
where we used the equilibrium relations. $EE^{HF_W}_{\cI = 0}$ is LAS if the determinant of the Jacobian, $ \det(\mathcal{J}(EE^{HF_W}_{\cI = 0}))$, is positive and the trace, $\Tr(\mathcal{J}(EE^{HF_W}_{\cI = 0}))$ is negative. We have:

\begin{multline*}
\det(\mathcal{J}(EE^{HF_W}_{\cI = 0})) = \dfrac{e m \lfw H_D}{(1 + \lfw \theta F_W)^2}\left(- \dfrac{m \lfw}{1 + \lfw \theta F_W} + m\beta (1-\alpha) r_F \left(1 -\dfrac{F_W}{(1-\alpha) K_F} \right) \right) F_W
\end{multline*}
and using equilibrium values and proposition \ref{propBeta}, we obtain that the determinant is positive, $\det(\mathcal{J}_(EE^{HF_W}_{\cI = 0})) > 0$.

The trace of $\mathcal{J}(EE^{HF_W}_{\cI = 0})$ is given by:
\begin{equation*}
\Tr(\mathcal{J}(EE^{HF_W}_{\cI = 0})) = - r_F(1+\beta m H_D) \dfrac{F_W}{K_F} +  \dfrac{ m \lfw^2 \theta H_D F_W}{(1 + \lfw \theta F_W)^2}.
\end{equation*}

It is negative if
\begin{subequations}
\begin{align}
r_F(1+\beta m H_D) \dfrac{F_W}{K_F} &>  \dfrac{ m \lfw^2 \theta H_D F_W}{(1 + \lfw \theta F_W)^2} \\
\dfrac{1+\beta m H_D}{H_D} \dfrac{r_F}{K_F} &>  \dfrac{ m \lfw^2 \theta }{(1 + \lfw \theta F_W)^2} \label{traceCoexistence1}
\end{align}
\end{subequations}

We have 
\begin{align*}
1+ \beta m H_D & = \dfrac{m\left(\dfrac{\lfw}{1 + \lfw \theta F_W^*} - \beta (1-\alpha) r_F + \beta r_F  \dfrac{F^*_{W}}{K_F}\right) + m \beta(1-\alpha)r_F\Big(1 - \dfrac{F^*_{W}}{K_F(1-\alpha)} \Big)}{m\left(\dfrac{\lfw}{1 + \lfw \theta F_W^*} - \beta (1-\alpha) r_F + \beta r_F  \dfrac{F^*_{W}}{K_F}\right)},  \\
&= \dfrac{m\dfrac{\lfw}{1 + \lfw \theta F_W^*}}{m\left(\dfrac{\lfw}{1 + \lfw \theta F_W^*} - \beta (1-\alpha) r_F + \beta r_F  \dfrac{F^*_{W}}{K_F}\right)}
\end{align*} 
and therefore, 

\begin{equation*}
\dfrac{1+ \beta m H_D}{H_D} = \dfrac{m\dfrac{\lfw}{1 + \lfw \theta F_W^*}}{(1-\alpha)r_F\Big(1 - \dfrac{F^*_{W}}{K_F(1-\alpha)} \Big)}
\end{equation*}
Injecting this expression in \eqref{traceCoexistence1}, we obtain that $\mathcal{J}(EE^{HF_W}_{\cI = 0}) < 0$ if

\begin{subequations}
\begin{align*}
\dfrac{m\dfrac{\lfw}{1 + \lfw \theta F_W^*}}{(1-\alpha)r_F\Big(1 - \dfrac{F^*_{W}}{K_F(1-\alpha)} \Big)} \dfrac{r_F}{K_F} &>  \dfrac{ m \lfw^2 \theta }{(1 + \lfw \theta F_W)^2} \\
\dfrac{1}{\Big((1-\alpha) K_F - {F^*_{W}} \Big)}  &>  \dfrac{ \lfw \theta }{(1 + \lfw \theta F_W)}, \\
\dfrac{(1 + \lfw \theta F_W)}{ \lfw \theta } & > (1-\alpha) K_F - F^*_{W}, \\
\dfrac{me}{ \lfw \theta \Big(me - \theta \lfw (\mu_D - f_D) \Big)} & > (1-\alpha) K_F - \dfrac{\mu_D - f_D}{\lfw (me - \theta (\mu_D - f_D))}.
\end{align*}
\end{subequations}
using the value of $F_W^*$.

Finally, we obtain that 
\begin{equation*}
\Tr(\mathcal{J}(EE^{HF_W}_{\cI = 0})) < 0 \Leftrightarrow \dfrac{me}{\theta (\mu_D - f_D)} + 1 > \mathcal{N}_{\cI = 0}.
\end{equation*}
\YD{c'est à dire
$$
\theta K_{F}(1-\alpha)\lambda_{F}<\dfrac{me+\theta(\mu_{D}-f_{D})}{me-\theta(\mu_{D}-f_{D})}.
$$
A priori, numériquement, cela devrait nous donner des valeurs de $\lambda_F$ assez réaliste pour avoir un cycle limite.... Très simple comme condition... si on compare à l'autre du Holling-type I! }

\YD{Après une question qui vient naturellement: retrouve t-on un résultat similaire si on reste avec un système à 3 équations sans faire du QSS....
}
\end{itemize}

\end{proof}

Now we asses the global stability of the equilibrium. 

\begin{prop}
If $\mathcal{N}_{\cI = 0} < 1$, that is if $EE^{F_W}$ is LAS, then it is Globally Asymptotically Stable (GAS).
\end{prop}

\begin{proof}
On the following, we assume $\mathcal{N}_{\cI = 0} < 1$. We consider a solution $H_D^s, F_W^s$ of system \eqref{QSSA, I=0} with initial condition on $\Omega$. Since $\Omega$ is an invariant region, and by increasing monotonicity of function $z(x) = \dfrac{x}{1 + a x}$, we have for any time $t$:

\begin{equation*}
\def\arraystretch{2}
\left\lbrace \begin{array}{l}
\dfrac{dH_D^s}{dt} \leq -(\mu_D - f_D) H_D^s + \dfrac{e m \lfw (1-\alpha) K_F}{1 + \lfw \theta (1-\alpha) K_F} H_D^s \\
\dfrac{dF_W^s}{dt} = (1-\alpha) (1+\beta m H_D^s) r_F \left(1 - \dfrac{F_W^s}{(1-\alpha)K_F} \right) F_W^s - m \dfrac{\lfw F_W^s}{1 + \lfw \theta F_W^s} H_D^s \\
\end{array} \right.
\end{equation*}

We consider the following sub-system:
\begin{equation}
\def\arraystretch{2}
\left\lbrace \begin{array}{l}
\dfrac{dH_D}{dt} = -(\mu_D - f_D) H_D + \dfrac{e m \lfw (1-\alpha) K_F}{1 + \lfw \theta (1-\alpha) K_F} H_D \\
\dfrac{dF_W}{dt} = (1-\alpha) (1+\beta m H_D) r_F \left(1 - \dfrac{F_W}{(1-\alpha)K_F} \right) F_W - m \dfrac{\lfw F_W}{1 + \lfw \theta F_W} H_D \\
\end{array} \right.
\label{EEF, limit system}
\end{equation}

We will apply theorem \ref{theoremVidyasagar} on this system, with $x = H_D$, $y = F_W$, $x^* = 0$ and $y^* = K_F(1- \alpha)$.

Since $\mathcal{N}_{\cI = 0} < 1$, it is straightforward to show that $x^*$ is GAS for system $\dfrac{dx}{dt} = f_1(x)$. It is also immediate to show that $y^*$ is GAS for system $\dfrac{dy}{dt} = f_{[2]}(x^*, y) = (1-\alpha) r_F \left(1 - \dfrac{y}{K_F(1-\alpha)} \right) $. 

Moreover, the trajectories of the solution of limit-system \label{EEF, limit system} with initial condition in $\Omega$ are bounded (proposition \ref{invariantRegion}). So, we can apply theorem \ref{theoremVidyasagar}, and we obtain that equilibrium $\Big(0, K_F(1-\alpha) \Big)$ is GAS on $\Omega$ for the limit system, and therefore for the original system \eqref{QSSA, I=0}.
\end{proof}

\begin{prop}
If $\mathcal{N}_{\cI = 0} > 1$ then $EE^{HF_W}$ exists. Moreover, if:
\begin{itemize}
\item $\theta K_{F}(1-\alpha)\lambda_{F}<\dfrac{me+\theta(\mu_{D}-f_{D})}{me-\theta(\mu_{D}-f_{D})}$, then $EE^{HF_W}$ is GAS.
\item $\theta K_{F}(1-\alpha)\lambda_{F} > \dfrac{me+\theta(\mu_{D}-f_{D})}{me-\theta(\mu_{D}-f_{D})}$ it exists a stable periodic orbit.
\end{itemize}
\end{prop}

\begin{proof}
We will use the Bendixson-Dulac criterion to show that no periodic or homoclinic loop exist. Then, by the Poincaré-Bendixson theorem, each solution will converge to an existing equilibrium. Since under the given conditions $EE^{HF_W}$ is the only stable equilibrium, all solutions converge to it, that is $EE^{HF_W}$ is globally asymptotically stable.

Following \cite{hsu_competing_1978}, we will consider the Dulac function $g(H_D, F_W) = \dfrac{1 + \lfw \theta F_W }{F_W} H_D^{\gamma - 1}$ where $\gamma > 0$ is to be determined. We have:

\begin{equation*}
f_1 \times g(H_D, F_W) = -(\mu_D - f_D) \dfrac{1 + \lfw \theta F_W}{F_W}H_D^\gamma + e m \lfw H_D ^\gamma
\end{equation*}

so

\begin{equation*}
\dfrac{\partial \Big(f_1 g \Big)}{\partial H_D} = \gamma \left(e m \lfw - (\mu_D - f_D) \dfrac{1 + \lfw \theta F_W}{F_W} \right) H_D^{\gamma-1}.
\end{equation*}

Moreover,
\begin{equation*}
f_2 \times g(H_D, F_W) = r_F(1-\alpha) (1 + \beta m H_D) \left(1 - \dfrac{F_W}{(1-\alpha)K_F}\right)(1+\lfw \theta F_W) H_D^{\gamma - 1} - m \lfw \theta H_D^\gamma,
\end{equation*}

so

\begin{align*}
\dfrac{\partial \Big(f_2 g \Big)}{\partial F_W} &= r_F(1-\alpha) (1 + \beta m H_D) \left(\dfrac{-1 + (1-\alpha)K_F \lfw \theta}{(1-\alpha)K_F}-  \dfrac{2 \lfw \theta}{(1-\alpha)K_F} F_W\right) H_D^{\gamma - 1}, \\
\dfrac{\partial \Big(f_2 g \Big)}{\partial F_W} &= r_F (1 + \beta m H_D) \left(\dfrac{-1 + (1-\alpha)K_F \lfw \theta}{K_F}-  \dfrac{2 \lfw \theta}{K_F} F_W\right) H_D^{\gamma - 1}
\end{align*}

Summing the two derivatives, we obtain:

\begin{align*}
\dfrac{\partial \Big(f_1 g \Big)}{\partial H_D} + \dfrac{\partial \Big(f_2 g \Big)}{\partial F_W} &= H_D^{\gamma - 1} \left( \gamma e m \lfw - \gamma (\mu_D - f_D) \dfrac{1 + \lfw \theta F_W}{F_W} + r_F(1 + \beta m H_D) \times \right. \\ & \left. \left(\dfrac{-1 + (1-\alpha)K_F \lfw \theta}{K_F}-  \dfrac{2 \lfw \theta}{K_F} F_W\right) \right) \\
\dfrac{\partial \Big(f_1 g \Big)}{\partial H_D} + \dfrac{\partial \Big(f_2 g \Big)}{\partial F_W} &= \dfrac{H_D^{\gamma - 1}}{F_W} \left( \dfrac{2 r_F(1+ \beta m H_D) F_W}{K_F} \Big(\dfrac{(1-\alpha)\lfw \theta K_F - 1}{2} - \lfw \theta F_W \Big) + \right. \\ & \left. \gamma \Big(e m \lfw - \theta \lfw(\mu_D - f_D) \Big) F_W - \mu_D - f_D \right) \\
\dfrac{\partial \Big(f_1 g \Big)}{\partial H_D} + \dfrac{\partial \Big(f_2 g \Big)}{\partial F_W} &=\dfrac{H_D^{\gamma - 1}}{F_W} \times \\&  \left( \dfrac{2 r_F(1+ \beta m H_D) F_W}{K_F} \Big(\dfrac{(1-\alpha) \lfw \theta K_F - 1}{2} - \lfw \theta F_W \Big) +  \gamma \Big(e m \lfw - \theta \lfw(\mu_D - f_D) \Big)( F_W - F_W^*) \right)
\end{align*}

We note 
\begin{multline}
P_{GAS}(F_W, H_D) := - \dfrac{2 r_F \lfw \theta}{K_F} (1 + \beta m H_D) F_W^2 + \dfrac{r_F}{K_F} \Big((1-\alpha)K_F \lfw \theta - 1\Big) (1 + \beta m H_D) F_W + \\ \gamma \lfw \Big(em - \theta (\mu_D - f_D)\Big) F_W - \gamma \lfw \Big(em - \theta (\mu_D - f_D)\Big) F_W^*,
\end{multline}

which is the quadratic term in brackets. For $0 < H_D \leq H_D^{max}$, $P_{GAS}(F_W, H_D)$ is maximum for $H_D = H_D^{max}$ and for
$$F_W = \bar{F_W} := \dfrac{ \Big((1-\alpha)K_F \lfw \theta - 1\Big)}{4 \lfw \theta} + \gamma \dfrac{\Big(em - \theta (\mu_D - f_D)\Big)}{4 r_F \theta (1 + m \beta H_D^{max})} K_F $$

We will show that it is possible to choose $\gamma$ such that \begin{equation}\label{inequalitiesGAS}
\dfrac{(1-\alpha) K_F \theta \lfw -1}{2 \lfw \theta} < \bar{F_W} < F_W^*
\end{equation} 
from which it follows that $P_F(H_D^{max}, \bar{F_W})$ is negative, and therefore $\dfrac{\partial \Big(f_1 g \Big)}{\partial H_D} + \dfrac{\partial \Big(f_2 g \Big)}{\partial F_W} < 0$ for $(H_D, F_W) \in \Omega$.

To show that we can choose $\gamma > 0$ such that the two inequalities \eqref{inequalitiesGAS} holds, it is sufficient to prove that $\dfrac{(1-\alpha) K_F \theta \lfw -1}{2 \lfw \theta} < F_W^*$. This condition is equivalent to
\begin{align*}
\dfrac{(1-\alpha) K_F \theta \lfw -1}{2 \lfw \theta} < F_W^* = \dfrac{\mu_D - f_D}{\lfw (em - \theta (\mu_D - f_D)} &\Leftrightarrow \dfrac{(1-\alpha) K_F \theta \lfw -1}{2 \lfw \theta} < \dfrac{\mu_D - f_D}{\lfw (em - \theta (\mu_D - f_D)} \\
&\Leftrightarrow \dfrac{(1-\alpha) K_F \theta \lfw}{2\theta} - \dfrac{em - \theta (\mu_D - f_D)}{2 \theta} < \mu_D - f_D \\
&\Leftrightarrow (1-\alpha) K_F \lfw \theta < \dfrac{em + \theta(\mu_D- f_D)}{em - \theta(\mu_D- f_D)}
\end{align*}
which is precisely the assumed condition.
\end{proof}

\subsection{Model analysis with immigration, $\cI > 0$}

In this section, we study the long term behavior of the system \eqref{QSSA} with $\cI > 0$. We start by the equilibrium existence.


\begin{prop}
The following results hold:
\begin{itemize}
\item The system \eqref{QSSA} admits a human-only equilibrium $EE^{H} = \Big(\dfrac{\cI}{\mu_D - f_D},0\Big)$ that always exist.
\item The system admits one equilibrium of coexistence $EE^{HF_W} = \Big(H_D^*, F_W^* \Big)$ if $$\mathcal{N}_{\cI > 0}:= \dfrac{r_F(1-\alpha)\Big({\dfrac{\mu_D - f_D}{m\cI}+\beta\Big)}}{\lfw}  > 1$$

\item The system admits two equilibrium of coexistence $EE^{HF_W} = \Big(H_D^*, F_W^* \Big)$ if (the conditions I to III and condition IV) or if (the conditions I to III and condition V-VI) are true

\marc{Je suis obligé de rajouter les conditions IV ou (V et VI) pour assurer que $H_D^* > 0$. Dans les cas précédents, j'arrive à montrer que c'est toujours le cas. Ici pas forcément lorsque $e m - \theta(\mu_D - f_D) >0$ }

\begin{itemize}
\item Condition I: $\mathcal{N}_{I>0} < 1$
\item Condition II: $\Big(em - \theta(\mu_D-f_D) \Big) + \dfrac{\mu_D-f_D + \beta m \cI}{\lfw(1-\alpha) K_F } < \beta \theta m \cI$
\item Condition III: $\Delta_{P_F} > 0$
\item Condition IV: $e m - \theta(\mu_D - f_D) < 0$
\item Condition V: $e m - \theta(\mu_D - f_D) >0$
\item Condition VI: $\dfrac{  \theta \beta m \cI - \Big(em - \theta(\mu_D-f_D) + \dfrac{\mu_D-f_D + \beta m \cI}{\lfw K_F(1-\alpha) } \Big) }{2 \Big( \theta \beta m \cI - (em - \theta(\mu_D-f_D)) \Big)} < \dfrac{\mu_D - f_D}{\lfw (1-\alpha) K_F \Big(e m - \theta (\mu_D - f_D)  \Big) }$
\end{itemize}


where $\Delta_{P_F}$, equal at

\begin{multline*}
\Delta_{P_F} = \left(r_F (1-\alpha)   \Big(em - \theta(\mu_D-f_D + \beta m \cI)  \Big) - \dfrac{r_F(\mu_D-f_D) }{\lfw K_F} \right)^2 + \left( \dfrac{\beta r_F m \cI}{\lfw K_F}\right)^2  
\\   +  2 \dfrac{r_F m \cI \Big(e m - \theta (\mu_D - f_D + \beta m \cI)\Big)}{K_F} \left(2 - \dfrac{r_F(1-\alpha)\beta }{\lfw} \left(1 - \dfrac{\mu_D - f_D}{\lfw K_F(1-\alpha) \Big(e m - \theta (\mu_D - f_D + \beta m \cI)\Big)}\right) \right),
\end{multline*}
is the discriminant of the following polynomial:

\begin{multline*}
P_F(X) := X^2 \left(\dfrac{r_F}{K_F} \Big(em - \theta(\mu_D-f_D + \beta m \cI) \Big) \right) - \\ 
X \left(r_F (1-\alpha)   \Big(em - \theta(\mu_D-f_D + \beta m \cI)  \Big) + \dfrac{r_F(\mu_D-f_D)}{\lfw K_F} + \dfrac{\beta r_F m \cI}{\lfw K_F} \right) + \\
 \left(\dfrac{(\mu_D - f_D)(1-\alpha) r_F}{\lfw} - m\cI\Big(1 - \dfrac{(1-\alpha)\beta r_F}{\lfw} \Big) \right)
 \end{multline*}

Moreover, if:
\begin{itemize}
\item $\mathcal{N}_{I>0}$ and $em - \theta(\mu_D - f_D) > \beta \theta m \cI$, $F_W^*$ is the lowest root of $P_F$
\item $\mathcal{N}_{I>0}$ and $em - \theta(\mu_D - f_D) < \beta \theta m \cI$, $F_W^*$ is the largest root of $P_F$
\item Condition (I to III and IV) or (I to III and V and VI) are true, $F_W^*$ are the two roots of $P_F$.
\end{itemize}
$H_D^*$ is always given by:
$$
H_D^* = \dfrac{\cI}{\mu_D - f_D - \dfrac{e\lfw m }{1 + \theta \lfw F_W^*} F_W^*}
$$

\end{itemize}
\end{prop}







\begin{proof}
To derive the equilibrium existence we solve \eqref{QSSA} with $\dfrac{d y}{dt} = 0$. Therefore, an equilibrium satisfies the system of equations:
\begin{equation}\label{systemEquilibre}
\left\lbrace \begin{array}{cll}
\def\arraystretch{2}
\cI + e m \dfrac{\lfw}{1 + \lfw \theta F_W^*} F_W^* H_D^* + (f_D - \mu_D) H_D^* = 0,&&\\
m H_D^*\Big(\dfrac{\lfw}{1 + \lfw \theta F_W^*} - (1-\alpha)r_F \beta + \dfrac{r_F \beta}{K_F}F_W^* \Big) - r_F(1-\alpha) \left(1- \dfrac{F_W^*}{(1-\alpha)K_F}\right)= 0& \mbox{or} & F^*_W = 0.
\end{array} \right.
\end{equation}


When $F_W^* = 0$, we recover the human only equilibrium $EE^{H} = \Big(\dfrac{\cI}{\mu_D - f_D},0\Big)$. When $F_W^* > 0$, the first equation gives $H_D^* = \dfrac{\cI}{\mu_D - f_D - \dfrac{e\lfw m }{1 + \theta \lfw F_W^*} F_W^*}$. Injecting this expression in the second equation gives:


\begin{equation*}
\dfrac{m\cI}{\mu_D - f_D - \dfrac{e\lfw m }{1 + \theta \lfw F_W^*} F_W^*}\left(\dfrac{\lfw}{1 + \lfw \theta F_W^*} - (1-\alpha)r_F \beta + \dfrac{r_F \beta}{K_F}F_W^* \right) - r_F(1-\alpha) \left(1- \dfrac{F_W^*}{(1-\alpha)K_F}\right)= 0 
\end{equation*}
Direct computations show that $F_W^*$ is solution of

\begin{equation*}
P_F(F_W^*) = 0
\end{equation*}
with 

\begin{multline}
P_F(X) := X^2 \left(\dfrac{r_F}{K_F} \Big(em - \theta(\mu_D-f_D + \beta m \cI) \Big) \right) - \\ 
X \left(r_F (1-\alpha)   \Big(em - \theta(\mu_D-f_D + \beta m \cI)  \Big) + \dfrac{r_F(\mu_D-f_D)}{\lfw K_F} + \dfrac{\beta r_F m \cI}{\lfw K_F} \right) + \\
 \left(\dfrac{(\mu_D - f_D)(1-\alpha) r_F}{\lfw} - m\cI\Big(1 - \dfrac{(1-\alpha)\beta r_F}{\lfw} \Big) \right)
\end{multline}

We are searching for the positive roots $F_i$ of $P_F$ which are positive, lower than $K_F(1- \alpha)$ and such that $H_D^* > 0 \Leftrightarrow \mu_D - f_D - \dfrac{e\lfw m}{1 + \theta \lfw F_i} F_i > 0$. Before looking for the roots, we can make several observations, justified in annex.

\begin{itemize}
\item Observation 1: $P_F((1-\alpha)K_F) = - m \cI < 0$
\item Observation 2: If $e m - \theta (\mu_D - f_D) < 0$, $H_D^* > 0$. However, if $e m - \theta (\mu_D - f_D) > 0$,  $H_D^* > 0$ if $\dfrac{\mu_D - f_D}{\lfw \Big(e m - \theta (\mu_D - f_D)\Big)} > F_W^*$
\item Observation 3: $P_F\left(\dfrac{\mu_D - f_D}{\lfw \Big(e m - \theta (\mu_D - f_D)} \right) < 0$
\item Observation 4: the positivity of the dominant coefficient of $P_F$ implies the negativity of its coefficient in $X$ \textit{ie} $a_2 > 0 \Rightarrow a_1 < 0$.
\end{itemize}

%\begin{align*}
%&P_F\left(\dfrac{\mu_D - f_D}{\lfw \Big(em - \theta (\mu_D - f_D) \Big)} \right) = \left(\dfrac{\mu_D - f_D}{\lfw \Big(em - \theta (\mu_D - f_D) \Big)} \right)^2\left(\dfrac{r_F}{K_F} \Big(em - \theta(\mu_D-f_D + \beta m \cI) \Big) \right)  \\ & -\left(\dfrac{\mu_D - f_D}{\lfw \Big(em - \theta (\mu_D - f_D) \Big)} \right)\left(r_F (1-\alpha)   \Big(em - \theta(\mu_D-f_D + \beta m \cI)  \Big) + \dfrac{r_F(\mu_D-f_D)}{\lfw K_F} + \dfrac{\beta r_F m \cI}{\lfw K_F} \right) \\ &+\left(\dfrac{(\mu_D - f_D)(1-\alpha) r_F}{\lfw} - m\cI\Big(1 - \dfrac{(1-\alpha)\beta r_F}{\lfw} \Big) \right) \\
%&P_F\left(\dfrac{\mu_D - f_D}{\lfw \Big(em - \theta (\mu_D - f_D) \Big)} \right) = \dfrac{-r_F \theta \beta m \cI}{K_F}\left(\dfrac{\mu_D - f_D}{\lfw \Big(em - \theta (\mu_D - f_D) \Big)} \right)^2 + \dfrac{(\mu_D - f_D) r_F (1-\alpha) \theta \beta m \cI}{\lfw (me - \theta (\mu_D - f_D) )} \\ &- \dfrac{(\mu_D - f_D) r_F \beta m I}{\lfw ^2 (me - \theta (\mu_D - f_D)) K_F} - m\cI + \dfrac{(1-\alpha) r_F \beta m \cI}{\lfw} \\
%&P_F\left(\dfrac{\mu_D - f_D}{\lfw \Big(em - \theta (\mu_D - f_D) \Big)} \right) = \dfrac{-r_F \theta \beta m \cI}{K_F}\left(\dfrac{\mu_D - f_D}{\lfw \Big(em - \theta (\mu_D - f_D) \Big)} \right)^2 - m \cI - \dfrac{(\mu_D - f_D) r_F \beta m \cI}{\lfw ^2 (me - \theta (\mu_D - f_D)) K_F} \\& + \dfrac{r_F(1-\alpha) \beta m \cI}{\lfw} \dfrac{me}{me - \theta (\mu_D - f_D)} \\
%&P_F\left(\dfrac{\mu_D - f_D}{\lfw \Big(em - \theta (\mu_D - f_D) \Big)} \right) = - m \cI - m\cI \dfrac{(\mu_D - f_D) r_F \beta }{\lfw ^2 (me - \theta (\mu_D - f_D)) K_F} \dfrac{em}{em - \theta(\mu_D - f_D)} \\& + \dfrac{r_F(1-\alpha) \beta m \cI}{\lfw} \dfrac{me}{me - \theta (\mu_D - f_D)} \\
%&P_F\left(\dfrac{\mu_D - f_D}{\lfw \Big(em - \theta (\mu_D - f_D) \Big)} \right) = - m \cI - m\cI \dfrac{em}{em - \theta(\mu_D - f_D)} \left(\dfrac{(\mu_D - f_D) r_F \beta }{\lfw ^2 (me - \theta (\mu_D - f_D)) K_F} - \dfrac{r_F(1-\alpha) \beta }{\lfw}\right) \\
%&P_F\left(\dfrac{\mu_D - f_D}{\lfw \Big(em - \theta (\mu_D - f_D) \Big)} \right) = - m \cI + m\cI \dfrac{em}{em - \theta(\mu_D - f_D)} \dfrac{r_F(1-\alpha) \beta}{\lfw} \left(1 - \dfrac{(\mu_D - f_D)}{\lfw (me - \theta (\mu_D - f_D)) (1-\alpha)K_F}\right) \\
%&P_F\left(\dfrac{\mu_D - f_D}{\lfw \Big(em - \theta (\mu_D - f_D) \Big)} \right) < 0 
%\end{align*}
%
%thanks to proposition \ref{propBeta}.

Now, we can search for the roots of $P_F(X) = a_2 X^2 + a_1 X + a_0$. Their numbers and signs depends on its coefficient sign. We distinguish different cases, summarized in the following table:


\begin{table}[!ht]
\centering
\begin{tabular}{c|c|c|c|c|c|c}
 Case & \multicolumn{4}{c|}{Sign of} & Number of positive roots   & \\
 & $a_0$ & $a_1$ & $a_2$ & $\Delta_{P_F}$ && \\
\hline
\textbf{Case 1} & \multirow{2}{*}{$>0$} & $>0$ & $< 0$ & $>0^*$ &One & It defines an equilibrium \\
\cline{3-7}
\textbf{Case 2} &  &  & $> 0$ & $>0^*$ &Two & The lowest root defines an equilibrium \\
\hline
\textbf{Case 3} & \multirow{5}{*}{$< 0$} &  & $> 0$ & $>0^*$ & One & No equilibrium exists \\
\cline{3-7}
\textbf{Case 4} &  & $<0$ & $< 0$ &  & Zero & No equilibrium exists \\
\cline{3-7}
\textbf{Case 5} & &\multirow{3}{*}{$ > 0$} & \multirow{3}{*}{$< 0$} & $<0$ &Zero &  No equilibrium exists\\
\cline{5-7}
\multirow{2}{*}{\textbf{Case 6}} & & & &\multirow{2}{*}{$>0$} &\multirow{2}{*}{Two} & The roots can define an equilibrium \\
& & & & & & under supplementary conditions
\end{tabular}
\caption{\centering $^*$: positivity of $\Delta_{P_F}$ is implied by previous conditions (obvious except for \textbf{Case 2}, see below). \newline Empty box means that the sign of the quantity is not useful to conclude}
\end{table}


\begin{itemize}
\item First, we assume that its constant coefficient is positive,\textit{ie}$a_0 = \dfrac{(\mu_D - f_D)(1-\alpha) r_F}{\lfw} - m\cI\Big(1 - \dfrac{(1-\alpha)\beta r_F}{\lfw} \Big) > 0 \Leftrightarrow \mathcal{N}_{\cI > 0} > 1$.

\begin{itemize}
\item \textbf{CASE 1: }We assume that the dominant coefficient $a_2$ is negative, \textit{ie} if $ em - \theta (\mu_D - f_D) < \theta \beta m \cI$.

In this case, $P_F$ admits one negative $F_1$ and one positive $F_2$ root. Moreover, $P_F$ is positive on $(F_1, F_2)$, and negative elsewhere. Since $P_F((1-\alpha) K_F) < 0$, we have $0 < F_2 < (1-\alpha) K_F$. Moreover, since $P_F\left(\dfrac{\mu_D - f_D}{\lfw \Big(e m - \theta (\mu_D - f_D)} \right) < 0$, we also have $F_2 < \dfrac{\mu_D - f_D}{\lfw \Big(e m - \theta (\mu_D - f_D)\Big)}$ when $e m - \theta (\mu_D - f_D) > 0$. Therefore, $H_D(F_2)$ is always positive (cf observation 2), and $F_2$ defines an equilibrium.

\medskip

\item \textbf{CASE 2: } We assume that the dominant coefficient $a_2$ is positive, \textit{ie} if $em - \theta (\mu_D - f_D) > \theta \beta m \cI$.

In this case, $P_F$ may admits 2 or 0 positive roots, depending on the sign of its discriminant $\Delta_{P_F}$. The computations below show that under the assumption $em - \theta (\mu_D - f_D) > \theta \beta m \cI$, we have $\Delta_{P_F} > 0$, which means that $P_F$ admits two positive roots, $F_1 < F_2$. 

Moreover, $P_F$ is negative on $(F_1, F_2)$, and positive elsewhere. Since $P_F((1-\alpha) K_F) < 0$, we have $0 < F_1 < (1-\alpha) K_F < F_2$. Consequently, $F_2$ can not define an equilibrium.

Moreover, since $P_F\left(\dfrac{\mu_D - f_D}{\lfw \Big(e m - \theta (\mu_D - f_D)} \right) < 0$, we also have $F_1 < \dfrac{\mu_D - f_D}{\lfw \Big(e m - \theta (\mu_D - f_D)\Big)}$ when $e m - \theta (\mu_D - f_D) > 0$. Therefore, $H_D(F_1)$ is always positive (cf observation 2), and $F_1$ defines an equilibrium.
\end{itemize}

\item Now, we assume that the constant coefficient is negative, \textit{ie} $\mathcal{N}_{\cI > 0} < 1$.

\begin{itemize}
\item \textbf{CASE 3: } We assume that the dominant coefficient is positive, \textit{ie} if $em - \theta (\mu_D - f_D) > \theta \beta m \cI$.

In this case, $P_F$ admits one negative $F_1$ and one positive $F_2$ root. Moreover, $P_F$ is negative on $(F_1, F_2)$, and positive elsewhere. Since $P_F((1-\alpha) K_F) < 0$, we have $(1-\alpha) K_F < F_2$, and neither $F_1$ or $F_2$ define an equilibrium.

\item Now, we assume that the dominant coefficient is negative, \textit{ie} if $em - \theta (\mu_D - f_D) < \theta \beta m \cI$.

In this case, $P_F$ may admits 2 or 0 positive roots, depending on the signs of its coefficient in $X$ and of its discriminant $\Delta_{P_F}$.
\begin{itemize}
\item \textbf{CASE 4: }If the coefficient in $X$ is negative, \textit{ie} if 
\begin{multline*}
- \left(r_F (1-\alpha)   \Big(em - \theta(\mu_D-f_D + \beta m \cI)  \Big) + \dfrac{r_F(\mu_D-f_D)}{\lfw K_F} + \dfrac{\beta r_F m \cI}{\lfw K_F}\right) < 0 \\
\Leftrightarrow em - \theta(\mu_D-f_D + \beta m \cI) + \dfrac{\mu_D-f_D + \beta m \cI}{\lfw (1- \alpha) K_F} < 0,
\end{multline*}
the eventual real roots are both negative.
\item \textbf{CASE 5: }If the coefficient in $X$ is positive and $\Delta_{P_F} < 0$, $P_F$ admits no real root.

\item \textbf{CASE 6: }If the coefficient in $X$ is positive and $\Delta_{P_F} > 0$, $P_F$ admits two real and positives roots, $F_1$ and $F_2$. Moreover, $P_F$ is positive on $(F_1, F_2)$, and negative elsewhere. Since $P_F((1-\alpha) K_F) < 0$ and $P_F'((1-\alpha) K_F) < 0$, we have $0 < F_1 < F_2 < (1-\alpha) K_F$.

\medskip

If $e m - \theta(\mu_D - f_D) < 0$, $F_1$ and $F_2$ define an equilibrium since $H_D^*(F_i) > 0$ (cf Observation 2). 

Otherwise, if $e m - \theta(\mu_D - f_D)  > 0$, since $P_F\left(\dfrac{\mu_D - f_D}{\lfw \Big(e m - \theta (\mu_D - f_D)} \right) < 0$, either $\dfrac{\mu_D - f_D}{\lfw \Big(e m - \theta (\mu_D - f_D)} < F_1 < F_2$ and $H_D^*(F_i) < 0$, for $i = 1,2$ either $\dfrac{\mu_D - f_D}{\lfw \Big(e m - \theta (\mu_D - f_D)} > F_2 > F_1$ and $H_D^*(F_i) > 0$, for $i = 1,2$. 

The condition $\dfrac{\mu_D - f_D}{\lfw \Big(e m - \theta (\mu_D - f_D)} > F_2 > F_1$ is equivalent to $P_F'\left(\dfrac{\mu_D - f_D}{\lfw \Big(e m - \theta (\mu_D - f_D)}\right) < 0$, that is $2a_2\left(\dfrac{\mu_D - f_D}{\lfw \Big(e m - \theta (\mu_D - f_D)}\right) + a_1 < 0$, which is equivalent at:

$$\dfrac{  \theta \beta m \cI - \Big(em - \theta(\mu_D-f_D) + \dfrac{\mu_D-f_D + \beta m \cI}{\lfw K_F(1-\alpha) } \Big) }{2 \Big( \theta \beta m \cI - (em - \theta(\mu_D-f_D)) \Big)} < \dfrac{\mu_D - f_D}{\lfw (1-\alpha) K_F \Big(e m - \theta (\mu_D - f_D)  \Big) }$$
\end{itemize}

\end{itemize}
\end{itemize}

The discriminant of $P_F$ is given by:

\begin{subequations}
\begin{align}
\Delta_{P_F} &= \left(r_F (1-\alpha)   \Big(em - \theta(\mu_D-f_D + \beta m \cI)  \Big) + \dfrac{r_F(\mu_D-f_D)}{\lfw K_F} + \dfrac{\beta r_F m \cI}{\lfw K_F} \right)^2 - \\ \nonumber & 4 \left(\dfrac{r_F}{K_F} \Big(em - \theta(\mu_D-f_D + \beta m \cI) \Big) \right) \times \left(\dfrac{(\mu_D - f_D)(1-\alpha) r_F}{\lfw} - m\cI\Big(1 - \dfrac{(1-\alpha)\beta r_F}{\lfw} \Big) \right) \\
\Delta_{P_F} &= \left(r_F (1-\alpha)   \Big(em - \theta(\mu_D-f_D + \beta m \cI)  \Big) - \dfrac{r_F(\mu_D-f_D) }{\lfw K_F} \right)^2 + \left( \dfrac{\beta r_F m \cI}{\lfw K_F}\right)^2 \\ \nonumber & + 2 \dfrac{r_F^2 (\mu_D - f_D) \cI \beta m}{\lfw^2 K_F^2} + 4 \dfrac{r_F m \cI \Big(e m - \theta (\mu_D - f_D + \beta m \cI)\Big)}{K_F} - 2\dfrac{r_F^2(1-\alpha) \beta m \cI\Big(e m - \theta (\mu_D - f_D + \beta m \cI)\Big)}{K_F \lfw} \\
\Delta_{P_F} &= \left(r_F (1-\alpha)   \Big(em - \theta(\mu_D-f_D + \beta m \cI)  \Big) - \dfrac{r_F(\mu_D-f_D) }{\lfw K_F} \right)^2 + \left( \dfrac{\beta r_F m \cI}{\lfw K_F}\right)^2 \label{discrminantPF} 
\\ \nonumber &  +  2 \dfrac{r_F m \cI \Big(e m - \theta (\mu_D - f_D + \beta m \cI)\Big)}{K_F} \left(2 - \dfrac{r_F(1-\alpha)\beta }{\lfw} \left(1 - \dfrac{\mu_D - f_D}{\lfw K_F(1-\alpha) \Big(e m - \theta (\mu_D - f_D + \beta m \cI)\Big)}\right) \right) 
\end{align}
\end{subequations}

Using the fact that $e m - \theta(\mu_D - f_D + \beta m \cI) > 0$, we obtain the following inequality:

\begin{align*}
1 - \dfrac{\mu_D - f_D}{\lfw K_F(1-\alpha) \Big(e m - \theta (\mu_D - f_D + \beta m \cI)\Big)} < 1 - \dfrac{\mu_D - f_D}{\lfw K_F(1-\alpha) me }.
\end{align*}
We use this inequality in \eqref{discrminantPF}, to obtain:

\begin{multline*}
\Delta_{P_F} > \left(r_F (1-\alpha)   \Big(em - \theta(\mu_D-f_D + \beta m \cI)  \Big) - \dfrac{r_F(\mu_D-f_D) }{\lfw K_F} \right)^2 + \left( \dfrac{\beta r_F m \cI}{\lfw K_F}\right)^2 \\  +  2 \dfrac{r_F m \cI \Big(e m - \theta (\mu_D - f_D + \beta m \cI)\Big)}{K_F} \left(2 - \dfrac{r_F(1-\alpha)\beta }{\lfw} \left(1 - \dfrac{\mu_D - f_D}{\lfw K_F(1-\alpha) e m }\right) \right)
\end{multline*}

Thanks to proposition \ref{propBeta}, we know that $2 - \dfrac{r_F(1-\alpha)\beta }{\lfw} \left(1 - \dfrac{\mu_D - f_D}{\lfw K_F(1-\alpha) e m }\right) > 1$, and consequently we obtain that
$$
\Delta_{P_F} > 0.
$$


We will compute $P_F\left(\dfrac{\mu_D - f_D}{\lfw \Big(em - \theta (\mu_D - f_D) \Big)} \right)$. We have that:

\begin{align*}
&P_F\left(\dfrac{\mu_D - f_D}{\lfw \Big(em - \theta (\mu_D - f_D) \Big)} \right) = \left(\dfrac{\mu_D - f_D}{\lfw \Big(em - \theta (\mu_D - f_D) \Big)} \right)^2\left(\dfrac{r_F}{K_F} \Big(em - \theta(\mu_D-f_D + \beta m \cI) \Big) \right)  \\ & -\left(\dfrac{\mu_D - f_D}{\lfw \Big(em - \theta (\mu_D - f_D) \Big)} \right)\left(r_F (1-\alpha)   \Big(em - \theta(\mu_D-f_D + \beta m \cI)  \Big) + \dfrac{r_F(\mu_D-f_D)}{\lfw K_F} + \dfrac{\beta r_F m \cI}{\lfw K_F} \right) \\ &+\left(\dfrac{(\mu_D - f_D)(1-\alpha) r_F}{\lfw} - m\cI\Big(1 - \dfrac{(1-\alpha)\beta r_F}{\lfw} \Big) \right) \\
&P_F\left(\dfrac{\mu_D - f_D}{\lfw \Big(em - \theta (\mu_D - f_D) \Big)} \right) = \dfrac{-r_F \theta \beta m \cI}{K_F}\left(\dfrac{\mu_D - f_D}{\lfw \Big(em - \theta (\mu_D - f_D) \Big)} \right)^2 + \dfrac{(\mu_D - f_D) r_F (1-\alpha) \theta \beta m \cI}{\lfw (me - \theta (\mu_D - f_D) )} \\ &- \dfrac{(\mu_D - f_D) r_F \beta m I}{\lfw ^2 (me - \theta (\mu_D - f_D)) K_F} - m\cI + \dfrac{(1-\alpha) r_F \beta m \cI}{\lfw} \\
&P_F\left(\dfrac{\mu_D - f_D}{\lfw \Big(em - \theta (\mu_D - f_D) \Big)} \right) = \dfrac{-r_F \theta \beta m \cI}{K_F}\left(\dfrac{\mu_D - f_D}{\lfw \Big(em - \theta (\mu_D - f_D) \Big)} \right)^2 - m \cI - \dfrac{(\mu_D - f_D) r_F \beta m \cI}{\lfw ^2 (me - \theta (\mu_D - f_D)) K_F} \\& + \dfrac{r_F(1-\alpha) \beta m \cI}{\lfw} \dfrac{me}{me - \theta (\mu_D - f_D)} \\
&P_F\left(\dfrac{\mu_D - f_D}{\lfw \Big(em - \theta (\mu_D - f_D) \Big)} \right) = - m \cI - m\cI \dfrac{(\mu_D - f_D) r_F \beta }{\lfw ^2 (me - \theta (\mu_D - f_D)) K_F} \dfrac{em}{em - \theta(\mu_D - f_D)} \\& + \dfrac{r_F(1-\alpha) \beta m \cI}{\lfw} \dfrac{me}{me - \theta (\mu_D - f_D)} \\
&P_F\left(\dfrac{\mu_D - f_D}{\lfw \Big(em - \theta (\mu_D - f_D) \Big)} \right) = - m \cI - m\cI \dfrac{em}{em - \theta(\mu_D - f_D)} \left(\dfrac{(\mu_D - f_D) r_F \beta }{\lfw ^2 (me - \theta (\mu_D - f_D)) K_F} - \dfrac{r_F(1-\alpha) \beta }{\lfw}\right) \\
&P_F\left(\dfrac{\mu_D - f_D}{\lfw \Big(em - \theta (\mu_D - f_D) \Big)} \right) = - m \cI + m\cI \dfrac{em}{em - \theta(\mu_D - f_D)} \dfrac{r_F(1-\alpha) \beta}{\lfw} \left(1 - \dfrac{(\mu_D - f_D)}{\lfw (me - \theta (\mu_D - f_D)) (1-\alpha)K_F}\right) \\
&P_F\left(\dfrac{\mu_D - f_D}{\lfw \Big(em - \theta (\mu_D - f_D) \Big)} \right) < 0 
\end{align*}

thanks to proposition \ref{propBeta}.

\end{proof}

\begin{prop}
\label{propLAS} The following results are valid.
\begin{itemize}
\item When $\mathcal{N}_{\cI > 0} < 1$, the human equilibrium $EE^{H}$ is LAS.
\end{itemize}
\end{prop}

\begin{proof}
\begin{multline*}
\mathcal{J}(H_D, F_W) = \\
\begin{bmatrix}
- (\mu_D-f_D) + e m \dfrac{\lfw F_W}{1 + \lfw \theta F_W}&  \dfrac{e m \lfw H_D}{(1 + \lfw \theta F_W)^2} \\
\left(- \dfrac{m \lfw}{1 + \lfw \theta F_W} + m\beta (1-\alpha) r_F \left(1 -\dfrac{F_W}{(1-\alpha) K_F} \right) \right) F_W  & r_F(1-\alpha)(1+\beta m H_D) \left( 1 - \dfrac{2F_W}{K_F(1-\alpha)} \right) -  \dfrac{ m \lfw H_D}{(1 + \lfw \theta F_W)^2}
\end{bmatrix}.
\end{multline*}
\begin{itemize}
\item At equilibrium $EE^{H}$, the Jacobian matrix is:
\begin{equation*}
\mathcal{J}(EE^{H}) =
\begin{bmatrix}
- (\mu_D-f_D) &  \dfrac{e m \lfw I}{\mu_D - f_D} \\
0 & r_F(1-\alpha)\left(1+ \dfrac{\beta m\cI}{\mu_D - f_D}\right)-  \dfrac{m \lfw \cI}{\mu_D - f_D}
\end{bmatrix}.
\end{equation*}
The eigenvalues are: $- (\mu_D-f_D) < 0$ and $r_F(1-\alpha)\left(1+ \dfrac{\beta m\cI}{\mu_D - f_D}\right)-  \dfrac{m \lfw \cI}{\mu_D - f_D}$ which is negative when $\mathcal{N}_{\cI > 0} < 1$

\item \marc{à continuer} At equilibrium $EE^{HF_W}$, the Jacobian matrix is:
\begin{equation*}
\mathcal{J}(EE^{HF_W}) =
\begin{bmatrix}
-(\mu_D -f_D) +e m \dfrac{\lfw F_W}{1 + \lfw \theta F_W} &  \dfrac{e m \lfw H_D}{(1 + \lfw \theta F_W)^2} \\
\left(- \dfrac{m \lfw}{1 + \lfw \theta F_W} + m\beta (1-\alpha) r_F \left(1 -\dfrac{F_W}{(1-\alpha) K_F} \right) \right) F_W  & - r_F(1+\beta m H_D) \dfrac{F_W}{K_F} +  \dfrac{ m \lfw^2 \theta H_D F_W}{(1 + \lfw \theta F_W)^2}
\end{bmatrix}.
\end{equation*}

$EE^{HF_W}$ is LAS if the trace of the Jacobian is negative and its determinant positive. After some computations (see dimensionless system), we obtain:

\begin{equation*}
\det(\mathcal{J}(H_D^*, F_W^*))  = \dfrac{F_W^*}{(1 + F_W^*)^2}\left( - P_F'(F_W^*) - a_2(F_W^*)^2  + a_0 \right)
\end{equation*}
where $a_2$ and $a_0$ are the coefficient of $P_F = a_2 X^2 + a_1 X + a_0$ and $P_F(F_W^*) = 0$.

We will use the different cases made before. $F_W^*$ is defined for case 1, 2 and 6.
\begin{itemize}
\item \textbf{Case 1:} In this case, $a_0 > 0$ and $a_2 < 0$. Moreover, we know that $F_W^*$ is the largest root of $P_F$, and therefore $P_F'(F_W^*) < 0$. Consequently, $\det(\mathcal{J}(H_D^*, F_W^*)) > 0$.
\item \textbf{Case 2:} In this case, $a_0 > 0$ and $a_2 > 0$. Moreover, we know that $F_W^*$ is the lowest root of $P_F$, and therefore $P_F'(F_W^*) < 0$. Moreover, using $F_W^* = \dfrac{-a_1}{2a_2} - \dfrac{\sqrt{a_1^2 - 4 a_2 a_0}}{2a_2}$, we obtain $- a_2(F_W^*)^2  + a_0 > 0$. Consequently, $\det(\mathcal{J}(H_D^*, F_W^*)) > 0$.
\item \textbf{Case 6:} In this case, $a_0 < 0$, $a_2 < 0$, $a_1 > 0$, $\Delta_{P_F} > 0$... \marc{à continuer}.
\end{itemize}





\end{itemize}
\end{proof}

\subsection{Dimensionless system}
To facilitate the computation, we consider the following dimensionless system:

\begin{equation}
\def\arraystretch{2}
\left\lbrace \begin{array}{l}
\dfrac{dh_D}{d\tau} = i - d h_D + n \dfrac{h_D f_W}{1 + f_W} \\
\dfrac{df_W}{d \tau} = (1+ b h_D) \left(1 - \dfrac{f_W}{k_F} \right) f_W - n \dfrac{h_D f_W}{1 + f_W}
\end{array} \right.
\label{0dimSystem}
\end{equation}

where $h_D = H_D \dfrac{\lfw \theta}{e}$, $f_W = F_W \lfw \theta$, $\tau = r_F(1-\alpha) t$, 
$$i = \dfrac{I \lfw \theta}{e r_F(1-\alpha)} \quad d = \dfrac{\mu_D - f_D}{r_F(1-\alpha)}$$
$$n = \dfrac{e m }{\theta r_F(1-\alpha)} \quad b = \dfrac{me \beta}{\lfw \theta}$$ and $$k_F = (1-\alpha) K_F \lfw \theta$$

The coexistence equilibrium is given by:

$$h_D^* = \dfrac{i}{d - \dfrac{n f_W}{1 + f_W}} \quad OR \quad h_D^* = \dfrac{1 - \dfrac{f_W}{k_F}}{\dfrac{n }{1 + f_W} - b (1 - \dfrac{f_W}{k_F})}$$

and $f_W^*$ is solution of:

$$P_f(f_W^*) = 0$$
where
\begin{equation}
P_f(X) := X^2 \dfrac{n - bi - d}{k_F} - X (n - bi - d + \dfrac{d + bi}{k_F}) + (d + bi - ni)
\end{equation}

A root $f_{W,i}$ of $P_f$ define an equilibrium if $0 < f_{W,i} < k_F$ and $d - \dfrac{n f_W}{1 + f_W} > 0$.

\begin{prop} The following holds true:
\begin{itemize}
\item If $d + bi - ni > 0 \Leftrightarrow \mathcal{N}_{\cI > 0}:= \dfrac{r_F(1-\alpha)\Big({\dfrac{\mu_D - f_D}{m\cI}+\beta\Big)}}{\lfw}  > 1$, the system admits a unique equilibrium of coexistence. $f_W^*$ is:
\begin{itemize}
\item if $n - bi - d > 0$ the lowest root of $P_f$ (which admits two positive roots)
\item if $n - bi - d < 0$ the largest root of $P_f$ (which admits one positive and one negative root)
\end{itemize}
\item $d + bi - ni  < 0$, the system admits 0 or 2 equilibrium of coexistence:
\begin{itemize}
\item if $n - bi - d > 0$, it admits no equilibrium
\item if $n - bi - d < 0$, it may admit two equilibrium of coexistence, depending on the sign of $\Delta_f$
\end{itemize}
\end{itemize}

\end{prop}


The Jacobian matrix of \eqref{0dimSystem} is given by:
\begin{equation}
\mathcal{J}(h_D, f_W) = \begin{bmatrix}
- d + \dfrac{nf_W}{1+f_W} & \dfrac{n}{(1 + f_W)^2}h_D \\
\Big(b (1 - \dfrac{f_W}{k_F}) - \dfrac{n}{1+f_W}\Big) f_W & (1+bh_D)(1 - \dfrac{2f_W}{k_F}) - \dfrac{n}{(1 + f_W)^2}h_D
\end{bmatrix}
\end{equation} 

At coexistence equilibrium, we have: $-d + \dfrac{nf_W}{1 + f_W} = -\dfrac{i}{h_D}$ and $(1+bh_D)(1 - \dfrac{2f_W}{k_F}) - \dfrac{n}{(1 + f_W)^2}h_D = -(1+b h_D) \dfrac{f_W}{k_F} + \dfrac{n h_D f_W}{(1+f_W)^2}$.
We have:

\begin{align*}
\det(\mathcal{J}(h_D^*, f_W^*)) &= i \dfrac{1 + bh_D}{h_D} \dfrac{f_W}{k_F} - \dfrac{n i f_W}{(1+f_W)^2} + \Big(\dfrac{n}{1+f_W} - b (1 - \dfrac{f_W}{k_F}) \Big) \dfrac{n f_W h_D}{(1+f_W)^2} \\
\det(\mathcal{J}(h_D^*, f_W^*)) &= \dfrac{i}{k_F}\dfrac{f_W}{h_D} + ib \dfrac{f_W}{k_F} - ni \dfrac{f_W}{(1+f_W)^2} +(1-\dfrac{f_W}{k_F})\dfrac{n f_W}{(1+f_W)^2} \\
\det(\mathcal{J}(h_D^*, f_W^*)) &= \dfrac{f_W}{(1 + f_W)^2} \left(\dfrac{i}{k_F}\dfrac{(1+f_W)^2}{h_D} + ib \dfrac{(1+f_W)^2}{k_F} - ni +(1-\dfrac{f_W}{k_F})n \right) \\
\det(\mathcal{J}(h_D^*, f_W^*)) &= \dfrac{f_W}{(1 + f_W)^2} \left(\Big(d - \dfrac{nf_W}{1 + f_W} \Big)\dfrac{(1+f_W)^2}{k_F} + ib \dfrac{(1+f_W)^2}{k_F} - ni +(1-\dfrac{f_W}{k_F})n \right)\\
\det(\mathcal{J}(h_D^*, f_W^*)) &= \dfrac{f_W}{(1 + f_W)^2} \left(\dfrac{d + ib}{k_F}(1+f_W)^2  - \dfrac{nf_W(1+f_W)}{k_F} - ni +(1-\dfrac{f_W}{k_F})n \right)\\
\det(\mathcal{J}(h_D^*, f_W^*)) &= \dfrac{f_W}{(1 + f_W)^2} \left(\dfrac{d + ib}{k_F} + 2\dfrac{d + ib}{k_F}f_W + \dfrac{d + ib}{k_F}f_W^2  - \dfrac{nf_W}{k_F} -\dfrac{nf_W^2}{k_F} - ni + n- n\dfrac{f_W}{k_F} \right)\\
\det(\mathcal{J}(h_D^*, f_W^*)) &= \dfrac{f_W}{(1 + f_W)^2} \left(\dfrac{d + ib}{k_F} + 2\dfrac{d + ib - n}{k_F}f_W + \dfrac{d + ib -n}{k_F}f_W^2  - ni + n \right)\\
\det(\mathcal{J}(h_D^*, f_W^*)) &= \dfrac{f_W}{(1 + f_W)^2} \left( \Big(\dfrac{d + ib}{k_F} + n + 2\dfrac{d + ib - n}{k_F}f_W\Big) + \dfrac{d + ib -n}{k_F}f_W^2  - ni \right)\\
\det(\mathcal{J}(h_D^*, f_W^*)) &= \dfrac{f_W}{(1 + f_W)^2} \left( \Big(\dfrac{d + ib}{k_F} + n -bi - d + 2\dfrac{d + ib - n}{k_F}f_W\Big) + \dfrac{d + ib -n}{k_F}f_W^2  - ni +bi + d \right)\\
\det(\mathcal{J}(h_D^*, f_W^*)) &= \dfrac{f_W}{(1 + f_W)^2} \left( - \Big(2\dfrac{n - bi -d}{k_F}f_W - (n - bi -d + \dfrac{d + ib}{k_F}) \Big) - \dfrac{n - bi - d}{k_F}f_W^2  - ni +bi + d \right) \\
\det(\mathcal{J}(h_D^*, f_W^*)) &= \dfrac{f_W}{(1 + f_W)^2} \left( - P_F'(f_W^*) - \dfrac{n - bi - d}{k_F}(f_W^*)^2  - ni +bi + d \right)
\end{align*}

We can also compute the trace of $\mathcal{J}(h_D^*, f_W^*)$. We have:

\begin{align*}
\Tr(\mathcal{J}(h_D^*, f_W^*)) &= - d + \dfrac{nf_W}{1+f_W} + (1+bh_D)(1 - \dfrac{2f_W}{k_F}) - \dfrac{n}{(1 + f_W)^2}h_D \\
\Tr(\mathcal{J}(h_D^*, f_W^*)) &= - d + \dfrac{nf_W}{1+f_W} -(1+b h_D) \dfrac{f_W}{k_F} + \dfrac{n h_D f_W}{(1+f_W)^2}\\
\end{align*}

In the special case where $\beta = 0 \Leftrightarrow b = 0$, using $h_D =  \dfrac{(1 - \dfrac{f_W}{k_F})(1+ f_W)}{n}$, we have:

\begin{align*}
\Tr(\mathcal{J}(h_D^*, f_W^*)) &= - d + \dfrac{nf_W}{1+f_W} - \dfrac{f_W}{k_F} + \dfrac{n f_W}{(1+f_W)^2} \times\dfrac{(1 - \dfrac{f_W}{k_F})(1+ f_W)}{n} \\
\Tr(\mathcal{J}(h_D^*, f_W^*)) &= \dfrac{-d + (n-d)f_W}{1+f_W} - \dfrac{f_W(1+f_W)}{k_F(1+f_W)} + \dfrac{f_W}{(1+f_W)}(1 - \dfrac{f_W}{k_F}) \\
\Tr(\mathcal{J}(h_D^*, f_W^*))(1+f_W) &= -\Big(d - (n-d)f_W\Big) + \Big( k_F - 2f_W - 1\Big)\dfrac{f_W}{k_F} \\
\end{align*}

Since $h_D^* = \dfrac{i}{d - \dfrac{n f_W}{1 + f_W}}$, we have $d - (n - d)f_W^* > 0$.

When:

\begin{itemize}
\item $n-d < 0 \Leftrightarrow em - \theta(\mu_D - f_D) < 0$, we have $f_W = \dfrac{k_F}{2} + \dfrac{d}{2(n-d)} + \dfrac{\sqrt{\Delta_f} k_F}{2(n-d)}$
\item $n-d > 0 \Leftrightarrow em - \theta(\mu_D - f_D) > 0$, we have $f_W = \dfrac{k_F}{2} + \dfrac{d}{2(n-d)} - \dfrac{\sqrt{\Delta_f} k_F}{2(n-d)}$
\end{itemize}

This gives:
$$
k_F - 2f_W - 1 = -\dfrac{d}{n-d} \mp \dfrac{\sqrt{\Delta_f} k_F}{(n-d)}
$$
and
$$
d - (n-d)f_W =  \dfrac{d}{2} - \dfrac{(n-d)k_F}{2} \mp \dfrac{\sqrt{\Delta_f} k_F}{2}
$$ and
$$
\dfrac{f_W}{k_F} = \dfrac{1}{2} + \dfrac{d}{2(n-d)k_F} \pm \dfrac{\sqrt{\Delta_f}}{2(n-d)}
$$
Therefore,

\begin{align*}
\Tr(\mathcal{J}(h_D^*, f_W^*))(1+f_W) &= -\Big(d - (n-d)f_W\Big) + \Big( k_F - 2f_W - 1\Big)\dfrac{f_W}{k_F} \\
\Tr(\mathcal{J}(h_D^*, f_W^*))(1+f_W) &= -\Big(\dfrac{d}{2} - \dfrac{(n-d)k_F}{2} \mp \dfrac{\sqrt{\Delta_f} k_F}{2}\Big) + \Big( -\dfrac{d}{n-d} \mp \dfrac{\sqrt{\Delta_f} k_F}{(n-d)}\Big)\dfrac{f_W}{k_F} \\
\end{align*}
We have:

\begin{align*}
\Big( -\dfrac{d}{n-d} \mp \dfrac{\sqrt{\Delta_f} k_F}{(n-d)}\Big)\dfrac{f_W}{k_F} &= -\dfrac{d}{2(n-d)} - \dfrac{d^2}{2(n-d)^2k_F} \mp \dfrac{d \sqrt{\Delta_f}}{2(n-d)^2} \mp \dfrac{\sqrt{\Delta_f} k_F}{2(n-d)} \mp \dfrac{d \sqrt{\Delta_f}}{2(n-d)^2} - \dfrac{\Delta_f k_F}{2(n-d)^2} \\
\Big( -\dfrac{d}{n-d} \mp \dfrac{\sqrt{\Delta_f} k_F}{(n-d)}\Big)\dfrac{f_W}{k_F} &= -\dfrac{d}{2(n-d)} - \dfrac{d^2}{2(n-d)^2k_F} \mp \dfrac{d \sqrt{\Delta_f}}{(n-d)^2} \mp \dfrac{\sqrt{\Delta_f} k_F}{2(n-d)} - \dfrac{\Delta_f k_F}{2(n-d)^2} \\
%\Big( -\dfrac{d}{n-d} \mp \dfrac{\sqrt{\Delta_f} k_F}{(n-d)}\Big)\dfrac{f_W}{k_F} &= \left(-\dfrac{d(n-d)}{2} - \dfrac{d^2}{2k_F} \mp d \sqrt{\Delta_f} \mp \dfrac{\sqrt{\Delta_f}(n-d) k_F}{2} - \dfrac{\Delta_f k_F}{2} \right) \dfrac{1}{(n-d)^2}
\end{align*}

Then,
\begin{align*}
\Tr(\mathcal{J}(h_D^*, f_W^*))(1+f_W) &= -\Big(\dfrac{d}{2} - \dfrac{(n-d)k_F}{2} \mp \dfrac{\sqrt{\Delta_f} k_F}{2}\Big) + \Big( -\dfrac{d}{n-d} \mp \dfrac{\sqrt{\Delta_f} k_F}{(n-d)}\Big)\dfrac{f_W}{k_F} \\
&= -\Big(\dfrac{d}{2} - \dfrac{(n-d)k_F}{2} \mp \dfrac{\sqrt{\Delta_f} k_F}{2}\Big) -\dfrac{d}{2(n-d)} - \dfrac{d^2}{2(n-d)^2k_F} \mp \dfrac{d \sqrt{\Delta_f}}{(n-d)^2} \mp \dfrac{\sqrt{\Delta_f} k_F}{2(n-d)} - \dfrac{\Delta_f k_F}{2(n-d)^2} \\
&= -\dfrac{d}{2} + \dfrac{(n-d)k_F}{2} \pm \dfrac{\sqrt{\Delta_f} k_F}{2}-\dfrac{d}{2(n-d)} - \dfrac{d^2}{2(n-d)^2k_F} \mp \dfrac{d \sqrt{\Delta_f}}{(n-d)^2} \mp \dfrac{\sqrt{\Delta_f} k_F}{2(n-d)} - \dfrac{\Delta_f k_F}{2(n-d)^2}\\
&= -\dfrac{d(n-d +1)}{2(n-d)} + \dfrac{(n-d)k_F}{2} \pm \dfrac{\sqrt{\Delta_f} k_F}{2(n-d)} \dfrac{n-d - 1}{1} - \dfrac{d^2}{2(n-d)^2k_F} \mp \dfrac{d \sqrt{\Delta_f}}{(n-d)^2}- \dfrac{\Delta_f k_F}{2(n-d)^2}
\end{align*}


% \begin{prop}
% When $\cI > 0$, the following result holds.
% \begin{itemize}
% \item System \eqref{QSSA eq} has a Human-only equilibrium $EE^{H} = \Big(\dfrac{\cI}{\mu_D - f_D}, 0, \dfrac{m \cI}{\mu_D - f_D} \Big)$ that always exists.
% \item When
% $$ \mathcal{N}_{\cI >0} :=  \dfrac{r_F(1-\alpha)\Big({\dfrac{\mu_D - f_D}{m\cI}+\beta\Big)}}{\lfw}  > 1,$$
% system \eqref{QSSA eq} has a unique coexistence equilibrium $EE^{HF_W}_{\cI = 0} = \Big(H^*_{D}, F^*_{W}, m H^*_{D} \Big)$
% where
% $$F^*_{W} = \dfrac{(1-\alpha)K_F}{2}\left(1 - \dfrac{\sqrt{\Delta_F}}{e(1-\alpha)r_F}\right) + \dfrac{\mu_D - f_D + \cI \beta m}{2\lfw m e},\quad
% H^*_{D} = \dfrac{(1-\alpha)r_F\Big(1 - \dfrac{F^*_{W}}{(1-\alpha)K_F} \Big)}{m\left(\lfw - \beta (1-\alpha) r_F + \beta r_F  \dfrac{F^*_{W}}{K_F}\right)},
% \quad 
% H^*_{W} = m H^*_{D}$$
% and
% $$
% \Delta_F = \left(e(1-\alpha)r_F - \dfrac{(\mu_D - f_D) r_F}{\lfw m K_F}\right)^2 + \dfrac{\cI \beta r_F}{\lfw K_F} \left(\dfrac{\cI \beta r_F}{\lfw K_F} + 2\dfrac{(\mu_D - f_D) r_F}{\lfw m K_F} + 2e(1-\alpha)r_F \right) + 4\dfrac{er_F}{K_F}  \cI\Big(1 - \dfrac{(1-\alpha)\beta r_F}{\lfw} \Big)
% $$
% \end{itemize} 
% \end{prop}

% The following proposition stands the local stability of the equilibrium.


% \begin{prop}
% When $\cI = 0$, system \eqref{QSSA eq} admits:
% \begin{itemize}
% \item A trivial equilibrium which is unstable.
% \item A Fauna equilibrium, which is LAS if $\mathcal{N}_{\cI = 0} < 1$.
% \item When $\mathcal{N}_{\cI = 0} > 1$, the coexistence equilibrium $EE^{HF_W}$ exists. It is LAS without condition.
% \end{itemize}

% When $\cI > 0$, system \eqref{QSSA eq} admits:
% \begin{itemize}
% \item A Human equilibrium, which is LAS if $\mathcal{N}_{\cI > 0} < 1$.
% \item When $\mathcal{N}_{\cI > 0} > 1$, the coexistence equilibrium $EE^{HF_W}$ exists. It is LAS without condition.
% \end{itemize}
% \end{prop}

% \begin{proof}

% To investigate the local stability of the equilibrium of the system \eqref{QSSA eq}, we can use its Jacobian matrix. It is given by:

% \begin{multline}
% \mathcal{J}_{QSSA}(H_D, F_W) = \\ \begin{bmatrix}
% -(\mu_D - f_D) + e \lfw m F_W & e \lfw m H_D \\
% m\left(-\lfw + \beta (1-\alpha) r_F \Big(1- \dfrac{F_W}{(1-\alpha)K_F} \Big) \right) F_W & (1-\alpha) (1+\beta m H_D) r_F \left(1 - \dfrac{2F_W}{K_F} \right) - \lfw m H_D
% \end{bmatrix}
% \end{multline}

% \begin{itemize}
% \item At equilibrium $TE$, the Jacobian is given by:
% \begin{equation*}
% \mathcal{J}_{QSSA}(TE) = \begin{bmatrix}
% -(\mu_D - f_D) &0 \\
% 0 & (1-\alpha)  r_F 
% \end{bmatrix}
% \end{equation*}
% $(1-\alpha) r_F > 0$ is a positive eigenvalue, an therefore $TE$ is unstable.

% \item At equilibrium $EE^{F_W}$, the Jacobian is given by: 
% \begin{equation*}
% \mathcal{J}_{QSSA}(EE^{F_W}) = \begin{bmatrix}
% -(\mu_D - f_D) + e\lfw m K_F(1-\alpha) &0 \\
% - m \lfw (1-\alpha)K_F & -(1-\alpha)  r_F 
% \end{bmatrix}
% \end{equation*}
% $-(1-\alpha)  r_F$ and $-(\mu_D - f_D) + e\lfw m K_F(1-\alpha)$ are the eigenvalues. They are both negative if $-(\mu_D - f_D) + e\lfw m K_F(1-\alpha) <0 \Leftrightarrow \mathcal{N}_{\cI = 0} < 1$

% \item At equilibrium $EE^{H}$, the Jacobian is given by:
% \begin{equation*}
% \mathcal{J}_{QSSA}(EE^{F_W}) = \begin{bmatrix}
% -(\mu_D - f_D) &  \dfrac{e \lfw m \cI}{\mu_D - f_D} \\
% 0 & (1-\alpha)\Big(1+ \dfrac{m \beta \cI}{\mu_D - f_D}\Big)  r_F -  \dfrac{\lfw m  \cI}{\mu_D - f_D}
% \end{bmatrix}
% \end{equation*}
% $-(\mu_D - f_D)$ and $(1-\alpha)\Big(1+ \dfrac{\cI}{\mu_D - f_D}\Big)  r_F -  \dfrac{\lfw m  \cI}{\mu_D - f_D}$ are the eigenvalues. They are both negative if $(1-\alpha)\Big(1+ \dfrac{m \beta \cI}{\mu_D - f_D}\Big)  r_F -  \dfrac{\lfw m  \cI}{\mu_D - f_D} <0 \Leftrightarrow \mathcal{N}_{\cI > 0} < 1$


% \item At equilibrium $EE^{HF_W}_{\cI \geq 0}$, the Jacobian is given by:

% \begin{align*}
% \mathcal{J}_{QSSA}(EE^{HF_W}_{\cI \geq 0}) &= \\&\begin{bmatrix}
% -(\mu_D - f_D) + e \lfw m F_W & e \lfw m H_D \\
% m\left(-\lfw + \beta (1-\alpha) r_F \Big(1- \dfrac{F_W}{(1-\alpha)K_F} \Big) \right) F_W & (1-\alpha) (1+\beta m H_D) r_F \left(1 - \dfrac{2F_W}{K_F} \right) - \lfw m H_D
% \end{bmatrix} \\
%  &\begin{bmatrix}
% -\dfrac{\cI}{H_D} & e \lfw m H_D \\
% m\left(-\lfw + \beta (1-\alpha) r_F \Big(1- \dfrac{F_W}{(1-\alpha)K_F} \Big) \right) F_W & -(1+\beta m H_D) r_F \dfrac{F_W}{K_F} 
% \end{bmatrix}
% \end{align*}

% using equilibrium formulas. $EE^{HF_W}_{\cI \geq 0}$ is LAS if the trace of $\mathcal{J}_{QSSA}(EE^{HF_W}_{\cI \geq 0}) $ is negative and its determinant positive. We have:

% \begin{equation*}
% \Tr(\mathcal{J}_{QSSA}(EE^{HF_W}_{\cI \geq 0})) = -\dfrac{\cI}{H_D} -(1+\beta m H_D) r_F \dfrac{F_W}{K_F} 
% \end{equation*}
% that is $\Tr(\mathcal{J}_{QSSA}(EE^{HF_W}_{\cI \geq 0})) < 0$.

% When $\cI = 0$, the determinant is simply:

% \begin{align*}
% \det(\mathcal{J}_{QSSA}(EE^{HF_W}_{\cI = 0})) &= - m^2 e \lfw \left(-\lfw + \beta (1-\alpha) r_F \Big(1- \dfrac{F_W}{(1-\alpha)K_F} \Big) \right) F_W H_D, \\
% &= m^2 e \lfw^2 \left(1 + \dfrac{\beta (1-\alpha) r_F}{\lfw} \Big(1- \dfrac{\mu_D - f_D}{me \lfw(1-\alpha)K_F} \Big) \right) F_W H_D, \\
% \end{align*}

% using the equilibrium value. The proposition \ref{propBeta} shows that this last expression is positive. Consequently $\det(\mathcal{J}_{QSSA}(EE^{HF_W}_{\cI = 0})) > 0$ and $EE^{HF_W}_{\cI = 0}$ is LAS.

% When $\cI > 0$, the determinant is:

% \begin{align*}
% \det(\mathcal{J}_{QSSA}(EE^{HF_W}_{\cI > 0})) &= \dfrac{\cI}{H_D} \dfrac{1 + \beta m H_D}{K_F} r_F F_W - m^2 e \lfw \left(-\lfw + \beta(1-\alpha)r_F \Big(1- \dfrac{F_W}{(1-\alpha) K_F} \Big) \right) F_W H_D  \\
% &= \dfrac{\cI}{H_D} \dfrac{1 + \beta m H_D}{K_F} r_F F_W + m^2 e \lfw \left(\lfw - \beta(1-\alpha)r_F + r_F \beta\dfrac{F_W}{ K_F} \Big) \right) F_W H_D  \\
% \end{align*}

% Using proposition \ref{propPF}, we have that $(1-\alpha)K_F - \dfrac{\lfw K_F}{\beta r_F} < F_W$. Consequently $\det(\mathcal{J}_{QSSA}(EE^{HF_W}_{\cI > 0})) > 0$ and $EE^{HF_W}_{\cI > 0}$ is also LAS.
% \end{itemize}

% \end{proof}

% The following propositions aim to assess the global stability of the equilibrium.

% \begin{prop}
% The system \eqref{QSSA eq} admits no limit cycle.
% \end{prop}

% \begin{proof}
% The Bendixson-Dulac criterion with the function $\phi(H_D, F_W) = \dfrac{1}{H_D F_W}$ will show the result. We note $f$ the right hand side of equations \eqref{QSSA eq}. We have:

% \begin{equation*}
% (\phi \times f_1)(H_D, F_W) = \dfrac{\cI}{H_D F_W} -\dfrac{\mu_D - f_D}{F_W} - e\lfw m
% \end{equation*} and therefore

% \begin{equation*}
% \dfrac{\partial (\phi f_1)}{\partial H_D}(H_D, F_W) = - \dfrac{\cI}{F_W H_D^2} <0
% \end{equation*}

% On the other hand, we also have:
% \begin{equation*}
% (\phi \times f_2)(H_D, F_W) = - m \lfw + \dfrac{(1-\alpha) (1+ \beta m H_D) r_F}{H_D} - \dfrac{(1+\beta m H_D) r_F}{H_D} \dfrac{F_W}{K_F}
% \end{equation*} and therefore

% \begin{equation*}
% \dfrac{\partial (\phi f_2)}{\partial F_W}(H_D, F_W) = - \dfrac{(1+\beta m H_D) r_F}{H_D K_F} <0
% \end{equation*}

% Consequently, $\dfrac{\partial (\phi f_1)}{\partial H_D} + \dfrac{\partial (\phi f_2)}{\partial F_W} < 0$ and according to the Bendixson-Dulac criterion, the system \eqref{QSSA eq} does not admit any limit cycle.

% \end{proof}


% \begin{prop}
% Under the assumptions of proposition , the equilibrium which are LAS are GAS.
% \end{prop}



























% \section{Model analysis in the case without immigration}
% In this section, we study the specific case where there is no immigration. This mean that the human population mainly dependents on hunt and food production to subsist. The system rewrites:
% \begin{equation}
% \def\arraystretch{2}
% \left\{ \begin{array}{l}
% \dfrac{dH_D}{dt}= e\dfrac{\lfw F_W}{1 + \lfw \theta F_W} H_W + (f_D - \mu_D) H_D - m_D H_D + m_W H_W, \\
% \dfrac{dF_W}{dt} = r_F(1- \alpha) (1+ \beta H_W) \left(1 - \dfrac{F_W}{(1-\alpha)K_F} \right) F_W - \dfrac{\lfw F_W}{1 + \lfw \theta F_W} H_W \\
% \dfrac{dH_W}{dt}= m_D H_D - m_W H_W 
% \end{array} \right.
% \label{equationsHDFWHW, cI=0}
% \end{equation}


% \subsection{Qualitative analysis - long term dynamics}
% On the following, we study the existence and stability of equilibrium of model \eqref{equationsHDFWHW, cI=0}. 


% \begin{prop}
% \label{theoremEquilibre, cI=0}
% The following results hold:
% \begin{itemize}
% \item System \eqref{equationsHDFWHW, cI=0} admits a trivial equilibrium $TE = \Big(0,0,0\Big)$ and a fauna-only equilibrium $EE^{F_W} = \Big(0, (1-\alpha)K_F, 0 \Big)$ that always exist.

% \item When
% $$
% \mathcal{N}_{\cI = 0} := \dfrac{m e \lfw (1-\alpha)K_F}{\mu_D - f_D} >1,
% $$ 
% then system \eqref{equationsHDFWHW, cI=0} admits a unique coexistence equilibrium $EE^{HF_W} = \Big(H^*_{D, \cI = 0}, F^*_{W, \cI = 0}, H^*_{W, \cI = 0} \Big)$ \\ 
% where 


% $$F^*_{W, \cI = 0} = \dfrac{\mu_D - f_D}{\lfw m e},
% \quad 
% H^*_{D, \cI = 0} = \dfrac{(1-\alpha)r_F\Big(1 - \dfrac{F^*_{W, \cI = 0}}{K_F(1-\alpha)} \Big)}{m\left(\lfw - \beta (1-\alpha) r_F + \beta r_F  \dfrac{F^*_{W, \cI = 0}}{K_F}\right)} ,
% \quad 
% H^*_{W, \cI = 0} = m H^*_{D, \cI = 0}.$$
% \end{itemize}
% \end{prop}

% \begin{proof}
% To derive the equilibrium we solve \eqref{equationsHDFWHW, cI=0} with $\dfrac{d y}{dt} = 0$. Therefore, an equilibrium satisfies the system of equations:
% \begin{equation}\label{system-equilibre, cI=0}
% \def\arraystretch{2}
% \left\lbrace \begin{array}{cll}
%  e \lfw m F_W^* + f_D - \mu_D = 0& \mbox{or} & H_D^* = 0,\\
% m H_D^*\Big(\lfw - (1-\alpha)r_F \beta + \dfrac{r_F \beta}{K_F}F_W^* \Big) + r_F \dfrac{F_W^*}{K_F} - (1-\alpha)r_F= 0& \mbox{or} & F^*_W = 0,\\
% H_W^* = \dfrac{m_D}{m_W} H_D^* = m H_D^*.&&
% \end{array} \right.
% \end{equation}
% When $H_D^*=0$ and $F_W^*=0$, we recover the trivial equilibrium $TE = \Big(0,0,0\Big)$. When $H_D^*=0$ and $F_W^*\neq0$, we obtain the fauna-only equilibrium $EE^{F_W} = \Big(0, K_F(1-\alpha), 0 \Big)$. Finally, when $H_D^*\neq0$ and $F_W^*\neq0$, direct computations lead to a unique set of values given by:
% $$F^*_{W} = \dfrac{\mu_D - f_D}{\lfw m e},
% \quad 
% H^*_{D} = \dfrac{(1-\alpha)r_F\Big(1 - \dfrac{F^*_{W}}{K_F(1-\alpha)} \Big)}{m\left(\lfw - \beta (1-\alpha) r_F + \beta r_F  \dfrac{F^*_{W}}{K_F}\right)} ,
% \quad 
% H^*_{W} = m H^*_{D}.$$

% Those values are biologically meaningful when $F_W^* \leq (1-\alpha) K_F$ and when $H_D^*$ is positive. The first inequality gives the constraint $\dfrac{\mu_D - f_D}{\lfw m e} \leq (1-\alpha)K_F$. From now, we assume it. The numerator of $H^*_{D}$ is non negative, and positive when $\dfrac{\mu_D - f_D}{\lfw m e} < (1-\alpha)K_F$. We need to check the sign of its denominator, which has to be positive. We have:

% \begin{align*}
% \lfw - \beta (1-\alpha) r_F + \beta r_F  \dfrac{F^*_{W}}{K_F} &= \lfw\Big(1 - \dfrac{\beta (1-\alpha) r_F}{\lfw} + \beta r_F  \dfrac{\mu_D - f_D}{\lfw^2 m e K_F} \Big) \\
% &= \lfw\left(1 - \dfrac{\beta (1-\alpha) r_F}{\lfw}\Big(1 -\dfrac{\mu_D - f_D}{\lfw m e K_F(1-\alpha)} \Big) \right)
% &\geq 0,
% \end{align*}

% thanks to proposition \ref{propBeta}. Therefore, the equilibrium of coexistence is biologically meaningful if $\dfrac{\mu_D - f_D}{\lfw m e} < (1-\alpha)K_F \Leftrightarrow 1 < \dfrac{\lfw (1-\alpha)K_F m e}{\mu_D - f_D}$

% \end{proof}

% Now, we look for the local asymptotic stability of the equilibrium.

% \begin{prop}\label{propLAS, cI=0} The following results are valid.
% \begin{itemize}
% \item The trivial equilibrium $TE$ is unstable.
% \item When $\mathcal{N}_{\cI = 0} < 1$, the fauna equilibrium, $EE^{F_W}$, is Locally Asymptotically Stable (LAS).
% \item When $\mathcal{N}_{\cI = 0} > 1$, the coexistence equilibrium, $EE^{HF_W}_{\cI =0}$, exists. It is LAS if $\Delta_{Stab} > 0$ where 
% \begin{multline*}
% \Delta_{Stab, \cI =0} = \Big(\mu_D - f_D + m_D + (1+\beta H_W^*)r_F \dfrac{F_W^*}{K_F} + m_W\Big) \times \\ \big( \mu_D  -f_D + m_D + m_W \big) r_F(1+ \beta H_W^*) \dfrac{F^*_W}{K_F} - 
% m_D e \lfw (1- \alpha) r_F \left(1 - \dfrac{F_W^*}{(1- \alpha)K_F}\right) F_W^*,
% \end{multline*}
% and unstable when $\Delta_{Stab, \cI =0} < 0$
% \end{itemize}
% \end{prop}



% \begin{proof}
% To prove this theorem, we look at the Jacobian of system \eqref{equationsHDFWHW, cI=0}. It is given by:

% \begin{multline*}
% \mathcal{J}(H_D, F_W, H_W) = \\
% \begin{bmatrix}
% f_D-\mu_D - m_D & e \lfw H_W & e\lfw F_W + m_W \\
% 0 & r_F(1-\alpha)(1+\beta H_W) \left( 1 - \dfrac{2F_W}{K_F(1-\alpha)} \right) - \lfw H_W & \Big((1-\alpha)\beta r_F - \lfw \Big) F_W -  \dfrac{r_F\beta}{K_F} F_W^2\\
% m_D & 0 & -m_W
% \end{bmatrix}.
% \end{multline*}

% \begin{itemize}
% \item At equilibrium $TE$, we have:
% \begin{equation*}
% \mathcal{J}(TE) = \begin{bmatrix}
% f_D-\mu_D - m_D & 0 &  m_W \\
% 0 & r_F(1-\alpha)  &  0\\
% m_D & 0 & -m_W
% \end{bmatrix}.
% \end{equation*}
% and $r_F > 0$ is an eigenvalue of $\mathcal{J}(TE)$. So, $TE$ is unstable.
% \item At equilibrium $EE^{F_W}$, we have
% \begin{equation*}
% \mathcal{J}(EE^{F_W}) = \begin{bmatrix}
% f_D-\mu_D - m_D & 0 & e\lfw K_F(1-\alpha) + m_W \\
% 0 & -(1-\alpha)r_F  & -\lfw(1-\alpha)K_F  \\
% m_D & 0 & -m_W
% \end{bmatrix}.
% \end{equation*}

% The characteristic polynomial of $\mathcal{J}(EE^{F_W})$ is given by:
% \begin{equation*}
% \chi(X) = (X +r_F) \times \left(X^2 - X\Big(f_D - \mu_D - m_D - m_W \Big) + m_W(\mu_D - f_D) - m_D e \lfw K_F(1-\alpha) \right).
% \end{equation*}

% We need to determine the sign of the roots' real part of the second factor. Since the coefficient in $X$ is positive, the sign of their real part is determined by the sign of the constant coefficient.
% The roots have a negative real part if the constant coefficient is positive \textit{ie} if $\dfrac{m e \lfw K_F(1-\alpha)}{\mu_D - f_D} < 1 $, and a positive real part if $\dfrac{m e \lfw K_F(1-\alpha)}{\mu_D - f_D} > 1 $. Stability of $EE^{F_W}$ follows.

% \item Now, we look for the asymptotic stability of the equilibrium of coexistence $EE^{HF_W}_{\cI=0}$. The first part of the computations are common with the ones for proving the LAS of equilibrium $EE^{HF_W}_{\cI >0}$.   To keep some generality, we use notation $EE^{HF_W}$ for both $EE^{HF_W}_{\cI =0}$ and $EE^{HF_W}_{\cI >0}$. We have


% \begin{equation*}
% \mathcal{J}(EE^{H F_W}) = \begin{bmatrix}
% f_D -\mu_D - m_D & e \lfw H_W^* & e \lfw F^*_W +m_W \\
% 0 & -(1 + \beta H_W^*)r_F \dfrac{F_W^*}{K_F} & \big( (1-\alpha)\beta r_F - \lfw \big) F_W^* -  \dfrac{r_F\beta}{K_F} (F_W^*)^2 \\
% m_D & 0 & -m_W
% \end{bmatrix}.
% \end{equation*} 

% Its characteristic polynomial is given by: $\chi = X^3 + a_2 X^2 + a_1 X + a_0$. In particular, we know that $a_2 = - \Tr(\mathcal{J}(EE^{H F_W}))$ and $a_0 = - \det (\mathcal{J}(EE^{H F_W}))$.

% According to the Routh-Hurwitz criterion \marc{ref}, $EE^{H F_W}$ is LAS if $a_i > 0$ for $i=1,2,3$ and $a_2 a_1 - a_0 > 0$.

% We have:
% \begin{align}
% a_2 &= - \Tr\Big(\mathcal{J}(EE^{H F_W})\Big) \\
%  &= -(f_D - \mu_D - m_D - (1+\beta H_W^*)r_F \dfrac{F_W^*}{K_F} - m_W) \\
%  &= \mu_D - f_D + m_D + (1+\beta H_W^*)r_F \dfrac{F_W^*}{K_F} + m_W \label{expressionA2}
% \end{align}
% that is $a_2>0$. Coefficient $a_0$ is given by:

% \begin{subequations}
% \begin{align}
% a_0 &= -\det\Big(\mathcal{J}(EE^{H F_W})\Big), \\
% a_0 &= \Big(\mu_D + m_D -f_D \Big) m_W (1+\beta H_W^*) r_F \dfrac{F^*_W}{K_F}  - m_D (1 + \beta H_W^*) r_F \dfrac{F_W^*}{K_F}(e\lfw F_W^* + m_W) + \\
% \nonumber
% &  m_D e \lfw  \left((\lfw - (1-\alpha)\beta r_F)  + \dfrac{r_F\beta}{K_F} F_W^* \right)H_W^* F_W^* \\
% a_0 &= \Big(\mu_D -f_D \Big) m_W (1+\beta H_W^*) r_F \dfrac{F^*_W}{K_F}  - m_D e\lfw (1 + \beta H_W^*) r_F \dfrac{(F_W^*)^2}{K_F} + \\
% \nonumber
% &  m_D e \lfw \left((\lfw - (1-\alpha)\beta r_F)  + \dfrac{r_F\beta}{K_F} F_W^* \right)H_W^*F_W^* \\
% a_0 &= \Big(\mu_D -f_D \Big) m_W (1+\beta H_W^*) r_F \dfrac{F^*_W}{K_F}  - m_D e\lfw (1 + \beta H_W^*) r_F \dfrac{(F_W^*)^2}{K_F} + \\
% \nonumber
% &  m_D e \lfw (1- \alpha) r_F \left(1 - \dfrac{F_W^*}{(1- \alpha)K_F}\right) F_W^* \\
% a_0 &= e \lfw m_D r_F (1 + \beta H_W^*) \left(\dfrac{\mu_D -f_D }{e \lfw m} - F_W^*\right) \dfrac{F_W^*}{K_F} + m_D e \lfw (1- \alpha) r_F \left(1 - \dfrac{F_W^*}{(1- \alpha)K_F}\right) F_W^*  \\
% a_0 &= e \lfw m_D r_F \left(\dfrac{\mu_D -f_D }{e \lfw m K_F} - 2\dfrac{F_W^*}{K_F} + (1-\alpha) + \dfrac{\beta H_W^*}{K_F} \left(\dfrac{\mu_D -f_D }{e \lfw m} - F_W^*\right) \right) F_W^*  \label{expressionA0}
% \end{align}
% \end{subequations}

% When $\cI = 0$, we have:

% \begin{equation*}
% F_W^* = \dfrac{\mu_D - f_D}{\lfw m e}.
% \end{equation*} 
% Injecting this expression into \eqref{expressionA0}, we obtain:

% \begin{equation*}
% a_{0, \cI=0} = e \lfw m_D r_F  (1- \alpha) \left(1 - \dfrac{F_W^*}{(1- \alpha)K_F}\right) F_W^* 
% \end{equation*}
% that is $a_{0, \cI=0}>0$. The coefficient $a_1$ is given by:
% \begin{subequations}
% \begin{align}
% a_1 &= \big( \mu_D  -f_D + m_D) r_F(1+ \beta H_W^*) \dfrac{F^*_W}{K_F} + (\mu_D -f_D + m_D) m_W + r_F(1+ \beta H_W^*) \dfrac{F_W^*}{K_F} m_W - \\ \nonumber &m_D (e\lfw F^*_W + m_W), \\
% a_1 &= \big( \mu_D  -f_D + m_D + m_W) r_F(1+ \beta H_W^*) \dfrac{F^*_W}{K_F} + (\mu_D -f_D) m_W  - m_D e\lfw F^*_W, \\
% a_1 &= \big( \mu_D  -f_D + m_D + m_W) r_F(1+ \beta H_W^*) \dfrac{F^*_W}{K_F} + \left(\dfrac{\mu_D -f_D}{e\lfw m} - F_W^*\right) e \lfw m_D . \label{expressionA1}
% \end{align}
% \end{subequations}

% Again, using the expression of $F^*_W$ in the case where $\cI = 0$, we have:

% \begin{equation*}
% a_{1, \cI =0} = \big( \mu_D  -f_D + m_D + m_W) r_F(1+ \beta H_W^*) \dfrac{F^*_W}{K_F} .
% \end{equation*}
% and we do have $a_{1, \cI =0} > 0$.

% The first assumption of Rough-Hurwitz is verified, $a_{i, \cI =0} > 0$ for $i=1,2,3$. Therefore, the asymptotic stability of $EE^{HF_W,  \cI =0}$ only depends on the sign of $\Delta_{Stab}= a_2 a_1 - a_0$, which has to be positive. 
% \end{itemize}
% \end{proof}


% \begin{prop}
% In the special case where $\beta = 0$, condition $\Delta_{Stab, \cI =\beta =0} > 0$ is equivalent to $\dfrac{\lfw}{\lfw^*} < 1$ where
% \begin{multline*}
% \lfw^* = \\
%  \dfrac{\left[m_{W}(\mu_{D}-f_{D})+\big(\mu_{D}-f_{D}+m_{D}+m_{W})^{2}\right]\left(1+\sqrt{1+4\dfrac{(1-\alpha)m_{W}r_{F}\left(\mu_{D}-f_{D}\right)\big(\mu_{D}-f_{D}+m_{D}+m_{W})}{\left[m_{W}\dfrac{\mu_{D}-f_{D}}{e}+\big(\mu_{D}-f_{D}+m_{D}+m_{W})^{2}\right]^{2}}}\right)}{2em_D (1-\alpha) K_F }
% \end{multline*}
% \end{prop}

% \begin{proof}
% When $\beta = 0$, we have
% \begin{multline} \label{DeltaStab, generalCase}
% \Delta_{Stab, \cI =\beta = 0} =  \left(\mu_D - f_D + m_D + m_W + r_F\dfrac{F_W^*}{K_F} \right) \\ \times   \left( \mu_D -f_D + m_D + m_W \right) - m_D e \lfw \Big(K_F(1-\alpha) - F_W^* \Big),
% \end{multline}

% with $F_W^* = \dfrac{\mu_D - f_D}{\lfw m e}$. Therefore,

% \begin{multline*}
% \Delta_{Stab, \cI=\beta = 0} > 0 \\
% \Leftrightarrow \left(\mu_D - f_D + m_D + m_W + r_F \dfrac{\mu_D - f_D}{\lfw K_F m e} \right) \times   \left( \mu_D -f_D + m_D + m_W \right) > \\ m_D e \lfw \left(K_F(1-\alpha) - \dfrac{\mu_D - f_D}{\lfw m e} \right), \\
% \Leftrightarrow (\mu_D - f_D + m_D + m_W)^2 + r_F \dfrac{\mu_D - f_D}{\lfw K_F m e}  \times   \left( \mu_D -f_D + m_D + m_W \right) > \\ m_D e \lfw K_F(1-\alpha) - (\mu_D - f_D)m_W , \\
% \Leftrightarrow \lfw (\mu_D - f_D + m_D + m_W)^2 + r_F \dfrac{\mu_D - f_D}{K_F m e}  \times   \left( \mu_D -f_D + m_D + m_W \right) > \\ m_D e \lfw^2 K_F(1-\alpha) - \lfw (\mu_D - f_D)m_W , \\
% \Leftrightarrow 0 > \lfw^2 (1-\alpha) K_F  m_D e - \lfw \Big((\mu_D - f_D + m_D + m_W)^2 +(\mu_D - f_D)m_W \Big) - \\ \dfrac{r_F (\mu_D - f_D) }{K_F m e}  \big( \mu_D -f_D + m_D + m_W \big).\\
% \end{multline*}

% We define 
% \begin{multline*}
% P_{\Delta_{Stab, \cI= \beta = 0}}(X) := X^2 (1-\alpha) K_F  m_D e - X \Big((\mu_D - f_D + m_D + m_W)^2 +(\mu_D - f_D)m_W \Big) - \\ \dfrac{r_F (\mu_D - f_D) m_W}{K_F m_D e}  \big( \mu_D -f_D + m_D + m_W \big),
% \end{multline*} 

% such that we have 
% \begin{equation}
% \Delta_{Stab, \cI= \beta = 0} > 0 \Leftrightarrow P_{\Delta_{Stab, \cI= \beta = 0}}(\lfw) < 0.
% \label{equivalenceDeltaStabP}
% \end{equation}

% $P_{\Delta_{Stab, \cI = \beta = 0}}$ has a positive dominant coefficient, and its other coefficients are negative. So,  $P_{\Delta_{Stab, \cI= \beta = 0}}$ admits a unique positive root, noted $\lfw^*$, given by:
% \begin{multline}
% \lfw^* = \\
%  \dfrac{\left[m_{W}(\mu_{D}-f_{D})+\big(\mu_{D}-f_{D}+m_{D}+m_{W})^{2}\right]\left(1+\sqrt{1+4\dfrac{(1-\alpha)m_{W}r_{F}\left(\mu_{D}-f_{D}\right)\big(\mu_{D}-f_{D}+m_{D}+m_{W})}{\left[m_{W}\dfrac{\mu_{D}-f_{D}}{e}+\big(\mu_{D}-f_{D}+m_{D}+m_{W})^{2}\right]^{2}}}\right)}{2em_D (1-\alpha) K_F }
% \end{multline}

% Moreover, $P_{\Delta_{Stab, \cI = \beta = 0}}$ is negative on $\left[0, \lfw^* \right)$ and positive on $\left(\lfw ^*, +\infty \right)$. Using \eqref{equivalenceDeltaStabP}, we obtain that $EE^{HF_W}_{\cI = \beta = 0}$ is locally asymptotically stable if $\lfw  < \lfw ^*$.
% \end{proof}

% \marc{
% When $\beta > 0$, we obtain:
% \begin{equation*}
% \Delta_{Stab, \cI = 0} > 0 \Leftrightarrow 0 > a_5 \lfw ^5 + a_4 \lfw ^4 + a_3 \lfw ^3 + \beta a_2 \lfw ^2 + \beta a_1 \lfw ^1 + \beta a_0
% \end{equation*}
% with $a_5, a_1 > 0$ and $a_4, a_2, a_0 < 0$ and no clear sign for coefficient $a_3$. Therefore, the polynomial may have (if $a_3<0$ 2 or 0 positive roots) or (if $a_3 > 0$, 5, 3 or 1 positive roots).}

% %
% %The following intermediate table gives an overview of the long term behavior:
% %
% %\begin{table}[!ht]
% %\centering
% %\def\arraystretch{2}
% %\begin{tabular}{c|c|c|c|c}
% %$\cI$ & $\beta$ & $\dfrac{m e\lfw K_F(1-\alpha)}{\mu_D - f_D}$ &  $\dfrac{\lfw}{ \Big( \lfw \Big)^*}$ & \\
% %\hline
% %  \multirow{3}{*}{$=0$}&\multirow{3}{*}{$=0$}  & $ < 1$ & &$EE^{F_W}$ exists and is LAS. \\
% %   \cline{3-5}
% %& &\multirow{2}{*}{$ > 1$} & $<1$ &$EE^{HF_W}_{\cI=0}$ exists and is LAS.\\
% % \cline{4-5}
% % & & &$>1$ &$EE^{HF_W}_{\cI=0}$ exists and is unstable. 
% %\end{tabular}
% %\caption{\centering Intermediate table giving the known conditions of existence and asymptotic stability of equilibrium for system \eqref{equationsHDFWHW, cI=0}}
% %\end{table}
% We look now for global stability. The following propositions hold true:

% \begin{prop}\label{propEEFGAS}If 
% $$
% \mathcal{N}_{I =0} < 1,
% $$
% that is if equilibrium $EE^{F_W}$ is LAS, then it is globally asymptotically stable (GAS) on $\Omega$ for system \eqref{equationsHDFWHW, cI=0}.
% \end{prop}

% \begin{proof}
% In the following, we assume $ \mathcal{N}_{I =0} < 1$. We consider a solution $(H_D^s, F_W^s, H_W^s)$ of equations \eqref{equationsHDFWHW, cI=0} with initial conditions in $\Omega$. Using the fact that $\Omega$ is an invariant region, we have:

% \begin{equation}
% \def\arraystretch{2}
% \left\{ \begin{array}{l}
% \dfrac{dH^s_D}{dt} \leq e\lfw H^s_W K_F(1-\alpha) + (f_D - \mu_D) H^s_D - m_D H^s_D + m_W H^s_W , \\
% \dfrac{dF^s_W}{dt} = (1-\alpha)(1 + \beta H_W^s) r_F \left(1 - \dfrac{F^s_W}{K_F(1-\alpha)} \right) F^s_W - \lfw F^s_W H^s_W \\
% \dfrac{dH^s_W}{dt}= m_D H^s_D - m_W H^s_W 
% \end{array} \right.
% \end{equation}

% We consider the limit system, given by:
% \begin{equation}
% \def\arraystretch{2}
% \left\{ \begin{array}{l}
% \dfrac{dH_D}{dt} = \Big(e\lfw K_F(1-\alpha) + m_W\Big)H_W + (f_D - \mu_D - m_D) H_D \\
% \dfrac{dF_W}{dt} =(1-\alpha)(1 + \beta H_W) r_F \left(1 - \dfrac{F_W}{K_F(1-\alpha)} \right) F_W - \lfw F_W H_W \\
% \dfrac{dH_W}{dt}= m_D H_D - m_W H_W 
% \end{array} \right.
% \label{limitSystem}
% \end{equation}

% We will apply theorem \ref{theoremVidyasagar} on this system, with $x = (H_D, H_W)$, $y = F_W$, $x^* = (0,0)$ and $y^* = K_F(1- \alpha)$.

% We have that $x^*$ is GAS for system $\dfrac{dx}{dt} = f_{[1,3]}(x)$. Indeed, $x^*$ is the unique equilibrium of this system, and it is LAS since $\dfrac{\mu_D - f_D}{\lfw m e K_F(1-\alpha)} >1$. By applying the Bendixson-Dulac theorem \marc{ref}, we show that $\dfrac{dx}{dt} = f_{[1,3]}(x)$ does not admit any limit cycle. Therefore, the Poincarré theorem \marc{ref} shows that $(0, 0)$ is GAS for $\dfrac{dx}{dt} = f_{[1,3]}(x)$.

% It is quite immediate to show that $y^*$ is GAS for system $\dfrac{dy}{dt} = f_{[2]}(x^*, y)$. 

% Moreover, the trajectories of the solution of limit-system \eqref{limitSystem} with initial condition in $\Omega$ are bounded (proposition \ref{invariantRegion}). So, we can apply theorem \ref{theoremVidyasagar}, and we obtain that equilibrium $\Big(0, K_F(1-\alpha), 0 \Big)$ is GAS on $\Omega$ for the limit system, and therefore for the original system \eqref{equationsHDFWHW}


% \end{proof}


% \begin{prop}\label{LimitCycle, cI=0}
% If $\mathcal{N}_{I =0} > 1$, and if:
% \begin{itemize}
% \item $\Delta_{stab, \cI =0} > 0$, that is if equilibrium $EE^{HF_W}$ is LAS, then it is GAS on $\Omega$ for system \eqref{equationsHDFWHW, cI=0}.
% \item $\Delta_{stab, \cI =0} < 0$, system \eqref{equationsHDFWHW, cI=0} admits an orbitally asymptotically stable periodic solution.
% \end{itemize}

% Equivalently, when $\beta = 0$, we have the following ; if:
% \begin{itemize}
% \item $\lfw <  \lfw^*$, that is if equilibrium $EE^{HF_W}$ is LAS, then it is GAS on $\Omega$ for system \eqref{equationsHDFWHW, cI=0}.
% \item $\lfw  > \lfw^*$, system \eqref{equationsHDFWHW, cI=0} admits an orbitally asymptotically stable periodic solution.
% \end{itemize}
% \end{prop}

% \begin{proof}
% When $\mathcal{N}_{I =0} > 1$, system \eqref{equationsHDFWHW, cI=0} admits a unique positive and AS equilibrium, $EE^{HF_W}$.  

% When $\lfw \geq (1-\alpha)r_F \beta$, the equivalent system \eqref{equationshDfWhW} is competitive on $\Omega$. By applying theorem \ref{theorem: periodicASOrbit} to this system, we know that either $EE^{HF_W}$ is GAS, or it exists a asymptotically stable periodic solution. 
% \medskip

% On the other hand, when $\lfw < (1-\alpha)r_F \beta$, the equivalent system \eqref{equationshDfWhW} is only competitive on $\Omega_2$. Direct computations show that $EE^{HF_W} \in \Omega_2$. Therefore, we can apply theorem \ref{theorem: periodicASOrbit} to the equivalent system and on $\Omega_2$, and we know that either $EE^{HF_W}$ is GAS on $\Omega_2$, or it exists a asymptotically stable periodic solution in $\Omega_2$. 

% Moreover, we know, thanks to proposition \eqref{propDivisionOmega}, that any solutions starting in $\Omega_1 = \Omega \backslash \Omega_2$ enter in $\Omega_2$. Therefore, we can extend the previous result on the whole domain $\Omega$.

% Condition for stability is precisely $0 < \Delta_{stab, \cI =0}$, which is equivalent to $\lfw < \lfw^*$ when $\beta = 0$.
% \end{proof} 

% The results we obtained are summarized in the following table:
% \begin{table}[!ht]
% \centering
% \def\arraystretch{2}
% \begin{tabular}{c|c|c|c|c}
% $\cI$ &$\beta$ & $\mathcal{N}_{I =0}$ &  $\dfrac{\lfw}{  \lfw ^*}$ & \\
% \hline
% \multirow{4}{*}{$=0$}&\multirow{4}{*}{$=0$} & $ < 1$ & &$EE^{F_W}$ exists and is GAS.  \\
% \cline{3-5}
%  & & \multirow{3}{*}{$> 1$} & $ <1$ &$EE^{HF_W}_{\cI=0}$ exists and is GAS.\\
%  \cline{4-5}
%  & & &\multirow{2}{*}{$ > 1$} & $EE^{HF_W}_{\cI=0}$ exists and is unstable ; there is an asymptotically \\
% & & & &  stable periodic solution.
% \end{tabular}
% \caption{\centering Conditions of existence and asymptotic stability of equilibrium for system \eqref{equationsHDFWHW, cI=0}, when $\beta = 0$}
% \end{table}


% \begin{table}[!ht]
% \centering
% \def\arraystretch{2}
% \begin{tabular}{c|c|c|c|c}
% $\cI$ &$\beta$ & $\mathcal{N}_{I =0}$ &  $\Delta_{Stab, \cI =0}$ & \\
% \hline
% \multirow{4}{*}{$=0$}&\multirow{4}{*}{$>0$} & $ < 1$ & &$EE^{F_W}$ exists and is GAS.  \\
% \cline{3-5}
%  & & \multirow{3}{*}{$> 1$} & $ >0$ &$EE^{HF_W}_{\cI=0}$ exists and is GAS.\\
%  \cline{4-5}
%  & & &\multirow{2}{*}{$ <0 $} & $EE^{HF_W}_{\cI=0}$ exists and is unstable ; there is an asymptotically \\
% & & & &  stable periodic solution.
% \end{tabular}
% \caption{\centering Conditions of existence and asymptotic stability of equilibrium for system \eqref{equationsHDFWHW, cI=0}}
% \end{table}

% \subsection{Discussion}
% Inside this model, the parameters $\alpha$ and $\lfw$ represents the impact of human activities on their environment. It is interesting to interpret the previous results as a function of this parameters, in order to understand the consequences of an increase (or decrease) of hunt activities or environment destruction.

% First, we start by condition $\dfrac{m e \lfw  K_F(1-\alpha)}{\mu_D - f_D} > 1$, which is required for $EE^{HF_W}$ to exist. We propose the following definition:

% \begin{prop}\label{defLambdaMin, cI=0} We define 
% $$\lambda_{F, \cI=0}^{Min} := \dfrac{\mu_D - f_D}{m e K_F(1-\alpha)}$$
% such that 
% $$
% \text{$EE^{HF_W}_{\cI = 0}$ exists} \Leftrightarrow  \lfw > \lambda_{F, \cI=0}^{Min}.
% $$
% \end{prop}
% %\begin{proof}
% %The proof is straightforward: $EE^{HF_W}$ exists if $\dfrac{m e \lfw K_F(1-\alpha)}{\mu_D - f_D} > 1 \Leftrightarrow  \lfw> \dfrac{\mu_D - f_D}{ m e K_F(1-\alpha)} $.
% %\end{proof}

% This means that if the hunting rate is not sufficient, there is no co-existence possible, and even no steady state with a human population. This is due to the fact that when $\cI = 0$, only local activities ensure food intake. Consequently, if there is not enough hunt, the human population can not subsist.

% We can note that $\lambda_{F, \cI=0}^{Min}$ is a increasing function of the anthropization parameter $\alpha$ : the more anthropized the environment, the fewer wild animals there are, and the greater the hunting rate required. 

% \begin{prop}
% When $\beta = 0 $, we define
% \begin{multline*}
% \lambda_{F, \cI = \beta =0}^{Max}  = \\
% \dfrac{\left[m_{W}(\mu_{D}-f_{D})+\big(\mu_{D}-f_{D}+m_{D}+m_{W})^{2}\right]\left(1+\sqrt{1+4\dfrac{(1-\alpha)m_{W}r_{F}\left(\mu_{D}-f_{D}\right)\big(\mu_{D}-f_{D}+m_{D}+m_{W})}{\left[m_{W}\dfrac{\mu_{D}-f_{D}}{e}+\big(\mu_{D}-f_{D}+m_{D}+m_{W})^{2}\right]^{2}}}\right)}{2em_D (1-\alpha) K_F }
% \end{multline*}
% such that $$
% \text{$EE^{HF_W}_{\cI = \beta = 0}$ exists and is GAS} \Leftrightarrow \lambda_{F, \cI=0}^{Min} < \lfw < \lambda_{F, \cI =\beta =0}^{Max}
% .$$
% \end{prop}

% This result can be interpreted as follow: if the hunting rate is too high, the system's dynamic tends toward a limit cycle, which correspond to a classical predator-prey system.

% \begin{figure}[!ht]
% \centering
% \begin{subfigure}{0.49\textwidth}
% \centering
% \includegraphics[width=\textwidth]{SurfaceLambdaMinAlone.png}
% \caption{Only the surface $\lfw(\Kfa, m)^{min}$ is plot.}
% \end{subfigure}
% \begin{subfigure}{0.49\textwidth}
% \centering
% \includegraphics[width=\textwidth]{SurfaceLambdaMaxAlone.png}
% \caption{Both surfaces $\lfw(\Kfa, m)^{min}$ and $\lfw(\Kfa, m)^{max}$ are plot.}
% \end{subfigure}
% \hfill
% \begin{subfigure}{\textwidth}
% \includegraphics[width=1\textwidth]{BifurcationLambdaCurve.png}
% \caption{}
% \end{subfigure}
% \caption{\centering Bifurcation diagram for system \eqref{equationsHDFWHW, cI=0}. Two first figures are plot in the $(K_F(1-\alpha), m, \lfw)$ space. The third one is a cut for $m=0.2$. Other parameters value are $r_F = 0.8$, $K_F=3000$, $e=0.1$, $\cI=0$, $\mu_D = 0.017$, $f_D = 0.001$, $\beta =0$}
% \end{figure}


% \section{Model analysis in the case with immigration, $\cI > 0$} \label{section:with immigration}
% Now, we consider the case were food import occurs, \textit{ie} we assume $\cI > 0$. Since the human subsistence does not depend only on hunt, the system's dynamic and its interpretation change.

% \subsection{Theoretical analysis}
% \begin{prop} \label{equilibrium, I>0}
% The following results hold:
% \begin{itemize}
% \item System \eqref{equationsHDFWHW} has a Human-only equilibrium $EE^{H} = \Big(\dfrac{\cI}{\mu_D - f_D}, 0, \dfrac{m \cI}{\mu_D - f_D} \Big)$ that always exists.
% \item When
% $$ \mathcal{N}_{\cI >0} :=  \dfrac{r_F(1-\alpha)\Big({\dfrac{\mu_D - f_D}{m\cI}+\beta\Big)}}{\lfw}  > 1,$$
% system \eqref{equationsHDFWHW} has a unique coexistence equilibrium $EE^{HF_W}_{\cI = 0} = \Big(H^*_{D}, F^*_{W}, m H^*_{D} \Big)$
% where
% $$F^*_{W} = \dfrac{(1-\alpha)K_F}{2}\left(1 - \dfrac{\sqrt{\Delta_F}}{e(1-\alpha)r_F}\right) + \dfrac{\mu_D - f_D + \cI \beta m}{2\lfw m e},\quad
% H^*_{D} = \dfrac{(1-\alpha)r_F\Big(1 - \dfrac{F^*_{W}}{(1-\alpha)K_F} \Big)}{m\left(\lfw - \beta (1-\alpha) r_F + \beta r_F  \dfrac{F^*_{W}}{K_F}\right)},
% \quad 
% H^*_{W} = m H^*_{D}$$
% and
% $$
% \Delta_F = \left(e(1-\alpha)r_F - \dfrac{(\mu_D - f_D) r_F}{\lfw m K_F}\right)^2 + \dfrac{\cI \beta r_F}{\lfw K_F} \left(\dfrac{\cI \beta r_F}{\lfw K_F} + 2\dfrac{(\mu_D - f_D) r_F}{\lfw m K_F} + 2e(1-\alpha)r_F \right) + 4\dfrac{er_F}{K_F}  \cI\Big(1 - \dfrac{(1-\alpha)\beta r_F}{\lfw} \Big)
% $$
% \end{itemize} 
% \end{prop}

% \begin{proof}
% An equilibrium of system \eqref{equationsHDFWHW} satisfies the system of equations:
% \begin{equation}\label{systemEquilibre}
% \left\lbrace \begin{array}{cll}
% \cI + e \lfw m F_W^* H_D^* + (f_D - \mu_D) H_D^* = 0,&&\\
% F_W^* - \dfrac{(1-\alpha)K_F}{1 + \beta m H_D^*} \Big(1 - \dfrac{m(\lfw - (1-\alpha)\beta r_F) H^*_D}{(1-\alpha)r_F} \Big) = 0& \mbox{or} & F^*_W = 0,\\
% H_W^* = \dfrac{m_D}{m_W} H_D^* = m H_D^*.&&
% \end{array} \right.
% \end{equation}

% The solution of system \eqref{systemEquilibre} when $F_W^* = 0$ is the Human-only equilibrium $EE^{H} = \Big(\dfrac{\cI}{\mu_D - f_D}, 0, \dfrac{m \ \cI}{\mu_D - f_D} \Big)$.
% In the sequel, we assume that $F_W^* > 0$. In this case, $F^*_W$ is solution of the quadratic equation
% \begin{equation}
% P_F(X) := X^2 \left(\dfrac{er_F}{K_F} \right) - X \left(e(1-\alpha)r_F + \dfrac{(\mu_D - f_D) r_F}{\lfw m K_F} + \dfrac{\cI \beta r_F}{\lfw K_F} \right) + \left(\dfrac{(\mu_D - f_D)(1-\alpha) r_F}{\lfw m} - \cI\Big(1 - \dfrac{(1-\alpha)\beta r_F}{\lfw} \Big) \right) = 0.
% \end{equation}

% The polynomial $P_F$ is studied in proposition \ref{propPF}.
% It is shown that $P_F$ admits two real roots $F_1^* \leq F_2^*$, with $F_2^* > K_F(1- \alpha) > F_1^*$.

% To define an equilibrium, $F^*_W$ must be biologically meaningful, that is positive and lower than $(1-\alpha) K_F$. Therefore, $F_2^*$ is not biologically meaningful and $F_1^*$ it is only if it is positive, \textit{ie} only if  $\dfrac{(\mu_D - f_D) r_F}{\lfw m } > \cI\Big(1 - \dfrac{(1-\alpha)\beta r_F}{\lfw} \Big)$ (proposition \ref{propPF}).  $F_1^*$ is given by:
% $$F^*_1 = \dfrac{(1-\alpha)K_F}{2}\left(1 - \dfrac{\sqrt{\Delta_F}}{e(1-\alpha)r_F}\right) + \dfrac{\mu_D - f_D + \cI \beta m}{2\lfw m e}$$

% According to the first equation of system \eqref{systemEquilibre}, the value of $H_D^*$ at equilibrium is given by:

% $$
% H_D^* = \dfrac{\cI}{\mu_D - f_D - e \lfw m F_1^*}
% $$

% It is biologically meaningful if it is positive. Since $F_1^* < \dfrac{\mu_D - f_D}{e \lfw m}$ (proposition \ref{propPF}), it is the case. Finally, the equilibrium of coexistence exists if $\dfrac{(\mu_D - f_D) r_F}{\lfw m } > \cI\Big(1 - \dfrac{(1-\alpha)\beta r_F}{\lfw} \Big)$.
% \end{proof}



% Now, we look for the asymptotic stability of the equilibrium.

% \begin{prop}\label{propLAS} The following results are valid.
% \begin{itemize}
% \item When $\mathcal{N}_{\cI > 0} < 1$, the human equilibrium $EE^{H}$ is LAS.
% \item When $\mathcal{N}_{\cI > 0} > 1$, equilibrium of coexistence $EE^{HF_W}$  exists. it is LAS if 
% $$\Delta_{Stab, \cI > 0} > 0,$$  where 

% \begin{multline*}
% \Delta_{Stab, \cI > 0} = \left(\mu_D -f_D + m_D + (1+\beta H_W^*)r_F \dfrac{F_W^*}{K_F} + m_W  \right) \times \\ \left(\big( \mu_D  -f_D + m_D + m_W) r_F(1+ \beta H_W^*) \dfrac{F^*_W}{K_F} + \left(\dfrac{\mu_D -f_D}{e\lfw m} - F_W^*\right) e \lfw m_D \right) - \\m_D \lfw e r_F \left(\dfrac{\sqrt{\Delta_F}}{er_F} - \dfrac{\cI \beta}{\lfw K_F e} +  \dfrac{\beta H_W^*}{K_F} \left(\dfrac{\mu_D -f_D }{e \lfw m} - F_W^*\right)\right)  F^*_{W}
% \end{multline*}
% \end{itemize}
% \end{prop}

% \begin{proof}
% To assess the local stability or instability of the different equilibria, we look at the Jacobian matrix. The Jacobian of system \eqref{equationsHDFWHW} is given by

% \begin{multline*}
% \mathcal{J}(H_D, F_W, H_W) = \\
% \begin{bmatrix}
% f_D-\mu_D - m_D & e \lfw H_W & e\lfw F_W + m_W \\
% 0 & r_F(1-\alpha)(1+\beta H_W) \left( 1 - \dfrac{2F_W}{K_F(1-\alpha)} \right) - \lfw H_W & - (\lfw - (1-\alpha)\beta r_F) F_W -  \dfrac{r_F\beta}{K_F} F_W^2\\
% m_D & 0 & -m_W
% \end{bmatrix}.
% \end{multline*}


% \begin{itemize}
% \item At equilibrium $EE^{H}$, we have
% \begin{equation*}
% \mathcal{J}(EE^{H}) = \begin{bmatrix}
% f_D-\mu_D - m_D & e \lfw \dfrac{m \cI}{\mu_D - f_D} & m_W \\
% 0 & r_F(1-\alpha)(1+\beta\dfrac{m\cI}{\mu_D - f_D}) - \lfw\dfrac{m\cI}{\mu_D - f_D} & 0 \\
% m_D & 0 & -m_W
% \end{bmatrix}.
% \end{equation*}


% The characteristic polynomial of $\mathcal{J}(EE^{H})$ is given by:
% \begin{equation*}
% \chi(X) = \left(X - r_F(1-\alpha)(1+\beta\dfrac{m\cI}{\mu_D - f_D}) + \lfw\dfrac{m\cI}{\mu_D - f_D} \right) \times \left(X^2 - X\Big(f_D - \mu_D - m_D - m_W \Big) + m_W(\mu_D - f_D)\right).
% \end{equation*}

% The constant coefficient of the second factor and its coefficient in $X$ are positive. So, the roots of the second factor have a negative real part. Therefore, only the sign of $r_F(1-\alpha)(1+\beta\dfrac{m\cI}{\mu_D - f_D}) - \lfw\dfrac{m\cI}{\mu_D - f_D}$ determines the stability of $EE^{H}$. If it is negative, $EE^{H}$ is LAS and otherwise it is unstable.

% \item Now, we look for the asymptotic stability of the equilibrium of coexistence $EE^{HF_W}_{\cI > 0}$. We will reuse some of the computations done in the case $\cI = 0$, the Jacobian matrix being the same.

% The characteristic polynomial of $\mathcal{J}(EE^{H F_W})$ is given by: $\chi = X^3 + a_2 X^2 + a_1 X + a_0$, where expressions for $a_i$ are given by \eqref{expressionA2}, \eqref{expressionA1}, \eqref{expressionA0} respectively. According to the Routh-Hurwitz criterion \marc{ref}, $EE^{H F_W}$ are LAS if $a_i > 0$ for $i=1,2,3$ and $a_2 a_1 - a_0 > 0$.

% According to \eqref{expressionA2}, we have:
% \begin{equation*}
% a_2 = \mu_D -f_D + m_D + (1+\beta H_W^*)r_F \dfrac{F_W^*}{K_F} + m_W 
% \end{equation*}
% that is $a_2>0$. According to \eqref{expressionA0}, we have:
% \begin{equation*}
% a_0 = e \lfw m_D r_F \left(\dfrac{\mu_D -f_D }{e \lfw m K_F} - 2\dfrac{F_W^*}{K_F} + (1-\alpha) + \dfrac{\beta H_W^*}{K_F} \left(\dfrac{\mu_D -f_D }{e \lfw m} - F_W^*\right) \right) F_W^*
% \end{equation*}

% Using
% \begin{equation*}
% F_W^* = \dfrac{(1-\alpha)K_F}{2}\left(1 - \dfrac{\sqrt{\Delta_F}}{e(1-\alpha)r_F}\right) + \dfrac{\mu_D - f_D + \cI \beta m}{2\lfw m e},
% \end{equation*}
% we obtain
% \begin{equation*}
% a_0 = m_D \lfw e r_F \left(\dfrac{\sqrt{\Delta_F}}{er_F} - \dfrac{\cI \beta}{\lfw K_F e} +  \dfrac{\beta H_W^*}{K_F} \left(\dfrac{\mu_D -f_D }{e \lfw m} - F_W^*\right)\right)  F^*_{W},
% \end{equation*}

% Using proposition \eqref{propPF}, we know that $\dfrac{\mu_D -f_D }{e \lfw m} - F_W^* > 0$. We will show that  $\dfrac{\sqrt{\Delta_F}}{er_F} - \dfrac{\cI \beta}{\lfw K_F e}$ is also positive. According to the proposition \ref{equilibrium, I>0}, we have:

% \begin{multline*}
% \Delta_F = \left(e(1-\alpha)r_F - \dfrac{(\mu_D - f_D) r_F}{\lfw m K_F}\right)^2 + \dfrac{\cI \beta r_F}{\lfw K_F} \left(\dfrac{\cI \beta r_F}{\lfw K_F} + 2\dfrac{(\mu_D - f_D) r_F}{\lfw m K_F} + 2e(1-\alpha)r_F \right) + \\ 4\dfrac{er_F}{K_F}  \cI\Big(1 - \dfrac{(1-\alpha)\beta r_F}{\lfw} \Big)
% \end{multline*}
%  which gives $\Delta_F > \left(\dfrac{\cI \beta r_F}{\lfw K_F}\right)^2$ and so,

% \begin{equation*}
% \dfrac{\sqrt{\Delta_F}}{er_F} - \dfrac{\cI \beta}{\lfw K_F e} > 0.
% \end{equation*}

% Therefore, we obtain $a_0 > 0$.

% According to \eqref{expressionA1}, coefficient $a_1$ is given by:
% \begin{equation*}
% a_1 = \big( \mu_D  -f_D + m_D + m_W) r_F(1+ \beta H_W^*) \dfrac{F^*_W}{K_F} + \left(\dfrac{\mu_D -f_D}{e\lfw m} - F_W^*\right) e \lfw m_D .
% \end{equation*}

% which is positive, since we show that $\dfrac{\mu_D - f_D}{e \lfw m} > F^*_{W}$.

% The first assumption of the Rough-Hurwitz criteria is verified, $a_i > 0$ for $i=1,2,3$. Therefore, the local asymptotic stability of $EE^{HF_W}$ only depends on the sign of $\Delta_{Stab}= a_2 a_1 - a_0$, which has to be positive.
% \end{itemize}
% \end{proof}

% %For now, we have the following table:
% %\begin{table}[ht!]
% %\def\arraystretch{2}
% %\centering
% %\begin{tabular}{c|c|c|c}
% %$\cI$ & $\mathcal{N}_{\cI > 0} $ & $\Delta_{Stab, \cI > 0}$ & \\
% %\hline
% %\multirow{3}{*}{$>0$} & $<1$ & &$EE^{H}$ exists and is LAS \\
% %\cline{2-4}
% % & \multirow{2}{*}{$> 1$}  & $>0$ &$EE^{HF_W}_{\cI>0}$ exists and is LAS\\
% % \cline{3-4}
% % & & $ < 0$ & $EE^{HF_W}_{\cI>0}$ is unstable \\
% %\end{tabular}
% %\caption{Intermediate table summarizing the long term behavior}
% %\end{table}
% As before, we can complete it with information about global asymptotic stability and existence of limit cycles. 

% \begin{prop}
% The condition $\mathcal{N}_{I > 0} < 1$ implies $\lfw > (1-\alpha) \beta r_F$. Therefore, when $\mathcal{N}_{I > 0} < 1$, the equivalent system \eqref{equationshDfWhW} is competitive on $\Big\{z = (h_D, f_W, h_W) | 0 \leq h_D, f_W  \leq 0, h_W \leq 0 \Big\}$
% \end{prop}

% \begin{proof}
% Since $\mathcal{N}_{\cI > 0} = \dfrac{r_F(1-\alpha)}{\lfw}\dfrac{\mu_D - f_D}{m \cI} + \dfrac{r_F(1-\alpha) \beta}{\lfw}$, we have $\dfrac{r_F(1-\alpha) \beta}{\lfw} < \mathcal{N}_{\cI > 0}$. Therefore, $\mathcal{N}_{\cI > 0} < 1$ implies $\dfrac{r_F(1-\alpha) \beta}{\lfw} < 1$, which implies, according to proposition \ref{equivalentSystem}, that the equivalent system \eqref{equationshDfWhW} is competitive on $\Omega$.
% \end{proof}

% \begin{prop}
% If $$\mathcal{N}_{\cI > 0} < 1$$
% that is if equilibrium $EE^{H}$ is LAS, then it is GAS on $\Omega$ for system \eqref{equationsHDFWHW}.
% \end{prop}

% \begin{proof}
% When $\mathcal{N}_{\cI > 0} < 1$, it follows from the previous propositions that $EE^{H}$ is the only existing equilibrium, and it is LAS. Since the equivalent system \eqref{equationshDfWhW} is competitive on $\Omega$, $EE^{H}$ is GAS on $\Omega$.
% \end{proof}


% \begin{prop}
% If $\mathcal{N}_{\cI > 0} > 1$ and if 

% \begin{itemize}
% \item $\Delta_{Stab} > 0$, that is if equilibrium $EE^{HF_W}_{\cI >0}$ is LAS, then it is GAS on $\Omega$ for system \eqref{equationsHDFWHW}.
% \item $\Delta_{Stab} < 0$, system \eqref{equationsHDFWHW} admits an orbitally asymptotically stable periodic solution on $\Omega$
% \end{itemize}
% \end{prop}

% \begin{proof}
% If $\dfrac{r_F(1-\alpha) \beta}{\lfw} \leq 1$, then the equivalent system is competitive on $\Omega$ (proposition \ref{equivalentSystem}). Theorem \eqref{theorem: periodicASOrbit} gives the result.

% Now, we assume that $\dfrac{r_F(1-\alpha) \beta}{\lfw} > 1$. In this case, the equivalent system is competitive only on $\Omega_2$. As before, we will apply theorem \ref{theorem: periodicASOrbit}, but this time on $\Omega_2$. We need to verify that $\Omega_2$ satisfy the theorem's assumptions, specially that $EE^{HF_W}$ belongs to $\Omega_2$. According to proposition \ref{propPF}, we have $F_W^* > K_F(1-\alpha) - \dfrac{K_F \lfw}{r_F \beta} =  F_W^{compet}$, so it is the case. Therefore, either $EE^{HF_W}$ is GAS on $\Omega_2$, either there is an orbitally asymptotically stable periodic orbit in $\Omega_2$. Moreover, according to previous proposition, we know that any solution with positive initial conditions on $\Omega_1$ enters in $\Omega_2$. Therefore, we can extend the results obtained on $\Omega_2$ to $\Omega = \Omega_1 \cup \Omega_2$.

% \end{proof}


% \begin{table}[!ht]
% \def\arraystretch{2}
% \centering
% \begin{tabular}{c|c|c|c}
% $\cI$ & $\mathcal{N}_{\cI > 0} $ & $\Delta_{Stab, \cI > 0}$ & \\
% \hline
% \multirow{3}{*}{$>0$} & $<1$ & &$EE^{H}$ exists and is GAS on $\Omega$ \\
% \cline{2-4}
%  & \multirow{3}{*}{$> 1$}  & $>0$ &$EE^{HF_W}_{\cI>0}$ exists and is GAS on $\Omega$ \\
%  \cline{3-4}
%  & & \multirow{2}{*}{$ < 0$} & $EE^{HF_W}_{\cI>0}$ exists and is unstable ; there is an asymptotically \\
%  & & &  stable periodic solution. \\
% \end{tabular}
% \caption{\centering Conditions of existence and asymptotic stability of equilibrium for system \eqref{equationsHDFWHW}}
% \end{table}

% \subsection{Interpretation}

% \begin{prop}
% We define 
% $$\lambda_{F, \cI>0}^{Max} := r_F(1-\alpha)\Big({\dfrac{\mu_D - f_D}{m\cI}+\beta\Big)}$$
% such that 
% $$
% \text{$EE^{HF_W}_{\cI>0}$ exists} \Leftrightarrow  \lfw < \lambda_{F, \cI>0}^{Max}.
% $$
% \end{prop}

% When $\cI = 0$, the existence of the equilibrium of coexistence is implied by the condition $\lambda_{F, cI = 0}^{Min} < \lfw$, see definition \ref{defLambdaMin, cI=0}. When $\cI > 0$, we instead obtain a maximum bound for $\lfw$. This is due to the fact that when $\cI > 0$, a human population is always present (equilibrium $EE^{H}$ exists), and has to not over-hunt in order to preserve the wild fauna 
% % (we could rewrite the condition has $1 < r_F(1-\alpha)(\dfrac{1}{\lfw H^*_{W}} + \beta)$ : comparison between growth and intake).


% We can note that when $\cI > 0$, the existence of the equilibrium of coexistence does not depend on the level of anthropization, $K_F(1-\alpha)$ but the value of $F^*_W$ does. When $\cI = 0$, it is the opposite. 
% \marc{donc quand $\cI =0$, l'existence de la coexistence dépend la capacité du milieu, et la valeur de l'équilibre (=centre du cyle limite ?) uniquement de la chasse ; si on reprend les calculs, on doit vérifier ce seuil pour éviter que $H^*$ devienne négatif = on n'est pas sûr que la pop humaine puisse exister. Dans le cas $\cI > 0$, c'est l'existence de $F^*$ qui n'est pas sûr, et qui dépend de la pression de chasse.}




% \section{Numerical scheme }
% \begin{definition} A matrix $A \in \mathcal{M}_n (\mathbb{R})$ is called an $M$-matrix if\begin{itemize}
% \item all its off-diagonal term are negative
% \item and its real eigenvalues are positive.
% \end{itemize}

% If $A$ is a $M$-Matrix, then it is inverse positive, that is $A^{-1}$ exists and all its coefficients are positive.


% \end{definition}


% For a given $\Delta t>0$, we set $Y^n=\Big(H_D^n,F_W^n,H_W^n \Big)$ as an approximation of $y(t)=\Big(H_D(t),F_W(t),H_W(t)\Big)$ at $t=n\Delta t$, for $n=0,...,N$, where $T=N\Delta$.


% \subsection{Implicit NS scheme}
% We can consider the following implicit non standard scheme:

% \begin{equation} \label{NSImplicit scheme}
% \Big(I_3 - \phi(\Delta t) M(Y^n) \Big) Y^{n+1} = Y^{n} + \phi(\Delta t)V,
% \end{equation}
% where $I_3$ is the identity matrix, $V = \begin{bmatrix}
% \cI & 0 & 0
% \end{bmatrix}^T$ and 
% \begin{equation}
% M(Y^n) = \begin{bmatrix}
% f_D - \mu_D - m_D & e \lfw H_W^n & m_W \\
% 0 & r_F(1-\alpha)(1+\beta H_W^n)\left(1 - \dfrac{F_W^n}{K_F(1 - \alpha)} \right) - \lfw H_W^n & 0 \\
% m_D & 0 & -m_W
% \end{bmatrix}.
% \end{equation}


% If we choose $\phi(\Delta t)$ such that $I_3 - \phi(\Delta t) M(Y) $ is an $M$-matrix for all $Y$, we will obtain the conservation of the system's positivity. We have:


% \begin{multline}
% I_3 - \phi(\Delta t) M(Y)  = \\ \begin{bmatrix}
% 1 + \phi(\Delta t) \Big( \mu_D + m_D -f_D \Big) & - \phi(\Delta t) e \lfw H_W^n & -\phi(\Delta t) m_W \\
% 0 & 1 - \phi(\Delta t) \left((1-\alpha)(1+\beta H_W^n)r_F\left(1 - \dfrac{F_W^n}{K_F(1 - \alpha)} \right) - \lfw H_W^n \right)& 0 \\
% -\phi(\Delta t) m_D & 0 & 1 + \phi(\Delta t) m_W
% \end{bmatrix}.
% \end{multline}


% For all $Y \in \mathbb{R}^3_+$ and $\Delta t \geq 0$, the off-diagonals term are negative. Therefore, $I_3 - \phi(\Delta t) M(Y) $ is an $M$-matrix if its real eigenvalues are positive. Its characteristic polynomial is given by:
% \begin{multline}
% \chi = \left(X - 1 + \phi(\Delta t) \Big(r_F(1-\alpha)(1+\beta H_W^n)\Big(1 - \dfrac{F_W^n}{K_F(1 - \alpha)} \Big) - \lfw H_W^n \Big)\right) \times \\
% \left(X^2 - X \Big(2 + \phi(\Delta t) (\mu_D - f_D + m_D + m_W) \Big) + 1 + \phi(\Delta t) (\mu_D - f_D + m_D + m_W) + \phi(\Delta t)^2 m_W ( \mu_D - f_D) \right)
% \end{multline}
% The constant and dominant coefficient of the polynomial $$\left(X^2 - X \Big(2 + \phi(\Delta t) (\mu_D - f_D + m_D + m_W) \Big) + 1 + \phi(\Delta t) (\mu_D - f_D + m_D + m_W) + \phi(\Delta t)^2 m_W ( \mu_D - f_D) \right)$$ are positive, while its coefficient in $X$ is negative. Therefore, this polynomial admits either two complex and conjugate roots either two positive real roots.

% If we set
% \begin{equation}
% \phi(\Delta t) := \dfrac{1- e^{-Q \Delta t}}{Q}
% \end{equation}

% \marc{Si l'on suit la construction du schéma exact implicit pour $x' = a x$ , on devrait prendre
% \YD{
% \begin{equation}
% \phi(\Delta t) := \dfrac{e^{Q \Delta t} - 1}{Q}
% \end{equation}}
% mais il ne me semble pas y avoir de relation évidente entre les paramrètres ($r_F$, $\mu_D+m_D-f_D$ ou $m_W$) et $Q$ qui permettent d'avoir une M-matrix ; j'ai donc gardé la première formule, comme dans l'article ``Mathematical studies on the sterile insect technique for the Chikungunya disease and Aedes albopictus", Dumont \& Tchuenche)
% }

% with $Q \geq r_F(1 - \alpha)$, we have $1 - \phi(\Delta t)r_F(1-\alpha) \geq 0$ and the eigenvalue $$1 - \phi(\Delta t) \Big(r_F\Big(1 - \dfrac{F_W^n}{K_F(1 - \alpha)} \Big) - \lfw H_W^n \Big) = 1 - \phi(\Delta t)(1-\alpha)r_F + r_F \dfrac{1 + \beta H_W^n}{K_F}F^n_W + (\lfw - (1-\alpha) r_F  \beta) H^n_W $$ is positive. This show that $F(Y, \Delta t)$ is an $M$-matrix.

% \medskip
% A fixed point $Y^*$ of equation \eqref{NSImplicit scheme} verifies:
% \begin{align*}
% \Big(I_3 - \phi(\Delta t) M(Y^*) \Big) Y^* &= Y^* + \phi(\Delta t)V \\
%  M(Y^*) Y^* + V&= 0
% \end{align*}
% which is precisely the system of equations satisfied by the fixed points of the continuous system \eqref{equationsHDFWHW}. Therefore, the discrete and continuous system have the same fixed points.

% \YD{
% \begin{definition}
% A numerical scheme is called elementary stable whenever it has no other fixed points than those of the continuous system it approximates, the local stability of these fixed points is the same for both the discrete and the continuous dynamical systems for each value of $\Delta t$.
% \end{definition}
% Il faut donc vérifier cela....
% }

% \subsection{Explicit NSS \marc{a reprendre avec $\beta$ ?}}
%  We consider the following nonstandard scheme

% \begin{equation}
% \def\arraystretch{2}
% \left\{ \begin{array}{l}
% H_{D}^{n+1}=\Big(1-\phi(\Delta t) \left(\mu_{D}+m_{D}-f_{D}\right)\Big)H_{D}^{n}+ \phi(\Delta t)\Big(\cI+ \big(e\lambda_{F}F_{W}^{n+1} + m_{W}\big)H_{W}^{n}\Big)\\
% F_{W}^{n+1}=\dfrac{\left(1+\phi(\Delta t)r_{F}\right)}{1+\phi(\Delta t)\left(\dfrac{r_{F}}{K_{F}(1-\alpha)}F_{W}^{n}+\lambda_{F}H_{W}^{n}\right)}F_{W}^{n}\\ 
% H_{W}^{n+1}=\Big(1-\phi(\Delta t)m_{W}\Big)H_{W}^n+\phi(\Delta t)m_{D}H_{D}^n
% \end{array}\right.
% \end{equation}
% where $\phi(\Delta t)=\dfrac{1-e^{-Q\Delta t}}{Q}$, with $Q=\max\{\mu_D+m_D-f_D,m_W\}$. 
% It is straightforward to check that
% $$
% Y^n \geq 0 \Rightarrow Y^{n+1}\geq 0,\qquad \forall n\in \mathbb{N}.
% $$

% \section{Results}

% \begin{figure}[!ht]
% \centering
% \begin{subfigure}{0.48\textwidth}
% \centering
% \includegraphics[width=\textwidth]{LCI0.png}
% \caption{}
% \end{subfigure}
% \begin{subfigure}{0.48\textwidth}
% \centering
% \includegraphics[width=\textwidth]{LCI002.png}
% \caption{}
% \end{subfigure}
% \hfill
% \begin{subfigure}{0.49\textwidth}
% \includegraphics[width=1\textwidth]{LCI02.png}
% \caption{}
% \end{subfigure}
% \hfill
% \begin{subfigure}{0.49\textwidth}
% \includegraphics[width=1\textwidth]{LCI2.png}
% \caption{}
% \end{subfigure}
% \begin{subfigure}{0.49\textwidth}
% \includegraphics[width=1\textwidth]{LCI20.png}
% \caption{}
% \end{subfigure}
% \caption{\centering Orbits of solutions of system \eqref{equationsHDFWHW}, plot in the $(H_D+H_W ; F_W)$ plane, for different values of $\cI$. Other parameters value are $r_F = 4$, $K_F=3000$, $e=0.5$, $\alpha = 0.5$, $\mu_D = 0.017$, $f_D = 0.005$, $\beta =0$, $m_W = 0.4$, $m_D = 0.1$, $\lfw = 0.037$. The 4 first solutions converge toward a LC around $EE^{HF_W}$, the last one toward $EE^{H}$.}
% \end{figure}

% \newpage

% \section{Model with Holling-Type 2}
% \begin{equation}
% \def\arraystretch{2}
% \left\{ 
% \begin{array}{l}
% \dfrac{dH_D}{dt}= \cI + e\lfw H_W F_W + (f_D - \mu_D) H_D - m_D H_D + m_W H_W. \\
% \dfrac{dF_W}{dt} = r_F(1- \alpha) (1+ \beta H_W) \left(1 - \dfrac{F_W}{K_F(1-\alpha)} \right) F_W - \lfw \dfrac{H_W}{1 + \gamma H_W} F_W  \\
% \dfrac{dH_W}{dt}= m_D H_D - m_W H_W 
% \end{array} \right.
% \label{equationsHDFWHW2}
% \end{equation}

% Equilibrium when $\cI > 0$:
% \begin{equation}
% \def\arraystretch{2}
% \left\{ 
% \begin{array}{l}
% \cI + e\lfw m H_D F_W + (f_D - \mu_D) H_D = 0 \\
% r_F(1- \alpha) (1+ \beta H_W) \left(1 - \dfrac{F_W}{K_F(1-\alpha)} \right)  - \lfw \dfrac{H_W}{1 + \gamma H_W} = 0  \\
% H_W = m H_D
% \end{array} \right.
% \label{equationsHDFWHW2}
% \end{equation}

% Second equation gives:

% \begin{align*}
% r_F(1- \alpha) (1+ \beta H_W) \left(1 - \dfrac{F_W}{K_F(1-\alpha)} \right)  - \lfw \dfrac{H_W}{1 + \gamma H_W} &= 0 \\
% r_F(1- \alpha) ({1 + \gamma mH_D}) (1+ \beta m H_D) \left(1 - \dfrac{F_W}{K_F(1-\alpha)} \right)   &= \lfw mH_D \\
% - r_F(1- \alpha) ({1 + \gamma mH_D}) (1+ \beta m H_D) \dfrac{F_W}{K_F(1-\alpha)}  &= \lfw mH_D -r_F(1- \alpha) ({1 + \gamma mH_D}) (1+ \beta m H_D) \\
% \dfrac{F_W}{K_F(1-\alpha)}  &= 1 - \dfrac{\lfw mH_D}{r_F(1- \alpha) ({1 + \gamma mH_D}) (1+ \beta m H_D)}\\
% F_W  &= K_F(1-\alpha)\left(1- \dfrac{\lfw mH_D}{r_F(1- \alpha) ({1 + \gamma mH_D}) (1+ \beta m H_D)}\right)
% \end{align*}


% $F_W$ is positive if 

% $$
% \dfrac{\lfw}{r_F(1- \alpha)} mH_D <  1 + m (\gamma + \beta) H_D + m^2 \gamma \beta H_D^2
% $$

% $$
% 0 <  1 + \left(m (\gamma + \beta) - \dfrac{\lfw}{r_F(1- \alpha)} m \right) H_D + m^2 \gamma \beta H_D^2
% $$

% Injecting expression of $F_W$ on the first equation:
% $$
% \cI + e\lfw m H_D F_W + (f_D - \mu_D) H_D = 0
% $$
% $$
% \cI + e\lfw m H_D K_F(1-\alpha)\left(1- \dfrac{\lfw mH_D}{r_F(1- \alpha) ({1 + \gamma mH_D}) (1+ \beta m H_D)}\right) + (f_D - \mu_D) H_D = 0
% $$
% $$
% \cI + \left(f_D - \mu_D + e \lfw m K_F(1-\alpha) \right)H_D - e\lfw m H_D K_F(1-\alpha)\dfrac{\lfw mH_D}{r_F(1- \alpha) ({1 + \gamma mH_D}) (1+ \beta m H_D)} = 0
% $$

% \begin{multline*}
% \cI r_F(1- \alpha) ({1 + \gamma mH_D}) (1+ \beta m H_D) + r_F(1- \alpha) ({1 + \gamma mH_D}) (1+ \beta m H_D)\left(f_D - \mu_D +  e \lfw m K_F(1-\alpha) \right)H_D - \\ e\lfw m H_D K_F(1-\alpha)\lfw mH_D = 0
% \end{multline*}


% \begin{multline*}
% H_D^3 \left(r_F(1-\alpha) \gamma \beta m^2 \Big(f_D - \mu_D + e\lfw m K_F(1-\alpha) \Big) \right) \\
% + H_D^2 \left(e \lfw^2 m^2 K_F(1-\alpha) + \cI r_F(1-\alpha) \gamma \beta m^2 + r_F(1-\alpha) (\beta + \gamma) m \Big(f_D - \mu_D + e \lfw m K_F(1-\alpha) \Big) \right) + \\ H_D \left( \cI r_F(1-\alpha) \Big(\gamma + \beta\Big) m + r_F(1-\alpha)\Big(f_D - \mu_D + e\lfw m K_F(1-\alpha) \Big) \right)
% \\
% \cI r_F(1-\alpha) = 0
% \end{multline*}



%  \newpage






\bibliographystyle{plain}
\bibliography{Biblio/Math, Biblio/Context, Biblio/interactionsHumanEnvironmentModel}

\end{document}

