\documentclass{article}
\usepackage{graphicx,ulem} 
\usepackage{color}
\usepackage{amsfonts,amsmath}
\usepackage{amsthm}
\usepackage{empheq}
\usepackage{mathtools}
\usepackage{multirow}
%\usepackage{tikz}
\usepackage{titlesec}
\usepackage{caption}
%\usepackage{lscape}
\usepackage{graphicx}
\captionsetup{justification=justified}
\usepackage[toc,page]{appendix}
\usepackage{hyperref}
\usepackage{subcaption}
\usepackage{pdftricks}
\usepackage{xcolor}
\begin{psinputs}
\usepackage{amsfonts,amsmath}
	\usepackage{pstricks-add}
   \usepackage{pstricks, pst-node}
   \usepackage{multido}
   \newcommand{\lfw}{\lambda_{F}}
\end{psinputs}

\textheight240mm \voffset-23mm \textwidth160mm \hoffset-20mm

   \graphicspath{{./Images/}{../Schema}}
%\graphicspath{{Figures}}
\setcounter{secnumdepth}{4}
\titleformat{\paragraph}
{\normalfont\normalsize\bfseries}{\theparagraph}{1em}{}
\titlespacing*{\paragraph}
{0pt}{3.25ex plus 1ex minus .2ex}{1.5ex plus .2ex}

\newcommand{\lfd}{\lambda_{F, D}}
\newcommand{\lfw}{\lambda_{F}}
\newcommand{\Kfa}{K_{F,\alpha}}
\newcommand{\cI}{\mathcal{I}}

\newcommand{\marc}[1]{\textcolor{teal}{#1}}
\newcommand{\YD}[1]{\textcolor{magenta}{#1}}
\newcommand{\VY}[1]{\textcolor{blue}{#1}}

\DeclareMathOperator{\Tr}{Tr}
\newtheorem{theorem}{Theorem}
\newtheorem{prop}{Proposition}
\newtheorem{definition}{Definition}
\newtheorem{remark}{Remark}
\newtheorem{cor}{Corollary}
\newcommand*\phantomrel[1]{\mathrel{\phantom{#1}}}

\title{Zone invariante et limite sur $\beta$ }
\author{Marc Hétier, Yves Dumont  and Valaire Yatat-Djeumen}

\begin{document}

\maketitle
On considère le modèle suivant :
\begin{equation}
\def\arraystretch{2}
\left\{ 
\begin{array}{l}
\dfrac{dH_D}{dt}= \cI + e\lfw H_W F_W + (f_D - \mu_D) H_D - m_D H_D + m_W H_W. \\
\dfrac{dF_W}{dt} = r_F(1- \alpha) (1+ \beta H_W) \left(1 - \dfrac{F_W}{K_F(1-\alpha)} \right) F_W - \lfw F_W H_W \\
\dfrac{dH_W}{dt}= m_D H_D - m_W H_W 
\end{array} \right.
\label{equationsHDFWHW}
\end{equation}

qui donne, par changement de variable $h_W = -H_W$, $f_W = -F_W$ le modèle:


\begin{equation}
\left\{ \begin{array}{l}
\dfrac{dh_D}{dt}= \cI + e\lfw h_W f_W + (f_D - \mu_D) h_D - m_D h_D - m_W h_W, \\
\dfrac{df_W}{dt} = (1-\alpha)r_F \left(1 + \dfrac{1 - \beta h_W}{K_F(1-\alpha)}f_W \right) f_W + \Big(\lfw - (1-\alpha)\beta r_F \Big) f_W h_W, \\
\dfrac{dh_W}{dt}= -m_D h_D - m_W h_W.
\end{array} \right.
\label{equationshDfWhW}
\end{equation}


Lorsque 
\begin{equation}
(\mu_D - f_D) \dfrac{m_W}{m_D} > er_F (1-\alpha)^2 \beta \dfrac{K_F}{4},
\label{existenceZI}
\end{equation}
la zone invariante $$\Omega_2 = \Big\{\Big(H_D, F_W, H_W \Big) \in \mathbb{R}_+^3  \Big|H_D + H_W + eF_W \leq S^{max}, F_W \leq F_W^{max}, H_W \leq H_W^{max} \Big\},$$
 avec
$$
S^{max} = \dfrac{m_D + m_W}{m_D} \dfrac{\cI + \Big((\mu_D-f_D) + \dfrac{r_F}{4}(1-\alpha) \Big) e K_F (1-\alpha)}{(\mu_D - f_D) \dfrac{m_W}{m_D} - er_F(1-\alpha)^2 \beta \dfrac{K_F}{4}}
\quad
F_W^{max} = (1-\alpha)K_F,
$$
$$
H_W^{max} = \dfrac{\cI + \Big((\mu_D-f_D) + \dfrac{r_F}{4}(1-\alpha) \Big) e K_F (1-\alpha)}{(\mu_D - f_D) \dfrac{m_W}{m_D} - er_F(1-\alpha)^2 \beta \dfrac{K_F}{4}}
$$
est bien définie.


La condition d'existence  de la zone invariante, \eqref{existenceZI} donne:
\begin{align*}
(\mu_D - f_D) \dfrac{m_W}{m_D} &> er_F (1-\alpha)^2 \beta \dfrac{K_F}{4}, \\
\dfrac{\mu_D - f_D}{\dfrac{m_D}{m_W} e(1-\alpha) K_F } &> \dfrac{r_F (1-\alpha) \beta}{4}
\end{align*}

Dans le cas $\cI = 0$, l'équilibre de coexistence existe ssi:

\begin{equation}
\lfw > \dfrac{\mu_D - f_D}{\dfrac{m_D}{m_W} e (1-\alpha)K_F}
\end{equation}

Et donc, l'équilibre de coexistence et la zone invariante co-existe si:

$$
\lfw > \dfrac{\mu_D - f_D}{\dfrac{m_D}{m_W} e (1-\alpha)K_F} > \dfrac{r_F (1-\alpha) \beta}{4}
$$


On retrouve une inégalité entre $\lfw$ et $r_F(1- \alpha) \beta$.


Par ailleurs, le système $\eqref{equationshDfWhW}$ est compétitif pour $-(1-\alpha) K_F \leq f_W \leq 0$ ssi

\begin{align*}
\Big(\lfw - (1-\alpha)\beta r_F \Big) f_W - \beta r_F \dfrac{f_W^2}{K_F} &\leq 0 \\
\Big(\lfw - (1-\alpha)\beta r_F \Big) - \beta r_F \dfrac{f_W}{K_F} &\geq 0 \\
\lfw- (1-\alpha)\beta r_F \geq \beta r_F\dfrac{f_W}{K_F}
\end{align*}

et cette inégalité est vraie pour $f_W = -F_W \in [-(1-\alpha) K_F, 0]$ ssi $$\lfw- (1-\alpha)\beta r_F \geq 0$$


On peut noter que si l'on limite $\beta$ à $\beta < \dfrac{(\mu_D - f_D) \dfrac{m_W}{m_D}} {er_F (1-\alpha)^2K_F}$ (et non plus $\beta < \dfrac{4(\mu_D - f_D) \dfrac{m_W}{m_D}} {er_F (1-\alpha)^2 K_F}$), on obtient l'implication : (existence de la Zone invariante + de l'équilibre de coexistence)

$$
\lfw > \dfrac{\mu_D - f_D}{\dfrac{m_D}{m_W} e (1-\alpha)K_F} > r_F (1-\alpha) \beta
$$
donc compétitivité du système \eqref{equationshDfWhW}.

\end{document}

