\documentclass{article}
\usepackage{graphicx,ulem} 
\usepackage{color}
\usepackage{amsfonts,amsmath}
\usepackage{amsthm}
\usepackage{empheq}
\usepackage{mathtools}
\usepackage{multirow}
%\usepackage{tikz}
\usepackage{titlesec}
\usepackage{caption}
%\usepackage{lscape}
\usepackage{graphicx}
\captionsetup{justification=justified}
\usepackage[toc,page]{appendix}
\usepackage{hyperref}
\usepackage{subcaption}
\usepackage{pdftricks}
\usepackage{xcolor}
\begin{psinputs}
\usepackage{amsfonts,amsmath}
	\usepackage{pstricks-add}
   \usepackage{pstricks, pst-node}
   \usepackage{multido}
   \newcommand{\lfw}{\lambda_{F}}
\end{psinputs}

\textheight240mm \voffset-23mm \textwidth160mm \hoffset-20mm

   \graphicspath{{./Images/}{../Schema}}
%\graphicspath{{Figures}}
\setcounter{secnumdepth}{4}
\titleformat{\paragraph}
{\normalfont\normalsize\bfseries}{\theparagraph}{1em}{}
\titlespacing*{\paragraph}
{0pt}{3.25ex plus 1ex minus .2ex}{1.5ex plus .2ex}

\newcommand{\lfd}{\lambda_{F, D}}
\newcommand{\lfw}{\lambda_{F}}
\newcommand{\Kfa}{K_{F,\alpha}}
\newcommand{\cI}{\mathcal{I}}

\newcommand{\marc}[1]{\textcolor{teal}{#1}}
\newcommand{\YD}[1]{\textcolor{magenta}{#1}}
\newcommand{\VY}[1]{\textcolor{blue}{#1}}

\DeclareMathOperator{\Tr}{Tr}
\newtheorem{theorem}{Theorem}
\newtheorem{prop}{Proposition}
\newtheorem{definition}{Definition}
\newtheorem{remark}{Remark}
\newtheorem{cor}{Corollary}
\newcommand*\phantomrel[1]{\mathrel{\phantom{#1}}}

\title{Zone invariante et limite sur $\beta$ }
\author{Marc Hétier, Yves Dumont  and Valaire Yatat-Djeumen}

\begin{document}

\maketitle
On considère le modèle suivant :
\begin{equation}
\def\arraystretch{2}
\left\{ 
\begin{array}{l}
\dfrac{dH_D}{dt}= e\lfw H_W F_W + (f_D - \mu_D) H_D - m_D H_D + m_W H_W. \\
\dfrac{dF_W}{dt} = r_F(1- \alpha) (1+ \beta H_W) \left(1 - \dfrac{F_W}{K_F(1-\alpha)} \right) F_W - \lfw F_W H_W \\
\dfrac{dH_W}{dt}= m_D H_D - m_W H_W 
\end{array} \right.
\label{equationsHDFWHW, cI=0}
\end{equation}


\begin{prop}
\label{theoremEquilibre, cI=0}
The following results hold:
\begin{itemize}
\item System \eqref{equationsHDFWHW, cI=0} admits a trivial equilibrium $TE = \Big(0,0,0\Big)$ that always exists.
\item System \eqref{equationsHDFWHW, cI=0} admits a fauna-only equilibrium $EE^{F_W} = \Big(0, (1-\alpha)K_F, 0 \Big)$ that always exists.
\item When
$$
\mathcal{N}_{\cI = 0} >1,
$$ 
then system \eqref{equationsHDFWHW, cI=0} admits a unique coexistence equilibrium $EE^{HF_W} = \Big(H^*_{D, \cI = 0}, F^*_{W, \cI = 0}, H^*_{W, \cI = 0} \Big)$ \\ 
where 

\begin{equation*}
\mathcal{N}_{\cI = 0} := \dfrac{m e \lfw (1-\alpha)K_F}{\mu_D - f_D}
\end{equation*}

$$F^*_{W, \cI = 0} = \dfrac{\mu_D - f_D}{\lfw m e},
\quad 
H^*_{D, \cI = 0} = \dfrac{(1-\alpha)r_F\Big(1 - \dfrac{F^*_{W, \cI = 0}}{K_F(1-\alpha)} \Big)}{m\left(\lfw - \beta (1-\alpha) r_F + \beta r_F  \dfrac{F^*_{W, \cI = 0}}{K_F}\right)} ,
\quad 
H^*_{W, \cI = 0} = m H^*_{D, \cI = 0}.$$
\end{itemize}
\end{prop}

\begin{proof}
An equilibrium of system \eqref{equationsHDFWHW, cI=0} satisfies the system of equations:
\begin{equation}\label{system-equilibre, cI=0}
\left\lbrace \begin{array}{cll}
 e \lfw m F_W^* + f_D - \mu_D = 0& \mbox{or} & H_D^* = 0,\\
m H_D^*\Big(\lfw - (1-\alpha)r_F \beta - \dfrac{r_F \beta}{K_F}F_W^* \Big) + r_F \dfrac{F_W^*}{K_F} - (1-\alpha)r_F= 0& \mbox{or} & F^*_W = 0,\\
H_W^* = \dfrac{m_D}{m_W} H_D^* = m H_D^*.&&
\end{array} \right.
\end{equation}
When $H_D^*=0$ and $F_W^*=0$, we recover the trivial equilibrium $TE = \Big(0,0,0\Big)$. When $H_D^*=0$ and $F_W^*\neq0$, we obtain the fauna-only equilibrium $EE^{F_W} = \Big(0, K_F(1-\alpha), 0 \Big)$. 

Finally, when $H_D^*\neq0$ and $F_W^*\neq0$, direct computations lead to the following expression:
$$F^*_{W, \cI = 0} = \dfrac{\mu_D - f_D}{\lfw m e}, \quad
H^*_{D, \cI = 0} = \dfrac{(1-\alpha)r_F\Big(1 - \dfrac{F^*_{W, \cI = 0}}{K_F(1-\alpha)} \Big)}{m\left(\lfw - \beta (1-\alpha) r_F + \beta r_F  \dfrac{F^*_{W, \cI = 0}}{K_F}\right)} ,
\quad 
H^*_{W, \cI = 0} = m H^*_{D, \cI = 0}.$$


$F^*_{W, \cI = 0}$ is biologically meaningful whenever $\dfrac{\mu_D - f_D}{\lfw m e} \leq K_F(1-\alpha)$. From now we assume this.

$H^*_{D, \cI = 0}$ is biologically meaningful if it is positive. The numerator of $H^*_{D, \cI = 0}$ is non-negative since $F^*_{W, \cI = 0} = \dfrac{\mu_D - f_D}{\lfw m e} \leq K_F(1-\alpha)$ and positive if $\dfrac{\mu_D - f_D}{\lfw m e} < K_F(1-\alpha)$. 

From now, we assume $\dfrac{\mu_D - f_D}{\lfw m e} < K_F(1-\alpha)$.
We have to study the sign of the denominator of $H^*_{D, \cI = 0}$, which has to be positive:
The denominator is equal at:
\begin{align}
\lfw - \beta (1-\alpha) r_F + \beta r_F  \dfrac{F^*_{W, \cI = 0}}{K_F} &= \lfw - \beta (1-\alpha) r_F \Big(1 - \dfrac{F^*_{W, \cI = 0}}{(1-\alpha)K_F} \Big) \\
&= \lfw \left(1 - \dfrac{\beta (1-\alpha) r_F}{\lfw} \Big(1 - \dfrac{\mu_D - f_D}{\lfw m e(1-\alpha)K_F} \Big) \right) \label{denominator}
\end{align}

Using $\beta < \dfrac{4 (\mu_D - f_D)}{m e r_F (1-\alpha)^2 K_F}$ and since $1 - \dfrac{\mu_D - f_D}{\lfw m eK_F(1-\alpha)} > 0$, we have

\begin{align*}
\dfrac{\beta (1-\alpha) r_F}{\lfw} \Big(1 - \dfrac{\mu_D - f_D}{\lfw m e(1-\alpha)K_F} \Big) &< \dfrac{4 (\mu_D - f_D)}{m e (1-\alpha) K_F \lfw}\Big(1 - \dfrac{\mu_D - f_D}{\lfw m e(1-\alpha)K_F} \Big)
\end{align*}
A study of the function $g(x) = x(1-x)$ for $x \in (0,1)$ show that $g(x) \leq \dfrac{1}{4}$ and therefore 

\begin{align*}
\dfrac{\beta (1-\alpha) r_F}{\lfw} \Big(1 - \dfrac{\mu_D - f_D}{\lfw m e(1-\alpha)K_F} \Big) &< 4 \times \dfrac{1}{4} = 1 \\
0 &< 1 - \dfrac{\beta (1-\alpha) r_F}{\lfw} \Big(1 - \dfrac{\mu_D - f_D}{\lfw m e(1-\alpha)K_F} \Big)
\end{align*}

This means that the denominator of $H^*_{D, \cI = 0}$ (expression \eqref{denominator}) is positive.

Under the assumption $ \beta < \dfrac{4 (\mu_D - f_D)}{m e r_F (1-\alpha)^2 K_F}$, the equilibrium of coexistence exists if $\dfrac{\mu_D - f_D}{\lfw m e} < K_F(1-\alpha)$.
\end{proof}


\end{document}

