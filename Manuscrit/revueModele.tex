\documentclass[landscape, a3paper]{article}
\PassOptionsToPackage{dvipsnames}{xcolor}
\usepackage{graphicx,ulem} 
\usepackage{color, soul}
\usepackage{amsfonts,amsmath}
\usepackage{amsthm}
\usepackage{empheq}
\usepackage{mathtools}
\usepackage{multirow}
\usepackage{longtable}
\usepackage{titlesec}
\usepackage{caption}
\usepackage{array}
\usepackage{graphicx}
\captionsetup{justification=justified}
\usepackage[toc,page]{appendix}
\usepackage{hyperref}
\usepackage{subcaption}
\usepackage{xcolor}
\usepackage[left=1.5cm, right=2cm]{geometry}



%\textheight240mm \voffset-23mm \textwidth160mm \hoffset-20mm

   \graphicspath{{./Images/}{../Schema}}
%\graphicspath{{Figures}}
%\setcounter{secnumdepth}{4}
%\titleformat{\paragraph}
%{\normalfont\normalsize\bfseries}{\theparagraph}{1em}{}
%\titlespacing*{\paragraph}
%{0pt}{3.25ex plus 1ex minus .2ex}{1.5ex plus .2ex}


\newcommand{\marc}[1]{\textcolor{teal}{#1}}
\newcommand{\YD}[1]{\textcolor{magenta}{#1}}
\newcommand{\VY}[1]{\textcolor{blue}{#1}}


\newcommand{\Hlight}[1]{\textcolor{orange}{#1}}

\title{Revue Modèle Interactions Hommes-Environnement}
\author{Marc Hétier, Yves Dumont  and Valaire Yatat-Djeumen}

\begin{document}

\maketitle


\begin{longtable}{p{4cm}|c|p{5cm}|p{3cm}|p{5cm}|p{5cm}|p{8cm}|p{5cm}|}
Article Title & Type & Goal & Geographic Area & Hypothesis & Limits & Conclusion &Personal Comments\\
\hline
Modelling the Impact of Human Population and Its Associated Pressure on Forest Biomass and Forest-Dependent Wildlife Population \cite{fanuel_modelling_2023}
&  Review 
& 
& Intoduction mainly focuses on Tanzania 
& All along the review, the human population depends on forest biomass. 
& 7 out 8 categories consider population pressure or industrialization as variable.  & Socioeconomic status influences forest resource consumption patterns, thus, stratification of the human population based on economic status is a critical phenomenon in modelling human-nature interactions; \newline Use of sensitivity analysis recommended
& Distribute the different model in 8 different categories, according to the variables. In all the model considered, \Hlight{forest biomass is always a variable}, \Hlight{surely due to the geographic area.} \newline Category ``Forest biomass, wildlife, HPop, Pop pressure" seems the closest to our model \\

\hline

Modeling the effect of deforestation caused by human population pressure on wildlife species \cite{lata_modeling_2018}
& Model 
& Study the effect of deforestation caused by human pop. on wildlife 
&- 
& Forest, and HPop. follows logistic. \newline
Forest is linearly impacted by HPop (cutting, grazing..), and quadratically impacted by HPressure (agricultural needs; area cannot be used for regeneration) \newline
HPressure naturally decreases, and linearly increases with HPop \newline
Wildlife species are completely reliant on forestry resources. There are natural mortality and intraspecific competition, both increased by the HPressure 
& \Hlight{No positive feedback from human to wildlife. \newline
HPop only benefits from forest biomass, not from wildlife.}
 & To ensure the persistence of wildlife species, it is important that the population pressure be reduced 
 & The article is not very critical of model formulation, included and non-included terms \newline
 \Hlight{The human carrying capacity does not depend on resources, and is fixed}
 \\
 \hline
 
Mathematical model to study the impact of anthropogenic activities on forest biomass and forest-dependent wildlife population \cite{fanuel_mathematical_2024}
& Model 
& Study the impact of anthropogenic activities on forest biomass and forest-dependent wildlife populations 
& \Hlight{A region where inhabitants heavily rely on the forests resources for there subsistence, and not from hunt.} & Forest, wildlife biomass and HPop follow logistic equation \newline Growth and capacity of wildlife are function of forest biomass \newline
Forest biomass is decreased by wildlife, HPop (linearly) and HActivities (quadratic) \newline
Wildlife is decreased by HPop and HActivities (both linearly) \newline
\Hlight{Predation from wildlife to HPop}: either predation or destruction of resources
& \Hlight{No positive feedback from human to wildlife. \newline
HPop only benefits from forest biomass, not from wildlife.} \newline No real data \newline Integrate Spacial Dynamics ?

&
& \Hlight{The human carrying capacity does not depend on resources}


\\ \hline
Modelling the depletion of forestry resources by population and population pressure augmented industrialization \cite{dubey_modelling_2009}
& Model
& Study the depletion of forestry resources caused by HPop and industrialization
& Area where there are both domestic and industrial uses of forestry resources (eg northern India)
& Forest biomass (\Hlight{essetially wood}) follows a logistic, and is linearly depleted by both HPop and industrialization
\newline
HPop can growth (two cases) either logistically, either \Hlight{by immigration}, and use forest biomass as resources
\newline
Human Pressure : essetially here to increase industrialization
\newline
Level of Industrialization increases thanks to human pressure and forest biomass, \Hlight{is controlled by external measures applied by Government agencies} (term $-\theta_0 I$)
& No wildlife is considered \newline
& 
& Why not consider immigration and logistic growth in the same time ?


\\ \hline

Impact of human activities on forest resources and wildlife population \cite{pathak_impact_2021}
& Model
& Trace the causes and consequences of human activities on the depletion of forestry resources
& India
& Forest biomass follows a logistic equation ; there are depletion due to wildlife, human population and (quadratically) by human activities \newline
Wildlife growth depend on forest biomass, natural mortality, intraspecific competition and depletion due to HPop and Hactivities \newline
HPop growth logistically, and \Hlight{by use of forest biomass only}. Additional mortality due \Hlight{to human death caused by wild animals} \newline
HActivities growth with HPop and the use of Forest biomass
& An equilibrium point with no HPop but HActivities exists
\newline
HPop only benefits from forest biomass, not from wildlife.

&
&The fact that an equilibrium point with no HPop but HActivities exists let think \Hlight{that humans do not live on the considered area (=forest area), but only work on it}

\\ \hline \hline

Human-induced oscillations in a network landscape model  \cite{della_marca_human-induced_2022}
&  Model
& Investigate the role of human factor in the resilience of environmental systems
& Application to Northern Italy
& \Hlight{Consider different region, separated} by natural or anthropic barrier. \newline
For each regions, two main variables are consider : green area and a proportion of ecologists which work to protect the green area. \newline
Green area follows a logistic, with a degradation caused by the proportion of non-environmentalists, and a degradation \Hlight{due to the fragmented landscape} \newline
Ecologists population follow a logistic equation, with a growth term depending on the utility gain to protect the area : \Hlight{evolutionnary game dynamics}
& Ecosystem dynamics are really simplified \newline
Future model could add migration terms, from one region to others. \newline
\Hlight{Study the long term behavior, time dependent parameter could be more appropriate}
& Find configuration where stable periodic oscillations exist, as typical of resilient territories \newline
When several land units are connecting, oscillations in one unit may induce oscillation in others : highlight the role of connectivity


& An economist's point of view on human-environment system \newline
Example of different patches


\\ \hline \hline

Agricultural land use and the sustainability of social-ecological systems \cite{bengochea_paz_agricultural_2020}
& Model 
& Explore the effects of different agricultural land use strategies on long-term human-environment dynamics. 
& Rather occidental ? 
& HPop growths logistically, \Hlight{with a carrying capacity depending on the available resources} \newline
Take into account ecosystem services and land remission \newline
Type of agriculture (intense or not) controlled by a parameter
& Humans can not \Hlight{change of strategies over time} (adaptive strategies)
&
& It is more a human-land model, questioning past or future society's collapse. It is also a very general model.
\\ \hline

The Dynamics of Human–Environment Interactions in the Collapse of the  Classic Maya \cite{roman_dynamics_2018}
& Model
& Explore whether societal dynamics and depletion of natural resources can explain the fall of Maya
& Central America 16th century
& The HPop is divided in several roles. \Hlight{HPop growth is related to food available, as the repartition along the different categories (builders, workers in swidden/intensive agriculture)}. \newline
The environment is considered only through the amount of food it can produce. It follows a logistic equation. There is an additional intake from all the population. Rainfall increases the growth rate and the capacity\newline
A last variable exists, for the number of monuments built. It has no influence on the other variables.
& 
& A simple model can fit pretty well with historical data \newline
A shift from swidden to intensive agricultural practice may explain the decline \newline
Drought is probably not to be considered as a major contributor
& The HPop is categorized in different types of workers, which have different impact on the environment. 
\newline
The model isolate one parameter (harvesting rate per capita of intensive agriculture) as a bifurcation parameter. \newline
It is more a human-land model, questioning past or future society's collapse.
\\ \hline

\end{longtable}




\newpage

\bibliographystyle{plain}
\bibliography{Biblio/Math, Biblio/Context, Biblio/interactionsHumanEnvironmentModel}

\end{document}

