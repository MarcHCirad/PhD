\documentclass{article}
\usepackage{graphicx} 
\usepackage{color}
\usepackage{amsfonts,amsmath}
\usepackage{multirow}

\title{Éléments Bibliographiques}
\author{Marc Hétier}
\begin{document}

\maketitle

\section{Composition alimentaire dans trois populations forestières de la région cotière du Cameroun}
\textbf{FROM : l'alimentation en forêt tropicale : interactions...}
Peuples qui vivent dans la forêt doivent y trouver de quoi se nourrir. Mais nourriture saisonnière, possible qu'ils ne trouvent pas tout le temps en qtt suffisante. Etude entre 1983-1986 dans la région de Campo de trois populations qui ont choisi des strats différents : les Yassa (pêche et agri), les Mvae(chasse et agri) et les Bakola (chasse, cueillette et agri rudi).

Calcul des rations individuelles, afin de voir quel groupe (enfants, femmes, hommes) satisfait ou non ses besoins. Rations principalement constitué de manioc, produits chasse/pêche.

Origine de la nourriture : \textbf{la part des ressources des cueillettes est très faible, même pour les pygmées Bakolas}. Les produits de la forêt servent davantage de condiments que d'aliments. \textbf{Autour de 80\% des protéines sont d'origines animales, et de 1\% de cueillette, pour ces populations}. Données aussi sur l'importation des protéines : autour de 2\%.

Si on raisonne en terme de calories : \%80 proviennent de production locale, 13\% de la chasse/pêche, 5\% d'importation.

\section{Les pygmées camerounais face à l'insuffisance des produits alimentaires végétaux de la forêt équatoriale}
\textbf{FROM : l'alimentation en forêt tropicale : interactions...}

Les Baka, Bakola, Medzan ont des difficultés en ravitaillement en féculents. Auparavant approvisionnement auprès d'autres peuples, mais système colonial a boulversé. \textbf{Questions se posent : est une nécessité de recourir à des produits agricoles, ou est ce qu'une revalorisation de la collecte s'impose ?}

Bakola vivent au coeur des massifs forestiers, Baka en bord de route. Enquête de juillet 1983 à juin 1985.
\textbf{Densité d'environ 1.4 personne/km2}

Ressources disponibles dans la forêt : jeune pousse/feuilles, amandes, fruits, tubercules, champignons, miel. Pas disponibles tout le temps !Et difficile à conserver.
\textbf{Exceptés les amandes de.., les denrées disponibles en permanences ou conservables }
Discussion sur les densité et production des espèces à graines et des ignames.
\textbf{mentionne des sorties de récoltes courtes ($<1$km) et d'autres plus longues, 2à3 jours}

Constation finale : manque cruel de plantes féculents, surabondance d'amande et de graine grasse. Souligne qu'il y a une densité trop importante par rapport au rendement naturel. \textbf{Nécessité de se tourner vers l'agriculture.}



\section{Consequences of survey method for estimating hunters'
harvest rates}

Les données pour mesurer le nombre de proies capturées ou l'effort de chasse 
sont récoltés selon plusieurs méthodologies différentes, à chaque fois sujette à des bieais ou des erreurs. On note : hunter follows, interview de chasseurs, journal de bord, village/camp monitoring. Le terme "effort de chasse" est aussi mesuré de plusieurs façons différentes : nbr de jours de chasse, distance parcourue, nbr de pièges

\section{Interpreting long-term trends in bushmeat harvest in southeast Cameroon}
\textbf{Attention : Zone d'étude différente de ce qu'on regarde ? Ici villages très isolés.}


Une surchasse a pour effet de rendre prébondérant les animaux de petites tailles. D'où l'utilisation de la measure "Mean Body Mass Indicator". Un autre indicateur "Catch per Unit Effort" est utilisé comme proxy pour évaluer le niveau d'une population. Peut être mesuré par le temps passé à la chasse (compliqué), ou bien par le nombre de proie ramenées. C'est ce qui est fait ici, dans 3 villages du SE du Cameroun.


Description climatique de la zone d'étude. Avec \textbf{Nombre d'habitants, nombre de chasseurs} sur 3 périodes différentes (2003,2009,2016). Données recueillis par interlocuteur sur place, membre de la communauté.

Calcul du nbr de prise moyen par jour (moyenne sur un mois); et du MBMI.

Beaucoup d'ongulés, de rongeurs dans les proies. Présence de chiens pour accompagner les chasseurs.

Discussion des résultats observées : pas de changement sur le MBMI, peu de changement sur les CPHD mensuel (mais selon les villages ou méthodes (fusils/pièges) oui)

\textbf{Temps passé à la chasse estimé à 4.60h/week selon source (trouvable ??)}

\section{Carrying Capacity Limits to Sustainable Hunting in Tropical Forests}
But du chapitre : déterminer une capacité limite pour les populations vivants dans des fôrets tropicales, en supposant que le but est de limiter la perte de diversité et d'environnement.

Attention : ici, focus uniquement sur la chasse. Il n'est jamais question de cueillette.

D'autres sources de protéines (pêches, élevages) ne sont pas considérées car pas prioritaires dans la nourriture (cf plus bas).

Populations humaines vivent depuis longtemps dans des forêts tropicales, ont contribués à la biodiversité actuelle (espèce animale pas résistante ont disparus). Ajd, chasse est tjr un moyen de subsitance important pour les pops locales, mais les produits de la  chasse sont aussi vendus. Proies préférées sont les gds mammifères : densité de pop. et poids avantageux.

Densité d'animaux dépend du type d'habita : dans une forêt "evergreen" moins importante que dans une prairie : cf table (entrée pour ongulés, primates, rongeurs). \textbf{Donnée pour une population animal max en kg/km2}

Calcul de la production annuelle pour chaque espèce : procution à comprendre comme le nbr d'animal qui est ajouté à la population chaque année.
La population est déterminée par son régime alimentaire (moins de carnivore au km2 que d'herbivore) et par le poids de chaque individu (moins d'éléphant que de rongeurs)
Utilise un modèle précédemment publié pour calculer la prod annuelle de chaque espèce. Se base sur un site non perturbé (Manu National Park, Perou). On estime que chaque population est à sa capacité maximale (site undisturbed). Puis on dit que seulement un pourcentage (0.2 pour les longues, 0.4 pour les courtes, 0.6 pour les très courtes durées de vies) peuvent être ramassées.
D'autres méthodes existent : étude des prélèvements par les hommes dans des réserves, ou par des prédateurs naturels. \textbf{On obtient à chaque fois un taux de prélèvement annuel autour de 200kg/km2}

En divisant ce taux de prélèvement par les besoin nutritifs d'un humain ou par la consommation réelle (\textbf{donnée de consommation de viande en tableau pour différents peuples}), et en prenant en compte que toute la viande n'est pas comestible, on obtient une \textbf{densité max de 1 personne/km2}. Dans les populations observées, la densité est souvent moindre.

\textbf{La cueillette n'est pas présentée comme source importante de nourriture : trop saisonnale, consomme trop de temps et d'énergie.} 

Réflexion sur les changements apportés par l'agriculture : permet une augmentation notable des densités de population grâce à ça. De plus, l'abandon de champs permet la constitution de forêt secondaire, plus favorable à la vie animale, qui augmente aussi.

MAIS les populations agricoles sont aussi reliées à des marchés,  où la viande sauvage est vendu, au lieu d'être mangée. C'est surtout la viande préférrée qui y est vendu. Ainsi, la viande dispo par km2 pour les populations rurales n'augmentent pas, et peut même diminuer.
La culture seule permet quand même des densitées humaines plus importante. Cette population va continuer à chasser. La densité de population humaine raisonnable pour maintenir la conservation des espèces ne doit pas non plus excéder 1p/km2.

Augmenter cette densité en gardant la chasse sustainable : manger moins de viande, chasser des proies moins rentables mais plus abondantes/renouvelables, chasser les prédateurs. Aucune des trois n'est réaliste.

Dernière partie rapelle la non homogéniété spatiale des populations humaines. Les animaux sont donc tués localement, et les populations animales externes constitutent un réservoir qui permet de repeuplé les zones dévastées, et les densités humaines peuvent être légèrement augmentées.


\section{Long-term trends in wildlife community structure and functional diversity in a village hunting zone in southeast Cameroon}

Une étude sur la l'abondance de la biodiversité, la richesse d'espèce et la diversité fonctionelle a été menée, entre 2002 et 2016. La biodiversité a commencé a chuté dès 2009, le reste dès 2016 et dans le même temps les activité de chasse ont augmentées. Ce sont particulièrement les grands mammifères qui sont impactés (elephant et gds singes). Cela peut conduire a une catastrophe écologique.

La chasse peut être la plus gde cause de déclin de la faune sauvage sous les tropiques. Intensification de la chasse due à l'augmentation de la pop. humaine.
Demande plus importante de bushmeat car conflit.

Importance de chaque espèce dans un milieu; espèce keystone; important même pour des acitivités humaines de loisir, pour les évènements climatiques...

Surveiller la diversité des différents rôles dans un écosystème soumis à une pression de chasse permet d'acquérir des connaissances sur la résilience de cet écosytème.

L'objectif de ce papier est de présenter les résultats de surveillance de mamifères dans le sud est du cameroon, dans une zone soumise à la chasse.

Lieu : sud est du cameroun, sur 3 communes
Transect, collecte visuelle

Constat : diminution du nombre d'espèce, et globalement, pour chaque espèce du nombre d'individus.
Disscussion sur quels sont les individus viés par les chasseurs : les adultes.
On rappelle que différentes espèces vont réagir différent à la pression de prélèvement. (taux de prélèvement sustainable pour primates : 1à 4 \%)
On note aussi que due à la diminution de la diversité fonctionelle, la végétation va se trouver impacter.

Délai présent entre la diminution de la biodiversité et la diminution de la richesse en espèces et de la diversité fonctionnelle



\section{SDM Review}

Species Distribution Models (SDM), also known as ecological niche models \cite{barker_species_2022}, "seeks to predict the spatial patterns of species occurrence, i.e. where a species is likely to be found" \cite{beery_species_2021}. They are tools for implementation of conservation plans for threatned species or for monitoring invasive species \cite{deneu_very_2022}, but also to improve traditional risk maps \cite{barker_species_2022}.
Basically, SDM allow to obtain a map of presence probability. Same kind of maps can be given by field experts, or by raw species observation data as explained in \cite{beery_species_2021}. This article, a review of the different SDM concepts and terminology is made. Data availability, technical challenges and pitfalls are also discuss. Basically, SDM are based on 3 keys components : 
\begin{itemize}
\item dataset of species observation.
\item a location representation : a map which goes from real space to environmental space. From each real coordinate, it associate several environmental parameters.
\item a model based on those environmental parameters, which goes from environmental space to probability space.
\end{itemize}
Environmental parameters, also called covariates, can be of several types \cite{beery_species_2021} : climatic, pedologic, vegetation indices, land use, human influence..

\medskip
It exists several types of SDM, and different algorithms are used to implement them. SDM can be based on data related to presence-absence or presence only, single or multi-species. Algorithm can be statistical (logistic regression for instance), machine learning methods. One of the most popular software used for this last class is \textit{MaxEnt} \cite{phillips_maximum_2006}. Note that even if \textit{MaxEnt} relies on machine learning concept, a statistical explanation of it can be found in \cite{elith_statistical_2011}.

The article \cite{lippi_trends_2023} proposes a review on mosquito species distribution modeling. Among other interesting observations, the authors give the main environmental predictors used for mosquito SDM. They are : 
\begin{itemize}
\item Measures of temperature
\item Precipitation
\item Land cover and land use
\item Elevation
\end{itemize}
The article \cite{lippi_trends_2023} also gives some insights for top predictors for each taxonomic group.

Among the article reviewed by \cite{lippi_trends_2023}, two retained my attention. The first one, published in 2009, concerns \textit{anopheles} mosquitoes in Cameroon \cite{ayala_habitat_2009}. This study was conducted to characterize the distribution of malaria vectors. They sampled mosquitoes across the all Cameroon (475 442 km2), and also used previous recorded data. In total, records in 386 villages were used. Concerning the environmental data, the authors used raster map (1km2 per pixel) and 4 classes of environmental descriptors :
\begin{itemize}
\item topographic variables (source : Shuttle Radar Topographic) : elevation, slope, aspect, hydrographic network (min distance to water body)
\item climatic variables (source : LocClim database, from FAO) : rainfall, temperature, evapo-transpiration, relative humidity, hours of sunlight, wind speed
\item habitat variables (source Global Land Cover 2000 Project) : dense evergreen forest, deciduous woodland, forest/savanna mosaic, dry savannas, cropland
\item anthropic pressure (source Geonet Name Server and Dyepca Project) : min distance to localities or main road network
\end{itemize}
Using this descriptors, the authors conduced a Ecological Niche Factor Analysis (ENFA). ENFA produces two indices, indicating at which point the known ecological niche is specific inside the all region (marginality), and at which point the species is tolerant to other environment than its optimal one (specialization). The authors were able to produce a table indicating which eco-geographical variables play a role in the marginality/specialization of each sub-species of anopheles.
We can note that "Moreover, as found in other studies [32,63], eco-geographical variables (EGVs) related to human activity (distance to localities and roads) had the most important impact on the ecological niche of anthropophilic malaria vectors."

\par
The second article which retains my attention aims to predict the distribution of \textit{aedes albopictus} in the southeastern Pennsylvania, integrating both environmental and neighborhood factors. The study area was 21, 874 km2. The climate is globally uniform on it. Traps were placed in 8 801 different locations and allowed to get 129 476 records (March through November, 2001 to 2015). After processing, they used 1030 data presence for training, 441 for testing.

The neighborhood variables used are :
\begin{itemize}
\item rate of population below poverty
\item education index
\item median household income
\item housing density
\item vacant housing units
\item imperviousness of the surfaces
\item type of land cover
\end{itemize}

and the environmental variables used are :

\begin{itemize}
\item tree canopy
\item average precipitation for each month and for each semester
\item average temperature for each month and for each semester
\item elevation and slope
\item flow accumulation
\item enhanced vegetation indexes
\item normalized difference water index
\end{itemize}

The authors used MaxEnt to predict species distribution. They made 3 different simulations : the first one with environmental only variables, the second one with neighborhood variables only, and the third one with the 5 most significants variables of each set. The most important variables of the third model are: imperviousness, percent urban population, average enhanced vegetation index, temperature in the coldest quarter.
The authors conclude saying that neighborhood factors are not enough to correctly predict the distribution of anopheles. Instead, it is better to combine environmental factors with neighborhood ones than to only use environmental factors.

\medskip 
Finally, an article published in 2022 provide some guidelines for SDM applied to mosquitoes \cite{barker_species_2022}. Indeed, SDM are sensitive to predictor variable used, model building, and model evaluation considerations. Some recommendations have therefore been developed. Among them, there are: "include all regions within the species' environmental tolerances" - sometimes, due to private propriety or inaccessible area this can be difficult. Equivalently, one can consider more predictors which traduce the diversity of environment where live the species. Since mosquitoes can adapt quickly, considering the native or invaded area can not be sufficient to represent th entire niche. 

About predictors, temperature, precipitation or water availability are imperatives ones. In general, the spatial and temporal resolution of predictors must be justified by the expected response's resolution (sampling design or estimated spatial error). It should not be justified by the data availability. Interrogations about at which scale the species interacts with predictors is too often ignored or left without answers. Lattices, which are assumed to represent the environment's characteristic, are more reliable with equal sized grid.


\bibliographystyle{plain}


\newpage
\section{Socio-ecological modelling}
The models presented here are socio-ecological models and thus both environment and human population are variables. The notion of environment is traduced either by type of land (natural, conserved, agricultural..) as, for example, in \cite{henderson_ecological_2019, henderson_unequal_2021, bengochea_paz_agricultural_2020}, either by biodiversity as in \cite{lafuite_time-delayed_2017}. Human population growth (both natality and mortality) depends on natural resources. This dependence can take different form : direct (logistic growth with capacity given by resources available, with exponential term) \cite{bengochea_paz_agricultural_2020, lafuite_time-delayed_2017},  or indirect (using intermediate variables as technology) \cite{henderson_ecological_2019, henderson_unequal_2021}. Most of the models are general and do not consider specific time or location. It is not the case of the work done on \cite{roman_coupled_2017} and \cite{roman_dynamics_2018} which studied Easter island or classical Maya. Below, the strengths and weakness of each model are discussed.

\medskip
As discussed in \cite{henderson_ecological_2019}, population dynamics is commonly related to natural and agricultural resources, and more generally to food. In article \cite{bengochea_paz_agricultural_2020}, the authors use available food as the carrying capacity of the population dynamics, which follows a logistic curve. Even if this model has given good results when comparing to real data, it prevents to take into account demographic changes due to cultural or technological evolution, and this choice debated in the article. On the other hand, \cite{henderson_ecological_2019, roman_coupled_2017} prefers to specify birth and death rate, which depend on resources and wealth. This allows to model high natality and mortality rates when resources are rare, and low rates when resources are not limiting. Articles \cite{roman_coupled_2017, henderson_unequal_2021, roman_dynamics_2018} also consider migration between different spatial area or social class depending on where or if resources are available.

\medskip
Resources are always considering coming from agriculture. Agriculture performance may depend on the level of technology \cite{henderson_ecological_2019, lafuite_time-delayed_2017, bengochea_paz_agricultural_2020}, 
on types of agriculture (intensive or low) \cite{roman_dynamics_2018, bengochea_paz_agricultural_2020}.

Technology dynamics is often modeled by a logistic curve and is independent of others variables. This assumption is justified in \cite{lafuite_time-delayed_2017} only by the seek of simplicity. In \cite{henderson_ecological_2019}, the authors argue that there is no clear link between population's size and innovation or technological improvements. In \cite{henderson_unequal_2021}, the same authors chose to make technology dependent on human population density (denser populations make exchange easier), and on resources available. They also recall that technology does not increase the overall quantity of resources, but only the rate at which one can access to them.

When authors consider different types of agriculture, they distinct two cases: a low or a high intensive agriculture (in \cite{roman_dynamics_2018}, low intensive agriculture corresponds to swidden agriculture, while in \cite{bengochea_paz_agricultural_2020} it corresponds to agriculture on degraded land). Each type of agriculture provide different amount of food, and have different impacts on the environment. 

Article \cite{roman_dynamics_2018} models amount of available resource by a logistic function (after being used, resources take some times to recover) and add two mortality terms proportional to the harvest due to swidden or intensive agriculture. The mortality term due to intensive agriculture is higher than the one due to swidden agriculture. The article use parameters such that the regeneration of natural resources takes approx. 22 years, and intensive agriculture depletes resources in 7 years. Those parameters feat historical data.

Article \cite{bengochea_paz_agricultural_2020} focuses on the notion of services render by human population (high intensive case) or by environment (low intensive case) to agriculture. Intensive agriculture degrade lands, which then are enable to render environmental services to agriculture. Services rendered to agricultural land and recovery of degraded land are considered to occur only on the border between natural and agricultural/degraded land and they are assumed to be proportional to this border. Values of the proportional coefficients are not justified.

Article \cite{lafuite_time-delayed_2017}, even if not always clear in the model derivation, propose interesting relation between biodiversity, area and services. According to its sources, species richness and natural land are related by a power function ($SpeciesRichness = c \times (NaturalLand) ^ {z}$) and biodiversity and services are related by an other power function ($ EcoServicesToAgricultural = (Biodiversity)^\Omega$, with $\Omega < 1$).

\medskip
Two different types of migration are considered : migration between social classes \cite{henderson_unequal_2021, roman_dynamics_2018} and migration between spatial area \cite{henderson_unequal_2021, roman_coupled_2017}. Different social classes use different types of agriculture, while different area have different amount of resources. People change of area when resources are low, and of social classes when resources are sufficient/insufficient. 

\section{Ecosystem services}

Ecosystem services are defined in articles \cite{isbell_linking_2017, kremen_managing_2005} as the set of benefits that human can obtain from nature. Those benefits can occur at local, regional or global scale \cite{kremen_managing_2005, isbell_linking_2017} and include seed dispersal, pollination, purification of water, or even climate stability. The notion of ecosystem services is highly dependent to the one of ecosystem functioning \cite{isbell_linking_2017}, which corresponds to the fluxes of material and energy. Because ecosystem functioning is subject to human driven changes (biodiversity loss, habitat fragmentation, pollution, species invasion, climate change \cite{isbell_linking_2017}), ecosystem services are also threatened. Understand all the connections between human driven changes, ecosystem functioning and ecosystem services is not easy, due to the diversity of spatial and temporal scales at which they occur and the vast dimensions of diversity.

\medskip
Many studies (ref ? Tilman et al, Schwartz et al.. see \cite{kremen_managing_2005}) have focused on the link between species richness and ecosystem services, and on how to model it. In the article \cite{kremen_managing_2005} two relationships are proposed. 

\medskip
The first one models the aggregate ecosystem function ($F$) as the sum of the contribution of each ecosystem providers($f_i = \text{abundance}_i \times \text{efficiency}_i$) . This linear relation, proposed in (ref : Walker et al 1999), even if it allows to have a fine control on the importance of each contributors, requires to know a lot about each of them.
The other relation aims to traduce a "portfolio effect", which is "the stabilizing effect of diversity function that occurs simply through the net effect of random changes in species abundances" (ref : Tilman et al 1998). This relation takes the form of $$\text{biomass} = a \times \text{(richness)} ^b$$ as written in \cite{oconnor_general_2017}. This relation does not directly concern ecosystem services but biomass. However, the authors of \cite{oconnor_general_2017} says that when biomass can be considered as a good proxy for ecosystem services, this relation can also be used to model the production of ecosystem services.
Moreover, this article questions the generality of this relation, and use a large corpus to answer it. Their conclusion is that although the relation varies with trophic level and type of ecosystem (maritime, terrestrial), a value of $b = 0.26$ can be used. The authors of \cite{lafuite_time-delayed_2017} followed this recommendation, and applied to a socio-ecological model.

\medskip
Note that tools/software have also been developed, with the aim to spatialize ecosystems services. Some of them are listed in \cite{ochoa_tools_2017}. Some of them focus on specific services (water ressource management, fog interception) other are more general.

\bibliographystyle{plain}
\bibliography{SDM,SocioEcoModel, EcoServices}

\end{document}
