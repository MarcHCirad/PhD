 \documentclass{article}
\usepackage{graphicx} 
\usepackage{color}
\usepackage{amsfonts,amsmath}
\usepackage{amsthm}
\usepackage{empheq}
\usepackage{mathtools}
\usepackage{multirow}
\usepackage{tikz}
\usepackage{titlesec}
\usepackage{caption}
\usepackage{lscape}
\captionsetup{justification=justified}
\usepackage[toc,page]{appendix}

\textheight240mm \voffset-23mm \textwidth160mm \hoffset-20mm

\setcounter{secnumdepth}{4}
\titleformat{\paragraph}
{\normalfont\normalsize\bfseries}{\theparagraph}{1em}{}
\titlespacing*{\paragraph}
{0pt}{3.25ex plus 1ex minus .2ex}{1.5ex plus .2ex}

\newcommand{\lf}{\lambda_{FH}}
\newcommand{\lv}{\lambda_{VH}}
\newcommand{\lfa}{\lambda_{FH, A}}
\newcommand{\lva}{\lambda_{VH, A}}
\newcommand{\lfw}{\lambda_{FH, W}}
\newcommand{\lvw}{\lambda_{VH, W}}
\newcommand{\da}{\delta_A}
\newcommand{\dw}{\delta_W}
\newcommand{\dr}{\dfrac{\da}{\dw}}
\newcommand{\rd}{\dfrac{\dw}{\da}}
\newcommand{\RV}{R_0^V}
\newcommand{\RF}{R_0^F}
\newcommand{\NF}{\mathcal{N}_0^F}
\newcommand{\NV}{\mathcal{N}_0^V}
\newcommand{\NH}{\mathcal{N}_0^H}
\newcommand{\Fbeta}{F^*_\beta}
\newcommand{\Hbeta}{H^*_\beta}
\newcommand{\Vbeta}{V^*_\beta}
\newcommand{\VbetaF}{V^*_{\Fbeta, \beta}}
\newcommand{\FHterme}{\omega f + \lf}
\newcommand*\phantomrel[1]{\mathrel{\phantom{#1}}}

\title{Suivi Thèse Marc}
\author{Marc Hétier, Yves Dumont  and Valaire Yatat-Djeumen}

\begin{document}

\maketitle
\section{Plan}
\begin{enumerate}
\item Pour la zone anthropisée, seule, on va rester sur le modèle suivant:
\[
\left\{ \begin{array}{l}
\dfrac{dF_{a}}{dt}=r_{F}\dfrac{H_{a}}{H_{a}+L}\left(1-\dfrac{F_{a}}{K_{F}}\right)F_{a}-\mu_{F}F_{a}-\lambda_{FH,a}F_{a}H_{a},\\
\dfrac{dV_{a}}{dt}=r_{V}\dfrac{H_{a}}{H_{a}+L_{V}}\left(1-\dfrac{V_{a}}{K_{V}}\right)V_{a}-\mu_{V}V_{a}-\lambda_{VH,a}V_{a}H_{a},\\
\dfrac{dH_{a}}{dt}=r_{H}\left(1-\dfrac{H_{a}}{a_{a}F_{a}+b_{a}V_{A}+c}\right)\left(\dfrac{H_{a}}{\beta}-1\right)H_{a}.
\end{array}\right.
\]
L'étude se fait par étape:
\begin{enumerate}
\item On considère le cas $L=0$: c'est le cas que tu as déjà étudié et qui est déjà ``riche''.
\item Puis le cas $L>0$. Qu'est ce qui va changer par rapport au cas $L=0$
? Au moins, un équilibre de plus: $(0,0,0)$.
\end{enumerate}
\item Pour la zone non-anthropisée,, seule, on étudie les modèles
\begin{enumerate}
\item 
\[
\left\{ \begin{array}{l}
\dfrac{dF_{S}}{dt}=r_{F}\dfrac{V_{S}}{V_{S}+L_{S}}\left(1-\dfrac{F_{S}}{K_{F}}\right)F_{S},\\
\dfrac{dV_{S}}{dt}=r_{V}\left(1-\dfrac{V_{S}}{K_{V}}\right)V_{S}-\alpha V_{S}F_{S},
\end{array}\right.
\]
dont les équilibres seraient $(0,0)$, $(K_{F},0)$, $(0,K_{V})$
et $\left(K_{F},\left(1-\dfrac{\alpha}{r_{V}}K_{F}\right)K_{V}\right)$
si $1-\dfrac{\alpha}{r_{V}}K_{F}>0$. Sur ces 4 équilibres, seuls
2 sont stables....
\item 
\[
\left\{ \begin{array}{l}
\dfrac{dF_{S}}{dt}=r_{F}\left(1-\dfrac{F_{S}}{fV_{S}}\right)F_{S}\\
\dfrac{dV_{S}}{dt}=r_{V}\left(1-\dfrac{V_{S}}{K_{V}}\right)V_{S}-\alpha V_{S}F_{S}
\end{array}\right.
\]
Pour ce dernier modèle, les équilibres seraient $(0,K_{V})$ et $\left(f\dfrac{r_{V}K_{V}}{\alpha fK_{V}+r_{V}},\dfrac{r_{V}K_{V}}{\alpha fK_{V}+r_{V}}\right)$
\item Il faut distinguer quel serait le modèle le plus approprié... pour
la suite....
\end{enumerate}
\item Enfin, troisème étape, la dispersion humaine entre les patches anthropisée
et non-anthropisé:
\end{enumerate}
\[
\left\{ \begin{array}{l}
\dfrac{dF_{S}}{dt}=r_{F}\left(fV_W-\dfrac{F_{S}}{1}\right)F_{S}-\lambda_{FH}F_{S}H_{S}\\
\mbox{ou }\dfrac{dF_{S}}{dt}=r_{F}\dfrac{V_{S}}{V_{S}+L_{S}}\left(1-\dfrac{F_{S}}{K_{F}}\right)F_{S}-\lambda_{FH}F_{S}H_{S}\\
\dfrac{dV_{S}}{dt}=r_{V}\left(1-\dfrac{V_{S}}{K_{V}}\right)V_{S}-\lambda_{VH}V_{S}H_{S}-\alpha V_{S}F_{S}\\
\dfrac{dH_{S}}{dt}=\delta H_{a}-\delta_{s}H_{S}
\end{array}\right.
\]
\[
\left\{ \begin{array}{l}
\dfrac{dF_{a}}{dt}=r_{F}\dfrac{H_{a}}{H_{a}+L}\left(1-\dfrac{F_{a}}{K_{F}}\right)F_{a}-\mu_{F}F_{a}-\lambda_{FH,a}F_{a}H_{a}\\
\dfrac{dV_{a}}{dt}=r_{V}\dfrac{H_{a}}{H_{a}+L_{V}}\left(1-\dfrac{V_{a}}{K_{V}}\right)V_{a}-\mu_{V}V_{a}-\lambda_{VH,a}V_{a}H_{a}\\
\dfrac{dH_{a}}{dt}=r_{H}\left(1-\dfrac{H_{a}}{a_{a}F_{a}+b_{a}V_{A}+a_{s}F_{S}+b_{s}V_{S}+c}\right)\left(\dfrac{H_{a}}{\beta}-1\right)H_{a}-\delta H_{a}+\delta_{s}H_{S}
\end{array}\right.
\]



This model tries to take into account the interactions between human population ($H$), wild fauna and flora ($F$), and human-driven vegetation ($V$). 
We assume that human pick up resources from wild and human-driven vegetation. The rate per person of this collect is assumed to be decreasing with the quantity of resources.
Moreover, human-driven vegetation growth is depended on both services render by human (culture, watering...) and wild vegetation (pollination, moisture retention, wind protection). 
Deforestation by fire happen with intensity $\omega$ and frequency $f(H)$. Assuming that the combustible (dry grass) is available in enough quantity for a good fire's spatial dispersion, we can take $f(H) = fH$. Also, note that in the context of this study, fire is mainly used for hunting activities or to create a agricultural field. In both cases, the frequency and the intensity of those fires are low.
Conversely, wild biomass is a competitor for human driven vegetation.

Human maximum capacity is assumed to depend on resources (wild and human driven). This assumption is justified in \cite{fanta_equilibrium_2018}, used in \cite{bengochea_paz_agricultural_2020}.
Moreover, the human population is assumed have a reduced growth rate for low population size. This corresponds to a weak Allee effect. Some studies have already propose such phenomenon on human population ; see \cite{hamilton_human_2012} and \cite{vaesen_inbreeding_2019}.

All thus assumption lead to the following model :


\newpage
\section{Model in anthropized area}

\begin{equation}
\left\{ \begin{array}{l}
\dfrac{dF_{A}}{dt}=r_F \dfrac{H_A}{H_A+L_F}\left(1-\dfrac{F_A}{K_{F}}\right)F_A-\mu_{F}F_A-\lambda_{FH,A}F_AH_A,\\
\dfrac{dV_A}{dt}=r_{V}\dfrac{H_A}{H_A+L_{V}}\left(1-\dfrac{V_A}{K_{V}}\right)V_A-\mu_{V}V_A-\lambda_{VH,A}V_AH_A,\\
\dfrac{dH_A}{dt}=r_{H}\left(1-\dfrac{H_A}{a_{A}F_{A}+b_{A}V_{A}+c}\right)\left(\dfrac{H_A}{\beta}-1\right)H_A.
\end{array}\right.
\label{modelAnthropo}
\end{equation}

All the parameters are positifs, except $L_F$ which is non-negative. Moreover, we assume $\beta < c$.

\subsection{Sub-model}
There are two interesting sub model, which are $F_A-H_A$ and $V_A-H_A$. Since they are equivalent, we will only study sub-model $F_A-H_A$.

\begin{equation}
\left\{ \begin{array}{l}
\dfrac{dF_{A}}{dt}=r_F \dfrac{H_A}{H_A+L_F} \left(1-\dfrac{F_A}{K_{F}}\right)F_A-\mu_{F}F_A-\lambda_{FH,A}F_AH_A,\\
\dfrac{dH_A}{dt}=r_{H}\left(1-\dfrac{H_A}{a_{A}F_{A}+c}\right)\left(\dfrac{H_A}{\beta}-1\right)H_A.
\end{array}\right.
\label{submodelAnthropoFH}
\end{equation}

Set the following functions
\begin{equation}
T_F(H) = \dfrac{H}{H + L_F}\dfrac{r_F}{\mu_F + \lfa H}
\label{threshold:TF(H)}
\end{equation}
\begin{equation}
T_H(F) = \dfrac{a_A F + c }{\beta}
\label{threshold:TH(F)}
\end{equation}

\subsubsection{Equilibrium : existence}

System \eqref{submodelAnthropoFH} has the following equilibrium:
\begin{itemize}
\item A trivial equilibria $TE= (0,0)$ which always exists.
\item A faun equilibria $EE^{F_A} = (F_A^*, 0)$ where $F_A^* = K_F\left(1 - \dfrac{\mu_F}{r_F}\right)$. It exists if $L_F = 0$ and $1 < T_F(0)$.
\item A human equilibria $EE^{H_A}_\beta = (0, \beta)$ which always exists.
\item A faun-human equilibrium $EE^{F_AH_A}_\beta = \Big(F^*_{F_A, \beta}, \beta\Big)$ with $F^*_{F_A, \beta} = K_F \Big(1-\dfrac{\beta + L_F}{\beta}\dfrac{\mu_F + \lfa \beta}{r_F}\Big)$ which exists if $1 < T_F(\beta)$.
\item A human equilibria $EE^{H_A} = (0, c)$ which always exists.
\item One or two faun-human equilibrium $EE^{F_AH_A, i} = \Big(F^*_{F_AH_A, i}, H^*_{F_AH_A, i}\Big)$ 
%$F^*_{F_AH_A, i}$ is given by:
%$$ F^*_{F_AH_A, i} = K_F \left(1 - \dfrac{H^*_{F_AH_A, i} + L_F}{H^*_{F_AH_A, i}} \dfrac{\mu_F + \lfa H^*_{F_AH_A, i}}{r_F} \right)$$
%and its positive only if $1 < T_F(H^*_{F_AH_A, i})$.
%
%$H^*_{F_AH_A, i}$ is solution of
%\begin{equation}
%H^2 \left(1 + a_AK_F \dfrac{\lfa}{r_F} \right) - H \left( a_AK_F\left(1 - \dfrac{\mu_F + L_F \lfa}{r_F} \right) + c \right) + a_A K_F\dfrac{\mu_F L_F}{r_F} = 0
%\label{equilibreFAHA:equationHA}
%\end{equation}
%
%Note that when $L_F = 0$, this equation can be factorized by $H$ and $H^*_{F_AH_A, i}$ is solution of 
%\begin{equation}
%H \left(H \Big(1 + a_AK_F \dfrac{\lfa}{r_F}\Big) - a_AK_F\left(1 - \dfrac{\mu_F}{r_F} \right) - c \right) = 0
%\label{equilibreFAHA:equationHA, LF=0}
%\end{equation}
%
%We note $H^*_{F_AH_A, 2} = \dfrac{a_AK_F\left(1 - \dfrac{\mu_F}{r_F} \right) + c}{1 + a_AK_F \dfrac{\lfa}{r_F}}$ the unique positive solution of this equation. 
%
%Finally, when $L_F = 0$, there is only one equilibrium $EE^{F_AH_A}$ possible, which exists if $1 < T_F(H^*_{F_AH_A, 2}) \Leftrightarrow 1 < T_F(c)$.
%
%When $L_F > 0$, there is no such shortcut and we need to search for the real and positive solutions of equation \eqref{equilibreFAHA:equationHA}. Its discriminant is positive if:
%
%\begin{multline}
%\Delta_{F_AH_A} \geq 0 \Leftrightarrow \\
%\left( a_AK_F\left(1 - \dfrac{\mu_F + L_F \lfa}{r_F} \right) + c \right)^2 \geq  4 \left(1 + a_AK_F \dfrac{\lfa}{r_F} \right)a_A K_F\dfrac{\mu_F L_F}{r_F}
%\label{equilibreFAHA:discriminant}
%\end{multline}

where $H^*_{F_AH_A, i} = a_AF^*_{F_AH_A, i}+c$.
\end{itemize}
 $F^*_{F_AH_A, i}$ are solution of 
\begin{equation}
F^2 + F \left(\dfrac{K_F\Big(\dfrac{\lfa(2c+L_F) + \mu_F}{r_F} - 1\Big) + \dfrac{c}{a_A}}{1 + \dfrac{a_A K_F \lfa}{r_F}}  \right) + \dfrac{K_Fc}{a_A} \dfrac{\Big(\dfrac{c+L_F}{c} \dfrac{\lfa c + \mu_F}{r_F} - 1\Big)}{1 + \dfrac{a_A K_F \lfa}{r_F}} = 0
\label{equilibreFAHA:equationFA}
\end{equation}

Note that when $L_F = 0$, this equation can be factorized by $a_A F + c$ to obtain:
\begin{equation}
\dfrac{1}{a_A}\Big(a_A F + c\Big) \left(F - K_F \dfrac{1 - \dfrac{\mu_F + c \lfa}{r_F}}{1 + a_A K_F \dfrac{\lfa}{r_F}}\right) = 0
\label{equilibreFAHA:equationFA, LF = 0}
\end{equation}
and, in this case, it is clear that there is only one equilibria $EE^{F_AH_A}$ possible, which exists if $1 < T_F(c)$.
\\

When $L_F > 0$, there is no such shortcut and we need to search for the real and positive solutions of equation \eqref{equilibreFAHA:equationFA}. Its discriminant is positive if:

\begin{multline}
\Delta_{F_AH_A} \geq 0 \Leftrightarrow \\
\left(K_F \Big(\dfrac{\lfa(2c+L_F) + \mu_F}{r_F} - 1\Big) + \dfrac{c}{a_A} \right)^2 \geq  4 \dfrac{K_F}{a_A}  \Big(1 + \dfrac{a_A K_F \lfa}{r_F}\Big) \Big((c+L_F) \dfrac{\mu_F + \lfa c}{r_F} - c\Big)
\label{equilibreFAHA:discriminant2}
\end{multline}

If the right hand side is negative, \textit{ie} when $1< T_F(c)$, then the discriminant $\Delta_{F_AH_A}$ is positive, and the constant coefficient of \eqref{equilibreFAHA:equationFA} is negative. Thus, there is always one positive roots (and the other is negative).

Otherwise, when $T_F(c) < 1$, equation \eqref{equilibreFAHA:equationFA} has real and positive roots only if the discriminant is positive and if the coefficient in $F$ is negative. This last condition gives
\begin{subequations}
\begin{align}
& a_AK_F \Big(\dfrac{\lfa(2c+L_F) + \mu_F}{r_F}-1\Big) + c < 0 \\
%& \Leftrightarrow c \Big(1 + \dfrac{2 \lfa a_AK_F}{r_F} \Big) < a_AK_F \Big(1 - \dfrac{\mu_F + \lfa L_F}{r_F} \Big) \\
& \Leftrightarrow 1 < \dfrac{a_AK_F}{c} \dfrac{1 - \dfrac{\mu_F + \lfa L_F}{r_F}}{1 + \dfrac{2 \lfa a_A K_F}{r_F}}
\end{align}
\end{subequations}

In conclusion, if
\begin{equation}
1 < T_F(c)
\label{equilibreFAHA:conditionExistence1}
\end{equation}
there is one equilibrium, noted $EE^{F_AH_A}_2$ ; and if 
{\small
\begin{subequations}
    \begin{empheq}[left={\empheqlbrace\,}]{align}
&T_F(c) < 1 \\
&1 \leq \dfrac{\dfrac{c}{4a_A K_F}}{\dfrac{a_A K_F \lfa}{r_F} + 1}\dfrac{\left(\dfrac{a_A K_F}{c} \Big(\dfrac{\mu_F + \lfa L_F}{r_F} - 1\Big)+ \dfrac{2 \lfa a_A K_F}{r_F} + 1 \right)^2}{\dfrac{c+L_F}{c} \dfrac{\mu_F + \lfa c}{r_F} - 1} \\
& 1 < \dfrac{a_AK_F}{c} \dfrac{1 - \dfrac{\mu_F + \lfa L_F}{r_F}}{1 + \dfrac{2 \lfa a_AK_F}{r_F}}
    \end{empheq}
    \label{equilibreFAHA:conditionExistence2}
\end{subequations}
}
there are two equilibrium $EE^{F_AH_A}_1 \leq EE^{F_AH_A}_2$.

$F^{*}_{F_AH_A,i}$ are equal at
\begin{equation}
F^{*}_{F_AH_A,i} = -\dfrac{1}{2} \dfrac{ K_F\Big(\dfrac{\lfa(2c+L_F) + \mu_F}{r_F} - 1\Big) + \dfrac{c}{a_A}}{1 + \dfrac{a_A K_F \lfa}{r_F}}  \pm \dfrac{1}{2} \sqrt{\Delta_{F_AH_A}}
\label{equilibreFAHA:FA}
\end{equation}
where
\begin{equation}
\Delta_{F_AH_A} =
\left(\dfrac{K_F \Big(\dfrac{\lfa(2c+L_F) + \mu_F}{r_F} - 1\Big) + \dfrac{c}{a_A}}{1 + \dfrac{a_A K_F \lfa}{r_F}} \right)^2 -  4 \dfrac{K_Fc}{a_A}  \dfrac{\dfrac{c+L_F}{c} \dfrac{\mu_F + \lfa c}{r_F} - 1}{1 + \dfrac{a_A K_F \lfa}{r_F}} 
\end{equation}
and $H^*_{F_AH_A,i}$ are equal at
\begin{equation}
H^*_{F_AH_A,i} = a_A F^*_{F_AH_A,i} + c
\label{equilibreFAHA:HA}
\end{equation}

\subsubsection{Equilibrium : stability}

The asymptotic stability of those equilibrium can be investigated using the Jacobian matrix of the system.
It is given by:
\begin{equation}
\mathcal{J}(F_A,H_A) =  \begin{bmatrix}
r_F \dfrac{H_A}{H_A+L_F}(1-\dfrac{2F}{K_F}) - \lfa H_A - \mu_F & r_F \dfrac{L_F}{(H_A+L_F)^2}(1-\dfrac{F_A}{K_F})F_A  - \lfa F_A\\
r_H \dfrac{a_AH_A^2}{(a_AF_A+c)^2} (\dfrac{H_A}{\beta}-1) & r_H(1-\dfrac{H_A}{a_AF_A+c})(\dfrac{2H_A}{\beta}-1) - \dfrac{r_H H_A}{a_AF_A+c}(\dfrac{H_A}{\beta}-1)
\end{bmatrix}
\label{stabilityFAHA:jacobian}
\end{equation}

\begin{itemize}
\item At point $TE = (0, 0)$, the Jacobian is
\begin{itemize}
\item when $L_F = 0$:
\begin{equation}
\mathcal{J}(0,0) = \begin{bmatrix}
r_F-\mu_F & 0 \\
0 & -r_H
\end{bmatrix}
\end{equation}
and thus the eigenvalues are $r_F-\mu_F$ and $-r_H$. $TE$ is asymptotically stable (AS) if $T_F(0) < 1$.
\item when $L_F >0$:
\begin{equation}
\mathcal{J}(0,0) = \begin{bmatrix}
-\mu_F & 0 \\
0 & -r_H
\end{bmatrix}
\end{equation}
and thus the eigenvalues are $-\mu_F$ and $-r_H$. $TE$ is always AS.
\end{itemize}

\item At point $EE^{F_A}$ and when $L_F = 0$, the Jacobian is
\begin{equation}
\mathcal{J}(0, \beta) = \begin{bmatrix}
-r_F\dfrac{F_{F_A}^*}{K_F}& - \lfa F^*_{F_A}\\
0 & -r_H
\end{bmatrix}
\end{equation}
and $EE^{F_A}$ is AS.

\item At point $EE^{H_A}_\beta = \Big(0,\beta \Big)$, the Jacobian is
\begin{equation}
\mathcal{J}(0, \beta) = \begin{bmatrix}
r_F\dfrac{\beta}{\beta+L_F} - \lfa \beta - \mu_F & 0 \\
0 & r_H (1 - \dfrac{\beta}{c})
\end{bmatrix}
\end{equation}
Since $\beta < c$, $EE^{H_A}_\beta$ is never stable.

\item At point $EE^{F_AH_A}_\beta = \Big(F^*_{F_A,\beta},\beta \Big)$, the Jacobian is
\begin{equation}
\mathcal{J}(F^*_{F_A,\beta}, \beta) = \begin{bmatrix}
- r_F\dfrac{\beta}{\beta+L_F} F^*_{F_A,\beta} & \_ \\
0 & r_H (1 - \dfrac{\beta}{c})
\end{bmatrix}
\end{equation}
Eigenvalues appear on the diagonal. Since $\beta < c$, $EE^{F_AH_A}_\beta$ is never stable.


\item At point $EE^{H_A} = \Big(0,c \Big)$, the Jacobian is
\begin{equation}
\mathcal{J}(0,c) = \begin{bmatrix}
r_F \dfrac{c}{L_F +c} - \lfa c - \mu_F & 0 \\
r_H a_A (\dfrac{c}{\beta} - 1) & -r_H(\dfrac{c}{\beta} - 1)
\end{bmatrix}
\end{equation}
$EE^{H_A}$ is AS if $T_F(c) < 1$.

\item At point(s) $EE^{F_AH_A} = \Big(F^*_{F_AH_A}, a_A F^*_{F_AH_A} + c)$, the Jacobian is
\begin{equation}
\mathcal{J}(F^*_{F_AH_A}, H^*_{F_AH_A}) = \begin{bmatrix}
-r_F\dfrac{H^*_{F_AH_A}}{H^*_{F_AH_A} + L_F} \dfrac{V^*_{F_AH_A}}{K_F} & \Big(\dfrac{L_F\mu_F}{H^*_{F_AH_A}} - \lfa H^*_{F_AH_A}) \dfrac{F^*_{F_AH_A}}{H^*_{F_AH_A} + L_F} \\
r_H a_A (\dfrac{H^*_{F_AH_A}}{\beta} - 1) & -r_H(\dfrac{H^*_{F_AH_A}}{\beta} - 1)
\end{bmatrix}
\label{stabilityFAHA:jacobianFAHA}
\end{equation}
We used $r_F \dfrac{H^*_{F_AH_A}}{H^*_{F_AH_A} + L_F} \Big(1 - \dfrac{F^*_{F_AH_A}}{K_F} \Big)F^*_{F_AH_A} = \lfa H^*_{F_AH_A} F^*_{F_AH_A} + \mu_FF^*_{F_AH_A}$ to re-write the second coefficient.

$EE^{F_AH_A}$ is asymptotically stable if the Jacobian's trace is negative and its determinant is positive. We have:
\begin{multline}
det(\mathcal{J}(F^*_{F_AH_A}, H^*_{F_AH_A})) = \\ \Big(\dfrac{H^*_{F_AH_A}}{\beta} - 1\Big) \left( \Big(\dfrac{r_F}{K_F} + a_A \lfa \Big) (H^*_{F_AH_A})^2 - a_A L_F \mu_F \right) \dfrac{r_H F^*_{F_AH_A}}{H^*_{F_AH_A}(H^*_{F_AH_A} + L_F)}
\end{multline}
and
\begin{equation}
Tr(\mathcal{J}(F^*_{F_AH_A}, H^*_{F_AH_A})) = -r_F  \dfrac{H^*_{F_AH_A}}{H^*_{F_AH_A}+L_F}\dfrac{F^*_{F_AH_A}}{K_F} - r_H(\dfrac{H^*_{F_AH_A}}{\beta} - 1)
\end{equation}

Since $H^*_{F_AH_A} = a_A F^*_{F_AH_A} + c > c > \beta$, $Tr(\mathcal{J}(F^*_{F_AH_A}, H^*_{F_AH_A}))$ is always negative.

Moreover, the sign of $det(\mathcal{J}(F^*_{F_AH_A}, H^*_{F_AH_A}))$ is determined by the sign of $$\left( \Big(\dfrac{r_F}{K_F} + a_A\lfa\Big) (H^*_{F_AH_A})^2 - a_A L_F \mu_F \right)$$
\\

Using equation \eqref{equilibreFAHA:equationFA}, we can show that $H^*_{F_AH_A}$ are solutions (see appendix \ref{appendix:equilibreVH:equation H} for details) of

\begin{equation}
H^2 \left(1 + a_AK_F \dfrac{\lfa}{r_F} \right) - H \left( a_AK_F\left(1 - \dfrac{\mu_F + L_F \lfa}{r_F} \right) + c \right) + a_A K_F\dfrac{\mu_F L_F}{r_F} = 0
\label{equilibreFAHA:equationHA2}
\end{equation}

Since $H^*_{F_AH_A}$ are positive solutions of \eqref{equilibreFAHA:equationHA}, the term in $H$ is negative. Then, by using appendix \ref{appendice:ineq2nd}, we know that the term $\left( \Big(\dfrac{r_F}{K_F} + a_A\lfa\Big) (H^*_{F_AH_A})^2 - a_A L_F \mu_F \right)$ is positive for $H^*_{F_AH_A, 2}$ and negative for $H^*_{F_AH_A, 1}$.
\\

Finally, $det(\mathcal{J}(F^*_{F_AH_A}, H^*_{F_AH_A, 2}))$ is positive and $EE^{F_AH_A, 2}$ is AS while $det(\mathcal{J}(F^*_{F_AH_A}, H^*_{F_AH_A, 1}))$ is negative and $EE^{F_AH_A, 1}$ is not stable.
%\\
%
%So, on the first hand, the determinant of $\mathcal{J}(F^*_{F_AH_A,2}, H^*_{F_AH_A,2})$ is positive if $H^*_{F_AH_A, 2} > \beta$, and this condition also ensures that the trace is negative. In conclusion, $EE^{F_AH_A}_2$ is AS if $T_H(F^*_{F_AH_A, 2}) > 1$.
%\\
%
%On the other hand, the determinant of $\mathcal{J}(F^*_{F_AH_A, 1}, H^*_{F_AH_A, 1})$ is positive if $H^*_{F_AH_A, 1} < \beta$.
%The trace is negative at $EE^{F_AH_A}_1$  if 
%\begin{subequations}
%\begin{align}
%&-r_F  \dfrac{H^*_{F_AH_A, 1}}{H^*_{F_AH_A, 1}+L_F}\dfrac{F^*_{F_AH_A, 1}}{K_F} - r_H\Big(\dfrac{H^*_{F_AH_A, 1}}{\beta} - 1\Big) <0 \\
%& -r_F  H^*_{F_AH_A, 1} \dfrac{H^*_{F_AH_A, 1} - c}{a_A K_F} - r_H \Big(\dfrac{H^*_{F_AH_A, 1}}{\beta} - 1\Big)\Big(H^*_{F_AH_A, 1}+L_F \Big) <0 \\
%&(H^*_{F_AH_A, 1})^2 \Big(\dfrac{r_F}{K_Fa_A} + \dfrac{r_H}{\beta} \Big) + H^*_{F_AH_A, 1} \Big(-\dfrac{r_F c}{K_F	a_A} + \dfrac{r_H L_F}{\beta} - r_H \Big) - r_H L_F > 0
%\end{align}
%\end{subequations}
%
%Let define $\tilde{Tr}(H) := H^2\Big(\dfrac{r_F}{r_H}\dfrac{\beta}{K_F a_A} + 1 \Big) + H \Big(L_F - \beta - c \dfrac{r_F}{r_H}\dfrac{\beta}{K_F	a_A}\Big) - \beta L_F$. $\tilde{Tr}$ admits two real roots, one negative and one positive. Moreover, since $\tilde{Tr}(c) = r_H (c+ L_F) (\dfrac{c}{\beta}-1)$ and $\tilde{Tr}(\beta) = \beta^2 \dfrac{r_F}{K_F}(1 - \dfrac{c}{\beta})$, the intermediate value theorem allows to say that the positive root is in $(c, \beta)$. We note this positive root $H_{min}$.
%
%\begin{figure}[ht]
%\centering
%\begin{tikzpicture}
%\draw (0,-1.5)[->] -- (0,2.5)node[left]{$\tilde{Tr}(H)$};
%\draw (-0.25, 0)[->] -- (5,0)node[below]{$H$};
%\draw plot [domain=-0.25:4.5](\x, 0.1*\x^2 + 0.1*\x - 1);
%\draw (0.25, -0.1) -- (0.25, 0.1) node[above] {$c$};
%\draw (4.25, -0.1) -- (4.25, 0.1) node[above] {$\beta$};
%\draw (2.7015, -0.1)node[below] {$H_{min}$} -- (2.7015, 0.1) ;
%\end{tikzpicture}
%\caption{\centering Curve of $\tilde{Tr}$, which has two real roots, one positive and one negative. We are looking for conditions such that $\tilde{Tr}(H_{F_AH_A,1}^*) > 0$, with $H_{F_AH_A,1}^*\in (c, \beta)$. }
%\label{fig:Tr tilde}
%\end{figure}
%
%And since $\tilde{Tr}$ is monotone on $(c, \beta)$, we have $\tilde{Tr}(H_{F_AH_A,1}^*)$ if $H_{min} < H_{F_AH_A,1}^*$.
%\\
%
%Finally, $EE^{F_AH_A}_1$ is AS if 
%\begin{equation}
%H_{min} < H^*_{F_AH_A, 1} < \beta
%\end{equation}
%where 
%\begin{equation}
%H_{min} := -\dfrac{L_F - \beta - \dfrac{c r_F}{r_H}\dfrac{\beta}{K_F a_A}}{2\dfrac{r_F}{r_H}\dfrac{\beta}{K_F a_A} + 2} + \dfrac{\sqrt{\Big(L_F - \beta - \dfrac{r_F}{r_H}\dfrac{c \beta}{K_F a_A}\Big)^2 + 4 \beta L_F \Big(\dfrac{r_F\beta}{r_H K_F a_A} + 1\Big)}}{2\dfrac{r_F}{r_H}\dfrac{\beta}{K_F a_A} + 2} 
%\label{equilibreFAVA:Hmin}
%\end{equation}
\end{itemize}


\subsubsection{Summary of the long term dynamic}
Let recall the definition of function \eqref{threshold:TF(H)} and \eqref{threshold:TH(F)}:
\begin{equation}
T_F(H) = \dfrac{H}{H + L_F}\dfrac{r_F}{\mu_F + \lfa H}
\tag{\ref{threshold:TF(H)}}
\end{equation}
\begin{equation}
T_H(F) = \dfrac{a_A F + c }{\beta}
\tag{\ref{threshold:TH(F)}}
\end{equation}

\begin{equation}
T_{\Delta_{FH}} = \dfrac{\dfrac{c}{4a_A K_F}}{\dfrac{a_A K_F \lfa}{r_F} + 1}\dfrac{\left(\dfrac{a_A K_F}{c} \Big(\dfrac{\mu_F + \lfa L_F}{r_F} - 1\Big)+ \dfrac{2 \lfa a_A K_F}{r_F} + 1 \right)^2}{\dfrac{c+L_F}{c} \dfrac{\mu_F + \lfa c}{r_F} - 1}
\end{equation}
and
\begin{equation}
T_R = \dfrac{a_AK_F}{c} \dfrac{1 - \dfrac{\mu_F + \lfa L_F}{r_F}}{1 + \dfrac{2 \lfa a_AK_F}{r_F}}
\end{equation}

All those thresholds are used in the following table to summarize the different cases:

\begin{table}[!ht]
\centering
\caption{\centering Stability and existence for subsystem $F_A-H_A$ when $L_F = 0$}
\begin{tabular}{c|c|c}
$T_F(0)$  & $T_F(c)$ & Equilibrium \\
\hline
 $<1$ & &  $TE$, $EE^{H_A}$ \\
\cline{1-3}
  \multirow{2}{*}{$1<$}  & $<1$&  $EE^{F_A}$, $EE^{H_A}$ \\
& $1<$ & $EE^{F_A}$, $EE^{F_AH_A}_2$ \\
\hline
\end{tabular}
\end{table}
%
%\begin{table}[!ht]
%\centering
%\caption{\centering Stability and existence for subsystem $F_A-H_A$ when $L_F = 0$: condition in brackets are required conditions but implied by the other row's elements}
%\begin{tabular}{c|c|c|c|c|c}
%$T_H(0)$ & $T_F(0)$  & $T_F(\beta)$ & $T_F(c)$ & $T_H(F^*_{F_AH_A, 2})$  & Equilibrium \\
%\hline
%\multirow{3}{*}{$1<$} & $<1$ & & & & $TE$, $EE^{H_A}$ \\
%\cline{2-6}
% &  \multirow{2}{*}{$1<$}&  & $<1$& & $EE^{F_A}$, $EE^{H_A}$ \\
% & & & $1<$ & &$EE^{F_A}$, $EE^{F_AH_A}_2$ \\
%\hline
%\multirow{5}{*}{$<1$}& $<1$ & & && $TE$, $EE^{H_A}_\beta$ \\
%\cline{2-6}
% & \multirow{4}{*}{$1<$} &\multirow{2}{*}{$<1$} &$1<$ & $1<$& $EE^{F_A}$, $EE^{H_A}_\beta$, $EE^{F_AH_A}_2$ \\
% \cline{4-6}
% & & &  & &  $EE^{F_A}$, $EE^{H_A}_\beta$ \\
%\cline{3-6}
% & &\multirow{2}{*}{$<1$} &$(1<)$ & $1<$& $EE^{F_A}$, $EE^{F_AH_A}_\beta$, $EE^{F_AH_A}_2$ \\
% \cline{4-6}
% & & &  & &  $EE^{F_A}$, $EE^{F_AH_A}_\beta$ \\
% \hline
%\end{tabular}
%\end{table}

\begin{table}[!ht]
\centering
\caption{\centering Stability and existence for subsystem $F_A-H_A$ when $L_F > 0$}
\begin{tabular}{c|c|c}
 $T_F(c)$ & $T_{\Delta_{F_AH_A}}$  & Equilibrium \\
 & and $T_R$ &  \\
\hline
 \multirow{2}{*}{$<1$} &$<1$& $TE, EE^{H_A}, EE^{F_AH_A}_2$ \\
 & & $TE, EE^{H_A}$ \\
 \cline{1-3}
$1<$ & & $TE, EE^{F_AH_A}_2$ \\
\hline
\end{tabular}
\end{table}

%\begin{table}[!ht]
%\centering
%\caption{\centering Stability and existence for subsystem $F_A-H_A$ when $L_F > 0$: condition in brackets are required conditions but implied by the other row's elements}
%\begin{tabular}{c|c|c|c|c|c|c}
%$T_H(0)$ & $T_F(\beta)$ & $T_F(c)$ & $T_{\Delta_{F_AH_A}}$ & $T_H(F^*_{F_AH_A,1})$  & $T_H(F^*_{F_AH_A,2})$ & Equilibrium \\
% & & & and $T_R$ & and $T_{H_{min}}$  &  &  \\
%\hline
%\multirow{3}{*}{$1<$} & & \multirow{2}{*}{$<1$} &$<1$& &$(1<)$ &$TE, EE^{H_A}, EE^{F_AH_A}_2$ \\
% & & & & & &$TE, EE^{H_A}$ \\
% \cline{2-7}
% & &$1<$ & & & $(1<)$ &$TE, EE^{F_AH_A}_2$ \\
%\hline
%\multirow{10}{*}{$<1$} & \multirow{5}{*}{$<1$} & \multirow{3}{*}{$<1$} &\multirow{3}{*}{$<1$} & \multirow{2}{*}{$<1$} & $1<$ & $TE, EE^{H_A}_\beta, EE^{F_AH_A}_1, EE^{F_AH_A}_2$ \\
% & & & & & $<1$ & $TE, EE^{H_A}_\beta, EE^{F_AH_A}_1$ \\
% \cline{5-7}
% & & & & & $1<$ & $TE, EE^{H_A}_\beta, EE^{F_AH_A}_2$ \\
% \cline{3-7}
% & & $1<$ & & & $1<$ & $TE, EE^{H_A}_\beta, EE^{F_AH_A}_2$ \\
% \cline{3-7}
% & & & & & & $TE, EE^{H_A}_\beta$ \\
% \cline{2-7}
%& \multirow{5}{*}{$1<$} & \multirow{3}{*}{$<1$} &\multirow{3}{*}{$<1$} & \multirow{2}{*}{$<1$} & $1<$ & $TE, EE^{F_AH_A}_\beta, EE^{F_AH_A}_1, EE^{F_AH_A}_2$ \\
% & & & & & $<1$ & $TE, EE^{F_AH_A}_\beta, EE^{F_AH_A}_1$ \\
% \cline{5-7}
% & & & & & $1<$ & $TE, EE^{F_AH_A}_\beta, EE^{F_AH_A}_2$ \\
% \cline{3-7}
% & & $1<$ & & & $1<$ & $TE, EE^{F_AH_A}_\beta, EE^{F_AH_A}_2$ \\
% \cline{3-7}
% & & & & & & $TE, EE^{F_AH_A}_\beta$ \\
% \hline
%\end{tabular}
%\end{table}

\subsubsection{Limit cycle}
We will show that the system $F_A-H_A$ \textbf{does not admit a limit cycle}.

To do so, we will consider two subset $\Omega_1 = \{(F_A,H_A) \in [0, +\infty) \times [\beta, +\infty)\}$ and $\Omega_2 = \{(F_A,H_A) \in [0, +\infty) \times [0, \beta])\}$ . 


On $\Omega_1$, we will use the Bendixson-Dulac theorem (see \cite{farkas_1994_periodic}, page 137). Consider the function

\begin{align}
\phi : (0, +\infty) &\times (\beta, +\infty) \rightarrow \mathbf{R} \label{limit cylce:phiFH}
\\
\nonumber
(F_A,H_A) & \mapsto \dfrac{1}{F_A H_A (H_A/\beta - 1)}
\end{align}

Multiplying the right hand side of system $F_A-H_A$ \eqref{submodelAnthropoFH} by this function, and taking the derivative with respect to $F$ or $H$ we obtain :
\begin{subequations}
\begin{align}
&\dfrac{\partial f_{1,FH} \times \phi}{\partial F}(F,H) = - \dfrac{r_F}{K_F} \dfrac{1}{H \big(\dfrac{H}{\beta}-1 \big)} \\
&\dfrac{\partial f_{2,FH} \times \phi}{\partial H}(F,H) = - \dfrac{r_H}{(aF + c) F}
\end{align}
\end{subequations}

Then, for $(F, H) \in \overset{\circ}{\Omega}_1$
\begin{equation}
\Big(\dfrac{\partial f_{1,FH} \times \phi}{\partial F} + \dfrac{\partial f_{2,FH} \times \phi}{\partial H}\Big) (F, H) < 0
\end{equation}
and, according to Bendixson-Dulac theorem, there is no limit cycle on $\Omega_1$.

On $\Omega_2$, we will distinguish two cases. First, we assume that $\beta < c$ . Then, we have on $\Omega_2$:
\begin{equation}
f_{2}(F_A,H_A) = r_H \underset{0<}{\underbrace{\Big(1 - \dfrac{H_A}{a_AF_A + c} \Big)}}\underset{<0}{\underbrace{\Big(\dfrac{H_A}{\beta} -1\Big)}} H_A < 0
\end{equation}
since $\beta \leq a_AF_A + c$. $H_A(t)$ is thus decreasing on $\Omega_2$, and there is no limit cycle on $\Omega_2$.
Now, we assume that $c < \beta$. By looking the vector fields in all the different configurations, it is possible to show that there is no limit cycle either.

\newpage

\subsection{Overall model}
We know consider the overall model, given by equations \eqref{modelAnthropo}:
\begin{equation}
\left\{ \begin{array}{l}
\dfrac{dF_{A}}{dt}=r_F\dfrac{H_A}{H_A + L_F} \left(1-\dfrac{F_A}{K_{F}}\right)F_A-\mu_{F}F_A-\lambda_{FH,A}F_AH_A,\\
\dfrac{dV_A}{dt}=r_{V}\dfrac{H_A}{H_A+L_{V}}\left(1-\dfrac{V_A}{K_{V}}\right)V_A-\mu_{V}V_A-\lambda_{VH,A}V_AH_A,\\
\dfrac{dH_A}{dt}=r_{H}\left(1-\dfrac{H_A}{a_{A}F_{A}+b_{A}V_{A}+c}\right)\left(\dfrac{H_A}{\beta}-1\right)H_A.
\end{array}\right.
\tag{\ref{modelAnthropo}}
\end{equation}

Set the following function:
\begin{equation}
T_F(H) = \dfrac{H}{H + L_F}\dfrac{r_F}{\mu_F + \lfa H}
\tag{\ref{threshold:TF(H)}}
\end{equation}
\begin{equation}
T_V(H) = \dfrac{H}{H + L_V} \dfrac{r_V}{\mu_V + \lva H}
\label{threshold:TV(H)}
\end{equation}


\subsubsection{Equilibrium}

Model \eqref{modelAnthropo} admits the following equilibrium:
\begin{itemize}
\item $TE = (0,0,0)$.
\item A faun equilibrium $EE^{F_A} = (F_A^*, 0, 0)$ which exists if $L_F = 0$ and if $1 < T_F(0)$.
\item A human equilibrium $EE^{H_A}_\beta = \Big(0,0,\beta \Big)$.
\item A faun-human equilibrium $EE^{F_AH_A}_\beta = \Big(F^*_{F_A, \beta}, 0, \beta\Big)$ where $F^*_{F_A, \beta} = K_F \Big(1-\dfrac{\beta + L_F}{\beta}\dfrac{\mu_F + \beta \lfa}{r_F} \Big)$. It exists if $1 < T_F(\beta)$.
\item A vegetation-human equilibrium $EE^{V_AH_A}_\beta = (0, V_{V_A, \beta}^*, \beta)$ with $V_{V_A, \beta}^* = K_V \Big( 1- \dfrac{\beta + L_V}{\beta} \dfrac{\mu_V + \lva \beta}{r_V} \Big)$. This equilibria exists when $1 < T_V(\beta)$.
\item A faun-vegetation-human equilibrium $EE^{F_AV_AH_A}_\beta = (F^*_{F_A, \beta}, V_{V_A, \beta}^*, \beta)$. This equilibria exists when $1 < T_F(\beta)$ and $1 < T_V(\beta)$.
\item A human equilibria $EE^{H_A} = (0, 0, c)$ which always exists.
\item One or two faun-human equilibrium $EE^{F_AH_A}_i = \Big(F^*_{F_AH_A, i}, 0, a_A F^*_{F_AH_A, i}+c\Big)$ 
\item One or two vegetation-human equilibrium $EE^{V_AH_A}_i = \Big(0, V^*_{V_AH_A, i}, bV^*_{V_AH_A, i}+c \Big)$.
\item One or two faun-vegetation-human equilibrium $EE^{F_AV_AH_A}_i$. Conditions of existence are studied below. 
\end{itemize}


Equilibrium $EE^{F_AV_AH_A}_i$ are given by 

\begin{multline}
EE^{F_AV_AH_A}_i =\\ \left(K_F \Big(1 - \dfrac{H^*_{F_AV_AH_A,i} + L_F}{H^*_{F_AV_AH_A,i}}\dfrac{\mu_F + \lfa H^*_{F_AV_AH_A,i}}{r_F}\Big), \right. \\ \left. \vphantom{\dfrac{H}{H}} K_V \Big(1 - \dfrac{H^*_{F_AV_AH_A,i} + L_V}{H^*_{F_AV_AH_A,i}}\dfrac{\mu_V + \lva H^*_{F_AV_AH_A,i}}{r_V}\Big), \right. \\ \left. \vphantom{\dfrac{H}{H}} H^*_{F_AV_AH_A,i} \right)
\end{multline}
$H^*_{F_AV_AH_A,i}$ are solutions of

\begin{multline}
H^2 \left( 1+ a_AK_F \dfrac{\lfa}{r_F} + b_A K_V \dfrac{\lva}{r_V} \right) - \\
H \left(a_A K_F \Big(1-\dfrac{\mu_F + L_F\lfa}{r_F}\Big) + b_A K_V \Big(1-\dfrac{\mu_V + L_V\lva}{r_V}\Big) +c \right) + \\
a_AK_F\dfrac{L_F \mu_F}{r_F} + b_A K_V \dfrac{\mu_V L_V}{r_V} = 0
\label{equilibreFAVAHA:equationHA}
\end{multline}


The discriminant of this equation is positive if:
\begin{multline}
\Delta_{F_AV_AH_A} \geq 0 \Leftrightarrow \left(a_A K_F \Big(1-\dfrac{\mu_F + L_F\lfa}{r_F}\Big) + b_A K_V \Big(1-\dfrac{\mu_V + L_V\lva}{r_V}\Big) +c \right)^2 \geq \\ 4 \left(a_AK_F\dfrac{L_F \mu_F}{r_F} + b_A K_V \dfrac{\mu_V L_V}{r_V} \right)\left( 1+ a_AK_F \dfrac{\lfa}{r_F} + b_A K_V \dfrac{\lva}{r_V} \right)
\end{multline}

When it is the case, equation \eqref{equilibreFAVAHA:equationHA} admits two real solutions, which are positive if
\begin{equation}
0 < a_A K_F \Big(1-\dfrac{\mu_F + L_F\lfa}{r_F}\Big) + b_A K_V \Big(1-\dfrac{\mu_V + L_V\lva}{r_V}\Big) + c
\end{equation}

Moreover, the positive solutions must satisfy $1 < T_F(H^*_{F_AV_AH_A,i})$ and $1 < T_V(H^*_{F_AV_AH_A,i})$ in order to get positive values of $F^*_{F_AV_AH_A,i}$ and $V^*_{F_AV_AH_A,i}$.

Finally, condition of existence are given by:
\begin{subequations}
    \begin{empheq}[left={\empheqlbrace\,}]{align}
&0 \leq \Delta_{F_AV_AH_A} \\
&0 < a_A K_F \Big(1-\dfrac{\mu_F + L_F\lfa}{r_F}\Big) + b_A K_V \Big(1-\dfrac{\mu_V + L_V\lva}{r_V}\Big) + c \\
&1 < T_F(H^*_{F_AV_AH_A,i}) \\
&1 < T_V(H^*_{F_AV_AH_A,i})
    \end{empheq}
\end{subequations}

\subsubsection{Stability}
The stability of those equilibrium can be determined using the Jacobian matrix of the system. It is given by:
{\footnotesize
\begin{multline}
\mathcal{J}(F,V,H) = \\ \begin{bmatrix}
\dfrac{r_F H}{H+L_F} \Big(1-\dfrac{2F}{K_F} \Big) - \mu_F - \lfa H & 0 & \dfrac{r_F L_F}{(H+L_F)^2}(1-\dfrac{F}{K_F})F  - \lfa F \\
0 &  \dfrac{r_V H}{H+L_V}(1-\dfrac{2V}{K_V}) - \lva H - \mu_V & \dfrac{r_V L_V}{(H+L_V)^2}(1-\dfrac{V}{K_V})V  - \lva V\\
\dfrac{r_H a_A H^2}{(a_AF+b_AV+c)^2} (\dfrac{H}{\beta}-1) & \dfrac{r_H b_A H^2}{(a_AF+b_AV+c)^2} (\dfrac{H}{\beta}-1) & r_H(1-\dfrac{H}{a_A F+b_A V+c})(\dfrac{2H}{\beta}-1) - \dfrac{r_H \Big(\dfrac{H}{\beta}-1\Big)H}{a_AF+b_AV+c}
\end{bmatrix}
\label{modelAnthropo:jacobian}
\end{multline}
}


\begin{itemize}
\item At point $TE$, the jacobian is:
\begin{itemize}
\item when $L_F = 0$:
\[
\mathcal{J}(0,0,0) = \begin{bmatrix}
r_F- \mu_F & 0 & 0 \\ 0 &- \mu_V & 0 \\ 0 & 0 & -r_H
\end{bmatrix}
\] and $TE$ is AS if $T_F(0) < 1$.
\item when $L_F > 0$:
\[
\mathcal{J}(0,0,0) = \begin{bmatrix}
- \mu_F & 0 & 0 \\ 0 &- \mu_V & 0 \\ 0 & 0 & -r_H
\end{bmatrix}
\] 
and $TE$ is always AS
\end{itemize}

\item at point $EE^{F_A}$, when $L_F = 0$:
\[
\mathcal{J}(F^*_A,0,0) = \begin{bmatrix}
- r_F \dfrac{F_A^*}{K_F} & 0 & -\lfa F^*_A \\ 0 &- \mu_V & 0 \\ 0 & 0 & -r_H
\end{bmatrix}
\]
and $EE^{F_A}$ is AS.
\item at point $EE^{H_A}_\beta$:
\[
\mathcal{J}(0,0,\beta) = \begin{bmatrix}
\dfrac{r_F \beta}{\beta + L_F} - \mu_F - \lfa \beta & 0 & 0 \\ 0 &\dfrac{r_V \beta}{\beta + L_V} - \mu_V - \lva \beta & 0 \\ 0 & 0 & r_H \Big(1-\dfrac{\beta}{c}\Big)
\end{bmatrix}
\]
Since $\beta < c$, $EE^{H_A}_\beta$ is never stable.
\item At point $EE^{F_AH_A}_\beta$:
\[
\mathcal{J}(F^*_{F_A, \beta},0,\beta) = \begin{bmatrix}
-\dfrac{r_F \beta}{\beta + L_F}\dfrac{F^*_{F_A, \beta}}{K_F} & 0 & \dfrac{r_F L_F}{(\beta+L_F)^2}\left(1-\dfrac{F^*_{F_A, \beta}}{K_F}\right)F^*_{F_A, \beta}  - \lfa F^*_{F_A, \beta} \\ 
0 &\dfrac{r_V \beta}{\beta + L_V} - \mu_V - \lva \beta & 0 \\ 
0 & 0 & r_H \Big(1-\dfrac{\beta}{a_AF^*_{A, \beta} + c}\Big)
\end{bmatrix}
\]
Since $\beta < c$, $EE^{F_AH_A}_\beta$ is never stable.
\item At point $EE^{V_AH_A}_\beta$:
\[
\mathcal{J}(0,V^*_{V_A, \beta},\beta) = \begin{bmatrix}
\dfrac{r_F \beta}{\beta + L_F} - \mu_F - \lfa \beta & 0 & 0\\
0 & -\dfrac{r_V \beta}{\beta + L_V}\dfrac{V^*_{V_A, \beta}}{K_V} & \dfrac{r_V L_V}{(\beta+L_V)^2}\left(1-\dfrac{V^*_{V_A, \beta}}{K_V}\right)V^*_{V_A, \beta}  - \lva V^*_{V_A, \beta} \\
0 & 0 & r_H \Big(1-\dfrac{\beta}{b_AV^*_{V_A, \beta} + c}\Big)
\end{bmatrix}
\]
Since $\beta < c$, $EE^{V_AH_A}_\beta$ is never stable.
\item At point $EE^{F_AV_AH_A}_\beta$:
\[
\mathcal{J}(F^*_{F_A, \beta},V^*_{V_A, \beta},\beta) = \begin{bmatrix}
-\dfrac{r_F \beta}{\beta + L_F}\dfrac{F^*_{F_A, \beta}}{K_F} & 0 & \dfrac{r_F L_F}{(\beta+L_F)^2}\left(1-\dfrac{F^*_{F_A, \beta}}{K_F}\right)F^*_{F_A, \beta}  - \lfa F^*_{F_A, \beta}\\
0 & -\dfrac{r_V \beta}{\beta + L_V}\dfrac{V^*_{V_A, \beta}}{K_V} & \dfrac{r_V L_V}{(\beta+L_V)^2}\left(1-\dfrac{V^*_{V_A, \beta}}{K_V}\right)V^*_{V_A, \beta}  - \lva V^*_{V_A, \beta} \\
0 & 0 & r_H \Big(1-\dfrac{\beta}{a_AF^*_{F_A, \beta} + b_AV^*_{V_A, \beta} + c}\Big)
\end{bmatrix}
\]
Since $\beta < c$, $EE^{F_AV_AH_A}_\beta$ is never stable.
\item At point $EE^{H_A} = (0, 0, c)$:
\[
\mathcal{J}(0,0,c) = \begin{bmatrix}
\dfrac{r_F c}{c + L_F} - \mu_F - \lfa c & 0 & 0\\
0 &\dfrac{r_V c}{c + L_V} - \mu_V - \lva c & 0\\
r_H a_A \Big(\dfrac{c}{\beta}-1\Big) & r_H b_A \Big(\dfrac{c}{\beta}-1\Big) & -r_H \Big(\dfrac{c}{\beta}-1\Big)
\end{bmatrix}
\]
and $EE^{H_A}$ is AS if $T_F(c) < 1$ and $T_V(c) < 1$.
\item At point $EE^{F_AH_A}_i$:
\begin{multline*}
\mathcal{J}(F^*_{F_AH_A, i},0,H^*_{F_AH_A, i}) = \\ \begin{bmatrix}
-\dfrac{r_F H^*_{F_AH_A, i}}{H^*_{F_AH_A, i} + L_F}\dfrac{F^*_{F_AH_A, i}}{K_F} & 0 & \left(\dfrac{r_F L_F \left(1-\dfrac{F^*_{F_AH_A, i}}{K_F}\right)}{(H^*_{F_AH_A, i}+L_F)^2}  - \lfa\right) F^*_{F_AH_A, i} \\
0 &\dfrac{r_V H^*_{F_AH_A, i}}{H^*_{F_AH_A, i} + L_V} - \mu_V - \lva H^*_{F_AH_A, i} & 0\\ 
 r_H a_A \Big(\dfrac{H^*_{F_AH_A, i}}{\beta}-1\Big) & r_H b_A \Big(\dfrac{H^*_{F_AH_A, i}}{\beta}-1\Big) & -r_H \Big(\dfrac{H^*_{F_AH_A, i}}{\beta}-1\Big)
\end{bmatrix}
\end{multline*}
The characteristic polynomial of this matrix is given by:
\begin{multline}
\chi\Big(\mathcal{J}(F^*_{F_AH_A, i},0,H^*_{F_AH_A, i})\Big) = \left(X - \dfrac{r_V H^*_{F_AH_A, i}}{H^*_{F_AH_A, i} + L_V} + \mu_V + \lva H^*_{F_AH_A, i}\right) \times \\
\begin{vmatrix}
X + \dfrac{r_F H^*_{F_AH_A, i}}{H^*_{F_AH_A, i} + L_F}\dfrac{F^*_{F_AH_A, i}}{K_F} & \left(\dfrac{r_F L_F \left(1-\dfrac{F^*_{F_AH_A, i}}{K_F}\right)}{(H^*_{F_AH_A, i}+L_F)^2}  - \lfa\right) F^*_{F_AH_A, i} \\
 -r_H a_A \Big(\dfrac{H^*_{F_AH_A, i}}{\beta}-1\Big) & X + r_H \Big(\dfrac{H^*_{F_AH_A, i}}{\beta}-1\Big)
\end{vmatrix}
\end{multline}
This 2 by 2 determinant is the characteristic polynomial of the Jacobian's sub-system $F_A-H_A$, and its stability has already been studied: see equation \eqref{stabilityFAHA:jacobianFAHA}.
Thus, we can use the result obtained to say that:
\begin{itemize}
\item $EE^{F_AH_A}_2$ is AS if $T_V(H^*_{F_AH_A, 2}) < 1$.
\item $EE^{F_AH_A}_1$ is never stable.
\end{itemize}
 

\item At point $EE^{V_AH_A}_i$:
\begin{multline*}
\mathcal{J}(0, V^*_{V_AH_A, i},H^*_{V_AH_A, i}) = \\ \begin{bmatrix}
\dfrac{r_F H^*_{V_AH_A, i}}{H^*_{V_AH_A, i} + L_F} - \mu_F - \lfa H^*_{V_AH_A, i} & 0 & 0\\ 
0& -\dfrac{r_V H^*_{V_AH_A, i}}{H^*_{V_AH_A, i} + L_V}\dfrac{V^*_{V_AH_A, i}}{K_V} & \left(\dfrac{r_V L_V \left(1-\dfrac{V^*_{V_AH_A, i}}{K_V}\right)}{(H^*_{V_AH_A, i}+L_V)^2}  - \lva\right) V^*_{V_AH_A, i} \\
 r_H a_A \Big(\dfrac{H^*_{V_AH_A, i}}{\beta}-1\Big) & r_H b_A \Big(\dfrac{H^*_{V_AH_A, i}}{\beta}-1\Big) & -r_H \Big(\dfrac{H^*_{V_AH_A, i}}{\beta}-1\Big)
\end{bmatrix}
\end{multline*}
The characteristic polynomial of this matrix is given by:
\begin{multline}
\chi\Big(\mathcal{J}(0, V^*_{V_AH_A, i},H^*_{V_AH_A, i})\Big) = \left(X - \dfrac{r_F H^*_{V_AH_A, i}}{H^*_{V_AH_A, i} + L_F} + \mu_F + \lfa H^*_{V_AH_A, i}\right) \times \\
\begin{vmatrix}
X + \dfrac{r_V H^*_{V_AH_A, i}}{H^*_{V_AH_A, i} + L_V}\dfrac{V^*_{V_AH_A, i}}{K_V} & \left(\dfrac{r_V L_V \left(1-\dfrac{V^*_{V_AH_A, i}}{K_V}\right)}{(H^*_{V_AH_A, i}+L_V)^2}  - \lva\right) V^*_{V_AH_A, i} \\
-r_H a_A \Big(\dfrac{H^*_{V_AH_A, i}}{\beta}-1\Big) & X + r_H \Big(\dfrac{H^*_{V_AH_A, i}}{\beta}-1\Big)
\end{vmatrix}
\end{multline}
This 2 by 2 determinant is the Jacobian's characteristic polynomial of the sub-system $F_A-H_A$, and its stability has already been studied: see equation \eqref{stabilityFAHA:jacobianFAHA}.
Thus, we can use the result obtained to say that:
\begin{itemize}
\item $EE^{V_AH_A}_2$ is AS if $T_F(H^*_{V_AH_A, 2}) < 1$.
\item $EE^{V_AH_A}_1$ is never stable.
\end{itemize}

\item At point $EE^{F_AV_AH_A}_i$:
\begin{multline*}
\mathcal{J}(F^*_{F_AV_AH_A, i}, V^*_{F_AV_AH_A, i}, H^*_{F_AV_AH_A, i}) = \\ \small{ \begin{bmatrix}
-\dfrac{r_F H^*_{F_AV_AH_A, i}}{H^*_{F_AV_AH_A, i} + L_F}\dfrac{F^*_{F_AV_AH_A, i}}{K_F} & 0 & \left(\dfrac{L_F\mu_F}{H^*_{F_AV_AH_A, i}} - \lfa H^*_{F_AV_AH_A, i}\right) \dfrac{F^*_{F_AV_AH_A, i}}{H^*_{F_AV_AH_A, i}+L_F} \\
0& -\dfrac{r_V H^*_{F_AV_AH_A, i}}{H^*_{F_AV_AH_A, i} + L_V}\dfrac{V^*_{F_AV_AH_A, i}}{K_V} & \left(\dfrac{L_V\mu_V}{H^*_{F_AV_AH_A, i}} - \lva H^*_{F_AV_AH_A, i} \right) \dfrac{V^*_{F_AV_AH_A, i}}{H^*_{F_AV_AH_A, i} + L_V} \\
 r_H a_A \Big(\dfrac{H^*_{F_AV_AH_A, i}}{\beta}-1\Big) & r_H b_A \Big(\dfrac{H^*_{F_AV_AH_A, i}}{\beta}-1\Big) & -r_H \Big(\dfrac{H^*_{F_AV_AH_A, i}}{\beta}-1\Big)
\end{bmatrix}}
\end{multline*}

The characteristic polynomial of $\mathcal{J}(F^*_{F_AV_AH_A}, V^*_{F_AV_AH_A}, H^*_{F_AV_AH_A})$ is 
\begin{equation}
\chi_{F_AV_AH_A} = X^3 + q_{2,i} X^2 + q_{1,i} X + q_{0,i}
\end{equation}
with

\begin{subequations}
\begin{equation}
q_{2,i} = r_F \dfrac{H^*_{F_AV_AH_A, i}}{H^*_{F_AV_AH_A, i}+L_F} \dfrac{F^*_{F_AV_AH_A, i}}{K_F} + r_V \dfrac{H^*_{F_AV_AH_A, i}}{H^*_{F_AV_AH_A, i} + L_V}\dfrac{V^*_{F_AV_AH_A, i}}{K_V} + r_H\Big(\dfrac{H^*_{F_AV_AH_A, i}}{\beta} - 1\Big)
\end{equation}
\begin{multline}
q_{1,i} = r_F \dfrac{H^*_{F_AV_AH_A, i}}{H^*_{F_AV_AH_A, i}+ L_F} \dfrac{F^*_{F_AV_AH_A, i}}{K_F} \times r_V \dfrac{H^*_{F_AV_AH_A, i}}{H^*_{F_AV_AH_A, i} + L_V}\dfrac{V^*_{F_AV_AH_A, i}}{K_V} + \\ 
r_F \dfrac{H^*_{F_AV_AH_A, i}}{H^*_{F_AV_AH_A, i}+ L_F} \dfrac{F^*_{F_AV_AH_A, i}}{K_F} \times r_H \Big(\dfrac{H^*_{F_AV_AH_A, i}}{\beta} - 1\Big) +  \\ 
r_V \dfrac{H^*_{F_AV_AH_A, i}}{H^*_{F_AV_AH_A, i} + L_V}\dfrac{V^*_{F_AV_AH_A, i}}{K_V} \times r_H\Big(\dfrac{H^*_{F_AV_AH_A, i}}{\beta} - 1\Big) - \\
 a_A r_H \Big( \dfrac{H^*_{F_AV_AH_A, i}}{\beta} -1 \Big) \left(\dfrac{L_F\mu_F}{H^*_{F_AV_AH_A, i}} - \lfa H^*_{F_AV_AH_A, i}\right) \dfrac{F^*_{F_AV_AH_A, i}}{H^*_{F_AV_AH_A, i}+L_F} - \\
b_A r_H \Big( \dfrac{H^*_{F_AV_AH_A, i}}{\beta} -1 \Big) \left(\dfrac{L_V\mu_V}{H^*_{F_AV_AH_A, i}} - \lva H^*_{F_AV_AH_A, i}\right) \dfrac{V^*_{F_AV_AH_A, i}}{H^*_{F_AV_AH_A, i}+L_V}
\end{multline}
\begin{multline}
q_{0,i} = \dfrac{F^*_{F_AV_AH_A, i}V^*_{F_AV_AH_A, i}}{\Big(H^*_{F_AV_AH_A, i} + L_V\Big) \Big(H^*_{F_AV_AH_A, i} + L_F\Big)} \dfrac{r_Hr_Fr_V}{K_FK_V} \times \\ \left( \Big(H^*_{F_AV_AH_A, i}\Big)^2 \Big(1 + a_AK_F\dfrac{\lfa}{r_F} + b_AK_V \dfrac{\lva}{r_V} \Big) - \Big(a_AK_F \dfrac{L_F\mu_F}{r_F} + b_A K_V \dfrac{\mu_VL_V}{r_V} \Big)\right) \times \\ \left(\dfrac{H^*_{F_AV_AH_A, i}}{\beta}-1 \right)
\end{multline}
\end{subequations}

According to the Ruth-Hurwitz criterion, the equilibrium is AS if $q_2 > 0$, $q_0 > 0$ and $q_2 q_1 - q_0 > 0$. We will look for the sign of those quantity.

First, let start by the sign of $q_0$.
Using appendix \ref{appendice:ineq2nd} and equation \eqref{equilibreFAVAHA:equationHA}, we know that quantity $\left( \Big(H^*_{F_AV_AH_A}\Big)^2 \Big(1 + a_AK_F\dfrac{\lfa}{r_F} + b_AK_V \dfrac{\lva}{r_V} \Big) - \Big(a_AK_F \dfrac{L_F\mu_F}{r_F} + b_A K_V \dfrac{\mu_VL_V}{r_V} \Big)\right)$ is positive for $H^*_{F_AV_AH_A, 2}$ and negative for $H^*_{F_AV_AH_A,1}$. Then, $q_{0,2}$ is positive if $\beta < H^*_{F_AV_AH_A, 2}$ and $q_{0,1}$ is positive if $H^*_{F_AV_AH_A, 1} < \beta$.


Since $H^*_{F_AV_AH_A, 1} > c > \beta$, $EE^{F_AV_AH_A}_1$ is never stable.

Condition $\beta < H^*_{F_AV_AH_A, 2}$ is also sufficient to ensure $q_{2,2} > 0$.

I introduce notation:
\begin{subequations}
\begin{equation}
R_F = r_F \dfrac{H^*_{F_AV_AH_A, 2}}{H^*_{F_AV_AH_A, 2}+L_F} \dfrac{F^*_{F_AV_AH_A, 2}}{K_F}
\end{equation}
\begin{equation}
R_V = r_V \dfrac{H^*_{F_AV_AH_A, 2}}{H^*_{F_AV_AH_A, 2} + L_V}\dfrac{V^*_{F_AV_AH_A, 2}}{K_V}
\end{equation}
\begin{equation}
R_H = r_H\Big(\dfrac{H^*_{F_AV_AH_A, 2}}{\beta} - 1\Big)
\end{equation}
\end{subequations}
We have then:
\begin{subequations}
\begin{equation}
q_{2,2} = R_F + R_V + R_H
\end{equation}
\begin{multline}
q_{0,2} = \dfrac{R_FR_VR_H}{(H^*_{F_AV_AH_A, 2})^2} \times \\ \left((H^*_{F_AV_AH_A, 2})^2 \Big(1 + aK_F\dfrac{\lfa}{r_F} + bK_V \dfrac{\lva}{r_V}\Big) - \Big(aK_F \dfrac{L_F\mu_F}{r_F} + bK_V \dfrac{L_V \mu_V}{r_V}\Big) \right)
\end{multline}
\begin{multline}
q_{1,2} = R_FR_V + R_FR_H\left(1 - a \Big(\dfrac{L_F\mu_F}{\lfa}-(H^*_{F_AV_AH_A, 2})^2\Big) \dfrac{K_F\lfa}{r_F(H^*_{F_AV_AH_A, 2})^2}\right) + \\ R_VR_H \left( 1 - b\Big(\dfrac{L_V\mu_V}{\lva}-(H^*_{F_AV_AH_A, 2})^2\Big) \dfrac{K_V \lva}{(H^*_{F_AV_AH_A, 2})^2r_V}\right)
\end{multline}
\end{subequations}

and
\begin{subequations}
\begin{multline}
q_{2,2}q_{1,2} = R_F^2R_V + R_FR_V^2 + \\
\left(1 - a \Big(\dfrac{L_F\mu_F}{\lfa}-(H^*_2)^2\Big) \dfrac{K_F\lfa}{r_F(H^*_2)^2}\right) \left(R_F^2R_H + R_FR_H^2\right) + \\
\left( 1 - b\Big(\dfrac{L_V\mu_V}{\lva}-(H^*_2)^2\Big) \dfrac{K_V \lva}{(H^*_2)^2r_V}\right) \left(R_V^2R_H + R_VR_H^2\right) + \\
R_FR_VR_H \left(3 + b K_V \dfrac{\lva}{r_V} + a K_F \dfrac{\lfa}{r_F} - b \dfrac{K_V}{(H^*_2)^2}\dfrac{L_V \mu_V}{r_V} - a \dfrac{K_F}{(H^*_2)^2} \dfrac{L_F \mu_F}{r_F} \right)
\end{multline}
\begin{multline}
q_{2,i}q_{1,i} = R_F^2R_V + R_FR_V^2 + \\
\left(1 - a \Big(\dfrac{L_F\mu_F}{\lfa}-(H^*_2)^2\Big) \dfrac{K_F\lfa}{r_F(H^*_2)^2}\right) \left(R_F^2R_H + R_FR_H^2\right) + \\
\left( 1 - b\Big(\dfrac{L_V\mu_V}{\lva}-(H^*_2)^2\Big) \dfrac{K_V \lva}{r_V (H^*_2)^2}\right) \left(R_V^2R_H + R_VR_H^2\right) + \\
2 R_FR_VR_H + q_{0,2}
\end{multline}
\end{subequations}

We have then :

\begin{multline}
q_{2,2}q_{1,2} -q_{0,2} = R_F^2R_V + R_FR_V^2 + 2 R_FR_VR_H + \\
\left(1 - a \Big(\dfrac{L_F\mu_F}{\lfa}-(H^*_2)^2\Big) \dfrac{K_F\lfa}{r_F(H^*_2)^2}\right) \left(R_F^2R_H + R_FR_H^2\right) + \\
\left( 1 - \dfrac{b K_V}{(H^*_2)^2r_V}\Big(L_V\mu_V- \lva(H^*_2)^2\Big) \right) \left(R_V^2R_H + R_VR_H^2\right) 
\end{multline}
\end{itemize}

\paragraph{Analyse de $q_{2,2}q_{1,2} -q_{0,2}$ : piste 2}

\begin{multline*}
q_{2,2}q_{1,2} -q_{0,2} = R_F^2R_V + R_FR_V^2 + 2 R_FR_VR_H + \\
\left(1 - a \Big(\dfrac{L_F\mu_F}{\lfa}-(H^*_2)^2\Big) \dfrac{K_F\lfa}{r_F(H^*_2)^2}\right) \left(R_F^2R_H + R_FR_H^2\right) + \\
\left( 1 - \dfrac{b K_V}{(H^*_2)^2r_V}\Big(L_V\mu_V- \lva(H^*_2)^2\Big) \right) \left(R_V^2R_H + R_VR_H^2\right) 
\end{multline*}

On note que 
\begin{align*}
&\left(1 - a \Big(\dfrac{L_F\mu_F}{\lfa}-(H^*_2)^2\Big) \dfrac{K_F\lfa}{r_F(H^*_2)^2}\right) > 0 \\
\Leftrightarrow &(H^*_2)^2 \dfrac{r_F}{K_F} + a_A \Big( (H^*_2)^2\lfa - L_F \mu_F\Big) > 0
\end{align*}
 

On peut définir la fonction 
\begin{equation}
S_F(H) = H^2 \dfrac{r_F}{K_F} + a_A \Big(H^2 \lfa  - L_F \mu_F\Big)
\end{equation}
Lorsque $S_F(H^*_2) > 0$ et $S_V(H^*_2) > 0$, on a $q_2q_1 - q_0 > 0$.

Quand $(H_2^*)^2 \geq \dfrac{L_F \mu_F}{\lfa}$, le terme $S_F(H^*_2)$ est positif.

Quand $(H_2^*)^2 < \dfrac{L_F \mu_F}{\lfa}$, on montre (cf dessous) que c'est $S_V(H^*_2)$ qui est positif.

On peut donc distinguer 3 cas : 
\begin{itemize}
\item $(H_2^*)^2 \geq \max \left( \dfrac{L_F \mu_F}{\lfa}, \dfrac{L_V \mu_V}{\lva} \right)$ : alors $S_F(H^*_2) > 0$, $S_V(H^*_2) > 0$ et $q_2q_1 - q_0 > 0$.
\item $\min\left( \dfrac{L_F \mu_F}{\lfa}, \dfrac{L_V \mu_V}{\lva} \right) > (H_2^*)^2$ : alors $S_F(H^*_2) > 0$, $S_V(H^*_2) > 0$ et $q_2q_1 - q_0 > 0$.
\item $\max \left( \dfrac{L_F \mu_F}{\lfa}, \dfrac{L_V \mu_V}{\lva} \right) > (H_2^*)^2 \geq \min\left( \dfrac{L_F \mu_F}{\lfa}, \dfrac{L_V \mu_V}{\lva} \right)$ : alors ???
\end{itemize}

Preuve que $(H_2^*)^2 < \dfrac{L_V \mu_V}{\lva}$, implique $S_F(H^*_2) > 0$:


\begin{align*}
& S_F(H^*_2) > 0 \\
\Leftrightarrow & (H_2^*)^2 \left(1 + \dfrac{a_A K_F \lfa}{r_F} \right) > \dfrac{a_A K_F \mu_F L_F}{r_F} \\
\Leftrightarrow & H^*_2 \left( a_A K_F \Big(1-\dfrac{\mu_F +L_F \lfa}{r_F}\Big) + b_A K_V \Big(1-\dfrac{\mu_V +L_V \lva}{r_V}\Big) + c \right) - \dfrac{b_A K_V \lva}{r_V  }\Big(H^*_2\Big)^2 - \\ & \dfrac{a_A K_F \mu_F L_F}{r_F} - \dfrac{b_A K_V \mu_V L_V}{r_V} > \dfrac{a_A K_F \mu_F L_F}{r_F} \text{ en utilisant l'équation \eqref{equilibreFAVAHA:equationHA} vérifiée par $H^*$} \\
\Leftrightarrow & H^*_2 \left( a_A K_F \Big(1-\dfrac{\mu_F +L_F \lfa}{r_F}\Big) + b_A K_V \Big(1-\dfrac{\mu_V +L_V \lva}{r_V}\Big) + c \right) > \\ & 2 \dfrac{a_A K_F \mu_F L_F}{r_F} + \dfrac{b_A K_V \mu_V L_V}{r_V} + \dfrac{b_A K_V \lva}{r_V  }\Big(H^*_2\Big)^2
\end{align*}

Or, comme $(H_2^*)^2 < \dfrac{L_V \mu_V}{\lva}$, on a 
$$
2 \dfrac{a_A K_F \mu_F L_F}{r_F} + 2 \dfrac{b_A K_V \mu_V L_V}{r_V} > 2 \dfrac{a_A K_F \mu_F L_F}{r_F} + \dfrac{b_A K_V \mu_V L_V}{r_V} + \dfrac{b_A K_V \lva}{r_V  }\Big(H^*_2\Big)^2  
$$

et on sait (en utlisant l'équation \eqref{equilibreFAVAHA:equationHA} vérifiée par $H^*$) que 
$$
H^*_2 \left( a_A K_F \Big(1-\dfrac{\mu_F +L_F \lfa}{r_F}\Big) + b_A K_V \Big(1-\dfrac{\mu_V +L_V \lva}{r_V}\Big) + c \right) > 2 \dfrac{a_A K_F \mu_F L_F}{r_F} + 2 \dfrac{b_A K_V \mu_V L_V}{r_V} 
$$

\paragraph{Analyse de $q_{2,2}q_{1,2} -q_{0,2}$ : piste 1}
On a 
\begin{align*}
&\left(1 - a \Big(\dfrac{L_F\mu_F}{\lfa}-(H^*_2)^2\Big) \dfrac{K_F\lfa}{r_F(H^*_2)^2}\right) > 0 \\
\Leftrightarrow &(H^*_2)^2 \Big(\dfrac{r_F}{K_F} + a_A \lfa \Big) - a_AL_F \mu_F > 0
\end{align*}
 

On peut définir les fonctions 
\begin{equation}
S_F(H) = H^2 \Big(\dfrac{r_F}{K_F} + a_A \lfa \Big) - a_AL_F \mu_F
\end{equation}
et 
\begin{equation}
S_V(H) = H^2 \Big(\dfrac{r_V}{K_V} + b_A \lva \Big) - b_AL_V \mu_V
\end{equation}


On sait que si $S_V(H^*_{F_AV_AH_A,2}) > 0$ et $S_F(H^*_{F_AV_AH_A,2}) > 0$ alors $q_2q_1 - q_0 > 0$.

On sait aussi que $S_V(H^*_{V_AH_A,2}) > 0$ et $S_F(H^*_{F_AH_A,2}) > 0$.

Est ce que $H^*_{V_AH_A,2} < H^*_{F_AV_AH_A,2}$ ??

quand $L_i = 0$,  $S_i(H) > 0$ 




\newpage

\begin{landscape}
\begin{table}
\centering
\caption{Summary of conditions of existence and stability for all the possible equilibrium excepting $EE^{FVH}$}
{\small
\begin{tabular}{c|c|c|c|c|c|c|c|c|c|c|c|c|c|c}
Equilibrium & $T_H(0, 0)$ & $T_H(F^*_{FH},0)$ & $T_H(0,V^*_{VH,i})$ & $T_F(0)$ & $T_F(\beta)$ & $T_F(c)$& $T_F(H^*_{VH, i})$ & $T_V(\beta, 0)$ & $T_V(\beta, F^*_\beta)$ & $T_V(c, 0)$ & $T_V(c, F^*_{FH})$ & $T_{\Delta_{VH}}$ & $T_{V?}$ & $\dfrac{ H^*_{VH, 1}}{H_{min}} $ \\ \hline
$TE$ & & & & $<1$ & & & & & & & & & \\ \hline
$EE^{H_A}_\beta$& $<1$ & & & & $<1$ & & & $<1$ & & & & & \\ \hline
$EE^{F_A}$ & & & & $1<$ & & & & & & & & & \\ \hline
$EE^{F_AH_A}_\beta$ & $<1$ & & & & $1<$ & & & & $<1$ & & & & \\ \hline
$EE^{V_AH_A}_\beta$ & $<1$ & & & & $<1$ & & & $1<$ & & & & & \\ \hline
$EE^{F_AV_AH_A}_\beta$ & $<1$ & & & & $1<$ & & & & $1<$ & & & & \\ \hline
$EE^{H_A} $ & $1<$  & & & & & $<1$ & & & & $<1$ & & & \\ \hline
$EE^{F_AH_A} $ & & $1<$  & & & & $1<$ & & & & & $<1$ & & \\ \hline
$EE^{V_AH_A}_2 $ & & & $1<$ & & & & $<1$ & & & $1<$ & & & & \\ \hline
$EE^{V_AH_A}_2 $ & & & $1<$ & & & & $<1$ & & & $<1$ & &$1\leq$& $1<$ & \\ \hline
$EE^{V_AH_A}_1 $ & & & $<1$ & & & & $<1$ & & & $<1$ & &$1\leq$& $1<$ & $1<$\\ \hline
\end{tabular}
}
\end{table}
\end{landscape}

\section{Model in non anthropized area}
%\subsection{First version, with a fixed capacity $K_F$}
%\begin{equation}
%\left\lbrace \begin{array}{l}
%\dfrac{dF_W}{dt} = r_F \dfrac{V_W}{V_W + L_S} \left(1 - \dfrac{F_W}{K_F}\right) F_W \\
%\dfrac{dV_W}{dt} = r_V \left(1 - \dfrac{V_W}{K_V}\right) V_W - \alpha V_W F_W
%\end{array} \right.
%\label{modelWild1}
%\end{equation}
%
%The model's Jacobian is given by:
%\begin{equation}
%\mathcal{J}(F_W, V_W) = \begin{bmatrix}
%r_F \dfrac{V_W}{V_W + L_S} \left(1 - \dfrac{2F_W}{K_F}\right) & r_F \dfrac{L_S}{\Big(V_W + L_S \Big)^2} \left(1 - \dfrac{F_W}{K_F}\right) F_W \\
%-\alpha V_W & r_V \left(1 - \dfrac{2V_W}{K_V}\right) - \alpha F_W
%\end{bmatrix}
%\end{equation}
%
%\subsubsection{Equilibrium and stability}
%Model \eqref{modelWild1} admits the following equilibrium:
%\begin{itemize}
%\item A trivial equilibrium $TE = \Big(0,0\Big)$. At this point, the Jacobian eigenvalues are $0$ and $r_V$, and $TE$ is never stable.
%\item A equilibria of vegetation only, $EE^{V_W} = \Big(0, K_V\Big)$. At this point, the Jacobian eigenvalues are $r_F \dfrac{K_V}{K_V + L_V}$ and $-r_V$ and thus, $EE^{V_W}$ is never stable.
%\item A equilibria of fauna only, $EE^{F_W} = \Big(F^*, 0\Big)$, whith $F^* \in (0, +\infty)$. At this point, the Jacobian eigenvalues are $0$ and $r_V - \alpha F^*$. This equilibria is then stable if $\dfrac{r_V}{\alpha F^*} < 1$.
%\item A fauna-vegetation, $EE^{F_WV_W} = \left(K_F, K_V \Big(1 - \dfrac{\alpha K_F}{r_V} \Big) \right)$. This equilibria exists if $1 < \dfrac{r_V}{\alpha K_F}$. At this point, the Jacobian eigenvalues are $-r_V \dfrac{V_{F_WV_W}^*}{V^*_{F_WV_W}+ L_S}$ and $-r_V \dfrac{V^*_{F_WV_W}}{K_V}$. Thus, this equilibria is always asymptotically stable.
%\end{itemize}
%
%In summary, when $\dfrac{r_V}{\alpha K_F} < 1$, all the equilibrium $(F^*, 0)$ with $\dfrac{r_V}{\alpha}< F^*$ are stable.
%
%When $1 < \dfrac{r_V}{\alpha K_F}$, $EE^{F_WV_W}$ exists and is AS. Moreover, all the equilibrium $(F^*, 0)$ with $K_F < \dfrac{r_V}{\alpha} < F^*$ are stable.
%
%\begin{table}[ht]
%\centering
%\begin{tabular}{c|c}
%$\dfrac{r_V}{\alpha K_F}$ & Equilibrium \\
%\hline
%$<1$ & $(F^*, 0)$ with $\dfrac{r_V}{\alpha} < F^*$ are stable \\
%\hline
%$1 <$ & $EE^{F_WV_W}$ AS ;  $(F^*, 0)$ with $K_F < \dfrac{r_V}{\alpha} < F^*$ stable
%\end{tabular}
%\end{table}

%\subsubsection{Limit cycle}
%We can use the function $\phi(F_W, V_W) = \dfrac{1}{F_W V_W}$ and the Bendixson-Dulac theorem to show that model \eqref{modelWild1} does not admit any limit cycle.
%So, when $1 < \dfrac{r_V}{\alpha K_F}$, $EE^{F_WV_W}$ is GAS on $\{(F_W, V_W), 0 < F_W < \dfrac{r_V}{\alpha}, 0 < V_W\}$
\subsection{Second version : capacity $K_F(V_W) = f V_W$}

\begin{equation}
\left\lbrace \begin{array}{l}
\dfrac{dF_W}{dt} = r_F \dfrac{V_W}{V_W + L_W} \left(1 - \dfrac{F_W}{f V_W}\right) F_W \\
\dfrac{dV_W}{dt} = r_V \left(1 - \dfrac{V_W}{K_V}\right) V_W - \alpha V_W F_W
\end{array} \right.
\label{modelWild1}
\end{equation}

The model's Jacobian is given by:
\begin{equation}
\mathcal{J}(F_W, V_W) = \begin{bmatrix}
r_F \dfrac{V_W}{V_W + L_W} \left(1 - \dfrac{2F_W}{fV_W}\right) & r_F F_W \dfrac{L_W f + F_W}{f (V_W + L_W)^2} \\
-\alpha V_W & r_V \left(1 - \dfrac{2V_W}{K_V}\right) - \alpha F_W
\end{bmatrix}
\end{equation}

\subsubsection{Equilibrium and stability}
Model \eqref{modelWild1} admits the following equilibrium:
\begin{itemize}
\item TE = (0,0). At this point, the Jacobian's eigenvalues are $0$ and $r_V$. $TE$ is thus never AS.
\item A equilibria of vegetation only, $EE^{V_W} = \Big(0, K_V\Big)$. At this point, the Jacobian eigenvalues are $r_F \dfrac{K_V}{K_V + L_W}$ and $-r_V$ and thus, $EE^{V_W}$ is never stable.
\item A fauna-vegetation, $EE^{F_WV_W} = \left(fV^*_{F_WV_W}, V^*_{F_WV_W} \right)$ where $V^*_{F_WV_W} = K_V \dfrac{1}{1+K_V\dfrac{\alpha f}{r_V}}$. At this point, the Jacobian is:
\begin{equation}
\mathcal{J}(F_{F_WV_W}^*, V_{F_WV_W}^*) = \begin{bmatrix}
-r_F \dfrac{V_{F_WV_W}^*}{V_{F_WV_W}^* +L_W}  & r_F f \dfrac{V_{F_WV_W}^*}{L_V + V_{F_WV_W}^*}\\
- \alpha V_{F_WV_W}^* & -r_V \dfrac{V_{F_WV_W}^*}{K_V}
\end{bmatrix}
\end{equation}
The trace of $\mathcal{J}(F_{F_WV_W}^*, V_{F_WV_W}^*)$ is negative and its determinant is positive. Thus, $EE^{F_WV_W}$ is asymptotically stable.
\end{itemize}

\subsubsection{Limit cycle}
Let define the functions $\phi(F_W, V_W) = \dfrac{1}{F_W V_W}$ on $ \Omega = \{(F_W, V_W) \ F_W > 0, V_W > 0\}$, $f_1(F_W, V_W) = r_F \dfrac{V_W}{V_W + L_W} \left(1 - \dfrac{F_W}{f V_W}\right) F_W$ and $f_2 = r_V \left(1 - \dfrac{V_W}{K_V}\right) V_W - \alpha V_W F_W$.

For $(F_W, V_W) \in \Omega$, we have
\begin{equation}
\dfrac{\partial (\phi f_1)}{\partial F_W} = - \dfrac{r_F}{fV_W (V_W + L_V)}
\end{equation}
and
\begin{equation}
\dfrac{\partial (\phi f_2)}{\partial V_W} = - \dfrac{r_V}{K_V F_W}
\end{equation}
The sum of these two quantities is negative on $\Omega$, and we can use the Bendixson-Dulac's theorem to say that model \eqref{modelWild1} does not admit any limit cycle.

This also implies, by the Poincarré theorem, that $EE^{F_WV_W}$ is globally asymptotically stable on $\Omega$.


\section{Model with human migration}
We can add human migration between the anthropized and wild area. The model becomes:

\begin{subequations}
\begin{equation}
\left\lbrace \begin{array}{l}
\dfrac{dF_{a}}{dt}=r_{F}\dfrac{H_{A}}{H_{A}+L_F}\left(1-\dfrac{F_{A}}{K_{F}}\right)F_{A}-\mu_{F}F_{A}-\lfa F_{A} H_{A}\\
\dfrac{dV_{A}}{dt}=r_{V}\dfrac{H_{A}}{H_{A}+L_V}\left(1-\dfrac{V_{A}}{K_{V}}\right)V_A-\mu_{V}V_A-\lfa V_{A}H_{A}\\
\dfrac{dH_{a}}{dt}=r_{H}\left(1-\dfrac{H_A}{a_AF_A+b_{A}V_{A}+a_{W}F_{W}+b_{W}V_{W}+c}\right)\left(\dfrac{H_{A}}{\beta}-1\right)H_{A}-\da H_{A}+\dw H_{W}
\end{array} \right.
\end{equation}
\begin{equation}
\left\lbrace \begin{array}{l}
\dfrac{dF_W}{dt} = r_F \dfrac{V_W}{V_W + L_W} \left(1 - \dfrac{F_W}{f V_W}\right) F_W - \lfw H_W F_W\\
\dfrac{dV_W}{dt} = r_V \left(1 - \dfrac{V_W}{K_V}\right) V_W - \alpha V_W F_W - \lvw H_W V_W \\
\dfrac{dH_W}{dt} = \da H_A - \dw H_W
\end{array} \right.
\end{equation}
\end{subequations}

At equilibrium, $H_W =  \dfrac{\da}{\dw} H_A$. We consider the ratio $\delta = \dfrac{\da}{\dw} < 1$, meaning both that the population in the wild area is lower than the one in the anthropized area, and that the time spend in wild area is lower than the time spend in the anthropized area.

\subsection{$H_W = \dfrac{\da}{\dw} H_A$, $V_A = 0$, $F_A = 0$}
First, we consider the subsystem  $H_A-F_W-V_W-H_W$ with $H_W = \delta H_A$.
Equations are:

\begin{subequations}
\begin{equation}
\left\lbrace \begin{array}{l}
\dfrac{dH_{a}}{dt}=r_{H}\left(1-\dfrac{H_A}{a_{W}F_{W}+b_{W}V_{W}+c}\right)\left(\dfrac{H_{A}}{\beta}-1\right)H_{A}
\end{array} \right.
\end{equation}
\begin{equation}
\left\lbrace \begin{array}{l}
\dfrac{dF_W}{dt} = r_F \dfrac{V_W}{V_W + L_W} \left(1 - \dfrac{F_W}{f V_W}\right) F_W - \lfw H_A F_W\\
\dfrac{dV_W}{dt} = r_V \left(1 - \dfrac{V_W}{K_V}\right) V_W - \alpha V_W F_W - \lvw H_A V_W
\end{array} \right.
\end{equation}
\end{subequations}
where we rename $\lvw = \delta \lvw$ and $ \lfw = \delta \lfw$.

\subsubsection{Equilibrium}
Possible equilibrium are:

\begin{itemize}
\item $TE = \Big(0,0,0 \Big)$
\item $EE^{V_W} = \Big(0,0 ,K_V \Big)$
\item $EE^{F_WV_W} = \Big(0, fV^*_{F_WV_W},V^*_{F_WV_W} \Big)$
\item $EE^{H_A}_\beta = \Big(\beta,0,0 \Big)$
\item $EE^{H_AV_W}_\beta = \Big(\beta,0,V_{\beta V_W}^* \Big)$ where $V_{\beta V_W}^* = K_V \Big(1 - \dfrac{\lvw \beta}{r_V} \Big)$. This equilibrium exists if $\dfrac{r_V}{\lvw \beta}>1$
\item $EE^{H_AV_WF_W}_\beta = \Big(\beta,F_{\beta V_WF_W}^*,V_{\beta V_WF_W}^* \Big)$ where
$$V_{\beta V_WF_W}^* = K_V \Big(1 -\dfrac{\lvw \beta}{r_V} \Big) - K_V \dfrac{\alpha}{r_V} F_{\beta V_WF_W}^* $$
and
$$
F_{\beta V_WF_W}^* = f \Big(1 - \dfrac{\lfw \beta}{r_F}\Big) V_{V_WF_W, \beta}^* - fL_W \dfrac{\lfw \beta}{r_F}
$$
Using these two expressions leads to:
$$F_{\beta V_WF_W}^* = \dfrac{fK_V\Big(1 - \dfrac{\lfw \beta}{r_F}\Big)\Big(1 - \dfrac{\lvw \beta}{r_V}\Big) - fL_W \dfrac{\lfw \beta}{r_F}}{1 + f K_V \dfrac{\alpha}{r_V} \Big(1 - \dfrac{\lfw \beta}{r_F}\Big)} $$
and
$$V_{\beta V_WF_W}^* = K_V \dfrac{1 + \dfrac{\beta}{r_V}\Big(\dfrac{\alpha \lfw f L_W}{r_F} - \lvw \Big)}{1 + fK_V\dfrac{\alpha}{r_V}\Big(1 - \dfrac{\lfw \beta}{r_F}\Big)}
$$
The two first expressions show that:
\begin{itemize}
\item $V_{\beta V_WF_W}^* < K_V$
\item $V_{\beta V_WF_W}^* < V_{W \beta}^*$
\item $\Big(1 -\dfrac{\lvw \beta}{r_V} \Big) > 0$ ; otherwise $V_{\beta V_WF_W}^*$ is negative
\item $\Big(1 - \dfrac{\lfw \beta}{r_F}\Big) > 0$; otherwise $F_{\beta V_WF_W}^*$ is negative
\item $V_{\beta V_WF_W}^* +K_V \dfrac{\alpha}{r_V} F_{\beta V_WF_W}^*  = V_{\beta V_W}^*$
\end{itemize}
$EE^{H_AV_WF_W}_\beta$ exists if:
\begin{itemize}
\item $\Big(1 - \dfrac{\lfw \beta}{r_F}\Big) > 0$
\item $\Big(1 -\dfrac{\lvw \beta}{r_V} \Big) > 0$ 
\item $\dfrac{K_V}{L_W}\Big(1 - \dfrac{\lvw \beta}{r_V}\Big) \Big(\dfrac{r_F}{\lfw \beta} - 1\Big) > 1$
\end{itemize}

\item $EE^{H_A} = \Big(c, 0, 0\Big)$
\item $EE^{H_AV_W} = \Big(bV^*_{H_AV_W} + c, 0, V^*_{H_AV_W}\Big)$ where $V^*_{H_AV_W} = K_V \dfrac{1-\dfrac{\lvw c}{r_V}}{1 + K_V \dfrac{\lvw b}{r_V}}$. It exists if $\dfrac{\lvw c}{r_V} < 1$.

NB : puisque $\beta < c$, 
\begin{itemize}
\item existence de $EE^{H_AV_W}$ implique existence de $EE^{H_AV_W}_\beta$
\item si les deux existent, $V^*_{\beta V_W} > V_W^*$ et inverse pour la pop humaine
\end{itemize}
\end{itemize}

It is clear that
\begin{itemize}
\item
\begin{equation}
V^*_{V_W} = K_V > V^*_{V_WF_W}\:\:\: , \:\:\:V^*_{\beta V_W}
\end{equation}
and 
\begin{equation}
V^*_{\beta V_W} > V^*_{\beta V_W F_W} \:\:\: , \:\:\: V^*_{H_A V_W}
\end{equation}
\item When $\dfrac{1 + \dfrac{K_V b \lvw}{r_V}}{1 + \dfrac{K_V f \alpha}{r_V}} > 1 - \dfrac{\lvw c }{r_V}$, we also have
\begin{equation}
V^*_{V_WF_W} > V^*_{H_A V_W}
\end{equation}
Question : which criteria to go from $EE^{H_AV_WF_W}$ to $EE^{H_AV_W}$ ?
\begin{itemize}
\item a priori soit $\lfw$ trop important
\item soit $\lvw$ trop important ?? plutôt $EE^{H_A}$ dans ce cas
\end{itemize}

\item $EE^{H_AF_WV_W} = \Big(a F^*_{H_AF_WV_W} + bV^*_{H_AF_WV_W} + c, F^*_{H_AF_WV_W}, V^*_{H_AF_WV_W}\Big)$ where
\begin{equation}
V^*_{H_AF_WV_W} = V^*_{H_AV_W} - K_V \dfrac{\dfrac{\lvw a_W + \alpha}{r_V}}{1 + \dfrac{K_V b \lvw}{r_V}} F^*_{H_AF_WV_W}
\end{equation}
and $F^*_{H_AF_WV_W}$ is solution of (I note $p = K_V \dfrac{\dfrac{\lvw a_W + \alpha}{r_V}}{1 + \dfrac{K_V b \lvw}{r_V}}$):

\begin{multline}
F^2 \Big(p \lfw (b_Wp - a_W) \Big) + \\
F \left(r_F (\dfrac{1}{f} + p) + \lfw (a_W(V^*_{H_AV_W} + L_W) - 2 b_W p V^*_{H_AV_W} - (b_W L_W + c) p \right) + \\
\lfw \Big( b_W (V^*_{H_AV_W})^2 + c L_W + bL_W V^*_{H_AV_W} + c V^*_{H_AV_W} \Big) - r_F V^*_{H_AV_W} = 0
\end{multline}
On peut réécrire les termes:
\begin{multline}
F^2 \Big(p \lfw (b_Wp - a_W) \Big) + \\
F \left(r_F (\dfrac{1}{f} + p) - \lfw \big((b_Wp-a)(V^*_{H_AV_W} + L_W) + p (bV^*_{H_AV_W} + c)\big) \right) + \\
\lfw \Big(b_W V^*_{H_AV_W} + c \Big) \Big(V^*_{H_AV_W} + L_W \Big) - r_F V^*_{H_AV_W} = 0
\end{multline}
On a 
$$
b_W p - a_W = \dfrac{\dfrac{b_W K_V \alpha}{r_V} - a_W}{1 + \dfrac{K_V b \lvw}{r_V}}
$$

Regarder les solutions dans des cas extrêmes ?? $\lfw = 0$ ou grand, $\alpha =0$ ou grand...

\end{itemize}


\subsubsection{Stability}
The Jacobian is 
{\footnotesize
\begin{multline}
\mathcal{J}(H_A,F_W,V_W) = \\
\begin{bmatrix}
r_H(1-\dfrac{H_A}{a_W F_W+b_W V_W+c})(\dfrac{2H_A}{\beta}-1) - \dfrac{r_H \Big(\dfrac{H}{\beta}-1\Big)H}{a_WF_W+b_WV_W+c} & \dfrac{r_H a_W H^2}{(a_WF_W+b_WV_W+c)^2} (\dfrac{H}{\beta}-1) \\
-\lfw F_W & r_F \dfrac{V_W}{V_W + L_W} \left(1 - \dfrac{2F_W}{fV_W}\right) -\lfw H_A  \\
-\lvw V_W & -\alpha V_W
\end{bmatrix} \\
\begin{bmatrix}
 \dfrac{r_H b_W H^2}{(a_WF_W+b_WV_W+c)^2} (\dfrac{H}{\beta}-1) \\
r_F F_W \dfrac{L_W f + F_W}{f (V_W + L_W)^2} \\
 r_V \left(1 - \dfrac{2V_W}{K_V}\right) - \alpha F_W - \lvw H_A 
\end{bmatrix}
\end{multline}}

\begin{itemize}
\item At point $TE$:
\begin{equation}
\mathcal{J}(0,0,0) =
\begin{bmatrix}
-r_H &  & 0 \\
0 & 0 & 0 \\
0 & 0 & r_V
\end{bmatrix}
\end{equation}
\item At point $EE^{V_W}$:
\begin{equation}
\mathcal{J} =
\begin{bmatrix}
-r_H & 0 & 0 \\
0 & r_F \dfrac{K_V}{K_V + L_W} & 0 \\
- \lvw K_V & - \alpha K_V & -r_V
\end{bmatrix}
\end{equation}
\item At point $EE^{F_WV_W}$:
\begin{equation}
\mathcal{J} =
\begin{bmatrix}
-r_H & 0 & 0 \\
-\lfw F_{F_WV_W}^* & -r_F \dfrac{V_{F_WV_W}^*}{V_{F_WV_W}^* +L_W}  & r_F f \dfrac{V_{F_WV_W}^*}{L_V + V_{F_WV_W}^*}\\
-\lvw V_{F_WV_W}^* & - \alpha V_{F_WV_W}^* & -r_V \dfrac{V_{F_WV_W}^*}{K_V}
\end{bmatrix}
\end{equation}
Eigenvalues are all negatives, and $EE^{F_WV_W}$ is AS

\item At point $EE^{H_A}_\beta$,
\begin{equation}
\mathcal{J} =
\begin{bmatrix}
r_H \Big(1 - \dfrac{\beta}{c} \Big) & 0 & 0 \\
0 & -\lfw \beta  & 0 \\
0 & 0 & r_V - \lvw \beta
\end{bmatrix}
\end{equation}

\item At point $EE^{H_AV_W}_\beta$,
\begin{equation}
\mathcal{J} =
\begin{bmatrix}
r_H \Big(1 - \dfrac{\beta}{c} \Big) & 0 & 0 \\
0 & r_F \dfrac{V^*_{\beta V_W}}{V^*_{\beta V_W} + L_W} -\lfw \beta  & 0 \\
-\lvw V^*_{\beta V_W} & -\alpha V^*_{\beta V_W} & -r_V \dfrac{V^*_{\beta V_W}}{K_V}
\end{bmatrix}
\end{equation}

\item At point $EE^{H_AF_WV_W}_\beta$,
\begin{equation}
\mathcal{J} =
\begin{bmatrix}
r_H \Big(1 - \dfrac{\beta}{c} \Big) & 0 & 0 \\
-\lfw F^*_{\beta F_WV_W} & -r_F \dfrac{F^*_{\beta F_WV_W} }{f(V^*_{\beta F_WV_W}  + L_W)}  & r_F \Big(1 - \dfrac{\lfw \beta}{r_F} \Big) \dfrac{F^*_{\beta F_WV_W} }{L_W + V^*_{\beta F_WV_W} } \\
-\lvw V^*_{\beta F_WV_W}  & -\alpha V^*_{\beta F_WV_W}  & -r_V \dfrac{V^*_{\beta F_WV_W} }{K_V}
\end{bmatrix}
\end{equation}


\item At point $EE^{H_A}$,
\begin{equation}
\mathcal{J} =
\begin{bmatrix}
-r_H \Big(\dfrac{c}{\beta}-1\Big) & a_W \Big(\dfrac{c}{\beta}-1\Big) & b_W \Big(\dfrac{c}{\beta}-1\Big) \\
0 & -\lfw c  & 0 \\
0 & 0 & r_V - \lvw c
\end{bmatrix}
\end{equation}

\item At point $EE^{H_AV_W}$,
\begin{equation}
\mathcal{J} =
\begin{bmatrix}
-r_H \Big(\dfrac{bV^*_{H_AV_W} + c}{\beta}-1\Big) & a_W \Big(\dfrac{bV^*_{H_AV_W} + c}{\beta}-1\Big) & b_W \Big(\dfrac{bV^*_{H_AV_W} + c}{\beta}-1\Big) \\
0 & r_F \dfrac{V^*_{H_AV_W}}{V^*_{H_AV_W} + L_W}-\lfw (bV^*_{H_AV_W} + c)  & 0 \\
-\lvw V^*_{H_AV_W} & -\alpha V^*_{H_AV_W} & -r_V \dfrac{V^*_{H_AV_W}}{K_V} 
\end{bmatrix}
\end{equation}

Eigenvalues are $r_F\dfrac{V^*_{H_AV_W}}{V^*_{H_AV_W} + L_W}-\lfw (bV^*_{H_AV_W} + c)$, and the roots of
$$
X^2 + \left(r_H \Big(\dfrac{bV^*_{H_AV_W} + c}{\beta}-1\Big) + r_V \dfrac{V^*_{H_AV_W}}{K_V}  \right) X + \Big(\dfrac{bV^*_{H_AV_W} + c}{\beta}-1\Big) \left( r_H \dfrac{r_V}{K_V} +  \lvw b_W \right) V^*_{H_AV_W}
$$


\item At point $EE^{H_AF_WV_W}$, 
\begin{multline}
\mathcal{J} = \\
\begin{bmatrix}
-r_H \Big(\dfrac{H^*_{H_AF_WV_W}}{\beta}-1\Big) & a_W \Big(\dfrac{H^*_{H_AF_WV_W}}{\beta}-1\Big) & b_W \Big(\dfrac{H^*_{H_AF_WV_W}}{\beta}-1\Big) \\
-\lfw F^*_{H_AF_WV_W} & -r_F \dfrac{F^*_{H_AF_WV_W}}{f(V^*_{H_AF_WV_W} + L_W)} &
r_F F^*_{H_AF_WV_W} \dfrac{L_W f + F^*_{H_AF_WV_W}}{f(V^*_{H_AF_WV_W} + L_W)^2}\\
-\lvw V^*_{H_AF_WV_W} & -\alpha V^*_{H_AF_WV_W} & -r_V \dfrac{V^*_{H_AF_WV_W}}{K_V} 
\end{bmatrix}
\end{multline}
\end{itemize}

\section{Moustique}

\begin{equation}
\dfrac{dA}{dt} = b \dfrac{F_W}{F_W + L_S} \left(1 - \dfrac{A}{\beta V_W + c} \right) A - \mu A
\end{equation}
\begin{itemize}
\item $\beta V_W$ : site de ponte, service que $V_W$ fait pour $A$
\item mortalité densité dépendante $ - \mu A^2$
\item $b$ nbr d'individus qui émergent ; on peut calculer
\item même équation dans zone A ; avec une capacité différente car autre type de gîte, qui sont artificiels et fournis par $H_A$. Ou bien constante ? -> peut être utilisé en terme de contrôle
\item Important de montrer qu'il y a des facteurs différents différents dans zone $S$ et zone $A$
\item Rajouter $A$ dans le sous modèle $V_W$ $F_W$.
\item DYnamique du moustique plus rapide que celles en $V,F_W$; de toute manière l'équation $A$ est dead-end = système découplé, donc équilibre GAS conservé.
\item Etape d'après avec $A$ infecté ou non, $A_S$ and $A_I$. Pas de transmission verticale. DIstinguer $F_{W, I}$ et $F_{W, S}$. Virus considéré et faune : est ce qu'il y a une résistance/rémission/immunité -> biblio + demander à CPaupy
\end{itemize}
Si migration humaine 
\begin{equation}
\dfrac{dA}{dt} = b \dfrac{F_W + H_W}{F_W + L_S + H_W} \left(1 - \dfrac{A}{\beta V_W + c} \right) A - \mu A
\end{equation}
\begin{itemize}
\item Rajouter un facteur devant $H_W$ ? Car préférence de proie potentielle
\end{itemize}


\newpage

\begin{appendices}
\section{Inequality on second degree polynomial \label{appendice:ineq2nd}}

Consider the equation
\begin{equation}
X^2 + B X + C = 0
\label{appendice:eq}
\end{equation}
with positive discriminant $\Delta = B^2 - 4C \geq 0$. Let note $$X_{1} = \dfrac{-B}{2} - \dfrac{1}{2} \sqrt{B^2 - 4 C}$$ and $$X_2 = \dfrac{-B}{2} + \dfrac{1}{2} \sqrt{B^2 - 4 C}$$  the roots of \eqref{appendice:eq}.

When $B < 0$ and $0 < C$, the following inequality hold
\begin{equation}
(X_1) ^ 2 - C < 0 \:\:\text{ and } (X_2)^2 - C > 0
\end{equation}

Indeed, we have :
\begin{subequations}
\begin{align}
&(X_i) ^2 = \dfrac{1}{4} \Big(B^2 \mp 2 B\sqrt{B^2 - 4C} + B^2 - 4C \Big) \\
&(X_i)^2 - C = \dfrac{1}{4} \Big(2(B^2 - 4C) \mp 2B \sqrt{B^2 - 4C} \Big) \\
&(X_i)^2 - C = \dfrac{1}{2} \sqrt{B^2 - 4C} \Big( \sqrt{B^2 - 4C} \mp B \Big)
\end{align}
\end{subequations}
and when $B < 0$ and $0 < C$, we have $ -B = \sqrt{B^2} > \sqrt{B^2 - 4C}$.

\section{$H_{F_AV_AH_A}^*$ is higher than $H_{F_AH_A}^*$ and $H_{V_AH_A}^*$}
At equilibrium, the system equation are:
\begin{equation}
\left\lbrace \begin{array}{l}
r_{H}\left(1-\dfrac{H_A^*}{a_{A}F^*_{A}+b_{A}V^*_{A}+c}\right)\left(\dfrac{H^*_A}{\beta}-1\right)H^*_A = 0, \\
r_F\dfrac{H^*_A}{H_A^* + L_F} \left(1-\dfrac{F_A^*}{K_{F}}\right)F_A^*-\mu_{F}F_A^*-\lambda_{FH,A}F_A^*H_A^* = 0, \\
r_{V}\dfrac{H_A^*}{H_A^*+L_{V}}\left(1-\dfrac{V_A^*}{K_{V}}\right)V_A^*-\mu_{V}V_A^*-\lambda_{VH,A}V_A^*H_A^* = 0.
\end{array} \right.
\label{modelAnthropo:eq}
\end{equation}

We assume $H_A^* = a_AF_A^* + b_AV_A^* + c$, $F^*_A > 0$ and $V_A^* \geq 0$. Then, $H_A^*$ satisfies the equation

\begin{equation}
 \left(1 + a K_F \dfrac{\lfa}{r_F} \right)\left(H^*_A\right)^2 - \left(a_A K_F \Big(1 - \dfrac{\mu_F + \lfa L_F}{r_F} + c + b_A V^*_A \right) H^*_A + aK_F \dfrac{L_F\mu_F}{r_F} = 0
\end{equation}

When the discriminant of this equation is positive,


\end{appendices}

\bibliographystyle{plain}
\bibliography{bibSuivi2}

\end{document}
