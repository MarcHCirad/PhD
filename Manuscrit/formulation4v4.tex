\documentclass{article}
\usepackage{graphicx,ulem} 
\usepackage{color}
\usepackage{amsfonts,amsmath}
\usepackage{amsthm}
\usepackage{empheq}
\usepackage{mathtools}
\usepackage{multirow}
%\usepackage{tikz}
\usepackage{titlesec}
\usepackage{caption}
%\usepackage{lscape}
\usepackage{graphicx}
\captionsetup{justification=justified}
\usepackage[toc,page]{appendix}
\usepackage{hyperref}
\usepackage{subcaption}
\usepackage{pdftricks}
\usepackage{xcolor}
\begin{psinputs}
\usepackage{amsfonts,amsmath}
	\usepackage{pstricks-add}
   \usepackage{pstricks, pst-node}
   \usepackage{multido}
   \newcommand{\lfw}{\lambda_{F}}
\end{psinputs}

\textheight240mm \voffset-23mm \textwidth160mm \hoffset-20mm

\graphicspath{{./Images/}{../Schema}}
%\graphicspath{{Figures}}
\setcounter{secnumdepth}{4}
\titleformat{\paragraph}
{\normalfont\normalsize\bfseries}{\theparagraph}{1em}{}
\titlespacing*{\paragraph}
{0pt}{3.25ex plus 1ex minus .2ex}{1.5ex plus .2ex}

\newcommand{\lfd}{\lambda_{F, D}}
\newcommand{\lfw}{\lambda_{F}}
\newcommand{\Kfa}{K_{F,\alpha}}
\newcommand{\cI}{\mathcal{I}}
\newcommand{\N}{\mathcal{N}}

\newcommand{\marc}[1]{\textcolor{teal}{#1}}
\newcommand{\MH}[1]{\textcolor{teal}{#1}}
\newcommand{\YD}[1]{\textcolor{magenta}{#1}}
\newcommand{\VY}[1]{\textcolor{blue}{#1}}
\renewcommand{\epsilon}{\varepsilon}

\DeclareMathOperator{\Tr}{Tr}
\newtheorem{theorem}{Theorem}
\newtheorem{prop}{Proposition}
\newtheorem{definition}{Definition}
\newtheorem{remark}{Remark}
\newtheorem{cor}{Corollary}
\newcommand*\phantomrel[1]{\mathrel{\phantom{#1}}}

\title{Modèle Chasseur}
\author{Marc Hétier, Yves Dumont  and Valaire Yatat-Djeumen}

\begin{document}

\maketitle
%{\hypersetup{hidelinks}
%\tableofcontents}
%\newpage

\section{Introduction}

\begin{enumerate}
\item Contexte général : forêt tropical, population, environnement affecté : chasse et industrialisation. Perte de biodiv. Important car ..
\item  Intérêt de la modèlisation math
\item Nouveauté de l'étude : contrairement aux autres modèles, on considère une pop de chasseur : l'interaction se fait directement entre pop humaine et la faune sauvage
\item Zone d'étude considérée : sud cameroun, population principalement de chasseur, peu de ressources (nourriture) proviennent de la végétation
\item But de l'article : analyser l'impact de la (sur)-chasse et de l'anthropisation du milieu sur les populations (animales et humaines)
\item Plan de l'article

\begin{enumerate}
\item Présentation du modèle
\item Analyse théorique : équilibre, stabilité locale et globale
\item Effet de variation de paramètres : diagramme de bifurcation + interprétation
\item Présentation d'orbites
\item 
\item Conclusion
\end{enumerate}
\end{enumerate}
%\YD{Marc, ton niveau d'anglais.... si tu as des doutes utilise un "translator", ou https://www.deepl.com/fr/translator... mais on ne peut pas continuellement corriger et recorriger des formulations ou des expressions qui n'existent pas en anglais! Je doute que "defaunation" existe, même en français.... Il vaut mieux faire des phrases simples et compréhensibles. On n'a pas besoin de beaucoup de blabla....}
% \marc{Defaunation existe bien en anglais, ou est du moins utilisé par les auteurs qui parlent du sujet ; voir par exemple \cite{benitez-lopez_intact_2019} dont le titre est "Intact but empty forests? Patterns of hunting induced mammal defaunation in the tropics"...} 

Tropical forests are particularly rich ecosystems, in term of plants and animals diversity. They also provide resources for forest-based people, and other human populations living nearby. Those resources include, for example, food, \cite{avila_martin_food_2024}, medicine, cultural purposes \cite{kumar_marginal_2014} and sources of energy \cite{mangula_energy_2019}.

However, tropical forests are more and more degraded and fragmented by human settlement and activities, as the development of infrastructures (roadways, harbors, dams, ...), agriculture, and industrial complex (industrial palm grove, industrial logging...), etc. See 

Today, only 20\% of the remaining area are considered as intact, see \cite{benitez-lopez_intact_2019}. It is not only the vegetation which is endangered, but also the wildlife it shelters. Of course, the destruction of their habitat has indirect consequences on animal population, but they are also directly threaten by human activities, and specially by over-hunting \cite{wilkie_empty_2011, benitez-lopez_intact_2019}. In 2019, authors of \cite{benitez-lopez_intact_2019} estimated that the abundance of tropical mammal species had declined of 13\% in average, and predicted a decline of more than 70\% for mammals in West Africa. 

Defaunation is not harmless. In \cite{ripple_bushmeat_2016}, the authors recall that even partial defaunation have consequences on the environment and on the human population. For example, extinction of mammals affect the forest regeneration, since they play a role, among others, of seed disperser, see \cite{wright_bushmeat_2007, peres_dispersal_2016}.

Moreover, bushmeat is still an important source of food and income for some population, \cite{jones_incentives_2019}. This both means that hunt can not just be prohibited, but also that defaunation threaten the food security and livelihood conditions of those population. Therefore, it is necessary to tackle the subject of sustainable hunt.

It can be hard to determine if hunt, as it is practiced in a specific place, is sustainable or not. Firstly because getting knowledge about wildlife and hunt practice can not be done using remote-sensing \cite{peres_detecting_2006}, but also because specific field studies are hardly generalized to other places. However, we believe that mathematical modeling can help to overcome those problems. Indeed, mathematical models allow not only to synthesis and generalizes complex reality, but they also permit to determine parameters and thresholds of interest. These knowledge may in return facilitate future field studies, and help decision makers to take justified decisions.

Some mathematical models describing the human-environment interactions have already been studied, see for example the review \cite{fanuel_modelling_2023} and the references therein. \YD{A détailler.... Il faut donner les hypothèses et résultats principaux. Il ne suffit pas juste de balancer une phrase....} \marc{Oui bien sûr !! Je n'ai pas encore fini ce paragraphe !}

For this study, we consider a situation typical of South-Cameroon, where human population mainly live from hunt and agricultural resources. These populations do not threaten the vegetation (consummation of plant for food or medical uses are low compared \YD{to???}, see \cite{koppert_consommation_1996}). However, the forest has been \YD{(is or will be?)} impacted by industrial development: a deep sea harbor has been built near Kribi, industrial palm groves founded, roads created.

\section{Hunter Model}

We model the interactions between a hunter community and wilderness by a consumer-resource model. We consider two areas, one corresponding to a domestic area (typically a village, etc), the second to a wild area (Forest, etc).

Humans in the domestic area, $H_D$, are modeled by a consumer equation, based on available resource. The dynamic followed by $H_D$ can be separated in different terms. 
First there is a constant growth term, $\cI$, which represents immigration from other inhabited area. For example, the development of industrial complex bring new workers to the site.

Villagers are also able to produce a certain amount of food at the rate $f_D$. Note that $f_D$ can be understand as $f_D = e \lfd F_D$ where $F_D$ is a constant amount of cattle or agricultural resources. 
Moreover, we assume that $H_D$ has a natural death-rate, $\mu_D$.

Villagers come and go 
%\YD{???????? come and go?} \marc{come back and force était une coquille, l'expression voulue était come back and forth} \YD{donc "come and go". Merci d'utiliser des outils de traduction appropriés; demander à Yannnick pour de l'aide...}
between the domestic and wild area, mainly for hunt activities. This migration is modeled by a bilinear function $M(H_D, H_W) = m_W H_W - m_D H_D$, where $H_W$ is the human population present in the wild area. We note $m_D$ (respectively $m_W$) the migration rate from the domestic (resp. wild) area to the wild (resp. domestic) area. Note that we assume that the migration rates are such that $m = \dfrac{m_D}{m_W} < 1$. This assumption makes sense, because humans stay a shorter amount of time in the Wild area than in the Domestic area. On the first part of the study, we will consider that the migrations are fast compared to the other demographic processes. In this case, the function $M(H_D, H_W)$ writes $M(H_D, H_W) = \dfrac{1}{\epsilon} (m_W H_W - m_D H_D)$, with $\epsilon$ being small.
On the second part, we simply consider $M(H_D, H_W) = m_W H_W - m_D H_D$.
In all cases, the dynamic of $H_W$ corresponds to the migration process. 

Wild animals, $F_W$, hunted in the wild area are partially used to feed the villagers. Therefore, we consider a growth term $e \lfw H_W F_W$ \YD{(que représente ce terme? pourquoi sous cette forme?)}, where $e$ is a conversion rate \YD{(non, ce n'est plus un taux, c'est plutôt une proportion!)} between these hunted prey and the number of human it can nourish. This parameter encompasses both direct consumption of the meat by villagers, but also the fact that a large amount of bush-meat is not consumed by the village inhabitants, but is sold on city market and allow to buy other supply \cite{wilkie_bushmeat_1998}.


The dynamic of $F_W$ follows a logistic equation, with a carrying capacity, $K_F$, dependent on the surrounding vegetation. To take into account the level of anthropization of the habitat, we introduce the non-negative parameter $\alpha \in [0, 1)$. When $\alpha > 0$, the carrying capacity of the habitat is reduced of $\alpha \%$ from its original value. Anthropization may also have a negative impact on the animal's growth rate $r_F$. For the sake of simplicity, we model this impact in the same way, by multiplying the growth rate by $(1-\alpha)$.

We consider two different interactions between wild fauna and the population located in the wild area. On the one hand, wild fauna is hunted by humans present in the wild area. This is take into account by the functional response $\lfw H_W$, where $\lfw$ is the hunting rate and $H_W$ the number of hunter. The functional response is unbounded, to take into account the possibility of over-hunt. On the other hand, it is known that housing, culture and food supply may attract some mammals, specially rodents (\marc{donner des noms?}) and favor their reproduction (see for example \cite{dounias_foraging_2011, dobigny_zoonotic_2022}). We take this effect into account by multiplying the wild animal's growth rate, $r_F$, by the functional $(1 +  \beta H_W)$, where $\beta$ is the influence rate that human activities may have on wild animal's growth.

\begin{table}[ht]
\center
\begin{tabular}{|c|c|c|}
\hline 
Parameter & Description & Unit \\ 
\hline \hline
$t$ & Time & Year \\
\hline
$H_D$ & Humans in the domestic area & Ind \\
$H_W$ & Humans in the wild area & Ind \\
$F_W$ & Wild fauna & Ind \\
\hline
$e$ & Prey-food conversion \YD{(non, si $F_W$ et $H_W$ ont la même unité)} & - \\
$f_D$ & Food produced by human population & Year$^{-1}$ \\
$\mu_D$ & Human mortality rate  & Year$^{-1}$ \\
$m_D$ & Migration from domestic area to wild area & Year$^{-1}$ \\
$m_W$ & Migration from wild area to domestic area & Year$^{-1}$ \\
$r_F$ & Wild animal growth rate & Year$^{-1}$ \\
$K_F$ & Carrying capacity for wild fauna, fixed by the environment& Ind \\
$\alpha$ & Proportion of anthropized environment & - \\
$\beta$ & Positive impact from human activities to animal growth rate & Ind$^{-1}$  \\
$\lfw$ & Hunting rate & Ind/Year\\
$\mathcal{I}$ & Immigration rate &Ind/Year\\
\hline
\end{tabular}
\caption{Parameters and variables of the model}
\end{table}

Considering the characteristic times of humans and wildlife, the time scale considered all along the study will be the year.
\medskip

Finally, the model is given by the following equations:
\begin{equation}
\def\arraystretch{2}
\left\{ 
\begin{array}{l}
\dfrac{dH_D}{dt}= \cI + e\dfrac{\lfw F_W}{1 + \lfw \theta F_W}H_W + (f_D - \mu_D) H_D + M(H_D, H_W). \\
\dfrac{dF_W}{dt} = r_F(1- \alpha) (1+ \beta H_W) \left(1 - \dfrac{F_W}{K_F(1-\alpha)} \right) F_W - \dfrac{\lfw F_W}{1 + \lfw \theta F_W} H_W \\
\dfrac{dH_W}{dt}= - M(H_D, H_W) 
\end{array} \right.
\label{equationsHDFWHW}
\end{equation}
As state above, two expressions for  $M(H_D, H_W)$ will be considered, depending on the rapidity of the migration processes. In section \ref{section:Model analysis under the quasi steady state assumption}, they are considered as faster than other (demographic) processes, and therefore we use:
 $$M(H_D, H_W)  = \dfrac{1}{\epsilon}(m_W H_W - m_D H_D),$$ $\epsilon$ being the conversion parameter between the slow and fast processes. In the other parts of the study, we consider that the migration processes occur at the same speed than the other processes, and we simply use:
$$ M(H_D, H_W)  = m_W H_W - m_D H_D.$$

The presence of $\dfrac{1}{\epsilon}$ does not impact the validity or the expressions of the results presented in the following part, and it is therefore omitted.


\section{Existence and uniqueness of global solutions}
In this section, we state general results on system \eqref{equationsHDFWHW}:  existence of an invariant region, existence and uniqueness of global solutions.

We begin by proving the local existence and uniqueness of solutions of system \eqref{equationsHDFWHW}. The right hand side of equations \eqref{equationsHDFWHW} defines a function $f(y)$ (with $y = (H_D, F_W, H_W)$) which is of class $\mathcal{C}^1$ on $\mathbf{R}^3$. The theorem of Cauchy-Lipschitz ensures that model \eqref{equationsHDFWHW} admits a unique solution, at least locally, for any given initial condition, see \cite{walter_ordinary_1998}.

We need to add a constraint on the sign of $f_D - \mu_D$, to avoid infinite growth of human population, and ensure that the system is well defined. In the rest of this study, we will assume that $\mu_D-f_D > 0$. This means that the food produced by the human population living in the domestic area is not sufficient to ensure the permanence of the population. Hunt and/or immigration is necessary. 

The following proposition indicates a compact and invariant subset of $\mathbf{R}_+^3$, on which the solutions are bounded.

\begin{prop}\label{invariantRegion} 
Assume 
\begin{equation}
\beta < \dfrac{4(\mu_D - f_D)}{m e r_F (1-\alpha)^2 K_F} := \beta^*
\label{Beta}
\end{equation}
Then, the region
$$\Omega = \Big\{\Big(H_D, F_W, H_W \Big) \in \mathbb{R}_+^3  \Big|H_D + H_W + eF_W \leq S^{max}, F_W \leq F_W^{max}, H_W \leq H_W^{max} \Big\},$$
is a compact and invariant set for system \eqref{equationsHDFWHW}, 
where
$$
S^{max} = \Big(1 + \dfrac{m_D}{m_W} \Big) \dfrac{\cI + \left(\mu_D - f_D + \dfrac{r_F}{4}(1-\alpha) \right) e (1-\alpha)K_F }{\dfrac{\mu_D - f_D}{m} - er_F (1-\alpha)^2 \beta \dfrac{K_F}{4}},
\quad
F_W^{max} = (1-\alpha)K_F,
\quad
H_W^{max} = \dfrac{m_D}{m_D + m_W} S^{max}
$$
In particular, this means that any solutions of equations \eqref{equationsHDFWHW} with initial condition in $\Omega$ remains in $\Omega$.
\end{prop}
%
\begin{proof}
To prove this proposition we will use the notion of invariant region, see \cite{smoller_shock_1994}. Before, we introduce the variable $S = H_D + H_W + e F_W$. We have:

\begin{equation}
\dfrac{dS}{dt} = \cI + (f_D - \mu_D) \Big(S - H_W - eF_W \Big) + e (1-\alpha)(1+\beta H_W)r_F  \left(1 - \dfrac{F_W}{(1-\alpha)K_F} \right) F_W.
\end{equation}

With this new variable, the model writes:

\begin{equation}
\left\{ \begin{array}{l}
\dfrac{dS}{dt} = \cI + (f_D - \mu_D) \Big(S - H_W - e F_W \Big) + e(1-\alpha)(1+\beta H_W) r_F \left(1 - \dfrac{F_W}{(1-\alpha)K_F} \right) F_W, \\
\dfrac{dF_W}{dt} = (1-\alpha)(1+\beta H_W) r_F \left(1 - \dfrac{F_W}{K_F(1-\alpha)} \right) F_W - \dfrac{\lfw F_W}{1 + \lfw \theta F_W} H_W \\
\dfrac{dH_W}{dt}= m_D \left(S - eF_W\right) - (m_W + m_D) H_W 
\end{array} \right.
\label{equationsSFWHW}
\end{equation}


We define the function $g(z)$, with $z=\Big(S, F_W, H_W \Big)$ as the right hand side of this equations. We also introduce the following functions:
$$
G_1(z) = S - S^{max},
\quad
G_2(z) = F_W - F_W^{max},
\quad
G_3(z) = H_W - H_W^{max}
$$

Following \cite{smoller_shock_1994}, we will show that quantities $(\nabla G_1 \cdot g)|_{S = S^{max}}$, $(\nabla G_2 \cdot g)|_{F_W = F_W^{max}}$ and $(\nabla G_3 \cdot g)|_{H_W = H_W^{max}}$ are non-positive for $z \in \Omega_S = \Big\{ \Big(S, F_W, H_W \Big) \in (\mathbb{R})^3  \Big|S \leq S^{max}, F_W \leq F_W^{max}, H_W \leq H_W^{max} \Big\}$.

Using the fact that $\mu_D - f_D >0$ and $z\in \Omega_S$, we have:

\begin{align*}
(\nabla G_1 \cdot g)|_{S = S^{max}} &= \cI + (f_D - \mu_D) \Big(S^{max} - H_W - eF_W \Big) + e r_F(1-\alpha)(1+\beta H_W)  \left(1 - \dfrac{F_W}{(1-\alpha)K_F} \right) F_W, \\
&\leq \cI + (f_D - \mu_D) S^{max} + (\mu_D - f_D) H_W + \Big(\mu_D - f_D\Big) eF_W +er_F (1-\alpha)(1+\beta H_W) \dfrac{(1-\alpha)K_F}{4} \\
&  \leq \cI + (f_D - \mu_D) S^{max} + (\mu_D - f_D) \dfrac{m_D}{m_D + m_W} S^{max} + \Big(\mu_D - f_D\Big) e (1-\alpha)K_F + \\&er_F (1-\alpha)(1+\beta \dfrac{m_D}{m_D + m_W} S^{max}) \dfrac{(1-\alpha)K_F}{4}, \\
&  \leq \cI + \left( -(\mu_D - f_D) +  \dfrac{(\mu_D - f_D)m_D}{m_D + m_W} +  \dfrac{er_F (1-\alpha)^2\beta m_D}{4(m_D + m_W)}K_F \right)S^{max} +\\&  \Big(\mu_D - f_D + \dfrac{r_F}{4} (1-\alpha)\Big) e (1-\alpha)K_F,  \\
&  \leq \cI + \dfrac{m_D}{m_D + m_W}\left( -\dfrac{(\mu_D - f_D)}{m}  +  e\dfrac{r_F}{4} (1-\alpha)^2\beta K_F \right) S^{max}+\\&  \Big(\mu_D - f_D + \dfrac{r_F}{4} (1-\alpha)\Big) e (1-\alpha)K_F,  \\
&  \leq \cI - \dfrac{m_D}{m_D + m_W}\Big(1 + \dfrac{m_D}{m_W} \Big)\left( \cI + \left(\mu_D - f_D + \dfrac{r_F}{4}(1-\alpha) \right) e (1-\alpha)K_F \right)+\\&  \Big(\mu_D - f_D + r_F (1-\alpha)\Big) e (1-\alpha)K_F,  \\
&  \leq \cI - \left( \cI + \Big(\mu_D - f_D + \dfrac{r_F}{4}(1-\alpha) \Big) e (1-\alpha)K_F \right)+\\&  \Big(\mu_D - f_D + r_F (1-\alpha)\Big) e (1-\alpha)K_F,  \\
&\leq 0
\end{align*}

that is $(\nabla G_1 \cdot g)|_{S = S^{max}} \leq 0$. The two others inequalities are straightforward to obtain. We have:
\begin{align*}
(\nabla G_2 \cdot g)|_{F_W = F_W^{max}} &= r_F  \left(1 - \dfrac{K_F (1-\alpha)}{K_F (1-\alpha)}\right)K_F (1-\alpha)  - \lfw H_W K_F (1-\alpha), \\
(\nabla G_2 \cdot g)|_{F_W = F_W^{max}} & = - \lfw H_W K_F (1-\alpha), \\
(\nabla G_2 \cdot g)|_{F_W = F_W^{max}} & \leq 0.
\end{align*}

The computations for $(\nabla G_3 \cdot g)|_{H_W = H_W^{max}}$ give:

\begin{align*}
(\nabla G_3 \cdot g)|_{H_W = H_W^{max}} &= m_D (S - eF_W) - (m_W + m_D) H_W^{max}, \\
(\nabla G_3 \cdot g)|_{H_W = H_W^{max}} &= m_D (S - eF_W) - m_D S^{max}, \\
(\nabla G_3 \cdot g)|_{H_W = H_W^{max}} & \leq m_D (S - S^{max} -  eF_W), \\
(\nabla G_3 \cdot g)|_{H_W = H_W^{max}} & \leq 0.
\end{align*}

We have shown that $(\nabla G_1 \cdot g)|_{S = S^{max}} \leq 0$, $(\nabla G_2 \cdot g)|_{F_W = F_W^{max}} \leq 0$ and $(\nabla G_3 \cdot g)|_{H_W = H_W^{max}} \leq 0$ in  $\Omega_S$.  According to \cite{smoller_shock_1994}, this prove that $\Omega_S$ is an invariant region for system \eqref{equationsSFWHW}.

This also shows that the set  $\Big\{\Big(H_D, F_W, H_W \Big) \in \mathbb{R}^3  \Big|H_D + H_W + eF_W \leq S^{max}, F_W \leq F_W^{max}, H_W \leq H_W^{max} \Big\}$ is invariant for system \eqref{equationsHDFWHW}. 


Moreover, for any point $y \in \partial (\mathbb{R}_+)^3$, the vector field defined by $f(y)$ is either tangent or directed inward. Then, $\Omega$ is an invariant region for equations \eqref{equationsHDFWHW}. 

\end{proof}

The next proposition shown that equations \eqref{equationsHDFWHW} define a dynamical system on $\Omega$.

\begin{prop}
Equations \eqref{equationsHDFWHW} define a dynamical system on $\Omega$, that is, for any initial condition $(t_0, y)$ with $t_0 \in \mathbf{R}$ and $y \in \Omega$, it exists a unique solution of equations \eqref{equationsHDFWHW}, and this solution is defined for all $t \geq t_0$.
\end{prop}

\begin{proof}
We already prove that equations \eqref{equationsHDFWHW} admit, at least locally, a unique solution for every initial condition. Moreover, since $\Omega$ is an invariant region, the solutions with initial condition on $\Omega$ are bounded. Based on uniform boundedness, we deduce that solutions of system \eqref{equationsHDFWHW} with initial condition on $\Omega$ exists globally, for all $t\geq t_0$. Therefore, $\eqref{equationsHDFWHW}$ defines a dynamical system on $\Omega$.
\end{proof}

\section{Preliminary Results}

We start this section by providing some computational but useful result.

\begin{prop} \label{propBeta}
Assume $\beta>0$ and \eqref{Beta} holds, then the following inequality is true for all $D > 0$:
$$
\beta D \Big(1 - \dfrac{\mu_D - f_D}{me r_F (1-\alpha)^2K_F } D\Big) < 1.
$$
\end{prop}

\begin{proof}
When $\Big(1 - \dfrac{\mu_D - f_D}{me r_F (1-\alpha)^2K_F } D\Big) \leq 0$, the proposition is obviously true. When $\Big(1 - \dfrac{\mu_D - f_D}{me r_F (1-\alpha)^2K_F } D\Big) > 0$, we have:

\begin{equation*}
\beta D \Big(1 - \dfrac{\mu_D - f_D}{me r_F (1-\alpha)^2K_F } D\Big) < 4 \dfrac{\mu_D - f_D}{me r_F (1-\alpha)^2K_F } D \left(1 -\dfrac{\mu_D - f_D}{me r_F (1-\alpha)^2K_F } D \right)
\end{equation*}

It is straightforward to show that $x(1 - x) \leq \dfrac{1}{4}$ for $x \in \mathbb{R}$. Therefore,
$$
\beta D \Big(1 - \dfrac{\mu_D - f_D}{me r_F (1-\alpha)^2K_F } D\Big) < 1.
$$
\end{proof}

%\begin{prop} \label{propPF}
%We define the following polynomial, which is used all along section \ref{section:with immigration}:
%\begin{multline}
%P_F(X) := X^2 \left(\dfrac{er_F}{K_F} \right) - X \left(e(1-\alpha)r_F + \dfrac{(\mu_D - f_D) r_F}{\lfw m K_F} + \dfrac{\cI \beta r_F}{\lfw K_F} \right) + \\ \left(\dfrac{(\mu_D - f_D)(1-\alpha) r_F}{\lfw m} - \cI\Big(1 - \dfrac{(1-\alpha)\beta r_F}{\lfw} \Big) \right).
%\label{polynome-Feq}
%\end{multline} When $\beta < \beta^*$ and $\cI > 0$, the following results are true:
%
%\begin{itemize}
%\item $P_F\Big((1-\alpha)K_F\Big) = -\cI < 0$
%\item $P_F\Big(\dfrac{\mu_D - f_D}{\lfw m e}\Big) < 0$
%\item $P_F\Big((1-\alpha)K_F - \dfrac{K_F \lfw}{\beta r_F}\Big) > 0$
%\item $P_F$ admits two real roots, $F_1^* \leq F_2^*$. 
%\item If $\dfrac{(\mu_D - f_D) r_F}{\lfw m } \leq \cI\Big(1 - \dfrac{(1-\alpha)\beta r_F}{\lfw} \Big)$,  $F_2^*$ is positive and $F_1^*$ is non positive. If $\dfrac{(\mu_D - f_D) r_F}{\lfw m } > \cI\Big(1 - \dfrac{(1-\alpha)\beta r_F}{\lfw} \Big)$, $F^*_1$ and $F^*_2$ are positive. Moreover, $P_F$ is positive on $(-\infty, F_1^*)$, negative on $(F^*_1, F^*_2)$ and positive on $(F^*_2, +\infty)$.
%\item From the precedent points, it follows that $$(1-\alpha)K_F - \dfrac{K_F \lfw}{\beta r_F} \leq F^*_1 \leq (1-\alpha)K_F, \dfrac{\mu_D - f_D}{\lfw m e} \leq F_2^* $$
%\end{itemize}
%
%\end{prop}
%
%\begin{proof}
%We have:
%\begin{align*}
%P_F((1-\alpha) K_F) &= \Big((1-\alpha) K_F \Big)^2 \left(\dfrac{er_F}{K_F} \right) - (1-\alpha) K_F \left(e(1-\alpha)r_F + \dfrac{(\mu_D - f_D) r_F}{\lfw m K_F} + \dfrac{\cI \beta r_F}{\lfw K_F} \right) + \\ &\left(\dfrac{(\mu_D - f_D)(1-\alpha) r_F}{\lfw m} - \cI\Big(1 - \dfrac{(1-\alpha)\beta r_F}{\lfw} \Big) \right), \\
%&=(1-\alpha)^2 e K_F r_F - e(1-\alpha)^2 K_F r_F - \dfrac{(\mu_D - f_D) (1-\alpha) r_F}{\lfw m} - \dfrac{\cI \beta (1-\alpha)r_F}{\lfw}  + \\ &\dfrac{(\mu_D - f_D)(1-\alpha) r_F}{\lfw m} - \cI +\cI \dfrac{(1-\alpha)\beta r_F}{\lfw}, \\
%&= -\cI< 0.
%\end{align*}
%Then, 
%
%\begin{align*}
%P_F\Big(\dfrac{\mu_D - f_D}{\lfw m e}\Big) &= \left(\dfrac{\mu_D - f_D}{\lfw m e}\right)^2 \left(\dfrac{er_F}{K_F} \right) - \dfrac{\mu_D - f_D}{\lfw m e} \left(e(1-\alpha)r_F + \dfrac{(\mu_D - f_D) r_F}{\lfw m K_F} + \dfrac{\cI \beta r_F}{\lfw K_F} \right) + \\ & \left(\dfrac{(\mu_D - f_D)(1-\alpha) r_F}{\lfw m} - \cI\Big(1 - \dfrac{(1-\alpha)\beta r_F}{\lfw} \Big) \right), \\
%&= - \dfrac{(\mu_D - f_D) \cI \beta r_F}{e \lfw ^2 K_F} - \cI + \dfrac{\cI (1-\alpha)r_F \beta}{\lfw}, \\
%&= -\cI \left( 1 - \dfrac{(1-\alpha)r_F \beta }{\lfw} + \dfrac{(\mu_D - f_D)  \beta r_F}{e \lfw ^2 K_F} \right), \\
%&= -\cI \left( 1 - \dfrac{\beta(1-\alpha)r_F  }{\lfw}\Big(1 - \dfrac{(\mu_D - f_D) }{ m e \lfw (1-\alpha) K_F}\Big) \right), \\
%& < 0,
%\end{align*}
%thanks to proposition \ref{propBeta}. We have:
%
%\begin{align*}
%P_F\Big((1-\alpha)K_F - \dfrac{K_F \lfw}{\beta r_F}\Big) &= P_F\Big((1-\alpha)K_F\Big) + \Big(\dfrac{K_F \lfw}{\beta r_F}\Big)^2 \dfrac{er_F}{K_F} - 2(1-\alpha)K_F \dfrac{K_F \lfw}{\beta r_F}\dfrac{er_F}{K_F} + \\ &\left(e(1-\alpha)r_F + \dfrac{(\mu_D - f_D) r_F}{\lfw m K_F} + \dfrac{\cI \beta r_F}{\lfw K_F} \right) \dfrac{K_F \lfw}{\beta r_F}, \\
%&= -\cI + \dfrac{K_F \lfw^2}{\beta^2 r_F} - 2 \dfrac{(1-\alpha)K_F \lfw e}{\beta} +\dfrac{(1-\alpha)K_F \lfw e}{\beta} + \dfrac{\mu_D - f_D}{\beta m} + \cI, \\
%&= \dfrac{K_F \lfw^2}{\beta^2 r_F} -  \dfrac{(1-\alpha)K_F \lfw e}{\beta} + \dfrac{\mu_D - f_D}{\beta m}, \\
%&= \dfrac{K_F \lfw^2}{\beta^2 r_F} \left(1 - \dfrac{\beta (1-\alpha) r_F}{\lfw} \Big(1 - \dfrac{\mu_D - f_D}{m e \lfw K_F(1-\alpha)} \Big) \right), \\
%&> 0,
%\end{align*}
%using proposition \ref{propBeta}.
%
%To show the last point of the proposition, we start by computing the discriminant of $P_F$, $\Delta_F$. We have:
%\begin{align*}
%\Delta_F &= \left(e(1-\alpha)r_F + \dfrac{(\mu_D - f_D) r_F}{\lfw m K_F} + \dfrac{\cI \beta r_F}{\lfw K_F} \right)^2 - 4\dfrac{er_F}{K_F}  \left(\dfrac{(\mu_D - f_D)(1-\alpha) r_F}{\lfw m} - \cI\Big(1 - \dfrac{(1-\alpha)\beta r_F}{\lfw} \Big) \right), \\
%\Delta_F &= \left(e(1-\alpha)r_F - \dfrac{(\mu_D - f_D) r_F}{\lfw m K_F}\right)^2 + \dfrac{\cI \beta r_F}{\lfw K_F} \left(\dfrac{\cI \beta r_F}{\lfw K_F} + 2\dfrac{(\mu_D - f_D) r_F}{\lfw m K_F} + 2e(1-\alpha)r_F \right) + 4\dfrac{er_F}{K_F}  \cI\Big(1 - \dfrac{(1-\alpha)\beta r_F}{\lfw} \Big), \\
%\Delta_F & > 0.
%\end{align*}
%
%Therefore, $P_F$ admits two real roots. Their sign depends on the sign of the constant coefficient. $P_F$ admits:
%\begin{itemize}
%\item One non positive root $F^*_1$ and one positive root $F^*_2$ if $$\dfrac{(\mu_D - f_D)(1-\alpha) r_F}{\lfw m} - \cI\Big(1 - \dfrac{(1-\alpha)\beta r_F}{\lfw} \Big) \leq 0 \Leftrightarrow \dfrac{(\mu_D - f_D) r_F}{\lfw m } \leq \cI\Big(1 - \dfrac{(1-\alpha)\beta r_F}{\lfw} \Big).$$
%\item Two positive roots $F^*_1\leq  F^*_2$ if $\dfrac{(\mu_D - f_D) r_F}{\lfw m } > \cI\Big(1 - \dfrac{(1-\alpha)\beta r_F}{\lfw} \Big)$.
%\end{itemize}
%They are given by:
%
%\begin{equation*}
%F_i^* = \dfrac{K_F(1-\alpha)}{2}\left(1 \pm \dfrac{\sqrt{\Delta_F}}{e(1-\alpha)r_F}\right) + \dfrac{\mu_D - f_D}{2\lfw m e} + \dfrac{\cI \beta}{2\lfw e}, \quad i=1,2.
%\end{equation*}
%\end{proof}

%The following definitions will be used all along the theoretical study.
%
%
%\begin{definition}
%An open set $\mathcal{D} \in \mathbf{R}^n$ is said to be p-convex provided that for every $x, y \in \mathcal{D}$, with $x \leq y$, the line segment joining $x$ and $y$ belongs to $\mathcal{D}$.
%\end{definition}
%
%\begin{definition}\cite{kaszkurewicz_matrix_2012}
%A square matrix $A \in Mn (\mathbf{R})$ is said reducible if there exists a permutation matrix $P$ such that $P^ T AP$ is block triangular. Otherwise, $A$ is called irreducible.
%
%\marc{OR a square matrix $A \in Mn (\mathbf{R})$ is said irreducible if for each nonempty proper subset $I$ of $N = \{1, ..., n\}$, there exists $i \in I$ and $j \in N\backslash I$ such that $A_{i,j} \neq 0$}
%\end{definition}
%
%
%\begin{definition}\cite{smith_monotone_1995}
%A system of differential equations
%$$ \dfrac{d x}{dt} = g(x), \quad x \in \mathcal{D}$$
%where $\mathcal{D}$ is an open subset of $\mathbf{R}^n$ and $g$ is continuously differentiable in $\mathcal{D}$ is said 
%
%\begin{itemize}
%\item irreducible if the Jacobian matrix of $g$ at $x$, $\mathcal{J}_g(x)$ is irreducible.
%\item competitive if $\mathcal{J}_g(x)$ has non positive off-diagonal elements:
%$$ \dfrac{\partial g_i}{\partial x_j}(x) \leq 0, \quad i \neq j.
%$$
%\end{itemize}
%
%\end{definition}
%
%
%
%\begin{theorem}\cite{zhu_stable_1994}\label{theorem: periodicASOrbit}
%We consider the system of differential equations
%$$
%\dfrac{dx}{dt} = g(x), \quad x \in \mathcal{D}.
%$$
%If
%\begin{itemize}
%\item $\mathcal{D}$ is an open, $p$-convex subset of $\mathbf{R}^3$,
%\item $\mathcal{D}$ contains a unique equilibrium point $x^*$ and $\det(\mathcal{J}_g(x^*)) < 0$,
%\item $g$ is analytic in $\mathcal{D}$,
%\item the system is competitive and irreducible in $\mathcal{D}$,
%\item the system is dissipative: For each $x_0 \in \mathcal{D}$, the positive semi-orbit through $x_0$, $\phi^+(x_0)$ has a compact closure in $\mathcal{D}$ . Moreover, there exists a compact subset $\mathcal{B}$ of $\mathcal{D}$ with the property that for each $x_0 \in \mathcal{D}$, there exists $T(x_0) > 0$ such that $x(t, x_0) \in \mathcal{B}$ for $t \geq T(x_0)$.
%\end{itemize}
%
%then either $x^*$ is stable, or there exists at least one non-trivial orbitally asymptotically stable  periodic orbit in $\mathcal{D}$.
%\end{theorem}
%
%One of the assumptions of theorem \eqref{theorem: periodicASOrbit} is that the system of equations is competitive, and that is not the case of system \eqref{equationsHDFWHW}. However, we will see that this system is equivalent to a competitive system, on which we can apply the theorem, and thus get back the conclusions.
%
%\begin{prop} \label{equivalentSystem}
%System \eqref{equationsHDFWHW} is equivalent to an irreducible and dissipative system. Moreover, if we note $z = (h_D, f_W, h_W)$ the variables of the equivalent system, that is competitive on
%\begin{itemize}
%\item $\Big\{z = (h_D, f_W, h_W) | 0 \leq h_D, f_W  \leq 0, h_W \leq 0 \Big\}$ if $\lfw - (1-\alpha)\beta r_F \geq 0$
%\item $\Big\{z = (h_D, f_W, h_W) | 0 \leq h_D, f_W \leq -K_F(1-\alpha) + \dfrac{K_F \lfw}{\beta r_F} , h_W \leq 0 \Big\}$ if $\lfw - (1-\alpha)\beta r_F < 0$
%\end{itemize}
%
%\end{prop}
%
%\begin{proof}
%Following \cite{wang_predator-prey_1997}, we do the following change of variable: $h_D =  H_D$, $f_W = -F_W$ and $h_W = -H_W$.  The system \eqref{equationsHDFWHW} is transformed into:
%
%\begin{equation}
%\def\arraystretch{2}
%\left\{ \begin{array}{l}
%\dfrac{dh_D}{dt}= \cI + e\dfrac{\lfw f_W}{1 - \lfw \theta f_W} h_W + (f_D - \mu_D) h_D - m_D h_D - m_W h_W, \\
%\dfrac{df_W}{dt} = (1-\alpha)(1 - \beta h_W) r_F \left(1 + \dfrac{f_W}{K_F(1-\alpha)} \right) f_W + \dfrac{\lfw f_W}{1 - \lfw \theta f_W} h_W, \\
%\dfrac{dh_W}{dt}= -m_D h_D - m_W h_W 
%\end{array} \right.
%\label{equationshDfWhW}
%\end{equation}
%
%
%We note $\mathcal{D} = \Big\{z = (h_D, f_W, h_W) | 0 < h_D, f_W < 0, h_W < 0 \Big\}$, and $g(z)$ the right hand side of the system. It is clear that $\mathcal{D}$ is a $p$-convex set, in which $g$ is analytic. According to proposition \ref{invariantRegion}, it exists an invariant region $\tilde{\Omega}$ for system \eqref{equationshDfWhW}, which is a compact subset of $\mathcal{D}$. This show that the system is dissipative for initial condition in  $\tilde{\Omega}$.
%
%The Jacobian of $g$ is given by:
%
%\begin{multline*}
%\mathcal{J}_g(z) = \\ \begin{bmatrix}
%f_D -\mu_D - m_D & e \dfrac{\lfw}{(1 - \lfw \theta f_W)^2} h_W & e \dfrac{\lfw f_W}{(1 - \lfw \theta f_W)} - m_W \\
%0 & r_F (1-\alpha)(1-\beta h_W) \Big(1 + \dfrac{2 f_W}{K_F(1-\alpha)}\Big) + \dfrac{\lfw}{(1 - \lfw \theta f_W)}  h_W & \dfrac{\lfw f_W}{(1 - \lfw \theta f_W)} - (1-\alpha)\beta r_F f_W - \beta r_F \dfrac{f_W^2}{K_F}\\
%-m_D & 0 & -m_W
%\end{bmatrix}.
%\end{multline*}
%Therefore, it is clear that system \eqref{equationshDfWhW} is irreducible. The system is competitive if the non-diagonal term $\lfw f_W - (1-\alpha)\beta r_F f_W - \beta r_F \dfrac{f_W^2}{K_F}$ is non-positive. For $f_w \in (-\infty, 0]$, we have:
%
%\begin{align*}
%&\dfrac{\lfw f_W}{(1 - \lfw \theta f_W)} - (1-\alpha)\beta r_F f_W - \beta r_F \dfrac{f_W^2}{K_F} \leq 0 \\
%&\Leftrightarrow \dfrac{\lfw}{(1 - \lfw \theta f_W)} - (1-\alpha)\beta r_F - \beta r_F \dfrac{f_W}{K_F} \geq 0 \\
%&\Leftrightarrow \dfrac{\lfw}{(1 - \lfw \theta f_W)} - (1-\alpha)\beta r_F \geq \beta r_F \dfrac{f_W}{K_F}.
%\end{align*}

% Therefore, the system is competitive on $(-\infty, 0]$ if $\lfw - (1-\alpha)\beta r_F \geq 0$. 

% When $\lfw - (1-\alpha)\beta r_F<0$, previous computations show that the system is competitive only on $\Big(-\infty, -K_F(1-\alpha) + \dfrac{K_F \lfw}{\beta r_F}\Big]$. 
% \end{proof}

%\begin{prop}
%When $\dfrac{r_F(1-\alpha) \beta}{\lfw} > 1$, we subdivide $\Omega$ into
%$$
%\Omega = \Omega_1 \cup \Omega_2
%$$
%where
%$$
%\Omega_1 = \Big\{\Big(H_D, F_W, H_W \Big) \in \mathbb{R}_+^3  \Big|H_D + H_W + eF_W \leq S^{max}, 0 \leq F_W < F_W^{compet}, H_W \leq H_W^{max} \Big\},
%$$
%
%$$
%\Omega_2 = \Big\{\Big(H_D, F_W, H_W \Big) \in \mathbb{R}_+^3  \Big|H_D + H_W + eF_W \leq S^{max}, F_W^{compet} \leq F_W \leq F_W^{max}, H_W \leq H_W^{max} \Big\},
%$$
%and
%$F_W^{compet} = K_F(1-\alpha) - \dfrac{K_F \lfw}{\beta r_F} > 0
%$
%
%Any solution starting in $\Omega_1$ will enter in $\Omega_2$, which is an invariant region, on which the equivalent system \eqref{equationshDfWhW} is competitive.
%\end{prop}
%
%\begin{proof}
%We start by showing that $\Omega_2$ is an invariant region. In fact, since we already prove that $\Omega$ is an invariant region, we only need to show that 
%$\nabla G \cdot g _{|F_W = F_W^{compet}}(y) > 0$, for $y \in \Omega_2$ where $G = F_W - F_W^{min}$ and $g$ is the right hand side of system \eqref{equationsSFWHW}. We have:
%
%\begin{align*}
%\nabla G \cdot g _{|F_W = F_W^{compet}} &= r_F(1-\alpha)(1+\beta H_W) \left(1 - \dfrac{F_W^{compet}}{(1-\alpha) K_F} \right)F_W^{compet} - \lfw H_W F^{compet}_W \\
%&= \left(r_F(1-\alpha)(1+\beta H_W) \left(1 - \dfrac{(1-\alpha) K_F - \dfrac{K_F \lfw}{r_F \beta}}{(1-\alpha) K_F}\right) - \lfw H_W \right) F^{compet}_W \\
%&= \left((1+\beta H_W) \left( \dfrac{\lfw}{\beta}\right) - \lfw H_W \right) F^{compet}_W \\
%&= \dfrac{\lfw}{\beta} F_W^{compet} \\
%&> 0
%\end{align*}
%
%Therefore, $\Omega_2$ is an invariant region. Now, we show that any solution with positive initial condition in $ \Omega_1$ enter in $\Omega_2$. We consider $F_W \in (0, F_W^{compet}]$, and using the same computations than before, we obtain:
%
%\begin{align*}
%\dfrac{dF_W}{dt} = &r_F(1-\alpha)(1+\beta H_W) \left(1 - \dfrac{F_W}{(1-\alpha) K_F}\right)F_W - \lfw H_W  F_W, \\
%& \geq \left(r_F(1-\alpha)(1+\beta H_W) \left(1 - \dfrac{F_W^{compet}}{(1-\alpha) K_F}\right) - \lfw H_W  \right) F_W, \\
%& \geq \dfrac{\lfw}{\beta} F_W,\\
%&> 0
%\end{align*}
%This means that any solution with positive initial condition in $\Omega_1$ will enter in $\Omega_2$.
%\end{proof}

We also recall the following useful results.
\begin{theorem}\label{Vidyasagar Theorem} \cite{vidyasagar_decomposition_1980, dumont_mathematical_2012}
Consider the following $\mathcal{C}^1$ system
\begin{equation}
\def\arraystretch{2}
\left\{ \begin{array}{l}
\dfrac{dx}{dt} = f(x), \\
\dfrac{dy}{dt} = g(x, y) 
\end{array} \right.
\label{equationVidyasagar}
\end{equation}

with $(x, y) \in \mathbf{R}^n \times\mathbf{R}^m$. Let $(x^*, y^*)$ be an equilibrium point.
If $x^*$ is GAS in $\mathbf{R}^n$ for the system $\dfrac{dx}{dt} = f(x)$, and if $y^*$ is GAS in $\mathbf{R}^m$ for the system $\dfrac{dy}{dt} = g(x^*, y)$, then $(x^*, y^*)$ is (locally) asymptotically stable for system \eqref{equationVidyasagar}. Moreover, if all trajectories of \eqref{equationVidyasagar} are forward bounded, then $(x^*, y^*)$ is GAS for \eqref{equationVidyasagar}.
\end{theorem}

\begin{theorem} \cite{wiggins_introduction_2003} \label{PoincareBendixson Theorem}
Let $\mathcal{M}$ be a positively invariant region for the vector field containing a finite number of fixed points. Let $p \in \mathcal{M}$ and consider its $\omega$-limit set, $\omega(p)$. Then one of the following possibilities holds.
\begin{enumerate}
\item $\omega(p)$ is a fixed point;
\item $\omega(p)$ is a closed orbit;
\item $\omega(p)$ consists of a finite number of fixed points $p_1, \ldots, p_n$ and regular heteroclinic or homoclinic orbits joining them.
\end{enumerate}
\end{theorem}

\begin{theorem} \cite{banasiak_methods_2014}  \label{tikhonovTheorem}
We consider the following systems of ODEs,
\begin{equation}\label{orginalProbelm}
\def\arraystretch{2}
\left\lbrace \begin{array}{l}
\dfrac{d x}{dt} = f(t,x,y,\epsilon), \quad x(0) = x_0 \\
\epsilon \dfrac{d y}{dt} = g(t,x,y,\epsilon), \quad y(0) = y_0, \\
\end{array} \right.
\end{equation}and the following assumptions:
\begin{enumerate}
\item \textbf{Assumption 1:} Assume that the functions $f, g$:
$$
f : [0, T]\times \mathcal{\bar{U}} \times \mathcal{V} \times [0, \epsilon_0] \rightarrow \mathbb{R}^n
$$
$$
g : [0, T]\times \mathcal{\bar{U}} \times \mathcal{V} \times [0, \epsilon_0] \rightarrow \mathbb{R}^m
$$
are continuous and satisfy the Lipschitz condition with respect to the variables $x$ and $y$ in $[0, T]\times \mathcal{\bar{U}} \times \mathcal{V}$, where $\mathcal{\bar{U}}$ is a compact set in $\mathbb{R}^n$, $\mathcal{V}$ is a bounded open set in $\mathbb{R}^m$ and $T, \epsilon_0 > 0$.
\item \textbf{Assumption 2:} The corresponding degenerate system reads
\begin{equation} \label{degenerateSystem}
\def\arraystretch{2}
\left\lbrace \begin{array}{l}
\dfrac{d x}{dt} = f(t,x,y,0), \quad x(0) = x_0 \\
0 =  g(t,x,y,0)
\end{array} \right.
\end{equation}

Assume that there exists a solution $y = \phi(t, x) \in \mathcal{V}$ of the second equation of \eqref{degenerateSystem}, for $(t,x) \in [0, T]\times \mathcal{\bar{U}}$. The solution is such that
$$
\phi \in \mathcal{C}^0([0, T]\times \mathcal{\bar{U}} ; \mathcal{V})
$$
and is isolated in $[0, T]\times \mathcal{\bar{U}}$.
\item \textbf{Assumption 3:} Consider the following auxiliary equation:
\begin{equation}\label{auxiliaryEquation}
\dfrac{d \tilde{y}}{d \tau} =  g(t,x,\tilde{y},0),
\end{equation}
where $t$ and $x$ are treated as parameters.

Assume that the solution $\tilde{y}_0 := \phi(t, x)$ of equation  \eqref{auxiliaryEquation} is asymptocially stable, uniformly with respect to $(t,x) \in [0, T]\times \mathcal{\bar{U}}$.

\item \textbf{Assumption 4:} 
Consider the reduced equation:
\begin{equation}\label{reducedEquation}
\dfrac{d\bar{x}}{dt} = f(t,\bar{x},\phi(t,\bar{x}), 0), \quad \bar{x}(0) = x_0.
\end{equation}
Assume that the function $(t,x) \mapsto f(t,x,\phi(t,x), 0)$ satisfies the Lipschitz condition with respect to $x$ in $[0, T]\times \mathcal{\bar{U}}$. Assume moreover that there exists a unique solution $\bar{x}$ of equation \eqref{reducedEquation} such that $$\bar{x}(t) \in Int \mathcal{\bar{U}}, \quad \forall t \in (0,T).$$

\item \textbf{Assumption 5:} We consider the equation \eqref{auxiliaryEquation} in the particular case $t=0$ and $x = x_0$:
\begin{equation}\label{auxiliaryEquation, 0}
\dfrac{d \tilde{y}}{d \tau} =  g(0, x_0,\tilde{y},0), \quad \tilde{y}(0) = y_0
\end{equation}
Assume that $y_0$ belongs to the region of attraction of the solution $y = \phi(0, x_0)$ of equation $g(0, x_0,\tilde{y},0) = 0$.
\end{enumerate}

Let assumptions 1, 2, 3, 4, 5 be satisfied. There exists $\epsilon_0$ such that for any $\epsilon \in (0, \epsilon_0]$ there exists a unique solution $(x_\epsilon(t), y_\epsilon(t))$ of problem \eqref{orginalProbelm} on $[0,T]$ and
\begin{equation}
\def\arraystretch{2}
\left\lbrace \begin{array}{l}
\lim\limits_{\epsilon \rightarrow 0}{x_\epsilon(t)} = \bar{x}(t) \quad t \in [0,T] \\
\lim\limits_{\epsilon \rightarrow 0} y_\epsilon(t) = \phi(t, \bar{x}(t)) := \bar{y}(t) \quad t \in (0,T] \\
\end{array} \right.
\end{equation}
where $\bar{x}(t)$ is the solution of problem \eqref{reducedEquation}.
\end{theorem}


\section{Model analysis under the quasi steady state assumption} \label{section:Model analysis under the quasi steady state assumption}

In this section, we consider that the displacement between wild and domestic area are fast processes compared to the demographic processes. This is typically the case when they occur daily. In this case, we consider the migration function
$$
M(H_D, H_W)  = \dfrac{1}{\epsilon}(m_W H_W - m_D H_D).
$$
The system writes:
\begin{equation}
\def\arraystretch{2}
\left\{ 
\begin{array}{l}
\dfrac{dH_D}{dt}= \cI + e\dfrac{\lfw F_W}{1 + \lfw \theta F_W}H_W + (f_D - \mu_D) H_D + \dfrac{1}{\epsilon}(m_W H_W - m_D H_D). \\
\dfrac{dF_W}{dt} = r_F(1- \alpha) (1+ \beta H_W) \left(1 - \dfrac{F_W}{K_F(1-\alpha)} \right) F_W - \dfrac{\lfw F_W}{1 + \lfw \theta F_W} H_W \\
\dfrac{dH_W}{dt}= - \dfrac{1}{\epsilon}(m_W H_W - m_D H_D)
\end{array} \right.
\label{equationsHDFWHW, fast}
\end{equation}

It is a slow-fast system, which we will study under the quasi steady state assumption. The following proposition justified this assumption.

\begin{prop} 


For any initial condition $(H_{D,0}, F_{W, 0})$, the solution $(H_D, F_W)$  of the following system
\begin{equation}
\def\arraystretch{2}
\left\lbrace \begin{array}{l}
\dfrac{dH_D}{dt} = \dfrac{\cI}{1+m} + \dfrac{f_D - \mu_D}{1+m} H_D + \dfrac{e }{1+m} \dfrac{m \lfw F_W}{1 + \lfw \theta F_W}H_D \\
\dfrac{dF_W}{dt} = (1-\alpha) (1+\beta m H_D) r_F \left(1 - \dfrac{F_W}{(1-\alpha)K_F} \right) F_W - m \dfrac{\lfw F_W}{1 + \lfw \theta F_W} H_D \\
H_W(t) = m H_D(t)
\end{array} \right.
\label{QSSA}
\end{equation} is such that for any time $T > 0$, 

\begin{equation}
\def\arraystretch{2}
\left\lbrace \begin{array}{l}
\lim\limits_{ \epsilon \rightarrow 0}{H_{D, \epsilon}(t)} = H_D(t) \quad t \in [0,T] \\
\lim\limits_{ \epsilon \rightarrow 0} F_{W,  \epsilon}(t) = F_W(t)\quad t \in [0,T] \\
 \lim\limits_{ \epsilon \rightarrow 0} H_{W,  \epsilon}(t) = m H_D(t), \quad  t\in (0,T] \\
\end{array} \right.
\end{equation}
where $\Big(H_{D, \epsilon}, F_{W,  \epsilon}, H_{W,  \epsilon} \Big)$ is the solution of the system \eqref{equationsHDFWHW, fast}.

\end{prop}

\begin{proof}

We will apply theorem \ref{tikhonovTheorem} on system \eqref{equationsHDFWHW, fast}. Following \cite{banasiak_methods_2014} and to avoid division by zero, we write the system with variable $H = H_D + H_W$. It gives:

\begin{equation} \label{equationsHFWHW} 
\def\arraystretch{2}
\left\{ 
\begin{array}{l}
\dfrac{dH}{dt}= \cI + e\dfrac{\lfw F_W}{1 + \lfw \theta F_W}H_W + (f_D - \mu_D)(H - H_W), \\
\dfrac{dF_W}{dt} = r_F(1- \alpha) (1+ \beta H_W) \left(1 - \dfrac{F_W}{K_F(1-\alpha)} \right) F_W - \dfrac{\lfw F_W}{1 + \lfw \theta F_W} H_W \\
\epsilon \dfrac{dH_W}{dt}= m_D H - (m_D + m_W) H_W
\end{array} \right.
\end{equation}

The system \eqref{equationsHFWHW} is under the form of \eqref{orginalProbelm} with $x = (H, F_W)$, $y = H_W$,  

$$f(t,x,y,\epsilon) = \begin{bmatrix}
cI + e\dfrac{\lfw F_W}{1 + \lfw \theta F_W}H_W + (f_D - \mu_D) (H - H_W), \\
r_F(1- \alpha) (1+ \beta H_W) \left(1 - \dfrac{F_W}{K_F(1-\alpha)} \right) F_W - \dfrac{\lfw F_W}{1 + \lfw \theta F_W} H_W
\end{bmatrix}  $$
and $g(t,x,y,\epsilon) = m H - (1 + m)H_W $.

The functions $f(t, \cdot, \cdot, \epsilon)$ and $g(t, \cdot, \cdot, \epsilon)$ are continuously differentiable in arbitrary $\mathcal{\bar{U}}$ and $\mathcal{V}$ intervals. Therefore, assumption 1 of theorem \eqref{tikhonovTheorem} is satisfied. The equation $ g(t,x,y,0) =  0$ admits for solution $H_W = \dfrac{m}{1+m}H$. It is clearly continuous as a function of $t, H$ and isolated. The assumption 2 of theorem \eqref{tikhonovTheorem} is also satisfied.

The auxiliary equation is given by
\begin{equation*}
\dfrac{d \tilde{H_W}}{d \tau} = m_D H(t) - (m_W + m_D)\tilde{H_W}
\end{equation*}

for which the fixed point $\tilde{H_W} = \dfrac{m}{1+m}H(t)$ is globally uniformly asymptotically stable with respect to $H$ and $t$. Therefore, the assumptions 3 and 5 of theorem \eqref{tikhonovTheorem} holds true.

The reduced equation writes:
\begin{equation} \label{equationHFW}
\def\arraystretch{2}
\left\lbrace \begin{array}{l}
\dfrac{dH}{dt}= \cI + e\dfrac{\lfw F_W}{1 + \lfw \theta F_W}\dfrac{m}{1+m}H + (f_D - \mu_D)\dfrac{H}{1+m}, \\
\dfrac{dF_W}{dt} = r_F(1- \alpha) (1+ \beta \dfrac{m}{1+m}H ) \left(1 - \dfrac{F_W}{K_F(1-\alpha)} \right) F_W - \dfrac{\lfw F_W}{1 + \lfw \theta F_W} \dfrac{m}{1+m}H.  \\
\end{array} \right.
\end{equation}
We see that the assumption 4 of theorem \eqref{tikhonovTheorem} is satisfied.

Consequently, we can claim that solutions 
$\Big(H_{ \epsilon}, F_{W,  \epsilon}, H_{W,  \epsilon} \Big)$  of system \eqref{degenerateSystem} satisfies:
\begin{equation}
\def\arraystretch{2}
\left\lbrace \begin{array}{l}
\lim\limits_{ \epsilon \rightarrow 0}{H_{ \epsilon}(t)} = H(t) \quad t \in [0,T] \\
\lim\limits_{ \epsilon \rightarrow 0} F_{W,  \epsilon}(t) = F_W(t)\quad t \in [0,T] \\
 \lim\limits_{ \epsilon \rightarrow 0} H_{W,  \epsilon}(t) = \dfrac{m}{1+m}H(t), \quad  t\in (0,T] \\
\end{array} \right.
\end{equation}
for any $T > 0$, where $(H, F_W)$ are the solutions to \eqref{equationHFW}. 

\medskip
Finally, we come back to the original variables that are $H_D, F_W$ and $H_W$. Using $H = H_D + H_W$ and $H_W = \dfrac{m}{1+m}H$, we obtain $H_W = m H_D$ and $H_D = \dfrac{1}{1+m}	H$. Therefore, the system is:

\begin{equation}
\def\arraystretch{2}
\left\lbrace \begin{array}{l}
\dfrac{dH_D}{dt} = \dfrac{\cI}{1+m} + \dfrac{f_D - \mu_D}{1+m} H_D + \dfrac{e }{1+m} \dfrac{m \lfw F_W}{1 + \lfw \theta F_W}H_D \\
\dfrac{dF_W}{dt} = (1-\alpha) (1+\beta m H_D) r_F \left(1 - \dfrac{F_W}{(1-\alpha)K_F} \right) F_W - m \dfrac{\lfw F_W}{1 + \lfw \theta F_W} H_D \\
H_W(t) = m H_D(t)
\end{array} \right.
\end{equation}
\end{proof}

Even if this approximation holds only on finite time intervals, and when $\epsilon$ is small enough, it justify our interest in the reduced model. We will see that when $\theta > 0$ the approximation is good enough. However, when $\theta = 0 $, some possibility for the long term dynamics are lost. In this particular case, the equations being simplified, we will provide a full analysis of the three dimensional system, using the theory of monotone systems.


On the following section, in order to simplify the computations, we redefine $\cI = \dfrac{\cI}{1 + m}$, $f_D = \dfrac{f_D}{1+ m}$, $\mu_D = \dfrac{\mu_D}{1+ m}$ and $e = \dfrac{e}{1+m}$. The system under study is therefore:

\begin{equation}
\def\arraystretch{2}
\left\lbrace \begin{array}{l}
\dfrac{dH_D}{dt} = \cI + (f_D - \mu_D) H_D + e  \dfrac{m \lfw F_W }{1 + \lfw \theta F_W}H_D \\
\dfrac{dF_W}{dt} = (1-\alpha) (1+\beta m H_D) r_F \left(1 - \dfrac{F_W}{(1-\alpha)K_F} \right) F_W - m \dfrac{\lfw F_W}{1 + \lfw \theta F_W} H_D 
\end{array} \right.
\label{QSSA}
\end{equation}



\subsection{Model analysis in the case without immigration}
In this case, the system rewrites:

\begin{equation}
\def\arraystretch{2}
\left\lbrace \begin{array}{l}
\dfrac{dH_D}{dt} = (f_D - \mu_D) H_D + e  \dfrac{m \lfw F_W }{1 + \lfw \theta F_W}H_D \\
\dfrac{dF_W}{dt} = (1-\alpha) (1+\beta m H_D) r_F \left(1 - \dfrac{F_W}{(1-\alpha)K_F} \right) F_W - m \dfrac{\lfw F_W}{1 + \lfw \theta F_W} H_D 
\end{array} \right.
\label{QSSA, I=0}
\end{equation}

\begin{prop}
\label{theoremEquilibre, I=0}
The following results hold:
\begin{itemize}
\item System \eqref{QSSA, I=0} admits a trivial equilibrium $TE = \Big(0,0\Big)$\YD{,} and a fauna-only equilibrium\YD{,} $EE^{F_W}=\Big(0, (1-\alpha)K_F \Big)$\YD{,} always exist.

\item When
$$
%\mathcal{N}_{\theta} := \dfrac{me}{\theta(\mu_D - f_D)} > 1 \quad \text{and} \quad 
\mathcal{N}_{\cI = 0} := \dfrac{\lfw (1-\alpha)K_F\big(me - \theta (\mu_D - f_D) \big)}{\mu_D - f_D} >1,
$$ 
system \eqref{QSSA, I=0} admits a unique coexistence equilibrium\YD{,} $EE^{HF_W} = \Big(H^*_{D, \cI = 0}, F^*_{W, \cI = 0}\Big)$, where 
$$
F^*_{W, \cI = 0} = \dfrac{\mu_D - f_D}{\lfw \big(me - \theta (\mu_D - f_D) \big)  },
\quad 
H^*_{D, \cI = 0} = \dfrac{(1-\alpha)r_F\Big(1 - \dfrac{F^*_{W}}{K_F(1-\alpha)} \Big)}{m\left(\dfrac{\lfw}{1 + \lfw \theta F_W^*} - \beta (1-\alpha) r_F + \beta r_F  \dfrac{F^*_{W}}{K_F}\right)}
.$$
\end{itemize}
\end{prop}

\begin{proof}
To derive the equilibrium we solve \eqref{QSSA, I=0} with $\dfrac{d y}{dt} = 0$. Therefore, an equilibrium satisfies the system of equations:
\begin{equation}\label{systemEquilibre, I=0}
\def\arraystretch{2}
\left\lbrace \begin{array}{cll}
 e m \dfrac{\lfw F_W^*}{1 + \lfw \theta F_W^*} + f_D - \mu_D = 0& \mbox{or} & H_D^* = 0,\\
m H_D^*\Big(\dfrac{\lfw}{1 + \lfw \theta F_W^*} - (1-\alpha)r_F \beta + \dfrac{r_F \beta}{K_F}F_W^* \Big) - r_F(1-\alpha) \left(1- \dfrac{F_W^*}{(1-\alpha)K_F}\right)= 0& \mbox{or} & F^*_W = 0.
\end{array} \right.
\end{equation}
When $H_D^*=0$ and $F_W^*=0$, we recover the trivial equilibrium $TE = \Big(0,0\Big)$. When $H_D^*=0$ and $F_W^*\neq0$, we obtain the fauna-only equilibrium $EE^{F_W} = \Big(0, K_F(1-\alpha) \Big)$. Finally, when $H_D^*\neq0$ and $F_W^*\neq0$, direct computations lead to a unique set of values given by:
$$
F^*_{W} = \dfrac{\mu_D - f_D}{\lfw \big(me - \theta (\mu_D - f_D) \big)  },
\quad 
H^*_{D} = \dfrac{(1-\alpha)r_F\Big(1 - \dfrac{F^*_{W}}{K_F(1-\alpha)} \Big)}{m\left(\dfrac{\lfw}{1 + \lfw \theta F_W^*} - \beta (1-\alpha) r_F + \beta r_F  \dfrac{F^*_{W}}{K_F}\right)},
$$
provided that $F^*_W$ and $H_D^*$ are well defined, that is positive and biologically meaningful. 

Thus, to have $F^*_W$ positive and biologically meaningful, we need to have: $me - \theta (\mu_D - f_D) >0$ and  $ F_W^* < (1-\alpha) K_F$. That is
$$
\dfrac{me}{\mu_D - f_D} > \theta,
$$
and
$$
K_F(1-\alpha) > \dfrac{\mu_D - f_D}{ \lfw  \big(me - \theta (\mu_D - f_D) \big)}.
$$

$H_D^*$ is positive if its denominator is positive. Using $F^*_{W} = \dfrac{\mu_D - f_D}{\lfw \big(me - \theta (\mu_D - f_D) \big)  }$ and $\dfrac{1 + \lfw \theta F_W^*}{\lfw} = \dfrac{me}{\lfw \big(m e - \theta (\mu_D - f_D)\big)}$, the denominator of $H_D^*$ is equal at:
\begin{multline*}
\dfrac{\lfw}{1 + \lfw \theta F_W^*} - \beta (1-\alpha) r_F + \beta r_F  \dfrac{F^*_{W}}{K_F} = \\ \dfrac{\lfw}{1 + \lfw \theta F_W^*}\left(1 - \dfrac{r_F me \beta(1-\alpha)}{\lfw \big(m e - \theta (\mu_D - f_D)\big)}\left(1 - \dfrac{\mu_D - f_D}{\lfw \big(me - \theta (\mu_D - f_D) \big)(1-\alpha)K_F } \right)\right).
\end{multline*}

Thanks to proposition \ref{propBeta}, we know that this quantity is positive. Therefore, the equilibrium of coexistence is biologically meaningful if:

\begin{equation*}
\lfw \dfrac{ K_F(1-\alpha) \big(me - \theta (\mu_D - f_D) \big)}{\mu_D - f_D}> 1,
\end{equation*}

since this condition implies $\dfrac{me}{\mu_D - f_D} > \theta$.

\end{proof}


Now, we look for the local asymptotic stability of the equilibrium.

\begin{prop}\label{propLAS, I=0} The following results are valid for $\theta \geq 0$.
\begin{itemize}
\item The trivial equilibrium $TE$ is unstable.
\item When $\mathcal{N}_{\cI = 0} \leq 1$, the fauna equilibrium $EE^{F_W}$, is Locally Asymptotically Stable (LAS). It is unstable if $\mathcal{N}_{\cI = 0} > 1$. 
\item When $\mathcal{N}_{\cI = 0} > 1$, the coexistence equilibrium, $EE^{HF_W}_{\cI =0}$, exists. It is LAS if 
\begin{equation}
    \theta K_{F}(1-\alpha)\lambda_{F}<\dfrac{me+\theta(\mu_{D}-f_{D})}{me-\theta(\mu_{D}-f_{D})}
    \label{condition_EEHF_LAS}
\end{equation}
and unstable if $\theta K_{F}(1-\alpha)\lambda_{F} > \dfrac{me+\theta(\mu_{D}-f_{D})}{me-\theta(\mu_{D}-f_{D})}$.
\end{itemize}
\end{prop}

\begin{proof}
To prove this theorem, we look at the Jacobian of system \eqref{QSSA, I=0}. It is given by:

\begin{multline*}
\mathcal{J}(H_D, F_W) = \\
\begin{bmatrix}
- (\mu_D-f_D) + e m \dfrac{\lfw F_W}{1 + \lfw \theta F_W}&  \dfrac{e m \lfw H_D}{(1 + \lfw \theta F_W)^2} \\
\left(- \dfrac{m \lfw}{1 + \lfw \theta F_W} + m\beta (1-\alpha) r_F \left(1 -\dfrac{F_W}{(1-\alpha) K_F} \right) \right) F_W  & r_F(1-\alpha)(1+\beta m H_D) \left( 1 - \dfrac{2F_W}{K_F(1-\alpha)} \right) -  \dfrac{ m \lfw H_D}{(1 + \lfw \theta F_W)^2}
\end{bmatrix}.
\end{multline*}

\begin{itemize}
\item At equilibrium $TE$, we have:
\begin{equation*}
\mathcal{J}(TE) = \begin{bmatrix}
- (\mu_D-f_D) & 0  \\
0 & r_F(1-\alpha) 
\end{bmatrix}.
\end{equation*}
and $r_F(1-\alpha) > 0$ is an eigenvalue of $\mathcal{J}(TE)$. So, $TE$ is unstable.

\item At equilibrium $EE^{F_W}$, we have
\begin{equation*}
\mathcal{J}(EE^{F_W}) = \begin{bmatrix}
- (\mu_D-f_D) + \dfrac{em \lfw (1-\alpha)K_F}{1+ \lfw \theta (1-\alpha)K_F} & 0 \\
- \dfrac{m \lfw(1-\alpha)K_F }{1 + \lfw \theta (1-\alpha)K_F}  & -(1-\alpha)r_F 
\end{bmatrix}.
\end{equation*}

Eigenvalues of $\mathcal{J}(EE^{F_W})$ are $-(1-\alpha)  r_F < 0$ and $- (\mu_D-f_D) + \dfrac{em \lfw (1-\alpha)K_F}{1+ \lfw \theta (1-\alpha)K_F}$. 

\begin{itemize}
\item If $- (\mu_D-f_D) + \dfrac{em \lfw (1-\alpha)K_F}{1+ \lfw \theta (1-\alpha)K_F} < 0 \Leftrightarrow \mathcal{N}_{\cI = 0} < 1$, the two eigenvalues are negative and $EE^{F_W}$ is LAS.
\item If $\mathcal{N}_{\cI = 0} > 1$, the second eigenvalue is positive, and $EE^{F_W}$ is unstable.
\item If $\mathcal{N}_{\cI = 0} = 1$, the second eigenvalue is null, and $EE^{F_W}$ is degenerate. To assess the local stability, we could use the theory of center manifolds. However, we will show on proposition \ref{prop EEF GAS} that when $\mathcal{N}_{\cI = 0} \leq 1$, $EE^{F_W}$ is Globally Asymptotically Stable, and therefore LAS.
\end{itemize}
%
%
%In this case, the Jacobian matrix is equal at:
%\begin{equation*}
%\mathcal{J}(EE^{F_W}) = \begin{bmatrix}
%0 & 0 \\
%- \dfrac{m \lfw(1-\alpha)K_F }{1 + \lfw \theta (1-\alpha)K_F} & -(1-\alpha)r_F 
%\end{bmatrix}.
%\end{equation*}
%We will use the theory of center manifolds to determine the stability of $EE^{F_W}$. First, we transform the equilibrium $EE^{F_W}$ to the origin via the translation $(y_1, y_2)= \Big(H_D, F_W - (1-\alpha)K_F \Big)$. With these variables, the system becomes:
%
%\begin{multline}
%\begin{bmatrix}
%\dfrac{dy_1}{dt} \\ \dfrac{dy_2}{dt}
%\end{bmatrix} = 
%\mathcal{J}(EE^{F_W}) \begin{bmatrix}
%y_1 \\ y_2
%\end{bmatrix} + \\
%\begin{bmatrix}
%\dfrac{em \lfw}{(1 + \lfw \theta (1-\alpha) K_F)(1 + \lfw \theta ((1-\alpha) K_F + y_2)} y_1 y_2 \\
%-r_F(1-\alpha) \beta m y_1 y_2 - r_F(1-\alpha)(1+ \beta m y_1) \dfrac{y_2^2}{(1-\alpha)K_F} - \dfrac{m \lfw}{(1 + \lfw \theta (1-\alpha) K_F)(1 + \lfw \theta ((1-\alpha) K_F + y_2)} y_1 y_2
%\end{bmatrix}
%\end{multline}
%
%We will transform this system in its normal form. To do so, we introduce 
%$$U = \begin{bmatrix}
%e_1 & e_2
%\end{bmatrix}^{-1} = \begin{bmatrix}
%- \dfrac{\mu_D - f_D}{e (1-\alpha)r_F} & 0 \\ \dfrac{\mu_D - f_D}{e (1-\alpha)r_F} & 1
%\end{bmatrix}$$
% where $e_1$ and $e_2$ are
%two eigenvectors of $\mathcal{J}(EE^{F_W})$ associated to the eigenvalues $\lambda_1 = 0$ and $\lambda_2 = -(1-\alpha)r_F$. We note  and $(h_D, f_W)$ the variables on the normal basis. They are given by:
%
%\begin{equation*}
%\begin{bmatrix}
%h_D \\ f_W
%\end{bmatrix} = U \begin{bmatrix}
%y_1 \\ y_2
%\end{bmatrix} = \begin{bmatrix}
%- \dfrac{\mu_D - f_D}{e (1-\alpha)r_F} y_1 \\ \dfrac{\mu_D - f_D}{e (1-\alpha)r_F} y_1 + y_2
%\end{bmatrix} \quad \text{and equivalently} \quad \begin{bmatrix}
%y_1 \\ y_2
%\end{bmatrix} = \begin{bmatrix}
%- \dfrac{e (1-\alpha)r_F}{\mu_D - f_D} h_D \\ h_D + f_W
%\end{bmatrix}
%\end{equation*}
%
%
%On the normal basis, the system \eqref{QSSA, I=0} is given by:
%\begin{equation}
%\begin{bmatrix}
%\dfrac{dh_D}{dt} \\ \dfrac{df_W}{dt} 
%\end{bmatrix} = \begin{bmatrix}
%0 & 0 \\
%0 & -r_F(1-\alpha)
%\end{bmatrix} \begin{bmatrix}
%h_D \\ f_W 
%\end{bmatrix} + \begin{bmatrix}
%R_c(h_D, f_W) \\ R_s(h_D, f_W)
%\end{bmatrix}
%\end{equation}
%
%where 
%\begin{equation}
%R_c(h_D, f_W) := - \dfrac{em \lfw}{(1 + \lfw \theta (1-\alpha) K_F)(1 + \lfw \theta ((1-\alpha) K_F + h_D + f_W))} (h_D + f_W) h_D
%\end{equation}
%and
%\begin{equation}
%R_s(h_D, f_W) := 
%\end{equation}
%
%The stable eigenspace is given by $E^s = Vec([0, 1])$ and the center eigenspace is is given by $E^c = Vec([1,0])$. We will determine an approximation of the center manifold $h(h_D)$.
%We note
%$$
%f_W = h(h_D) = a_0 h_D + a_1 h_D +  a h_D ^2 + bh_D^3 + \mathcal{O}(h_D^4)
%$$
%its Taylor expansion to the 4th order.
%
%Using the fact that $h(0) = h'(0) = 0$, we have $a_0 = a_1 = 0$. Moreover, $h(h_D)$ is invariant by the system. Therefore, 
%$$
%\dfrac{df_W}{dt} = h'(h_D) \dfrac{dh_D}{dt}
%$$



\item At equilibrium $EE^{HF_W}_{\cI = 0}$, the Jacobian is given by:

\begin{multline*}
\mathcal{J}_(EE^{HF_W}_{\cI = 0}) = \\ \begin{bmatrix}
0 & \dfrac{e m \lfw H_D}{(1 + \lfw \theta F_W)^2}  \\
\left(- \dfrac{m \lfw}{1 + \lfw \theta F_W} + m\beta (1-\alpha) r_F \left(1 -\dfrac{F_W}{(1-\alpha) K_F} \right) \right) F_W &
- r_F(1+\beta m H_D) \dfrac{F_W}{K_F} +  \dfrac{ m \lfw^2 \theta H_D F_W}{(1 + \lfw \theta F_W)^2}
\end{bmatrix}
\end{multline*}
where we used the equilibrium relations. $EE^{HF_W}_{\cI = 0}$ is LAS if the determinant of the Jacobian, $ \det(\mathcal{J}(EE^{HF_W}_{\cI = 0}))$, is positive and the trace, $\Tr(\mathcal{J}(EE^{HF_W}_{\cI = 0}))$ is negative. We have:

\begin{align*}
\det(\mathcal{J}(EE^{HF_W}_{\cI = 0})) &= \dfrac{e m \lfw H_D^*}{(1 + \lfw \theta F_W^*)^2}\left(\dfrac{m \lfw}{1 + \lfw \theta F_W^*} - m\beta (1-\alpha) r_F \left(1 -\dfrac{F_W^*}{(1-\alpha) K_F} \right) \right) F_W^* 
\end{align*}

Using proposition \ref{propBeta} and the relations at equilibrium we have that:
$$m\beta (1-\alpha) r_F \left(1 -\dfrac{F_W^*}{(1-\alpha) K_F} \right) < \dfrac{\mu_D - f_D}{F_W^* e} = \dfrac{m \lfw}{1 + \lfw \theta F_W^*}.$$ 

Therefore, the determinant is positive, $\det(\mathcal{J}(EE^{HF_W}_{\cI = 0})) > 0$.

The trace of $\mathcal{J}(EE^{HF_W}_{\cI = 0})$ is given by:
\begin{equation*}
\Tr(\mathcal{J}(EE^{HF_W}_{\cI = 0})) = - r_F(1+\beta m H_D^*) \dfrac{F_W^*}{K_F} +  \dfrac{ m \lfw^2 \theta H_D^* F_W^*}{(1 + \lfw \theta F_W^*)^2}.
\end{equation*}

It is negative if
\begin{subequations}
\begin{align}
r_F(1+\beta m H_D^*) \dfrac{F_W}{K_F} &>  \dfrac{ m \lfw^2 \theta H_D^* F_W^*}{(1 + \lfw \theta F_W^*)^2} \\
\dfrac{1+\beta m H_D^*}{H_D^*} \dfrac{r_F}{K_F} &>  \dfrac{ m \lfw^2 \theta }{(1 + \lfw \theta F_W^*)^2} \label{traceCoexistence1}
\end{align}
\end{subequations}

We have 
\begin{align*}
1+ \beta m H_D^* & = \dfrac{m\left(\dfrac{\lfw}{1 + \lfw \theta F_W^*} - \beta (1-\alpha) r_F + \beta r_F  \dfrac{F^*_{W}}{K_F}\right) + m \beta(1-\alpha)r_F\Big(1 - \dfrac{F^*_{W}}{K_F(1-\alpha)} \Big)}{m\left(\dfrac{\lfw}{1 + \lfw \theta F_W^*} - \beta (1-\alpha) r_F + \beta r_F  \dfrac{F^*_{W}}{K_F}\right)},  \\
&= \dfrac{m\dfrac{\lfw}{1 + \lfw \theta F_W^*}}{m\left(\dfrac{\lfw}{1 + \lfw \theta F_W^*} - \beta (1-\alpha) r_F + \beta r_F  \dfrac{F^*_{W}}{K_F}\right)}
\end{align*} 
and therefore, 

\begin{equation*}
\dfrac{1+ \beta m H_D^*}{H_D^*} = \dfrac{m\dfrac{\lfw}{1 + \lfw \theta F_W^*}}{(1-\alpha)r_F\Big(1 - \dfrac{F^*_{W}}{K_F(1-\alpha)} \Big)}
\end{equation*}
Injecting this expression in \eqref{traceCoexistence1}, we obtain that $\Tr(\mathcal{J}(EE^{HF_W}_{\cI = 0})) < 0$ if

\begin{subequations}
\begin{align*}
\dfrac{m\dfrac{\lfw}{1 + \lfw \theta F_W^*}}{(1-\alpha)r_F\Big(1 - \dfrac{F^*_{W}}{K_F(1-\alpha)} \Big)} \dfrac{r_F}{K_F} &>  \dfrac{ m \lfw^2 \theta }{(1 + \lfw \theta F_W^*)^2} \\
\dfrac{1}{\Big((1-\alpha) K_F - {F^*_{W}} \Big)}  &>  \dfrac{ \lfw \theta }{(1 + \lfw \theta F_W)}, \\
\dfrac{(1 + \lfw \theta F_W)}{ \lfw \theta } & > (1-\alpha) K_F - F^*_{W}, \\
\dfrac{me}{ \lfw \theta \Big(me - \theta \lfw (\mu_D - f_D) \Big)} & > (1-\alpha) K_F - \dfrac{\mu_D - f_D}{\lfw (me - \theta (\mu_D - f_D))}.
\end{align*}
\end{subequations}
using the value of $F_W^*$.

Finally, we obtain that 
\begin{equation*}
\Tr(\mathcal{J}(EE^{HF_W}_{\cI = 0})) < 0 \Leftrightarrow \theta K_{F}(1-\alpha)\lambda_{F}<\dfrac{me+\theta(\mu_{D}-f_{D})}{me-\theta(\mu_{D}-f_{D})}.
\end{equation*}


Consequently, when  $\theta K_{F}(1-\alpha)\lambda_{F}<\dfrac{me+\theta(\mu_{D}-f_{D})}{me-\theta(\mu_{D}-f_{D})}$, $EE^{HF_W}_{\cI = 0}$ is LAS, and when $\theta K_{F}(1-\alpha)\lambda_{F}>\dfrac{me+\theta(\mu_{D}-f_{D})}{me-\theta(\mu_{D}-f_{D})}$, the equilibrium is unstable. 

When $\theta K_{F}(1-\alpha)\lambda_{F} = \dfrac{me+\theta(\mu_{D}-f_{D})}{me-\theta(\mu_{D}-f_{D})}$, the equilibrium $EE^{HF_W}_{\cI = 0}$ is degenerate. 

\end{itemize}

\end{proof}

Now we asses the global stability of the equilibrium. 

\begin{prop} \label{prop EEF GAS}
When $\mathcal{N}_{\cI = 0} \leq 1$, $EE^{F_W}$ is Globally Asymptotically Stable (GAS).
\end{prop}

\begin{proof}
First, we assume $\mathcal{N}_{\cI = 0} < 1$. In this case, according to the demonstration of proposition \ref{propLAS, I=0}, we know that $EE^{F_W}$ is LAS.

We consider a solution $H_D^s, F_W^s$ of system \eqref{QSSA, I=0} with initial condition on $\Omega$. Since $\Omega$ is an invariant region, and by increasing monotonicity of function $z(x) = \dfrac{x}{1 + a x}$, we have for any time $t$:

\begin{equation*}
\def\arraystretch{2}
\left\lbrace \begin{array}{l}
\dfrac{dH_D^s}{dt} \leq -(\mu_D - f_D) H_D^s + \dfrac{e m \lfw (1-\alpha) K_F}{1 + \lfw \theta (1-\alpha) K_F} H_D^s \\
\dfrac{dF_W^s}{dt} = (1-\alpha) (1+\beta m H_D^s) r_F \left(1 - \dfrac{F_W^s}{(1-\alpha)K_F} \right) F_W^s - m \dfrac{\lfw F_W^s}{1 + \lfw \theta F_W^s} H_D^s \\
\end{array} \right.
\end{equation*}

We consider the following limit system:
\begin{equation}
\def\arraystretch{2}
\left\lbrace \begin{array}{l}
\dfrac{dH_D}{dt} = -(\mu_D - f_D) H_D + \dfrac{e m \lfw (1-\alpha) K_F}{1 + \lfw \theta (1-\alpha) K_F} H_D \\
\dfrac{dF_W}{dt} = (1-\alpha) (1+\beta m H_D) r_F \left(1 - \dfrac{F_W}{(1-\alpha)K_F} \right) F_W - m \dfrac{\lfw F_W}{1 + \lfw \theta F_W} H_D \\
\end{array} \right.
\label{EEF, limit system}
\end{equation}

We will apply theorem \ref{Vidyasagar Theorem} on this system, with $x = H_D$, $y = F_W$, $x^* = 0$ and $y^* = K_F(1- \alpha)$.

Since $\mathcal{N}_{\cI = 0} < 1$, it is straightforward to show that $x^*$ is GAS for system $\dfrac{dx}{dt} = f_1(x)$. It is also immediate to show that $y^*$ is GAS for system $\dfrac{dy}{dt} = f_{[2]}(x^*, y) = (1-\alpha) r_F \left(1 - \dfrac{y}{K_F(1-\alpha)} \right) $. 

Moreover, the trajectories of the solution of limit-system \label{EEF, limit system} with initial condition in $\Omega$ are bounded (proposition \ref{invariantRegion}). So, we can apply theorem \ref{Vidyasagar Theorem}, and we obtain that $EE^{F_W}$ is GAS in $\Omega$ for the limit system, and therefore for the original system \eqref{QSSA, I=0}.

Now we assume that $\mathcal{N}_{\cI = 0} = 1$. By definition of $\mathcal{N}_{\cI = 0}$, this is equivalent to $\mu_D - f_D = \dfrac{me \lfw (1-\alpha) K_F}{1 + \theta \lfw (1-\alpha)K_F}$. In this case, the system \eqref{QSSA, I=0} writes:

\begin{equation}
\def\arraystretch{2}
\left\lbrace \begin{array}{l}
\dfrac{dH_D}{dt} = \left(\dfrac{e m \lfw F_W}{1 + \lfw \theta F_W}- \dfrac{e m \lfw (1-\alpha) K_F}{1 + \lfw \theta (1-\alpha) K_F} \right) H_D \\
\dfrac{dF_W}{dt} = (1-\alpha) (1+\beta m H_D) r_F \left(1 - \dfrac{F_W}{(1-\alpha)K_F} \right) F_W - m \dfrac{\lfw F_W}{1 + \lfw \theta F_W} H_D \\
\end{array} \right.
\label{QSSA, I=0, Ni = 1}
\end{equation}

For any solution $\Big(H_D^s, F_W^s\Big)$ of system \eqref{QSSA, I=0, Ni = 1} with initial condition on $\Omega$, $F_W^{s}(t) \leq (1-\alpha) K_F$ for all $t\geq 0$. Since the function $z(x) = \dfrac{x}{1 + a x}$ is increasing, we obtain that $\dfrac{dH_D^s}{dt} < 0$ for all $t \geq 0$. 
Using the same reasoning than before, we obtain that $EE^{F_W}$ is GAS in $\Omega$ for system \eqref{QSSA, I=0, Ni = 1}.
\end{proof}

From now on, we will assume that $\mathcal{N}_{\cI = 0} > 1$.
\begin{prop}
A co-existence equilibrium, $EE^{HF_W}$, always exists. Moreover, if $\theta K_{F}(1-\alpha)\lambda_{F}<\dfrac{me+\theta(\mu_{D}-f_{D})}{me-\theta(\mu_{D}-f_{D})}$, then $EE^{HF_W}$ is GAS.
\end{prop}

\begin{proof}
We will use the Bendixson-Dulac criterion to show that no periodic or homoclinic loop exists. Then, by the Poincaré-Bendixson theorem, each solution will converge to an existing equilibrium. From Proposition \ref{propLAS, I=0}, under condition \eqref{condition_EEHF_LAS}, $EE^{HF_W}$ is the only LAS equilibrium. Consequently, all solutions will converge to it, that is $EE^{HF_W}$ is GAS.

Following \cite{hsu_competing_1978}, we will consider the Dulac function $g(H_D, F_W) = \dfrac{1 + \lfw \theta F_W }{F_W} H_D^{\gamma - 1}$ where $\gamma > 0$ is to be determined. We have:

\begin{equation*}
f_1 \times g(H_D, F_W) = -(\mu_D - f_D) \dfrac{1 + \lfw \theta F_W}{F_W}H_D^\gamma + e m \lfw H_D ^\gamma
\end{equation*}
so
\begin{equation*}
\dfrac{\partial \Big(f_1 g \Big)}{\partial H_D} = \gamma \left(e m \lfw - (\mu_D - f_D) \dfrac{1 + \lfw \theta F_W}{F_W} \right) H_D^{\gamma-1}.
\end{equation*}

Moreover,
\begin{equation*}
f_2 \times g(H_D, F_W) = r_F(1-\alpha) (1 + \beta m H_D) \left(1 - \dfrac{F_W}{(1-\alpha)K_F}\right)(1+\lfw \theta F_W) H_D^{\gamma - 1} - m \lfw \theta H_D^\gamma,
\end{equation*}
so
\begin{equation*}
\dfrac{\partial \Big(f_2 g \Big)}{\partial F_W} = r_F (1 + \beta m H_D) \left(\dfrac{-1 + (1-\alpha)K_F \lfw \theta}{K_F}-  \dfrac{2 \lfw \theta}{K_F} F_W\right) H_D^{\gamma - 1}
\end{equation*}

Summing the two derivatives, we obtain:

\begin{align*}
\dfrac{\partial \Big(f_1 g \Big)}{\partial H_D} + \dfrac{\partial \Big(f_2 g \Big)}{\partial F_W} &= H_D^{\gamma - 1} \left( \gamma e m \lfw - \gamma (\mu_D - f_D) \dfrac{1 + \lfw \theta F_W}{F_W} + r_F(1 + \beta m H_D) \times \right. \\ & \left. \left(\dfrac{-1 + (1-\alpha)K_F \lfw \theta}{K_F}-  \dfrac{2 \lfw \theta}{K_F} F_W\right) \right) \\
\dfrac{\partial \Big(f_1 g \Big)}{\partial H_D} + \dfrac{\partial \Big(f_2 g \Big)}{\partial F_W} &= \dfrac{H_D^{\gamma - 1}}{F_W} \left( \dfrac{2 r_F(1+ \beta m H_D) F_W}{K_F} \Big(\dfrac{(1-\alpha)\lfw \theta K_F - 1}{2} - \lfw \theta F_W \Big) + \right. \\ & \left. \gamma \Big(e m \lfw - \theta \lfw(\mu_D - f_D) \Big) F_W - (\mu_D - f_D) \right) \\
\dfrac{\partial \Big(f_1 g \Big)}{\partial H_D} + \dfrac{\partial \Big(f_2 g \Big)}{\partial F_W} &=\dfrac{H_D^{\gamma - 1}}{F_W} \times \\&  \left( \dfrac{2 r_F(1+ \beta m H_D) F_W}{K_F} \Big(\dfrac{(1-\alpha) \lfw \theta K_F - 1}{2} - \lfw \theta F_W \Big) +  \gamma \Big(e m \lfw - \theta \lfw(\mu_D - f_D) \Big)( F_W - F_W^*) \right) \\
\dfrac{\partial \Big(f_1 g \Big)}{\partial H_D} + \dfrac{\partial \Big(f_2 g \Big)}{\partial F_W} &= \dfrac{H_D^{\gamma - 1}}{F_W} \times P_{GAS}(F_W, H_D)
\end{align*}

where we noted
\begin{multline}
P_{GAS}(F_W, H_D) := - \dfrac{2 r_F \lfw \theta}{K_F} (1 + \beta m H_D) F_W^2 + \dfrac{r_F}{K_F} \Big((1-\alpha)K_F \lfw \theta - 1\Big) (1 + \beta m H_D) F_W + \\ \gamma \lfw \Big(em - \theta (\mu_D - f_D)\Big) F_W - \gamma \lfw \Big(em - \theta (\mu_D - f_D)\Big) F_W^*.
\end{multline}

We will show that we can choose a value of $\gamma$, say $\gamma^*$, such that the maximum value of $P_{GAS}$ in $\Omega = \{(H_D, F_W) \ 0 \leq F_W \leq K_F(1-\alpha) , 0 \leq H_D \leq H_D^{max}\}$ is negative, making $\dfrac{\partial \Big(f_1 g \Big)}{\partial H_D} + \dfrac{\partial \Big(f_2 g \Big)}{\partial F_W}$ negative in $\Omega$. By the Dulac criterion, there will be no periodic orbit or homoclinic loop in this region.

In fact, we will study $P_{GAS}$ not in $\Omega$ but in $\Omega^* = \{(H_D, F_W) \ 0 \leq F_W \leq F^{max, \gamma^*} , 0 \leq H_D \leq H_D^{max}\}$ where $F^{max, \gamma^*} \geq (1-\alpha)K_F$ will depend on $\gamma^*$.  

$P_{GAS}$ is continuous in $\Omega^*$, which is a compact set. Therefore, it admits a maximum on it.

\bigskip
We note $\Big(\bar{H_D}, \bar{F_W}\Big) \in \Omega^* $ a point at which $P_{GAS}$ is maximal. First, we show that $\Big(\bar{H_D}, \bar{F_W}\Big) \notin \mathring{\Omega^*}$. Assume the opposite. Then, the necessary local conditions gives  $\dfrac{\partial P_{GAS}}{\partial H_D}(\bar{H_D}, \bar{F_W}) = \dfrac{\partial P_{GAS}}{\partial F_W}(\bar{H_D}, \bar{F_W}) = 0$. 
Direct computations leads to

$$
\bar{F_W} = \dfrac{ \Big((1-\alpha)K_F \lfw \theta - 1\Big)}{4 \lfw \theta} + \gamma \dfrac{\Big(em - \theta (\mu_D - f_D)\Big)}{4 r_F \theta (1 + m \beta \bar{H_D})} K_F
$$

and 
$$
1 + \beta m \bar{H_D} = \dfrac{K_F \lfw \gamma(em - \theta (\mu_D - f_D)}{r_F (K_F(1-\alpha) \lfw \theta - 1)}
$$

For $\epsilon > 0$, we note: 

$$ 
\bar{F_W^\epsilon} =\dfrac{ \Big((1-\alpha)K_F \lfw \theta - 1\Big)}{4 \lfw \theta} + \gamma \dfrac{\Big(em - \theta (\mu_D - f_D)\Big)}{4 r_F \theta (1 + m \beta (\bar{H_D} + \epsilon))} K_F.$$

We will show that $P_{GAS}(\epsilon) := P_{GAS}(\bar{F_W^\epsilon}, \bar{H_D} + \epsilon) > P_{GAS}(\bar{F_W},\bar{H_D})$, meaning that $P_{GAS}(\bar{F_W},\bar{H_D})$ is not a maximum. First, we compute some intermediate terms.

\begin{align*}
\Big(1 + \beta m ( \bar{H_D} + \epsilon)\Big) \bar{F_W^\epsilon} &= (1+\beta m (\bar{H_D} + \epsilon))\dfrac{ \Big((1-\alpha)K_F \lfw \theta - 1\Big)}{4 \lfw \theta} + \gamma \dfrac{\Big(em - \theta (\mu_D - f_D)\Big)}{4 r_F \theta} K_F \\
& = (1 + \beta m  \bar{H_D}) \bar{F_W} + \beta m \epsilon \dfrac{(1-\alpha)K_F \lfw \theta - 1}{4 \lfw \theta} ,
\end{align*}


\begin{align*}
\Big(1 + \beta m ( \bar{H_D} + \epsilon)\Big) (\bar{F_W^\epsilon})^2 &= (1+\beta m (\bar{H_D} + \epsilon)) \left(\dfrac{ \Big((1-\alpha)K_F \lfw \theta - 1\Big)}{4 \lfw \theta} \right)^2 + \\& 2 \dfrac{\Big((1-\alpha)K_F \lfw \theta - 1\Big)\Big(em - \theta (\mu_D - f_D)\Big) \gamma K_F}{16 \lfw \theta^2 r_f} \\ & + \left(\gamma \dfrac{\Big(em - \theta (\mu_D - f_D)\Big)}{4 r_F \theta} K_F \right)^2 \dfrac{1}{ 1 + m \beta (\bar{H_D} + \epsilon)} \\
&= (1 + \beta m \bar{H_D}) (\bar{F_W})^2 + \\ & \beta m \epsilon \left(\left(\dfrac{ \Big((1-\alpha)K_F \lfw \theta - 1\Big)}{4 \lfw \theta} \right)^2 -  \dfrac{\left(\gamma \dfrac{\Big(em - \theta (\mu_D - f_D)\Big)}{4 r_F \theta} K_F \right)^2}{ (1 + m \beta (\bar{H_D} + \epsilon)(1 + m \beta \bar{H_D})} \right),
\end{align*}

and
\begin{align*}
\bar{F_W^\epsilon} = \bar{F_W} -m \beta \epsilon  \dfrac{\gamma K_F\Big(em - \theta (\mu_D - f_D)\Big)}{4 r_F \theta(1 + m \beta (\bar{H_D} + \epsilon))(1 + m \beta \bar{H_D})}.
\end{align*}

We now use these equalities to compute $P_{GAS}(\epsilon)$.

\begin{align*}
P_{GAS}(\epsilon) &= P_{GAS}(\bar{F_W}, \bar{H_D}) + \dfrac{r_F}{K_F} \beta m \epsilon \dfrac{\Big((1-\alpha)K_F \lfw \theta - 1\big)^2}{4 \lfw \theta} \\ & - \dfrac{2 r_F \lfw \theta}{K_F} \beta m \epsilon \left(\left(\dfrac{ \Big((1-\alpha)K_F \lfw \theta - 1\Big)}{4 \lfw \theta} \right)^2 -  \dfrac{\left(\gamma \dfrac{\Big(em - \theta (\mu_D - f_D)\Big)}{4 r_F \theta} K_F \right)^2}{ (1 + m \beta (\bar{H_D} + \epsilon)(1 + m \beta \bar{H_D})} \right) \\ &
- m \beta \epsilon  \dfrac{\lfw \gamma^2 K_F\Big(em - \theta (\mu_D - f_D)\Big)^2}{4 r_F \theta(1 + m \beta (\bar{H_D} + \epsilon)(1 + m \beta \bar{H_D})}\\
P_{GAS}(\epsilon) &= P_{GAS}(\bar{F_W}, \bar{H_D}) + m \beta \epsilon  \dfrac{r_F \Big((1-\alpha)K_F \lfw \theta - 1\Big)^2}{8 K_F \lfw \theta} 
- m \beta \epsilon  \dfrac{\lfw \gamma^2 K_F\Big(em - \theta (\mu_D - f_D)\Big)^2}{8 r_F \theta(1 + m \beta (\bar{H_D} + \epsilon))(1 + m \beta \bar{H_D})} \\
P_{GAS}(\epsilon) &= P_{GAS}(\bar{F_W}, \bar{H_D}) +\dfrac{m \beta \epsilon}{8 \theta}  \dfrac{r_F}{K_F} \dfrac{\Big((1-\alpha)K_F \lfw \theta - 1\Big)^2}{\lfw } 
- \dfrac{m \beta \epsilon}{8 \theta}  \dfrac{\lfw K_F \gamma^2 \Big(em - \theta (\mu_D - f_D)\Big)^2}{r_F (1 + m \beta \bar{H_D})^2 (1 + \dfrac{m \beta \epsilon}{1 + \beta m \bar{H_D}})} \\
\end{align*}

\begin{equation}\label{PGAS epsilon}
P_{GAS}(\epsilon) = P_{GAS}(\bar{F_W}, \bar{H_D}) +\dfrac{m \beta \epsilon}{8 \theta}  \dfrac{r_F}{K_F} \dfrac{\Big((1-\alpha)K_F \lfw \theta - 1\Big)^2}{\lfw } \left(1 - \dfrac{1}{1 + \dfrac{m \beta \epsilon}{1 + \beta m \bar{H_D}}} \right) 
\end{equation}

and therefore 
$$
P_{GAS}(\epsilon) > P_{GAS}(\bar{H_D}, \bar{F_W}).
$$

This means that $(\bar{H_D}, \bar{F_W})$ is not a local maximum for $P_{GAS}$. Consequently, $P_{GAS}$ does not reach its maximum on $\mathring{\Omega^*}$ but on $\bar{\Omega^*} =  \left\lbrace\big(F_W^{max, \gamma^*}, H_D \big) | 0 \leq H_D \leq \bar{H_D} \right\rbrace \cup \left\lbrace\big(0, H_D \big) | 0 \leq H_D \leq \bar{H_D} \right\rbrace \cup \left\lbrace\big(F_W, \bar{H_D} \big) | 0 \leq F_W \leq F_W^{max, \gamma^*} \right\rbrace \cup \left\lbrace\big(F_W, 0 \big) | 0 \leq F_W \leq F_W^{max, \gamma^*} \right\rbrace$. We will search for the maximum on this set.

\medskip

For a fixed value of $H_D$, the maximum value of $P_{GAS}(F_W ; H_D)$ is obtained for 

$$F_W = \bar{F_W}(H_D) := \dfrac{ \big((1-\alpha)K_F \lfw \theta - 1\big)}{4 \lfw \theta} + \gamma \dfrac{\Big(em - \theta (\mu_D - f_D)\Big)}{4 r_F \theta (1 + m \beta H_D)} K_F $$

We now fix $\gamma^*$ such that

\begin{equation}\label{gamma *}
\dfrac{(1-\alpha) K_F \theta \lfw -1}{2 \lfw \theta} < \bar{F_W}(H_D^{max}) < F_W^*
\end{equation}

This is possible because we assumed $\theta K_{F}(1-\alpha)\lambda_{F}<\dfrac{me+\theta(\mu_{D}-f_{D})}{me-\theta(\mu_{D}-f_{D})} \Leftrightarrow \dfrac{(1-\alpha) K_F \theta \lfw -1}{2 \lfw \theta} < F_W^*$. 

We also set 
\begin{equation} \label{FWmax}
F_W^{max, \gamma^*} = \max((1-\alpha)K_F, \bar{F_W}_{\gamma = \gamma^*}).
\end{equation}

We will show that this choice of $F_W^{max, \gamma^*}$ ensure that the maximum of $P_{GAS}$ is reached at $\Big(H_D^{max}, \bar{F_W}(H_D^{max})\Big) \in \Omega^*$ (since $F_W^{max, \gamma^*} > (1-\alpha) K_F > F^*_W$, we do have $\bar{F_W}(H_D^{max}) < F_W^{max, \gamma^*}$) and the inequalities \eqref{gamma *} ensure that this maximum is negative.

First, we show that the maximum of $P_{GAS}$ is indeed reached at $H_D^{max}, \bar{F_W}(H_D^{max})$. Previous computations shows that 

\begin{equation} \label{max1}
\max_{0 \leq F_W \leq F_W^{max, \gamma^*}} P_{GAS}(F_W ; H_D^{max}) = P_{GAS}(\bar{F_W}(H_D^{max}), H_D^{max}) > P_{GAS}(\bar{F_W}(0), 0) = \max_{0 \leq F_W \leq F_W^{max, \gamma^*}} P_{GAS}(F_W ; 0).
\end{equation}

(take $\epsilon = H_D^{max}$ and $\bar{H_D} = 0$ in equation \eqref{PGAS epsilon}).

Moreover,
\begin{multline*}
P_{GAS}(F_W^{max, \gamma^*}, H_D) = (1+ \beta m H_D) \dfrac{r_F}{K_F}F_W^{max, \gamma^*} \Big((1-\alpha)K_F - 2F_W^{max, \gamma^*} - 1\Big) +  \\
\gamma^* \Big(e m \lfw - \theta \lfw(\mu_D - f_D) \Big)( F_W^{max, \gamma^*} - F_W^*).
\end{multline*}

Since $F_W^{max, \gamma^*} > (1-\alpha)K_F$, the quantity $(1-\alpha)K_F - 2F_W^{max, \gamma^*} - 1$ is negative.
Therefore, the maximum of $P_{GAS}$ on $\left\lbrace\big((1-\alpha)K_F, H_D \big) | 0 \leq H_D \leq H_D^{max} \right\rbrace$ is reached for $H_D = 0$.

Consequently, we have 
\begin{equation} \label{max2}
P_{GAS}(\bar{F_W}(0) ; 0) = \max_{0 \leq F_W \leq F_W^{max, \gamma^*}} P_{GAS}(F_W ; 0) > P_{GAS}(F_W^{max, \gamma^*} , 0) = \max_{0 \leq H_D \leq H_D^{max}} P_{GAS}(F_W^{max, \gamma^*}, H_D).
\end{equation}

On $\left\lbrace\big(0, H_D \big) | 0 \leq H_D \leq \bar{H_D} \right\rbrace$, $P_{GAS}$ is constant and equal at:

$$P_{GAS}(0, H_D) = - \gamma \lfw \Big(em - \theta (\mu_D - f_D)\Big) F_W^* = P_{GAS}(0, 0).$$

Therefore,
\begin{equation}\label{max3}
P_{GAS}(\bar{F_W}(0) ; 0) = \max_{0 \leq F_W \leq F_W^{max, \gamma^*}} P_{GAS}(F_W ; 0) > P_{GAS}(0, 0) = \max_{0 \leq H_D \leq \bar{H_D}} P_{GAS}(H_D; 0).
\end{equation}

Considering \eqref{max1}, \eqref{max2}, and \eqref{max3}, we proved that the maximum value of $P_{GAS}$ is obtained for $H_D = H_D^{max}$ and $F_W = \bar{F_W}( H_D^{max})$. The maximum is equal at:

\begin{multline*}
P_{GAS}(\bar{F_W}(H_D^{max}) ,  H_D^{max}) = \dfrac{2 r_F(1+ \beta m \bar{H_D}) \bar{F_W}(H_D^{max})}{K_F} \Big(\dfrac{(1-\alpha) \lfw \theta K_F - 1}{2} - \lfw \theta \bar{F_W}(H_D^{max}) \Big) +  \\ \gamma^* \Big(e m \lfw - \theta \lfw(\mu_D - f_D) \Big)( \bar{F_W}(H_D^{max}) - F_W^*),
\end{multline*}

Since we choose  $\gamma^*$ such that \begin{equation}\label{inequalitiesGAS}
\dfrac{(1-\alpha) K_F \theta \lfw -1}{2 \lfw \theta} < \bar{F_W}(H_D^{max}) < F_W^*
\end{equation} 
it is negative. This ends the proof.


%\YD{
%\item[(b)] From Proposition \ref{propLAS, I=0}, we know that all equilibrium are unstable when $\theta K_{F}(1-\alpha)\lambda_{F} > \dfrac{me+\theta(\mu_{D}-f_{D})}{me-\theta(\mu_{D}-f_{D})}$. Thus,  according to the Poincaré-Bendixson theorem, the solution necessarily converges to a periodic-orbit. 
%}
%
%\YD{Si le cycle est stable alors on a une bifurcation de Hopf super-critique.... C'est en relation avec le signe du "nombre de Lyapunov", il me semble, non? Sans cela on ne peut rien dire.... Valaire?}
%\end{itemize}
\end{proof}


%\begin{prop} \marc{a remplacer éventuellement par la proposition suivante}
%If $\mathcal{N}_{\cI = 0} > 1$ the system \eqref{QSSA, I=0} admits a supercritical Hopf bifurcation for 
%$$\lfw = \lfw^c := \dfrac{me + \theta(\mu_D - f_D)}{\theta (1-\alpha) (K_F me - \theta(\mu_D - f_D))}.$$ 
%Therefore, for $\lfw > \lfw^c$, the system \eqref{QSSA, I=0} admits a stable periodic orbit.
%\end{prop}
%
%\begin{proof}
%At equilibrium $EE^{HF_W}_{\cI = 0}$, the Jacobian is:
%\begin{multline*}
%\mathcal{J}(EE^{HF_W}_{\cI = 0}) = \\ \begin{bmatrix}
%0 & \dfrac{e m \lfw H_D^*}{(1 + \lfw \theta F_W^*)^2}  \\
%\left(- \dfrac{m \lfw}{1 + \lfw \theta F_W^*} + m\beta (1-\alpha) r_F \left(1 -\dfrac{F_W^*}{(1-\alpha) K_F} \right) \right) F_W^* &
%- r_F(1+\beta m H_D^*) \dfrac{F_W^*}{K_F} +  \dfrac{ m \lfw^2 \theta H_D^* F_W^*}{(1 + \lfw \theta F_W^*)^2}
%\end{bmatrix}
%\end{multline*}
%On the following, we consider $\lfw$ as a bifurcation parameter. We note $r_{1,2} = a(\lfw) \pm i b (\lfw)$ the two eigenvalues of $\mathcal{J}(EE^{HF_W}_{\cI = 0})$.
%We proof on proposition \ref{propLAS, I=0} that its determinant is always positive, and that the trace is null if $\lfw = \lfw^c$. 
%
%Therefore, when $\lfw = \lfw^c$, $a(\lfw^c) = 0$ and $b (\lfw^c) \neq 0$. To derive a supercritical Hopf Bifurcation, we also have to verify the transversality property, that is $\dfrac{d a}{d \lfw}(\lfw^c) \neq 0$.
%
%We have that:
%$$
%a(\lfw) = \dfrac{1}{2} \Tr(\mathcal{J}(EE^{HF_W}_{\cI = 0})) 
%$$
%
%$$
%a(\lfw) = \dfrac{\mu_D - f_D}{em} \left(- \dfrac{m}{(1-\alpha) K_F - \dfrac{\mu_D - f_D}{\lfw (me - \theta (\mu_D - f_D))}} + \lfw \dfrac{me - \theta (\mu_D - f_D)}{me} \right) H_D^*
%$$
%
%\marc{reste encore à remplace $H_D^*$, ce qui va faire apparaître qlq $\lfw$ supplémentaires}
%
%\end{proof}


\begin{prop}
If $\mathcal{N}_{\cI = 0} > 1$ and if $\theta K_{F}(1-\alpha)\lambda_{F} > \dfrac{me+\theta(\mu_{D}-f_{D})}{me-\theta(\mu_{D}-f_{D})}$, the system \eqref{QSSA, I=0} admits a stable periodic orbit. 
\end{prop}

\begin{proof}
According to the Poincaré-Bendixson theorem \ref{PoincareBendixson Theorem}, any trajectory admits as $\omega$-limit set a fixed point, a heteroclinic or homoclinic orbit, or a closed orbit. We will show that the three first cases are impossible.
When $\mathcal{N}_{\cI = 0} > 1$ and $\theta K_{F}(1-\alpha)\lambda_{F} > \dfrac{me+\theta(\mu_{D}-f_{D})}{me-\theta(\mu_{D}-f_{D})}$ the three system's equilibrium, $TE$, $EE^{F_W}$ and $EE^{HF_W}$, are unstable. Therefore, a trajectory in $\Omega$ can not admit a fixed point as $\omega$-limit set.

The set $\Omega_{H_D=0} = \{(0, F_W) | 0 \leq F_W \leq (1-\alpha)K_F \}$ is the unstable manifold of $TE$ and the stable manifold of $EE^{F_W}$. 
A homoclinic or heteroclinic loop lies in the intersection of a stable and unstable manifold of a fixed point. Therefore, if it exists a loop involving $TE$ or $EE^{F_W}$, it will pass through $\Omega_{H_D=0}$. However, $\Omega_{H_D=0}$ is an invariant of system \eqref{QSSA, I=0}. Therefore, no loop can go through $\Omega_{H_D=0}$. Consequently, a trajectory in $\Omega$ can not admit a homoclinic or heteroclinic loop joining $TE$ or $EE^{F_W}$ as $\omega$-limit set.

So, a trajectory in $\Omega$ can only admit a homoclinic loop connecting $EE^{HF_W}$ to itself or a periodic orbit around $EE^{HF_W}$. Only a saddle point can generate a homoclinic loop. We show in proposition that $\det(\mathcal{J}(EE^{HF_W})) > 0$ and therefore, $EE^{HF_W}$ is not a saddle point and can not generate a homoclinic loop. 

Finally, any trajectory in $\Omega$ admits a periodic orbit around $EE^{HF_W}$ as $\omega$-limit set.
\end{proof}

\YD{\begin{remark}
It is interesting to notice that when $\theta=0$, which corresponds to the Holling-type I case, the endemic equilibrium is always GAS, in other words, no limit-cycle can exist.
\end{remark}
}

\subsubsection{Summary of the dynamic}

\begin{table}[!ht]
\centering
\def\arraystretch{2}
\begin{tabular}{c|c|c|c}
$\cI$ & $\N_{\cI=0}$ &  $\theta K_F(1-\alpha) \lfw \dfrac{me - \theta (\mu_D - f_D)}{me + \theta(\mu_D - f_D)}$ & \\
\hline
\multirow{4}{*}{$=0$} & $ \leq 1$ & &$EE^{F_W}$ exists and is GAS.  \\
\cline{2-4}
 & \multirow{3}{*}{$> 1$} & $<1$ &$EE^{HF_W}_{\cI=0}$ exists and is GAS.\\
 \cline{3-4}
 & & \multirow{2}{*}{$ > 1$} & $EE^{HF_W}_{\cI=0}$ exists and is unstable ; there is an asymptotically \\
 & & &  stable periodic solution.
\end{tabular}
\caption{\centering Conditions of existence and asymptotic stability of equilibrium for system \eqref{QSSA, I=0}}
\end{table}

\newpage 
\subsection{Model analysis with immigration, $\cI > 0$}

In this section, we study the long term behavior of the system \eqref{QSSA} with $\cI > 0$. We start by the equilibrium existence. Let set
\begin{equation}
\mathcal{N}_{\cI > 0}:= \dfrac{r_F(1-\alpha)\Big({\dfrac{\mu_D - f_D}{m\cI}+\beta\Big)}}{\lfw}.
\label{N_Igt0}
\end{equation}


\begin{prop}
The following results hold:
\begin{itemize}
\item System \eqref{QSSA} always admits a human-only equilibrium $EE^{H} = \Big(\dfrac{\cI}{\mu_D - f_D},0\Big)$.

\item When $\mathcal{N}_{\cI > 0}>1$, system \eqref{QSSA} admits one coexistence equilibrium, $EE^{HF_W} = \Big(H_D^*, F_W^* \Big)$. 

\item When $\mathcal{N}_{\cI > 0} = 1$, system \eqref{QSSA} admits one coexistence equilibrium, $EE^{HF_W} = \Big(H_D^*, F_W^* \Big)$ when condition I is true.

\item When $\mathcal{N}_{\cI > 0}<1$, system (\ref{QSSA}) admits one or two coexistence equilibrium, $EE_{i}^{HF_W} = \Big(H_{D,i}^*, F_{W,i}^* \Big)$, when (conditions I, II, and III) or when (conditions I, II and condition IV-V) are true. The number of coexistence equilibrium depend if there is equality (one equilibrium) or inequality (two equilibrium) in condition II.

\begin{itemize}
\item Condition I: $\Big(em - \theta(\mu_D-f_D) \Big) + \dfrac{\mu_D-f_D + \beta m \cI}{\lfw(1-\alpha) K_F } < \beta \theta m \cI$
\item Condition II: $\Delta_{P_F} \geq 0$
\item Condition III: $e m < \theta(\mu_D - f_D) $
\item Condition IV: $e m >\theta(\mu_D - f_D)$
\item Condition V: $\dfrac{  \theta \beta m \cI - \Big(em - \theta(\mu_D-f_D) + \dfrac{\mu_D-f_D + \beta m \cI}{\lfw K_F(1-\alpha) } \Big) }{2 \Big( \theta \beta m \cI - (em - \theta(\mu_D-f_D)) \Big)} < \dfrac{\mu_D - f_D}{\lfw (1-\alpha) K_F \Big(e m - \theta (\mu_D - f_D)  \Big) }$
\end{itemize}


where $\Delta_{P_F}$, equal at 

\begin{multline*}
\begin{array}{ll}
\Delta_{P_F}=&\left(r_{F}(1-\alpha)\Big(em-\theta(\mu_{D}-f_{D}+\beta m\mathcal{I})\Big)-\dfrac{r_{F}}{\lambda_{F}K_{F}}\left((\mu_{D}-f_{D})+\beta m\mathcal{I}\right)\right)^{2}+\\
&4\dfrac{r_{F}}{K_{F}}\Big(em-\theta(\mu_{D}-f_{D}+\beta m\mathcal{I})\Big)m\mathcal{I}
\end{array}
\end{multline*}
is the discriminant of the following polynomial:

\begin{multline*}
P_F(X) := X^2 \left(\dfrac{r_F}{K_F} \Big(em - \theta(\mu_D-f_D + \beta m \cI) \Big) \right) - \\ 
X \left(r_F (1-\alpha)   \Big(em - \theta(\mu_D-f_D + \beta m \cI)  \Big) + \dfrac{r_F(\mu_D-f_D)}{\lfw K_F} + \dfrac{\beta r_F m \cI}{\lfw K_F} \right) + \\
 \left(\dfrac{(\mu_D - f_D + m \cI \beta)(1-\alpha) r_F}{\lfw} - m\cI \right)
\end{multline*}

Moreover
\begin{itemize}
\item When $\mathcal{N}_{I>0} > 1$ and $em - \theta(\mu_D - f_D) \geq \beta \theta m \cI$, $F_W^*$ is the lowest root of $P_F$.
\item When $\mathcal{N}_{I>0} > 1$ and $em - \theta(\mu_D - f_D) < \beta \theta m \cI$, $F_W^*$ is the largest root of $P_F$.
\item When $\mathcal{N}_{I>0} = 1$ and condition I holds true, $F_W^*$ is the largest root of $P_F$.
\item When $\mathcal{N}_{I>0} < 1$ condition (I, II and III) or (I, II, IV and V) hold true, $F_{W,1}^*$ and $F_{W,2}^*$ are the two positive roots of $P_F$.
\end{itemize}
$H_D^*$ is always given by:
$$
H_D^* = \dfrac{\cI}{\mu_D - f_D - \dfrac{e\lfw m }{1 + \theta \lfw F_W^*} F_W^*}
$$

\end{itemize}
\end{prop}


\begin{proof}
To find all equilibrium, we solve \eqref{QSSA} with $\dfrac{d y}{dt} = 0$. Therefore, an equilibrium satisfies the system of equations:
\begin{equation}\label{systemEquilibre}
\left\lbrace \begin{array}{lll}
\def\arraystretch{2}
\cI + e m \dfrac{\lfw}{1 + \lfw \theta F_W^*} F_W^* H_D^* + (f_D - \mu_D) H_D^* = 0,&&\\
m H_D^*\Big(\dfrac{\lfw}{1 + \lfw \theta F_W^*} - (1-\alpha)r_F \beta + \dfrac{r_F \beta}{K_F}F_W^* \Big) - r_F(1-\alpha) \left(1- \dfrac{F_W^*}{(1-\alpha)K_F}\right)= 0& \mbox{or} & F^*_W = 0.
\end{array} \right.
\end{equation}


When $F_W^* = 0$, we recover the human only equilibrium $EE^{H} = \Big(\dfrac{\cI}{\mu_D - f_D},0\Big)$. When $F_W^* > 0$, the first equation leads to $H_D^* = \dfrac{\cI}{\mu_D - f_D - \dfrac{e\lfw m }{1 + \theta \lfw F_W^*} F_W^*}$, provided that the denominator is well posed. Injecting this expression in the second equation gives:


\begin{equation*}
\dfrac{m\cI}{\mu_D - f_D - \dfrac{e\lfw m }{1 + \theta \lfw F_W^*} F_W^*}\left(\dfrac{\lfw}{1 + \lfw \theta F_W^*} - (1-\alpha)r_F \beta + \dfrac{r_F \beta}{K_F}F_W^* \right) - r_F(1-\alpha) \left(1- \dfrac{F_W^*}{(1-\alpha)K_F}\right)= 0 
\end{equation*}

After some straightforward computations, this is equivalent to find the (positive) root(s) of $P_F(X)=a_2X^2+a_1X+a_0$, where
$$
\begin{array}{l}
a_2=\dfrac{r_F}{K_F} \Big(em - \theta(\mu_D-f_D + \beta m \cI) \Big),\\ 
a_1=- \left(r_F (1-\alpha)   \Big(em - \theta(\mu_D-f_D + \beta m \cI)  \Big) + \dfrac{r_F(\mu_D-f_D)}{\lfw K_F} + \dfrac{\beta r_F m \cI}{\lfw K_F} \right),\\
a_0= \dfrac{(\mu_D - f_D + \beta m \cI)(1-\alpha) r_F}{\lfw} - m\cI .
\end{array}
$$
We are searching for the positive roots $F_i^*$ of $P_F$ that are lower than $K_F(1- \alpha)$ and such that $\dfrac{e\lfw m}{1 + \theta \lfw F_i^*} F_i^* < \mu_D - f_D$. Before looking for the roots, we can make several observations, justified in the appendix.

\begin{itemize}
\item Observation 2: When $e m - \theta (\mu_D - f_D) < 0$, then $H_D^* > 0$. When $e m - \theta (\mu_D - f_D) > 0$ and $\dfrac{\mu_D - f_D}{\lfw \Big(e m - \theta (\mu_D - f_D)\Big)} > F_W^*$, then $H_D^* > 0$.
\item Observation 1: $P_F((1-\alpha)K_F) = - m \cI < 0$
\item Observation 3: $P_F\left(\dfrac{\mu_D - f_D}{\lfw \Big(e m - \theta (\mu_D - f_D)\Big)} \right) < 0$
\item Observation 4: $a_2 \geq 0$ implies $a_1 < 0$.
\end{itemize}

The number of roots and their signs depend on the coefficient signs. We distinguish different cases, summarized in table \ref{table2}, page \pageref{table2}.

\begin{itemize}
\item We first assume that $\mathcal{N}_{\cI > 0} > 1$. This is equivalent to $a_0>0$.

\begin{itemize}


\medskip

\item \textbf{CASE 1a:} Assume $a_2>0$. Then, $P_F$ may admits 2 or 0 positive roots, depending on the sign of its discriminant. $\Delta_{P_F}=a_1^2-a_0a_2$. A straightforward computation leads to

$$
\begin{array}{ll}
\Delta_{P_F}=&\left(r_{F}(1-\alpha)\Big(em-\theta(\mu_{D}-f_{D}+\beta m\mathcal{I})\Big)-\dfrac{r_{F}}{\lambda_{F}K_{F}}\left((\mu_{D}-f_{D})+\beta m\mathcal{I}\right)\right)^{2}+\\
&4\dfrac{r_{F}}{K_{F}}\Big(em-\theta(\mu_{D}-f_{D}+\beta m\mathcal{I})\Big)m\mathcal{I}.
\end{array}
$$

Thus, since $a_2>0$, we deduce that $\Delta_{P_F}>0$,
which means that $P_F$ admits two positive roots, $0<F_1^* < F_2^*$. 

Moreover, $P_F$ is negative in $(F_1^*, F_2^*)$, and positive elsewhere. Since $P_F((1-\alpha) K_F) < 0$, we necessarily have $0 < F_1^* < (1-\alpha) K_F < F_2^*$.
Moreover, since $P_F\left(\dfrac{\mu_D - f_D}{\lfw \Big(e m - \theta (\mu_D - f_D)\Big)} \right) < 0$, we also have $F_1^* < \dfrac{\mu_D - f_D}{\lfw \Big(e m - \theta (\mu_D - f_D)\Big)}$ when $e m - \theta (\mu_D - f_D) > 0$. Therefore, $H_{1,D}^*$ is well defined (cf observation 2), and $(H_{1,D}^*,F_1^*)$ is an equilibrium of system \eqref{QSSA}.

\item \textbf{CASE 1b:} Assume $a_2 = 0$. This implies $a_1 < 0$. Then, $P_F$ admits a unique (positive) root $F_1^* = \dfrac{-a_0}{a_1} = K_F(1-\alpha) - \dfrac{\lfw K_F \cI}{r_F e} < K_F(1-\alpha)$.

Moreover, $P_F$ is positive in $(-\infty, F_1^*]$. Since $P_F\left(\dfrac{\mu_D - f_D}{\lfw \Big(e m - \theta (\mu_D - f_D)\Big)} \right) < 0$, we also have $F_1^* < \dfrac{\mu_D - f_D}{\lfw \Big(e m - \theta (\mu_D - f_D)\Big)}$. Therefore, $H_{1,D}^*$ is well defined (cf observation 2), and $(H_{1,D}^*,F_1^*)$ is an equilibrium of system \eqref{QSSA}.

\item \textbf{CASE 2:} Assume $a_2<0$. Then $P_F$ admits one negative root, $F^*_1$, and one positive root, $F^*_2$. Moreover, $P_F$ is positive on $(F_1^*, F_2^*)$, and negative elsewhere. Since $P_F((1-\alpha) K_F) < 0$, we have $0 < F_2^* < (1-\alpha) K_F$. Moreover, since $P_F\left(\dfrac{\mu_D - f_D}{\lfw \Big(e m - \theta (\mu_D - f_D)\Big)} \right) < 0$, we deduce that $F_2^* < \dfrac{\mu_D - f_D}{\lfw \Big(e m - \theta (\mu_D - f_D)\Big)}$ when $e m - \theta (\mu_D - f_D) > 0$. Therefore, $H_{2,D}^*$ is well defined (cf observation 2), and $(H_{2,D}^*,F_2^*)$ is an equilibrium of system \eqref{QSSA}.

\end{itemize}

\item Assume $\mathcal{N}_{\cI > 0} = 1$. This is equivalent to $a_0 = 0$. 
\begin{itemize}
\item \textbf{CASE 3a} Assume $a_2 > 0$. Then $P_F$ admits two real roots, one being null, the other equal at $F_2^* = \dfrac{-a_1}{a_2} > 0$ thanks to observation 4. $P_F$ is negative in $(0, F_2^*)$. Since $P_F((1-\alpha)K_F) < 0$, we have $(1-\alpha)K_F < F_2^*$. No admissible coexistence equilibrium exists for system \eqref{QSSA}.
\item \textbf{CASE 3b} Assume $a_2 = 0$. Then $P_F$ does not admit a positive root.
\item  Assume $a_2 < 0$. Then $P_F$ admits two real roots, one being null, the other equal at $F_2^* = \dfrac{-a_1}{a_2}$. 
\begin{itemize}
\item \textbf{CASE 4} If $a_1 > 0$, since  $P_F$ is positive in $[0, F_2^*]$ and since $P_F((1-\alpha)K_F) < 0$ and $P_F\left(\dfrac{\mu_D - f_D}{\lfw \Big(e m - \theta (\mu_D - f_D)\Big)} \right) < 0$, $H_{2, D}^*$ and $F_2^*$ are well defined and $(H_{D, 2}^*, F_2^*)$ is an equilibrium for system \eqref{QSSA}.
\item \textbf{CASE 5} If $a_1 \leq 0$, $F_2^* \leq 0$. No admissible equilibrium exist for system \eqref{QSSA}.
\end{itemize} 
\end{itemize}


\item Assume $\mathcal{N}_{\cI > 0} < 1$. This is equivalent to $a_0<0$.

\begin{itemize}
\item \textbf{CASE 6:} Assume $a_2>0$. Then $P_F$ admits one negative root, $F_1^*$ and one positive root, $F_2^*$. Moreover, $P_F$ is negative on $(F_1^*, F_2^*)$, and positive elsewhere. Since $P_F((1-\alpha) K_F) < 0$, this means that $(1-\alpha) K_F < F_2^*$, such that $F^*_2$ is not and admissible root. No admissible equilibrium exists for system \eqref{QSSA}.
\item \textbf{CASE 7:} Assume $a_2 = 0$. This implies $a_1 < 0$. $P_F$ admits one negative root, so no admissible equilibrium exists for system \eqref{QSSA}.
\item Assume $a_2<0$. Then, in this case, $P_F$ may admits $2$ or $0$ positive roots, depending on the signs of $a_1$ and $\Delta_{P_F}$.
\begin{itemize}
\item \textbf{CASE 8:} Assume $a_1\leq0$. Then, since all coefficients are negative, all real roots are negative.
\item \textbf{CASE 9:} Assume $a_1>0$. When $\Delta_{P_F} < 0$, $P_F$ admits no real root.

\item \textbf{CASE 10:} Assume $a_1>0$. When $\Delta_{P_F} \geq 0$, $P_F$ admits two real and positive roots, $F_1^*$ and $F_2^*$ (with eventually $F_1^* = F_2^*$ when $\Delta_{P_F} = 0$). Moreover, $P_F$ is positive on $(F_1, F_2)$, and negative elsewhere. Since $P_F((1-\alpha) K_F) < 0$ and $P_F'((1-\alpha) K_F) < 0$, we have $0 < F_1^* < F^*_2 < (1-\alpha) K_F$.

\medskip

When $e m - \theta(\mu_D - f_D) < 0$, $F^*_1$ and $F^*_2$ are admissible, and thus system \eqref{QSSA} admits two admissible co-existence equilibrium $(H_{1,D}^*,F_1^*)$ and $(H_{2,D}^*,F_2^*)$ (cf Observation 2). 

Otherwise, if $e m - \theta(\mu_D - f_D)  > 0$, since $P_F\left(\dfrac{\mu_D - f_D}{\lfw \Big(e m - \theta (\mu_D - f_D)\Big)} \right) < 0$, either $\dfrac{\mu_D - f_D}{\lfw \Big(e m - \theta (\mu_D - f_D)} < F_1 < F_2$ and $H_D^*(F_i) < 0$, for $i = 1,2$ either $\dfrac{\mu_D - f_D}{\lfw \Big(e m - \theta (\mu_D - f_D)} > F_2 > F_1$ and $H_D^*(F_i) > 0$, for $i = 1,2$. 

The condition $\dfrac{\mu_D - f_D}{\lfw \Big(e m - \theta (\mu_D - f_D)} > F_2 > F_1$ is equivalent to $P_F'\left(\dfrac{\mu_D - f_D}{\lfw \Big(e m - \theta (\mu_D - f_D)}\right) < 0$, that is $2a_2\left(\dfrac{\mu_D - f_D}{\lfw \Big(e m - \theta (\mu_D - f_D)}\right) + a_1 < 0$, which is equivalent at:

$$
\dfrac{(1-\alpha) K_F}{2} + \dfrac{\mu_D - f_D + \beta m \cI}{2 \lfw (em - \theta (\mu_D - f_D + \beta m \cI))} < \dfrac{\mu_D - f_D}{\lfw (em - \theta (\mu_D - f_D))}
$$

$$
\dfrac{\lfw(1-\alpha) K_F}{2} + \dfrac{\mu_D - f_D + \beta m \cI}{2 (em - \theta (\mu_D - f_D + \beta m \cI))} < \dfrac{\mu_D - f_D}{em - \theta (\mu_D - f_D)}
$$

$$
\lfw(1-\alpha) K_F (em - \theta (\mu_D - f_D + \beta m \cI)) + \mu_D - f_D + \beta m \cI > 2\dfrac{(\mu_D - f_D) (em - \theta (\mu_D - f_D + \beta m \cI))}{em - \theta (\mu_D - f_D)}
$$

$$
\lfw(1-\alpha) K_F (em - \theta (\mu_D - f_D + \beta m \cI)) + \mu_D - f_D + \beta m \cI > 2(\mu_D - f_D) \Big(1 - \dfrac{\beta m \cI}{em - \theta (\mu_D - f_D)} \Big)
$$

$$
\lfw(1-\alpha) K_F (em - \theta (\mu_D - f_D)) - (\mu_D - f_D)  > (\lfw (1-\alpha)K_F \theta - 1) m \beta \cI - 2(\mu_D - f_D) \dfrac{\beta m \cI}{em - \theta (\mu_D - f_D)} 
$$

$$
\lfw(1-\alpha) K_F em + \Big(\lfw(1-\alpha) K_F \theta - 1\Big) (\mu_D - f_D)  > \Big(\lfw (1-\alpha)K_F \theta - 1 - \dfrac{2 (\mu_D- f_D)}{em - \theta (\mu_D - f_D)} \Big) m \beta \cI
$$

\end{itemize}

%\begin{figure}[!ht]
%    \centering
%    \includegraphics[scale = 0.5]{Figures/polynomePF.pdf}
%    \caption{Drawing of the different cases}
%    \label{fig:my_label}
%\end{figure}

\end{itemize}
\end{itemize}

\begin{table}[!ht]
\centering
\begin{tabular}{c|c|c|c|c|c|c}
 Case & \multicolumn{4}{c|}{Sign of} & Nb of positive roots & Equilibrium \\
 & $a_0$ & $a_1$ & $a_2$ & $\Delta_{P_F}$ && \\
\hline
\textbf{Case 1a} & \multirow{3}{*}{$>0$} & \multirow{2}{*}{$< 0^*$} & $> 0$ & $>0^*$ & $2$ & $(H_{1,D}^*,F_1^*)$ \\
\cline{4-7}
\textbf{Case 1b} &  &  & $= 0$ &  & $1$ & $(H_{1,D}^*,F_1^*)$ \\
\cline{3-7}
\textbf{Case 2} &  &  & $ < 0$ & $>0^*$ & $1$ & $(H_{2,D}^*,F_2^*)$ \\
\hline
\textbf{Case 3a}  & \multirow{4}{*}{$= 0$} & \multirow{2}{*}{$< 0^*$}& $> 0$ & $>0^*$ & 1 & -
\\
\cline{4-7}
\textbf{Case 3b}  & & & $=0$ &  & 1 & - \\
\cline{3-7}
\textbf{Case 4} & &$ > 0$ & \multirow{2}{*}{$ < 0$}& \multirow{2}{*}{$ > 0^*$} & 1 & $(H_{2,D}^*,F_2^*)$ \\
\cline{3-3} \cline{6-7}
\textbf{Case 5} & &$ \leq 0$ & &  & 0 & -
\\
\hline
\textbf{Case 6}  & \multirow{6}{*}{$ < 0$} & & $> 0$ & $>0^*$ & 1 & -
\\
\cline{3-7}
\textbf{Case 7} &  & $<0^*$ & $ = 0$ &  & 0 & - \\
\cline{3-7}
\textbf{Case 8} &  & $\leq0$ & \multirow{4}{*}{$ < 0$} &  & 0 & - \\
\cline{3-3} \cline{5-7}
\textbf{Case 9} &  & \multirow{3}{*}{$ > 0$} & &  $<0$& 0 & - \\
\cline{5-7}
\textbf{Case 10a} &  &  & &  $ = 0$& 1 & $(H_{1,D}^*,F_1^*)$ USC \\
\cline{5-7}
\textbf{Case 10b} &  &  & &  $ > 0$& 2 & $(H_{1,D}^*,F_1^*)$ and $(H_{2,D}^*,F_2^*)$ USC\\
\end{tabular}
\caption{\centering $^*$: implied by other conditions. \newline An empty box means that the sign of the quantity is not pertinent to conclude. \newline USC : Under Supplementary Conditions}
\label{table2}
\end{table}


%
%We will compute $P_F\left(\dfrac{\mu_D - f_D}{\lfw \Big(em - \theta (\mu_D - f_D) \Big)} \right)$. We have that:
%
%\begin{align*}
%&P_F\left(\dfrac{\mu_D - f_D}{\lfw \Big(em - \theta (\mu_D - f_D) \Big)} \right) = \left(\dfrac{\mu_D - f_D}{\lfw \Big(em - \theta (\mu_D - f_D) \Big)} \right)^2\left(\dfrac{r_F}{K_F} \Big(em - \theta(\mu_D-f_D + \beta m \cI) \Big) \right)  \\ & -\left(\dfrac{\mu_D - f_D}{\lfw \Big(em - \theta (\mu_D - f_D) \Big)} \right)\left(r_F (1-\alpha)   \Big(em - \theta(\mu_D-f_D + \beta m \cI)  \Big) + \dfrac{r_F(\mu_D-f_D)}{\lfw K_F} + \dfrac{\beta r_F m \cI}{\lfw K_F} \right) \\ &+\left(\dfrac{(\mu_D - f_D)(1-\alpha) r_F}{\lfw} - m\cI\Big(1 - \dfrac{(1-\alpha)\beta r_F}{\lfw} \Big) \right) \\
%&P_F\left(\dfrac{\mu_D - f_D}{\lfw \Big(em - \theta (\mu_D - f_D) \Big)} \right) = \dfrac{-r_F \theta \beta m \cI}{K_F}\left(\dfrac{\mu_D - f_D}{\lfw \Big(em - \theta (\mu_D - f_D) \Big)} \right)^2 + \dfrac{(\mu_D - f_D) r_F (1-\alpha) \theta \beta m \cI}{\lfw (me - \theta (\mu_D - f_D) )} \\ &- \dfrac{(\mu_D - f_D) r_F \beta m I}{\lfw ^2 (me - \theta (\mu_D - f_D)) K_F} - m\cI + \dfrac{(1-\alpha) r_F \beta m \cI}{\lfw} \\
%&P_F\left(\dfrac{\mu_D - f_D}{\lfw \Big(em - \theta (\mu_D - f_D) \Big)} \right) = \dfrac{-r_F \theta \beta m \cI}{K_F}\left(\dfrac{\mu_D - f_D}{\lfw \Big(em - \theta (\mu_D - f_D) \Big)} \right)^2 - m \cI - \dfrac{(\mu_D - f_D) r_F \beta m \cI}{\lfw ^2 (me - \theta (\mu_D - f_D)) K_F} \\& + \dfrac{r_F(1-\alpha) \beta m \cI}{\lfw} \dfrac{me}{me - \theta (\mu_D - f_D)} \\
%&P_F\left(\dfrac{\mu_D - f_D}{\lfw \Big(em - \theta (\mu_D - f_D) \Big)} \right) = - m \cI - m\cI \dfrac{(\mu_D - f_D) r_F \beta }{\lfw ^2 (me - \theta (\mu_D - f_D)) K_F} \dfrac{em}{em - \theta(\mu_D - f_D)} \\& + \dfrac{r_F(1-\alpha) \beta m \cI}{\lfw} \dfrac{me}{me - \theta (\mu_D - f_D)} \\
%&P_F\left(\dfrac{\mu_D - f_D}{\lfw \Big(em - \theta (\mu_D - f_D) \Big)} \right) = - m \cI - m\cI \dfrac{em}{em - \theta(\mu_D - f_D)} \left(\dfrac{(\mu_D - f_D) r_F \beta }{\lfw ^2 (me - \theta (\mu_D - f_D)) K_F} - \dfrac{r_F(1-\alpha) \beta }{\lfw}\right) \\
%&P_F\left(\dfrac{\mu_D - f_D}{\lfw \Big(em - \theta (\mu_D - f_D) \Big)} \right) = - m \cI + m\cI \dfrac{em}{em - \theta(\mu_D - f_D)} \dfrac{r_F(1-\alpha) \beta}{\lfw} \left(1 - \dfrac{(\mu_D - f_D)}{\lfw (me - \theta (\mu_D - f_D)) (1-\alpha)K_F}\right) \\
%&P_F\left(\dfrac{\mu_D - f_D}{\lfw \Big(em - \theta (\mu_D - f_D) \Big)} \right) < 0 
%\end{align*}

%thanks to proposition \ref{propBeta}.

\end{proof}

\begin{prop}
\label{propLAS} The following results are valid.
\begin{itemize}
\item When $\mathcal{N}_{\cI > 0} < 1$, the human equilibrium $EE^{H}$ is LAS.
\end{itemize}
\end{prop}

\begin{proof}
\begin{multline*}
\mathcal{J}(H_D, F_W) = \\
\begin{bmatrix}
- (\mu_D-f_D) + e m \dfrac{\lfw F_W}{1 + \lfw \theta F_W}&  \dfrac{e m \lfw H_D}{(1 + \lfw \theta F_W)^2} \\
\left(- \dfrac{m \lfw}{1 + \lfw \theta F_W} + m\beta (1-\alpha) r_F \left(1 -\dfrac{F_W}{(1-\alpha) K_F} \right) \right) F_W  & r_F(1-\alpha)(1+\beta m H_D) \left( 1 - \dfrac{2F_W}{K_F(1-\alpha)} \right) -  \dfrac{ m \lfw H_D}{(1 + \lfw \theta F_W)^2}
\end{bmatrix}.
\end{multline*}
\begin{itemize}
\item At equilibrium $EE^{H}$, the Jacobian matrix is:
\begin{equation*}
\mathcal{J}(EE^{H}) =
\begin{bmatrix}
- (\mu_D-f_D) &  \dfrac{e m \lfw I}{\mu_D - f_D} \\
0 & r_F(1-\alpha)\left(1+ \dfrac{\beta m\cI}{\mu_D - f_D}\right)-  \dfrac{m \lfw \cI}{\mu_D - f_D}
\end{bmatrix}.
\end{equation*}
The eigenvalues are: $- (\mu_D-f_D) < 0$ and $r_F(1-\alpha)\left(1+ \dfrac{\beta m\cI}{\mu_D - f_D}\right)-  \dfrac{m \lfw \cI}{\mu_D - f_D}$ which is negative when $\mathcal{N}_{\cI > 0} < 1$

\item At equilibrium $EE^{HF_W}$, the Jacobian matrix is:
\begin{equation*}
\mathcal{J}(EE^{HF_W}) =
\begin{bmatrix}
-(\mu_D -f_D) +e m \dfrac{\lfw F_W}{1 + \lfw \theta F_W} &  \dfrac{e m \lfw H_D}{(1 + \lfw \theta F_W)^2} \\
\left(- \dfrac{m \lfw}{1 + \lfw \theta F_W} + m\beta (1-\alpha) r_F \left(1 -\dfrac{F_W}{(1-\alpha) K_F} \right) \right) F_W  & - r_F(1+\beta m H_D) \dfrac{F_W}{K_F} +  \dfrac{ m \lfw^2 \theta H_D F_W}{(1 + \lfw \theta F_W)^2}
\end{bmatrix}.
\end{equation*}

$EE^{HF_W}$ is LAS if the trace of the Jacobian is negative and its determinant positive. The determinant is computed below, using the relations $-(\mu_D -f_D) +e m \dfrac{\lfw F_W^*}{1 + \lfw \theta F_W^*} = -\dfrac{\cI}{H_D^*}$ and $m H_D^*\Big(\dfrac{\lfw}{1 + \lfw \theta F_W^*} - (1-\alpha)r_F \beta + \dfrac{r_F \beta}{K_F}F_W^* \Big) = r_F(1-\alpha) \left(1- \dfrac{F_W^*}{(1-\alpha)K_F}\right)$.

\begin{align*}
\det(\mathcal{J}(H_D^*, F_W^*)) &= \dfrac{\cI r_F}{K_F} \dfrac{1 + \beta m H_D^*}{H_D^*}F_W^* - \dfrac{\cI m \lfw ^2 \theta F_W^*}{(1 + \lfw \theta F_W^*)^2} + \\ &\dfrac{e m \lfw H_D}{(1 + \lfw \theta F_W^*)^2} \left(\dfrac{m \lfw}{1 + \lfw \theta F_W^*} - m\beta (1-\alpha) r_F \left(1 -\dfrac{F_W^*}{(1-\alpha) K_F} \right) \right) F_W^* \\
\det(\mathcal{J}(H_D^*, F_W^*)) &= \dfrac{\cI r_F}{K_F} \beta m F_W^* +  \dfrac{\cI r_F}{K_F} \dfrac{F_W^*}{H_D^*} - \dfrac{\cI m \lfw ^2 \theta F_W^*}{(1 + \lfw \theta F_W^*)^2} + \\&  \dfrac{e m \lfw (1-\alpha) r_F}{(1 + \lfw \theta F_W^*)^2}  \left(1 -\dfrac{F_W^*}{(1-\alpha) K_F} \right)  F_W^* \\
\det(\mathcal{J}(H_D^*, F_W^*)) \dfrac{(1 + \lfw \theta F_W^*)^2}{F_W^*} &= \dfrac{\cI r_F}{K_F} \beta m (1 + \lfw \theta F_W^*)^2  +  \dfrac{\cI r_F}{K_F} \dfrac{(1 + \lfw \theta F_W^*)^2}{H_D^*} - \cI m \lfw ^2 \theta  + \\ &e m \lfw (1-\alpha) r_F  \left(1 -\dfrac{F_W^*}{(1-\alpha) K_F} \right)  \\
\det(\mathcal{J}(H_D^*, F_W^*)) \dfrac{(1 + \lfw \theta F_W^*)^2}{F_W^*} &= \dfrac{\cI r_F}{K_F} \beta m  + 2 \dfrac{\cI r_F \lfw \theta F_W^*}{K_F} + \dfrac{\cI r_F}{K_F} \beta m (\lfw \theta)^2 (F_W^*)^2  + \\ & \dfrac{r_F}{K_F} (1 + \lfw \theta F_W^*)^2 \Big((\mu_D -f_D) -e m \dfrac{\lfw F_W^*}{1 + \lfw \theta F_W^*}\Big)  \\ &- \cI m \lfw ^2 \theta  + e m \lfw (1-\alpha) r_F +  e m \lfw r_F \dfrac{F_W^*}{ K_F}   \\
\det(\mathcal{J}(H_D^*, F_W^*)) \dfrac{(1 + \lfw \theta F_W^*)^2}{F_W^*} &=  2\lfw \dfrac{r_F}{K_F} \Big(em - \theta(\mu_D-f_D + \beta m \cI) \Big) F_W^*  \\& -\lfw \left(r_F (1-\alpha)   \Big(em - \theta(\mu_D-f_D + \beta m \cI)  \Big) + \dfrac{r_F(\mu_D-f_D + \beta m \cI)}{\lfw K_F} \right) \\ & - \lfw^2 \theta \dfrac{r_F}{K_F} \Big(em - \theta(\mu_D-f_D + \beta m \cI) \Big) (F_W^*)^2 \\&
+ \lfw^2 \theta \left(\dfrac{(\mu_D - f_D + \beta m \cI)(1-\alpha) r_F}{\lfw} - m\cI\right)
\end{align*}







\begin{equation*}
\det(\mathcal{J}(H_D^*, F_W^*))  = \dfrac{ \lfw F_W^*}{(1 + \lfw \theta F_W^*)^2}\left( -P_F'(F_W^*) + \lfw \theta \Big(- a_2(F_W^*)^2  + a_0\Big) \right)
\end{equation*}
where $a_2$ and $a_0$ are the coefficient of $P_F = a_2 X^2 + a_1 X + a_0$ and $P_F(F_W^*) = 0$.

We will study the sign of $\det(\mathcal{J}(H_D^*, F_W^*))$ by considering the different cases describe in table \ref{table2}, page \pageref{table2}. Equilibrium $EE^{HF_W}$ exists for case 1a, 1b, 2, 4 10a and 10b.
\begin{itemize}
\item \textbf{Case 1a:} In this case, $a_0 > 0$, $a_2 > 0$ and $F_W^*$ is the lowest root of $P_F$. This implies that $P_F$ is decreasing at $F_W^*$ and therefore $P_F'(F_W^*) < 0$. Moreover, using $F_W^* = \dfrac{-a_1}{2a_2} - \dfrac{\sqrt{a_1^2 - 4 a_2 a_0}}{2a_2}$, we obtain $- a_2(F_W^*)^2  + a_0 > 0$. Consequently, $\det(\mathcal{J}(H_D^*, F_W^*)) > 0$.
\item \textbf{Case 1b:} In this case, $a_0 > 0$ $a_2 = 0$, $a_1 < 0$. This implies that $P_F$ is decreasing, and therefore $P_F'(F_W^*) < 0$. Consequently, $\det(\mathcal{J}(H_D^*, F_W^*)) > 0$.
\item \textbf{Case 2:} In this case, $a_0 > 0$ $a_2 < 0$, $a_1 < 0$ and $F_W^*$ is the largest root of $P_F$. $P_F$ is therefore decreasing at $F_W^*$ and therefore $P_F'(F_W^*) < 0$. Consequently, $\det(\mathcal{J}(H_D^*, F_W^*)) > 0$.
\item \textbf{Case 4:} In this case, $a_0 = 0$, $a_2 < 0$, $a_1 > 0$ and $F_W^*$ is the largest root of $P_F$. $P_F$ is therefore decreasing at $F_W^*$ and therefore $P_F'(F_W^*) < 0$. Consequently, $\det(\mathcal{J}(H_D^*, F_W^*)) > 0$.
\item \textbf{Case 10a:} In this case, $a_0 < 0$, $a_2 < 0$, $a_1 > 0$, $\Delta_{P_F} = 0$ and $F_W^*$ is the double root of $P_F$, $F_W^* = \dfrac{-a_1}{2 a_2}$. Therefore $P_F'(F_W^*) = 0$ and $- a_2(F_W^*)^2  + a_0 = 0$. Consequently, $\det(\mathcal{J}(H_D^*, F_W^*)) = 0$.
\item \textbf{Case 10b:} In this case, $a_0 < 0$, $a_2 < 0$, $a_1 > 0$, $\Delta_{P_F} > 0$. Moreover, two coexistence equilibrium may exists, $EE^{HF_W}_1$ and $EE^{HF_W}_2$ with $F_1^* < F_2^*$. Using $P_F(F_i^*) = 0$ for $i=1,2$, we obtain
\begin{align*}
- P_F'(F_i^*) +\lfw \theta (- a_2(F_i^*)^2  + a_0)  &= -(2a_2 F_i^* + a_1) + \lfw \theta \Big(- a_2(F_i^*)^2  + a_0\Big) \\
&= (-2a_2 + \lfw \theta a_1) F_i^* -a_1 + 2\lfw \theta a_0
 \end{align*}
Since $a_2 <0 < a_1$, $0 < (-2a_2 + \lfw \theta a_1)$ and 
\begin{equation}\label{inequalityDetF1F2}
- P_F'(F_2^*) + \lfw \theta (- a_2(F_2^*)^2  + a_0) > - P_F'(F_1^*) + \lfw \theta (- a_2(F_1^*)^2  + a_0).
\end{equation}


Using $F_2^* = \dfrac{-a_1}{2a_2} - \dfrac{\sqrt{a_1^2 - 4a_2a_0}}{2a_2}$, we obtain:
\begin{align*}
- P_F'(F_2^*) - a_2(F_2^*)^2  + a_0  &=(-2a_2 + \lfw \theta a_1) F_2^* -a_1 + 2\lfw \theta a_0\\
&= 2\lfw \theta a_0 + \sqrt{a_1^2 - 4a_2a_0} - \dfrac{\lfw \theta a_1^2}{2a_2} - \dfrac{\lfw \theta a_1\sqrt{a_1^2 - 4a_2a_0}}{2a_2}\\
&=\sqrt{a_1^2 - 4a_2a_0} \left(1 - \dfrac{\lfw \theta a_1}{2a_2} - \lfw \theta\dfrac{\sqrt{a_1^2 - 4a_2a_0}}{2a_2} \right)
\end{align*}

and, since $a_2 <0 < a_1$, $- P_F'(F_2^*) - a_2(F_2^*)^2  + a_0 > 0$.
Consequently, $\det(\mathcal{J}(H_{D,2}^*, F_{W,2}^*)) > 0$.


For $F_1^* = \dfrac{-a_1}{2a_2} + \dfrac{\sqrt{a_1^2 - 4a_2a_0}}{2a_2}$, we obtain:
\begin{align*}
- P_F'(F_1^*) - a_2(F_1^*)^2  + a_0  &=(-2a_2 + \lfw \theta a_1) F_1^* -a_1 + 2\lfw \theta a_0\\
&= 2\lfw \theta a_0 - \sqrt{a_1^2 - 4a_2a_0} - \lfw \theta \dfrac{a_1^2}{2a_2} + \lfw \theta \dfrac{a_1\sqrt{a_1^2 - 4a_2a_0}}{2a_2}\\
&=\sqrt{a_1^2 - 4a_2a_0} \left(-1 + \lfw \theta \dfrac{a_1}{2a_2} - \lfw \theta \dfrac{\sqrt{a_1^2 - 4a_2a_0}}{2a_2} \right)
\end{align*}

Moreover, since $a_1^2 > a_1^2 - 4a_2 a_0$ and $a_1 > 0 > a_2$, we have $\dfrac{a_1}{2a_2} < \dfrac{\sqrt{a_1^2 - 4 a_2 a_0}}{2a_2}$. This gives  $- P_F'(F_1^*) - a_2(F_1^*)^2  + a_0 < 0$ and consequently $\det(\mathcal{J}(H_{D,1}^*, F_{W,1}^*)) < 0$.

\end{itemize}

\begin{table}[!ht]
\centering
\begin{tabular}{c|c|c|c|c|c|c}
 Case & \multicolumn{4}{c|}{Sign of} &  Equilibrium & Sign of\\
 & $a_0$ & $a_1$ & $a_2$ & $\Delta_{P_F}$ & &$\det(\mathcal{J})$ \\
\hline
\textbf{Case 1a} & \multirow{3}{*}{$>0$} & \multirow{2}{*}{$< 0^*$} & $> 0$ & $>0^*$ & $(H_{1,D}^*,F_1^*)$ & $>0$ \\
\cline{4-7}
\textbf{Case 1b} &  &  & $= 0$ &  & $(H_{1,D}^*,F_1^*)$ &$>0$\\
\cline{3-7}
\textbf{Case 2} &  &  & $ < 0$ & $>0^*$ & $(H_{2,D}^*,F_2^*)$ &$>0$\\
\hline
\textbf{Case 4} & $=0$ &$ > 0$ & \multirow{1}{*}{$ < 0$}& \multirow{1}{*}{$ > 0^*$} & $(H_{2,D}^*,F_2^*)$ &$>0$ \\
\hline
\textbf{Case 10a} & \multirow{3}{*}{$ < 0$} & \multirow{3}{*}{$ > 0$} &\multirow{3}{*}{$ < 0$} &  $ = 0$& $(H_{1,D}^*,F_1^*)$ USC & $=0$ \\
\cline{5-7}
\multirow{2}{*}{\textbf{Case 10b}} &  &  & &  \multirow{2}{*}{$ > 0$}&  $(H_{1,D}^*,F_1^*)$ USC & $  < 0$\\
\cline{7-7}
 & & &  & & and $(H_{2,D}^*,F_2^*)$ USC & $ > 0$
\end{tabular}
\caption{\centering $^*$: implied by other conditions. \newline An empty box means that the sign of the quantity is not pertinent to conclude. \newline USC : Under Supplementary Conditions}
\label{table2}
\end{table}

We now compute the trace of $\mathcal{J}(H_D^*, F_W^*)$. It is given by:

\begin{align*}
\Tr(\mathcal{J}(H_D^*, F_W^*)) &= -(\mu_D -f_D) +e m \dfrac{\lfw F_W^*}{1 + \lfw \theta F_W^*} - r_F(1+\beta m H_D^*) \dfrac{F_W^*}{K_F} +  \dfrac{ m \lfw^2 \theta H_D^* F_W^*}{(1 + \lfw \theta F_W^*)^2} \\
&= -(\mu_D -f_D) +e m \dfrac{\lfw F_W^*}{1 + \lfw \theta F_W^*} - r_F  \dfrac{F_W^*}{K_F} + \left(-r_F \beta  + \dfrac{  \lfw^2 \theta }{(1 + \lfw \theta F_W^*)^2}\right)mF_W^* H_D^* \\
\times (1 + \lfw \theta F_W^*)^2&= (1 + \lfw \theta F_W^*)^2\left(-(\mu_D -f_D) +e m \dfrac{\lfw F_W^*}{1 + \lfw \theta F_W^*} - r_F  \dfrac{F_W^*}{K_F}\right) + \\& \left(-r_F \beta (1 + \lfw \theta F_W^*)^2  + \lfw^2 \theta\right)mF_W^* H_D^* \\
\Tr(\mathcal{J}(H_D^*, F_W^*))& \times (1 + \lfw \theta F_W^*)^2 = (1 + \lfw \theta F_W^*)\Big(-(\mu_D -f_D)(1 + \lfw \theta F_W^*) +e m \lfw F_W^* \Big) - (1 + \lfw \theta F_W^*)^2 r_F  \dfrac{F_W^*}{K_F} + \\& \left(-r_F \beta (1 + \lfw \theta F_W^*)^2  + \lfw^2 \theta\right)mF_W^* H_D^* \\
\Tr(\mathcal{J}(H_D^*, F_W^*)) &\times den_{H_D} \times (1 + \lfw \theta F_W^*)^2 = -(1 + \lfw \theta F_W^*)\Big((\mu_D -f_D)(1 + \lfw \theta F_W^*) - e m \lfw F_W^* \Big)^2 \\ &
- \Big((\mu_D -f_D)(1 + \lfw \theta F_W^*) - e m \lfw F_W^* \Big)(1 + \lfw \theta F_W^*)^2 r_F  \dfrac{F_W^*}{K_F} + \\& \left(-r_F \beta (1 + \lfw \theta F_W^*)^2  + \lfw^2 \theta\right)mF_W^* \cI (1+ \theta \lfw F_W^*) \\
\Tr(\mathcal{J}(H_D^*, F_W^*)) &\times den_{H_D} \times (1 + \lfw \theta F_W^*)= -\Big((\mu_D -f_D)(1 + \lfw \theta F_W^*) - e m \lfw F_W^* \Big)^2 \\&
- \Big((\mu_D -f_D)(1 + \lfw \theta F_W^*)  - e m \lfw F_W^* \Big)(1 + \lfw \theta F_W^*)r_F  \dfrac{F_W^*}{K_F} + \\& \Big(-r_F \beta (1 + \lfw \theta F_W^*)^2  + \lfw^2 \theta\Big)mF_W^* \cI  \\
\Tr(\mathcal{J}(H_D^*, F_W^*)) &\times den_{H_D} \times (1 + \lfw \theta F_W^*)= -\Big((\mu_D -f_D)(1 + \lfw \theta F_W^*) - e m \lfw F_W^* \Big)^2 \\&
- \Big((\mu_D -f_D)(1 + \lfw \theta F_W^*)  - e m \lfw F_W^* \Big)(1 + \lfw \theta F_W^*)r_F  \dfrac{F_W^*}{K_F} -r_F \beta (1 + \lfw \theta F_W^*)^2 mF_W^* \cI \\&  +\lfw^2 \theta mF_W^* \cI  \\
\end{align*}


\end{itemize}
\end{proof}

\newpage


\begin{align*}
P_{\Tr} := -\Big((\mu_D -f_D)(1 + \lfw \theta F_W^*) +e m \lfw F_W^* \Big)^2 - (1 + \lfw \theta F_W^*) r_F  \dfrac{F_W^*}{K_F} + \\& -r_F m \beta \cI (1 + \lfw \theta F_W^*)^2 F_W^* + \lfw^2 \theta m \cI F_W^*  
\end{align*}

\begin{equation*}
P_{\Tr} := a_3 (F_W^*)^3 +a_2 (F_W^*)^2+a_1 (F_W^*)+a_0
\end{equation*}

with
\begin{equation}
\def\arraystretch{2}
\left\lbrace \begin{array}{l}
a_0 = -(\mu_D - f_D) < 0 \\
a_1 = \lfw^2 \theta m \cI - r_F m \beta \cI - \dfrac{r_F}{K_F} - 2(\mu_D - f_D) \lfw \theta - 2 (\mu_D - f_D) em \lfw \\
a_2 = -2r_F m \beta	\cI \lfw \theta - \lfw \theta \dfrac{r_F}{K_F} - ((\mu_D - f_D) \lfw \theta +em \lfw)^2
\end{array} \right.
\end{equation}
\newpage

\section{Numerical Simulations}

\begin{table}[ht]
\centering
\begin{tabular}{|c|c|c|}
\hline 
Parameter & Value & Reference \\ 
\hline 
$e$ & & Assumed\\
$f_D$ & 0.0137 & \cite{koppert_consommation_1996}\\
$\mu_D$ & $1/60$ & \cite{ins_demographie}\\
$m_D$ & $0.483$ & \cite{avila_interpreting_2019}\\
$m_W$ & 24.3 & \cite{avila_interpreting_2019}\\
$r_F$ & $0.68$ & \cite{robinson_intrinsic_1986}\\
$K_F$ & 22725 & \cite{janson_ecological_1990} \\
$\alpha$ & $[0, 1)$ & parameter of interest; varies \\
$\beta$ & $\in [0, \beta^*)$ &  \\
$\lfw$ & - & parameter of interest; varies \\
$\mathcal{I}$ &  & \\
\hline
\end{tabular}

\caption{Parameters values}
\end{table}


\begin{table}[ht]
\centering
\begin{tabular}{|c|c|c|}
\hline 
Parameter & Value & Reference \\ 
\hline 
$e$ & & Assumed\\
$f_D$ & 0.0137 & \cite{koppert_consommation_1996}\\
$\mu_D$ & $1/60$ & \cite{ins_demographie}\\
$m_D$ & $0.483$ & \cite{avila_interpreting_2019}\\
$m_W$ & 24.3 & \cite{avila_interpreting_2019}\\
$r_F$ & $0.68$ & \cite{robinson_intrinsic_1986}\\
$K_F$ & 22725 & \cite{janson_ecological_1990} \\
$\alpha$ & $[0, 1)$ & parameter of interest; varies \\
$\beta$ & $\in [0, \beta^*)$ &  \\
$\lfw$ & - & parameter of interest; varies \\
$\mathcal{I}$ &  & \\
\hline
\end{tabular}

\caption{Parameters values}
\end{table}

\subsection{Numerical scheme}
We build a non standard scheme using the following equations:	
\begin{equation}
\def\arraystretch{2}
\left\lbrace \begin{array}{l}
\dfrac{H_D^{n+1} - H_D^n}{\phi(\Delta t)} = \cI  + (f_D - \mu_D - m_D) H_D^n + m_W H_W^n + \dfrac{e \lfw}{1 + \lfw \theta F_W^n} H_W^n F_W^{n+1} \\
\dfrac{F_W^{n+1} - F_W^n}{\phi(\Delta t)} = r_F(1-\alpha)(1+\beta H_W^n) \left(1 - \dfrac{F_W^{n+1}}{(1-\alpha)K_F}\right) F_W^n - \dfrac{e \lfw}{1 + \lfw \theta F_W^n} H_W^n F_W^{n+1} \\
\dfrac{H_W^{n+1} - H_W^n}{\phi(\Delta t)} = m_D H_D^n - m_W H_W^{n}
\end{array} \right.
\end{equation}

Direct computations lead to

\begin{equation}
\def\arraystretch{2}
\left\lbrace \begin{array}{l}
H_D^{n+1} = \Big(1 - \phi(\Delta t)(\mu_D - f_D + m_D)\Big)H_D^n + \phi(\Delta t) \Big(\cI  +(m_W  + \dfrac{e \lfw}{1 + \lfw \theta F_W^n} F_W^{n+1})H_W^n \Big) \\
F_W^{n+1} = \dfrac{1 + \phi(\Delta t) r_F(1-\alpha) (1 + \beta H_W^n)}{1 + \phi(\Delta t) \left(\dfrac{r_F(1 + \beta H_W^n)}{K_F}F_W^n + \dfrac{ \lfw H_W^n}{1 + \lfw \theta F_W^n}\right)}F_W^n \\
H_W^{n+1} = \phi(\Delta t) m_D H_D^n +(1 -\phi(\Delta t) m_W) H_W^{n}
\end{array} \right.
\end{equation}


We use the following time step function: $\phi(\Delta t) = \dfrac{1 - e^{-Q \Delta t}}{Q}$ where $Q = \max\{\mu_D - f_D + m_D, m_W\}$.

For the 2D equations, we use the following scheme:

\begin{equation}
\def\arraystretch{2}
\left\lbrace \begin{array}{l}
H_D^{n+1} = \Big(1 - \phi(\Delta t)(\mu_D - f_D)\Big)H_D^n + \phi(\Delta t) \Big(\cI  +\dfrac{m e \lfw}{1 + \lfw \theta F_W^n} F_W^{n+1} H_D^n \Big) \\
F_W^{n+1} = \dfrac{1 + \phi(\Delta t) r_F(1-\alpha) (1 + \beta H_W^n)}{1 + \phi(\Delta t) \left(\dfrac{r_F(1 + \beta m H_D^n)}{K_F}F_W^n + \dfrac{ \lfw m H_D^n}{1 + \lfw \theta F_W^n}\right)}F_W^n \\
H_W^{n+1} = m H_D^{n+1}
\end{array} \right.
\end{equation}




\subsection{Without Immigration}

\begin{figure}[!ht]
\centering
\includegraphics[scale=0.4]{HT2I0F.png}
\caption{Convergence toward $EE^{F}$ and comparison between 2D and 3D model}
\end{figure}

\begin{figure}[!ht]
\centering
\includegraphics[scale=0.4]{HT2I0HF.png}
\caption{Convergence toward $EE^{HF_W}$}
\end{figure}

\begin{figure}[!ht]
\centering
\includegraphics[scale=0.4]{HT2I0LC.png}
\caption{Convergence toward a limit cycle around $EE^{HF_W}$}
\end{figure}

\newpage

\subsection{With Immigration}

\begin{figure}[!ht]
\centering
\includegraphics[scale=0.4]{HT2_2eq.png}
\caption{Case 10b with $em - \theta(\mu_D - f_D) < 0$. Bi-stability between $EE^H$ and $EE^{HF_W}_2$.}
\end{figure}





\newpage




% \section{Dimensionless system}
% To facilitate the computation, we consider the following dimensionless system:

% \begin{equation}
% \def\arraystretch{2}
% \left\lbrace \begin{array}{l}
% \dfrac{dh_D}{d\tau} = i - d h_D + n \dfrac{h_D f_W}{1 + f_W} \\
% \dfrac{df_W}{d \tau} = (1+ b h_D) \left(1 - \dfrac{f_W}{k_F} \right) f_W - n \dfrac{h_D f_W}{1 + f_W}
% \end{array} \right.
% \label{0dimSystem}
% \end{equation}

% where $h_D = H_D \dfrac{\lfw \theta}{e}$, $f_W = F_W \lfw \theta$, $\tau = r_F(1-\alpha) t$, 
% $$i = \dfrac{I \lfw \theta}{e r_F(1-\alpha)} \quad d = \dfrac{\mu_D - f_D}{r_F(1-\alpha)}$$
% $$n = \dfrac{e m }{\theta r_F(1-\alpha)} \quad b = \dfrac{me \beta}{\lfw \theta}$$ and $$k_F = (1-\alpha) K_F \lfw \theta$$

% The coexistence equilibrium is given by:

% $$h_D^* = \dfrac{i}{d - \dfrac{n f_W}{1 + f_W}} \quad OR \quad h_D^* = \dfrac{1 - \dfrac{f_W}{k_F}}{\dfrac{n }{1 + f_W} - b (1 - \dfrac{f_W}{k_F})}$$

% and $f_W^*$ is solution of:

% $$P_f(f_W^*) = 0$$
% where
% \begin{equation}
% P_f(X) := X^2 \dfrac{n - bi - d}{k_F} - X (n - bi - d + \dfrac{d + bi}{k_F}) + (d + bi - ni)
% \end{equation}

% A root $f_{W,i}$ of $P_f$ define an equilibrium if $0 < f_{W,i} < k_F$ and $d - \dfrac{n f_W}{1 + f_W} > 0$.

% \begin{prop} The following holds true:
% \begin{itemize}
% \item If $d + bi - ni > 0 \Leftrightarrow \mathcal{N}_{\cI > 0}:= \dfrac{r_F(1-\alpha)\Big({\dfrac{\mu_D - f_D}{m\cI}+\beta\Big)}}{\lfw}  > 1$, the system admits a unique equilibrium of coexistence. $f_W^*$ is:
% \begin{itemize}
% \item if $n - bi - d > 0$ the lowest root of $P_f$ (which admits two positive roots)
% \item if $n - bi - d < 0$ the largest root of $P_f$ (which admits one positive and one negative root)
% \end{itemize}
% \item $d + bi - ni  < 0$, the system admits 0 or 2 equilibrium of coexistence:
% \begin{itemize}
% \item if $n - bi - d > 0$, it admits no equilibrium
% \item if $n - bi - d < 0$, it may admit two equilibrium of coexistence, depending on the sign of $\Delta_f$
% \end{itemize}
% \end{itemize}

% \end{prop}


% The Jacobian matrix of \eqref{0dimSystem} is given by:
% \begin{equation}
% \mathcal{J}(h_D, f_W) = \begin{bmatrix}
% - d + \dfrac{nf_W}{1+f_W} & \dfrac{n}{(1 + f_W)^2}h_D \\
% \Big(b (1 - \dfrac{f_W}{k_F}) - \dfrac{n}{1+f_W}\Big) f_W & (1+bh_D)(1 - \dfrac{2f_W}{k_F}) - \dfrac{n}{(1 + f_W)^2}h_D
% \end{bmatrix}
% \end{equation} 

% At coexistence equilibrium, we have: $-d + \dfrac{nf_W}{1 + f_W} = -\dfrac{i}{h_D}$ and $(1+bh_D)(1 - \dfrac{2f_W}{k_F}) - \dfrac{n}{(1 + f_W)^2}h_D = -(1+b h_D) \dfrac{f_W}{k_F} + \dfrac{n h_D f_W}{(1+f_W)^2}$.
% We have:

% \begin{align*}
% \det(\mathcal{J}(h_D^*, f_W^*)) &= i \dfrac{1 + bh_D}{h_D} \dfrac{f_W}{k_F} - \dfrac{n i f_W}{(1+f_W)^2} + \Big(\dfrac{n}{1+f_W} - b (1 - \dfrac{f_W}{k_F}) \Big) \dfrac{n f_W h_D}{(1+f_W)^2} \\
% \det(\mathcal{J}(h_D^*, f_W^*)) &= \dfrac{i}{k_F}\dfrac{f_W}{h_D} + ib \dfrac{f_W}{k_F} - ni \dfrac{f_W}{(1+f_W)^2} +(1-\dfrac{f_W}{k_F})\dfrac{n f_W}{(1+f_W)^2} \\
% \det(\mathcal{J}(h_D^*, f_W^*)) &= \dfrac{f_W}{(1 + f_W)^2} \left(\dfrac{i}{k_F}\dfrac{(1+f_W)^2}{h_D} + ib \dfrac{(1+f_W)^2}{k_F} - ni +(1-\dfrac{f_W}{k_F})n \right) \\
% \det(\mathcal{J}(h_D^*, f_W^*)) &= \dfrac{f_W}{(1 + f_W)^2} \left(\Big(d - \dfrac{nf_W}{1 + f_W} \Big)\dfrac{(1+f_W)^2}{k_F} + ib \dfrac{(1+f_W)^2}{k_F} - ni +(1-\dfrac{f_W}{k_F})n \right)\\
% \det(\mathcal{J}(h_D^*, f_W^*)) &= \dfrac{f_W}{(1 + f_W)^2} \left(\dfrac{d + ib}{k_F}(1+f_W)^2  - \dfrac{nf_W(1+f_W)}{k_F} - ni +(1-\dfrac{f_W}{k_F})n \right)\\
% \det(\mathcal{J}(h_D^*, f_W^*)) &= \dfrac{f_W}{(1 + f_W)^2} \left(\dfrac{d + ib}{k_F} + 2\dfrac{d + ib}{k_F}f_W + \dfrac{d + ib}{k_F}f_W^2  - \dfrac{nf_W}{k_F} -\dfrac{nf_W^2}{k_F} - ni + n- n\dfrac{f_W}{k_F} \right)\\
% \det(\mathcal{J}(h_D^*, f_W^*)) &= \dfrac{f_W}{(1 + f_W)^2} \left(\dfrac{d + ib}{k_F} + 2\dfrac{d + ib - n}{k_F}f_W + \dfrac{d + ib -n}{k_F}f_W^2  - ni + n \right)\\
% \det(\mathcal{J}(h_D^*, f_W^*)) &= \dfrac{f_W}{(1 + f_W)^2} \left( \Big(\dfrac{d + ib}{k_F} + n + 2\dfrac{d + ib - n}{k_F}f_W\Big) + \dfrac{d + ib -n}{k_F}f_W^2  - ni \right)\\
% \det(\mathcal{J}(h_D^*, f_W^*)) &= \dfrac{f_W}{(1 + f_W)^2} \left( \Big(\dfrac{d + ib}{k_F} + n -bi - d + 2\dfrac{d + ib - n}{k_F}f_W\Big) + \dfrac{d + ib -n}{k_F}f_W^2  - ni +bi + d \right)\\
% \det(\mathcal{J}(h_D^*, f_W^*)) &= \dfrac{f_W}{(1 + f_W)^2} \left( - \Big(2\dfrac{n - bi -d}{k_F}f_W - (n - bi -d + \dfrac{d + ib}{k_F}) \Big) - \dfrac{n - bi - d}{k_F}f_W^2  - ni +bi + d \right) \\
% \det(\mathcal{J}(h_D^*, f_W^*)) &= \dfrac{f_W}{(1 + f_W)^2} \left( - P_F'(f_W^*) - \dfrac{n - bi - d}{k_F}(f_W^*)^2  - ni +bi + d \right)
% \end{align*}

% The sign of $\det(\mathcal{J}(h_D^*, f_W^*))$ will depend on the sign of $em - \theta(\mu_D -f_D) - \theta m \beta I \equiv n - bi - d$.

% \begin{itemize}
% \item When $n - bi - d > 0$, the equilibrium exists if $- ni +bi + d > 0$. We know that $P_f$ is decreasing on $(-\infty, f_W^*]$. Therefore $- P_F'(f_W^*) > 0$. Moreover, we have
% $$f_W^* = \dfrac{(n - bi - d + \dfrac{d + bi}{k_F})}{2 (n - bi - d)} - \dfrac{\sqrt{(n - bi - d + \dfrac{d + bi}{k_F})^2 - 4(n - bi - d)(ni +bi + d) }}{2 (n - bi - d)}$$ 
% which gives $- \dfrac{n - bi - d}{k_F}(f_W^*)^2  - ni +bi + d > 0$.

% Consequently, $\det(\mathcal{J}(h_D^*, f_W^*)) > 0$.

% \item 

% \end{itemize}




\bibliographystyle{plain}
\bibliography{Biblio/Math, Biblio/Context, Biblio/interactionsHumanEnvironmentModel}

\end{document}

