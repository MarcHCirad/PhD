\documentclass{article}
\usepackage{graphicx} 
\usepackage{color}
\usepackage{amsfonts,amsmath}
\usepackage{amsthm}
\usepackage{empheq}
\usepackage{mathtools}
\usepackage{multirow}
\usepackage{tikz}
\usepackage{titlesec}
\usepackage{caption}
\usepackage{lscape}
\usepackage{graphicx}
\captionsetup{justification=justified}
\usepackage[toc,page]{appendix}
%\usepackage{hyperref}
\usepackage{subcaption}
\usepackage{bm}

\textheight240mm \voffset-23mm \textwidth160mm \hoffset-20mm

\graphicspath{{./Images/}{./Images/ComparisonBifurcationFAHA}}

\setcounter{secnumdepth}{4}
\titleformat{\paragraph}
{\normalfont\normalsize\bfseries}{\theparagraph}{1em}{}
\titlespacing*{\paragraph}
{0pt}{3.25ex plus 1ex minus .2ex}{1.5ex plus .2ex}

\newcommand{\marc}[1]{\textcolor{teal}{#1}}
\newcommand{\YD}[1]{\textcolor{magenta}{#1}}
\newcommand{\VY}[1]{\textcolor{blue}{#1}}

\DeclareMathOperator{\Tr}{Tr}
\newcommand{\lfd}{\lambda_{F, D}}
\newcommand{\lfw}{\lambda_{F}}
\newcommand{\Kfa}{K_{F,\alpha}}
\newcommand{\cI}{\mathcal{I}}
\newcommand*\phantomrel[1]{\mathrel{\phantom{#1}}}

\title{Paramètres Thèse Marc}
\author{Marc Hétier, Yves Dumont  and Valaire Yatat-Djeumen}

\begin{document}

\maketitle
\section{Model}

\begin{equation}
\def\arraystretch{2}
\left\{ 
\begin{array}{l}
\dfrac{dH_D}{dt}= \cI + e\lfw H_W F_W + (f_D - \mu_D) H_D - m_D H_D + m_W H_W. \\
\dfrac{dF_W}{dt} = r_F(1- \alpha) (1+ \beta H_W) \left(1 - \dfrac{F_W}{K_F(1-\alpha)} \right) F_W - \lfw F_W H_W \\
\dfrac{dH_W}{dt}= m_D H_D - m_W H_W 
\end{array} \right.
\label{equationsHDFWHW}
\end{equation}



\section{Parameters values}
\begin{table}[ht]
\begin{tabular}{|c|c|c|}
\hline 
Parameter & Description & Unit \\ 
\hline \hline
%$t$ & Time & Year \\
%\hline
%$H_D$ & Humans in the domestic area & Ind \\
%$H_W$ & Humans in the wild area & Ind \\
%$F_W$ & Wild fauna & Ind \\
%\hline
$e$ & Prey-food conversion & - \\
$f_D$ & Food produced by human population & Year$^{-1}$ \\
$\mu_D$ & Human mortality rate  & Year$^{-1}$ \\
$m_D$ & Displacement from domestic area to wild area & Year$^{-1}$ \\
$m_W$ & Displacement from wild area to domestic area & Year$^{-1}$ \\
$r_F$ & Wild animal growth rate & Year$^{-1}$ \\
$K_F$ & Carrying capacity for wild fauna, fixed by the environment& Ind \\
$\alpha$ & Proportion of anthropized environment & - \\
$\beta$ & Positive impact from human activities to animal growth rate & Ind$^{-1}$  \\
$\lfw$ & Hunting rate & Ind/Year\\
$\mathcal{I}$ & Immigration rate &Ind/Year\\
\hline
\end{tabular}
\caption{Parameters and variables of the model}
\end{table}

\begin{table}[ht]
\centering
\begin{tabular}{|c|c|c|c|c|}
\hline 
Param. & Description & Unit & Value & Reference \\ 
\hline 
$e$ & Prey-food conversion &- & Unit Order & Assumed\\
$f_D$ & Food produced&  Year$^-1$ &0.0164 & \cite{koppert_consommation_1996}\\
$\mu_D$ & Human mortality rate& Year$^-1$  & $1/50$ & \cite{ins_demographie}\\
$m_D$ & Displacement from D to W &Year$^-1$  &$7.83$ & \cite{avila_interpreting_2019}\\
$m_W$ &Displacement from W to D  &Year$^-1$  &24.3 & \cite{avila_interpreting_2019}\\
$r_F$ & Wild animal growth rate& Year$^-1$ & $3.35$ & \cite{robinson_intrinsic_1986}\\
$K_F$ & Carrying capacity for $F_W$ & Ind& 22725 & \cite{janson_ecological_1990} \\
\textcolor{red}{$\bm{\alpha}$}  & \textcolor{red}{\textbf{Prop. of anthropized environment}} &-&  $[0, 1)$ & Param of interest; varies \\
$\beta$ & Positive impact& Ind$^-1$ & $\in [0, \beta^*)$ &  \\
\textcolor{red}{$\bm{\lfw}$} & \textcolor{red}{\textbf{Hunting rate}} & Ind Year$^-1$ & - & Param of interest; varies \\
$\mathcal{I}$ &  Immigration rate& Ind Year$^-1$ & & Varies \\
\hline
\end{tabular}

\caption{Parameters values}
\end{table}


%\begin{table}[ht]
%\centering
%\begin{tabular}{|c|c|c|}
%\hline 
%Parameter & Value & Reference \\ 
%\hline 
%$e$ & & Assumed\\
%$f_D$ & 0.0137 & \cite{koppert_consommation_1996}\\
%$\mu_D$ & $1/60$ & \cite{ins_demographie}\\
%$m_D$ & $0.483$ & \cite{avila_interpreting_2019}\\
%$m_W$ & 24.3 & \cite{avila_interpreting_2019}\\
%$r_F$ & $0.68$ & \cite{robinson_intrinsic_1986}\\
%$K_F$ & 22725 & \cite{janson_ecological_1990} \\
%$\alpha$ & $[0, 1)$ & parameter of interest; varies \\
%$\beta$ & $\in [0, \beta^*)$ &  \\
%$\lfw$ & - & parameter of interest; varies \\
%$\mathcal{I}$ &  & \\
%\hline
%\end{tabular}
%
%\caption{Parameters values}
%\end{table}


\begin{table}[ht]
\centering
\begin{tabular}{|c|c|c|c|}
\hline 
Parameter & Unit & Value & Reference \\ 
\hline 
$e$ & - & Unit Order & Assumed\\
$f_D$ & Year$^-1$ &0.0164 & \cite{koppert_consommation_1996}\\
$\mu_D$ & Year$^-1$  & $1/50$ & \cite{ins_demographie}\\
$m_D$ & Year$^-1$  &$7.83$ & \cite{avila_interpreting_2019}\\
$m_W$ &Year$^-1$  &24.3 & \cite{avila_interpreting_2019}\\
$r_F$ & Year$^-1$ & $3.35$ & \cite{robinson_intrinsic_1986}\\
$K_F$ & Ind& 22725 & \cite{janson_ecological_1990} \\
$\alpha$ &-&  $[0, 1)$ & Parameter of interest; varies \\
$\beta$ & Ind$^-1$ & $\in [0, \beta^*)$ &  \\
$\lfw$ & Ind Year$^-1$ & - & Parameter of interest; varies \\
$\mathcal{I}$ &  Ind Year$^-1$ & & Varies \\
\hline
\end{tabular}

\caption{Parameters values}
\end{table}

\subsection{Complementary explication}
\begin{itemize}

\item Parameter $\mu_D$ : Life expectancy in Cameroon is around 60 years, so we just take $\mu_D = \dfrac{1}{60}$

\item Quantity $1/m_W$ corresponds to the average time spend in the wild area. Several articles record it, \cite{avila_interpreting_2019, jones_incentives_2019, jones_consequences_2020}. Authors of \cite{avila_interpreting_2019} talks about 4.6 hours per week (10 days per year), the ones of \cite{jones_incentives_2019} about 25 days per year, and the ones of \cite{jones_consequences_2020} about 5 days per year. We use a value of 15 days per year, giving $m_W = \dfrac{365}{15} = 24.3$

\item Parameter $m_D$ is estimated using the relation $m = \dfrac{H_W^*}{H_D^*}$ and $m_D = m \times m_W$. Let recall that $H_W^*$ and $H_D^*$ are not precisely the number of hunter and the number of non-hunter; they are the number of villagers (hunters) present in the wild area and the the number of villagers (hunter and non-hunter) present in the village. 
Authors of \cite{avila_interpreting_2019} registered the total populations $H^*$ of three villages in the Dja area (Southeast Cameroon), as well as the number of hunters $H_{hunter}$ for 2002, 2009 and 2015. The figures are given below:

\begin{table}[ht]
\centering
\begin{tabular}{c|ccc|ccc|ccc|ccc}
& \multicolumn{12}{c}{Village/year} \\
& \multicolumn{3}{c|}{Duomo Pierre} & \multicolumn{3}{c|}{Malen V} & \multicolumn{3}{c|}{Mimpala} & \multicolumn{3}{c}{Total} \\
& 2003 & 2009 & 2016  & 2003 & 2009 & 2016 & 2003 & 2009 & 2016 & 2003 & 2009 & 2016 \\
Total Village population size & 82 & 71 &85&143&163&152&98&81&71&323&315&308 \\
Total Number of Hunter & 12 & 18 & 29& 23&38&36&18&27&18&53&79&82
\end{tabular}
\end{table}

To ``convert" the previous data into $H_D^*$ and $H_W^*$, we divide the number of hunters by the time they spend in the forest (15 days per year). Therefore, we estimate $m$ by:

\begin{align*}
m = \dfrac{H_D^*}{H_W^*} \simeq \dfrac{H_{hunter} \dfrac{1}{15}}{H_{non hunter}}
\end{align*}

For 2003, we obtain $m \simeq = \dfrac{53\dfrac{1}{15}}{(323-53)} \simeq 0.013$ and therefore $m_D \simeq 0.31$. 
Over the three years, the average is $m_D = 0.483$.

\item For parameter $f_D$: Assuming that the population describes in \cite{koppert_consommation_1996} corresponds to the equilibrium of coexistence with no immigration $EE^{HF_W}_{\cI = 0}$, we follow the below reasoning.

The ratio $r$ between the food locally produced (quantity $f_D H_D^*$) and the resources coming from the hunt ($e \lfw F_W^* H_W^*$) can be estimated from data from \cite{koppert_consommation_1996}. Moreover, it writes:

\begin{align*}
r :=&  \dfrac{f_D H_D^*}{e \lfw F_W^* H_W^*} \\
=& \dfrac{f_D}{e \lfw \times \dfrac{\mu_D - f_D}{e m \lfw}} \times \dfrac{1}{m}\\
=& \dfrac{f_D}{\mu_D - f_D}
\end{align*}

Therefore, $f_D = \dfrac{r}{1 + r} \mu_D$ and we do have $f_D < \mu_D$, which is conformed to the model hypothesis.

We consider that the resources coming from the hunt correspond to the bush meat consumed and the imported food (obtained by selling bushmeat). According to \cite{koppert_consommation_1996}, this represents 18\% (resp 17\%; 17\%) of the calories by the Yassa (resp Mvae; Bakola). The local production furnishes 82\% (resp 81\%; 81\%) of the calories.

Therefore, we have $r = 4.56$ (resp $=4.76$; $=4.76$)using the calories data.

Using previously found value of $\mu_D$, we obtain $f_D = 0.0137$ (resp. $=0.0138$; $=0.0138$). Let recall that $\mu_D \simeq 0.017$.

 \begin{figure}[ht]
 \centering
 \includegraphics[width=0.8\textwidth]{DiagConsommation.png}
 \end{figure}

\item Parameter $r_F$ is estimated using data from \cite{robinson_intrinsic_1986}. This article provides intrinsic growth rates for 40 mammals of neo-tropics, \marc{(including non-african species that I did not remove yet)}. The mean over the 40 input is $r_F = 0.68$, the standard deviation is $1.78$.

\item Parameter $K_F$ is estimated using data from \cite{janson_ecological_1990}. The authors provide density of non flying mammals per km$^2$. The densities were measured in a biological station considered as an intact area. I sum up all the densities provided, assuming that all the species were co-living. It gives a density of $909$ animals per km$^2$. \marc{I think that we should multiply this number by the surface of the hunting area. Int the article \cite{avila_interpreting_2019}, the authors estimate that the overall hunting area for three villages (of Southeast Cameroon) was about 45-110 km$^2$, depending on the year. Taking an average area of $75$km$^2$ and dividing by 3 (three villages), this gives $K_F = 22725$}.




\end{itemize}



\newpage
\bibliographystyle{plain}
\bibliography{Biblio/Math, Biblio/Context, Biblio/WildArea}


\end{document}