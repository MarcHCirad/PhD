\documentclass{article}
\usepackage{authblk}
\usepackage{graphicx,ulem} 
\usepackage{color}
\usepackage{amsfonts,amsmath}
\usepackage{amsthm}
\usepackage{empheq}
\usepackage{mathtools}
\usepackage{multirow}
%\usepackage{tikz}
\usepackage{titlesec}
\usepackage{caption}
%\usepackage{lscape}
\usepackage{graphicx}
\captionsetup{justification=justified}
\usepackage[toc,page]{appendix}
\usepackage{hyperref}
\usepackage{subcaption}
\usepackage{pdftricks}
\usepackage[dvipsnames]{xcolor}
\begin{psinputs}
\usepackage{amsfonts,amsmath}
	\usepackage{pstricks-add}
   \usepackage{pstricks, pst-node}
   \usepackage{multido}
   \newcommand{\lfw}{\lambda_{F}}
\end{psinputs}

\textheight240mm \voffset-23mm \textwidth160mm \hoffset-20mm

 \graphicspath{{./Images/}{../Schema}{./Images/HT1/}}
%\graphicspath{{Figures}}
\setcounter{secnumdepth}{4}
\titleformat{\paragraph}
{\normalfont\normalsize\bfseries}{\theparagraph}{1em}{}
\titlespacing*{\paragraph}
{0pt}{3.25ex plus 1ex minus .2ex}{1.5ex plus .2ex}

\newcommand{\lfd}{\lambda_{F, D}}
\newcommand{\lfw}{\lambda_{F}}
\newcommand{\Kfa}{K_{F,\alpha}}
\newcommand{\cI}{\mathcal{I}}
\newcommand{\mW}{\tilde{m}_W}
\newcommand{\mD}{\tilde{m}_D}
\newcommand{\R}{\mathbb{R}}
\newcommand{\N}{\mathcal{N}}

\newcommand{\marc}[1]{\textcolor{teal}{#1}}
\newcommand{\YD}[1]{\textcolor{magenta}{#1}}
\newcommand{\VY}[1]{\textcolor{blue}{#1}}
\newcommand{\vdeux}[1]{\textcolor{orange}{#1}}
\newcommand{\vtrois}[1]{\textcolor{OliveGreen}{#1}}

\DeclareMathOperator{\Tr}{Tr}
\newtheorem{theorem}{Theorem}
\newtheorem{prop}[theorem]{Proposition}
\theoremstyle{definition}
\newtheorem{definition}[theorem]{Definition}
\theoremstyle{remark}
\newtheorem{remark}[theorem]{Remark}
\newtheorem{cor}[theorem]{Corollary}
\newcommand*\phantomrel[1]{\mathrel{\phantom{#1}}}

\title{Human-environment interactions in a tropical forest context: modeling, analysis and simulations}
\author[1,2,3]{Yves Dumont} 
\author[2]{Marc Hétier}
\author[2,4]{Valaire Yatat-Djeumen}

\affil[1]{CIRAD, UMR AMAP, 3P, F-97410 Saint Pierre, France}
\affil[2]{AMAP, Univ Montpellier, CIRAD, CNRS, INRA, IRD, Montpellier, France}
\affil[3]{University of Pretoria, Department of Mathematics and Applied Mathematics, Pretoria, South Africa}
\affil[4]{ENSPY, University of Yaoundé 1, Cameroon}
\begin{document}

\maketitle

\section{Evolution des cycles limites lorsque $\theta > 0$}

Valeurs des paramètres utilisées:
\begin{itemize}
\item $r_F = 3.75$
\item $K_F = 7250$
\item $\alpha = 0.5$
\item $\beta= 0.0$
\item $\lfw$ varies
\item $e=0.2$
\item $\cI=0.0$
\item $\theta$ varies
\item $\mu_D=0.02$
\item $f_D=0.00$
\item $m_D=2.10$
\item $m_W=7.30$
\end{itemize}

A chaque fois, le système est dans une configuration de cycle limite si $\lfw > \lfw^{max}$ avec:

\begin{multline*}
\lambda^{max}_{F, \theta = 0} := \left[m_{W}(\mu_{D}-f_{D})+\big(\mu_{D}-f_{D}+m_{D}+m_{W})^{2}\right] \times \\
 \dfrac{\left(1+\sqrt{1+4\dfrac{(1-\alpha)m_{W}r_{F}\left(\mu_{D}-f_{D}\right)\big(\mu_{D}-f_{D}+m_{D}+m_{W})}{\left[m_{W}\dfrac{\mu_{D}-f_{D}}{e}+\big(\mu_{D}-f_{D}+m_{D}+m_{W})^{2}\right]^{2}}}\right)}{2em_D (1-\alpha) K_F }.
\end{multline*}
et
$$
\lambda^{max}_{F, \theta > 0} =  \dfrac{me - \theta (\mu_D - f_D)}{me + \theta(\mu_D - f_D)}\dfrac{1}{\theta K_F(1-\alpha)}
$$

Prendre  $\lfw >> \lfw^{max}$ conduit à des problèmes de convergence numérique, sûrement du au fait que $F_W$ décroit très rapidement. Dans la suite, j'ai donc adapté la valeur de $\lfw $ pour rester proche de la valeur $\lfw^{max}$.

Pour:
\begin{itemize}
\item $\theta =0$, $\lfw^{max} = 0.0584$, $\lfw = 0.06$
\item $\theta =0.005$, $\lfw^{max} = 0.0553$, $\lfw = 0.06$
\item $\theta =0.1$, $\lfw^{max} = 0.00295$, $\lfw = 0.06$ puis $\lfw = 0.003$
\item $\theta = 1$, $\lfw^{max} = 0.00057$, $\lfw = 0.0006$
\end{itemize}

\begin{figure}[!ht]
\centering
\includegraphics[width=1\textwidth]{LCgradientTheta.png}
\end{figure}

\newpage

\subsection{$\theta = 0$}

\begin{figure}[!ht]
\centering
\includegraphics[width=1\textwidth]{LCTheta0.png}
\end{figure}

\begin{figure}[!ht]
\centering
\includegraphics[width=1\textwidth]{LCTheta01D.png}
\caption{$\theta = 0$, $\lfw = 0.06$, $\lfw^{max} = 0.0584$}
\end{figure}

\newpage

\subsection{$\theta = 0.005$}

\begin{figure}[!ht]
\centering
\includegraphics[width=1\textwidth]{LCTheta005.png}
\end{figure}

\begin{figure}[!ht]
\centering
\includegraphics[width=1\textwidth]{LCTheta0051D.png}
\caption{$\theta = 0.005$, $\lfw = 0.06$, $\lfw^{max} = 0.0553$}
\end{figure}

\newpage

\subsection{$\theta = 0.1$}

\begin{figure}[!ht]
\centering
\includegraphics[width=1\textwidth]{LCTheta01.png}
\end{figure}

\begin{figure}[!ht]
\centering
\includegraphics[width=1\textwidth]{LCTheta011D.png}
\caption{$\theta = 0.01$, $\lfw = 0.06$, $\lfw^{max} = 0.00295$}
\end{figure}

\newpage
\begin{figure}[!ht]
\centering
\includegraphics[width=1\textwidth]{LCTheta01B.png}
\end{figure}

\begin{figure}[!ht]
\centering
\includegraphics[width=1\textwidth]{LCTheta01B1D.png}
\caption{$\theta = 0.01$, $\lfw = 0.003$, $\lfw^{max} = 0.00295$}
\end{figure}


\newpage
\subsection{$\theta = 1$}

\begin{figure}[!ht]
\centering
\includegraphics[width=1\textwidth]{LCTheta1.png}
\end{figure}

\begin{figure}[!ht]
\centering
\includegraphics[width=1\textwidth]{LCTheta11D.png}
\caption{$\theta = 1$, $\lfw = 0.0006$, $\lfw^{max} = 0.00057$}
\end{figure}

\end{document}
