\documentclass{article}
\usepackage{graphicx} 
\usepackage{color}
\usepackage{amsfonts,amsmath}
\usepackage{amsthm}
\usepackage{empheq}
\usepackage{mathtools}
\usepackage{multirow}
\usepackage{tikz}
\usepackage{titlesec}
\usepackage{caption}
\usepackage{lscape}
\captionsetup{justification=justified}
\usepackage[toc,page]{appendix}

\textheight240mm \voffset-23mm \textwidth160mm \hoffset-20mm

\setcounter{secnumdepth}{4}
\titleformat{\paragraph}
{\normalfont\normalsize\bfseries}{\theparagraph}{1em}{}
\titlespacing*{\paragraph}
{0pt}{3.25ex plus 1ex minus .2ex}{1.5ex plus .2ex}

\newtheorem*{theo}{Theorem}
\newcommand{\yves}{\textcolor{magenta}}
\newcommand{\gf}{(r_F - \omega f - \mu_F)}
\newcommand{\gv}{(r_V- \mu_V)}
\newcommand{\lf}{\lambda_{FH}}
\newcommand{\lv}{\lambda_{VH}}
\newcommand{\RV}{R_0^V}
\newcommand{\RF}{R_0^F}
\newcommand{\NF}{\mathcal{N}_0^F}
\newcommand{\NV}{\mathcal{N}_0^V}
\newcommand{\NH}{\mathcal{N}_0^H}
\newcommand{\Fbeta}{F^*_\beta}
\newcommand{\Hbeta}{H^*_\beta}
\newcommand{\Vbeta}{V^*_\beta}
\newcommand{\VbetaF}{V^*_{\Fbeta, \beta}}
\newcommand{\FHterme}{\omega f + \lf}
\newcommand*\phantomrel[1]{\mathrel{\phantom{#1}}}

\title{Suivi Thèse Marc}
\author{Marc Hétier, Yves Dumont  and Valaire Yatat-Djeumen}

\begin{document}

\maketitle

This model tries to take into account the interactions between human population ($H$), wild fauna and flora ($F$), and human-driven vegetation ($V$). 
We assume that human pick up resources from wild and human-driven vegetation. The rate per person of this collect is assumed to be decreasing with the quantity of resources.
Moreover, human-driven vegetation growth is depended on both services render by human (culture, watering...) and wild vegetation (pollination, moisture retention, wind protection). 
Deforestation by fire happen with intensity $\omega$ and frequency $f(H)$. Assuming that the combustible (dry grass) is available in enough quantity for a good fire's spatial dispersion, we can take $f(H) = fH$. Also, note that in the context of this study, fire is mainly used for hunting activities or to create a agricultural field. In both cases, the frequency and the intensity of those fires are low.
Conversely, wild biomass is a competitor for human driven vegetation.

Human maximum capacity is assumed to depend on resources (wild and human driven). This assumption is justified in \cite{fanta_equilibrium_2018}, used in \cite{bengochea_paz_agricultural_2020}.
Moreover, the human population is assumed have a reduced growth rate for low population size. This corresponds to a weak Allee effect. Some studies have already propose such phenomenon on human population ; see \cite{hamilton_human_2012} and \cite{vaesen_inbreeding_2019}.

All thus assumption lead to the following model :

\begin{equation}    
\left\{ \begin{array}{l}
\dfrac{dF}{dt}=r_{F}\left(1-\dfrac{F}{K_{F}}\right)F-\omega f H F - \mu_F F -\lambda_{FH}g_F(F,V)F H \\
\dfrac{dV}{dt}=r_V \dfrac{H}{H + L_V} s(F) \left(1-\dfrac{V}{K_{V}}\right)V -\alpha FV-\mu_V V -\lambda_{VH} g_V(F,V) V H\\
\dfrac{dH}{dt}=r_H \left(1-\dfrac{H}{aF + bV + c} \right) (\dfrac{H}{\beta} - 1) H
\end{array}\right.
\label{model}
\end{equation}
where both $g_F$ and $g_V$ are decreasing in both $F$ and $V$.

\section{Simplified model}
First we can assume that $g_F = g_V = 1$ and that $s(F) = 1$. The model becomes :
\begin{equation}    
\left\{ \begin{array}{l}
\dfrac{dF}{dt}=r_{F}\left(1-\dfrac{F}{K_{F}}\right)F-\omega f H F - \mu_F F -\lf F H \\
\dfrac{dV}{dt}=r_V \dfrac{H}{H + L_V} \left(1-\dfrac{V}{K_{V}}\right)V -\alpha FV-\mu_V V -\lv V H\\
\dfrac{dH}{dt}=r_H \left(1-\dfrac{H}{aF + bV + c} \right)  (\dfrac{H}{\beta} - 1) H
\end{array}\right.
\label{model:simplified}
\end{equation}

\subsection{Sub-model $F-H$}
First, we can analyze the sub-model with variables $F$ and $H$ only. The model is:

\begin{equation}    
\left\{ \begin{array}{l}
\dfrac{dF}{dt}=r_{F}\left(1-\dfrac{F}{K_{F}}\right)F-\omega f H F - \mu_F F -\lf F H = f_{1, FH} (F,H) \\
\dfrac{dH}{dt}=r_H \left(1-\dfrac{H}{aF + c} \right)  (\dfrac{H}{\beta} - 1) H = f_{2,FH}(F,H)
\end{array}\right.
\label{model:submodelFH}
\end{equation}

\subsubsection{Equilibrium}
Equilibrium are obtained when the right hand side of the previous system is equal at 0. They are :
\begin{itemize}
\item $TE = (0,0)$
\item A forest equilibrium $EE^F = \Big(F^*, 0 \Big) $ where $F^* = K_F \Big(1 - \dfrac{\mu_F}{r_F}\Big)$. It exists when $\NF = \dfrac{r_F}{\mu_F} > 1$.
\item A human equilibrium $EE^H_\beta = \Big(0,\beta \Big)$
\item A forest-human equilibrium $EE^{FH}_\beta = \Big(F^*_\beta, \beta\Big)$ where $F^*_\beta = K_F \Big(1-\dfrac{\mu_F + \beta(\lf + \omega f)}{r_F} \Big)$.
This equilibrium exists when $\dfrac{r_F}{\mu_F + \beta(\lf + \omega f)} > 1$.
\item A human equilibrium $EE^H = \Big(0,c \Big)$
\item A forest-human equilibrium $EE^{FH} = \Big(F^*_{c}, aF^*_{FH} + c \Big)$ where $F^*_{FH} = K_F \left( \dfrac{1 - \dfrac{\mu_F}{r_F} - c \dfrac{\omega f + \lf}{r_F}}{1 + a K_F \dfrac{\omega f + \lf}{r_F}} \right)$. This equilibrium exists when $\dfrac{r_F}{\mu_F + c(\lf + \omega f)} > 1$.
\end{itemize}

\subsubsection{Stability}
The Jacobian matrix of the system is defined by :
\begin{equation}
\mathcal{J}(F,H) =  \begin{bmatrix}
r_F \Big(1-\dfrac{2F}{K_F} \Big) - \mu_F - (\lf + \omega f)H & -(\lf + \omega f)F \\
r_H \dfrac{aH^2}{(aF+c)^2} (\dfrac{H}{\beta}-1) & r_H(1-\dfrac{H}{aF+c})(\dfrac{2H}{\beta}-1) - \dfrac{r_H H}{aF+c}(\dfrac{H}{\beta}-1)
\end{bmatrix}
\label{stab:jacobianFH}
\end{equation}

To assess the stability of the different equilibrium, we look at the eigenvalues of the Jacobian.

\begin{itemize}
\item At point $TE = (0,0, 0)$, the Jacobian is
\begin{equation}
\mathcal{J}(0,0) = \begin{bmatrix}
r_F - \mu_F & 0 \\
0 & -r_H
\end{bmatrix}
\end{equation}
and thus the eigenvalues are $r_F - \mu_F$ and $-r_H$. $TE$ is asymptotically stable (AS) if $\NF < 1$.

\item At point $EE^{F} = (F^*, 0)$, the Jacobian is 
\begin{equation}
\mathcal{J}(F^*, 0, 0)
\begin{bmatrix}
-r_F \dfrac{F^F}{K_F} & -(\FHterme)F^F \\
0 & -r_H 
\end{bmatrix}
\end{equation}
and thus the eigenvalues are all negative. $EE^F$ is asymptotically stable.

\item At point $EE^H_\beta = \Big(0,\beta \Big)$, the Jacobian is
\begin{equation}
\mathcal{J}(0, \beta) = \begin{bmatrix}
r_F - \mu_F - (\FHterme) \beta & 0 \\
0  & r_H (1 - \dfrac{\beta}{c})
\end{bmatrix}
\end{equation}
$EE^H_\beta$ is AS if 

\begin{subequations}
    \begin{empheq}[left={\empheqlbrace\,}]{align}
    & \dfrac{r_F}{\mu_F + \beta (\FHterme)} < 1 \\
    & \dfrac{c}{\beta} < 1
    \end{empheq}
\end{subequations}

\item At point $EE^{FH}_\beta = \Big(F^*_\beta,\beta \Big)$, the Jacobian is
\begin{equation}
\mathcal{J}(F^*_\beta, \beta) = \begin{bmatrix}
-r_F \dfrac{F^*_\beta}{K_F} &  -(\FHterme) F^*_\beta \\
0 & r_H (1 - \dfrac{H^*_\beta}{c})
\end{bmatrix}
\end{equation}
$EE^{FH}_\beta$ is AS if $\dfrac{c}{\beta} < 1$

\item At point $EE^{H} = \Big(0,c\Big)$, the Jacobian is
\begin{equation}
\mathcal{J}(0,c) = \begin{bmatrix}
r_F - \mu_F - c(\FHterme) & 0 \\
r_H a (\dfrac{c}{\beta} - 1) & -r_H(\dfrac{c}{\beta} - 1)
\end{bmatrix}
\end{equation}
$EE^{H}$ is AS if
\begin{subequations}
    \begin{empheq}[left={\empheqlbrace\,}]{align}
    &\dfrac{r_F}{\mu_F + c (\FHterme)} < 1 \\
    & 1< \dfrac{c}{\beta} 
    \end{empheq}
\end{subequations}

\item At point $EE^{FH} = \Big(F^*_{FH}, aF^*_{FH}+c\Big)$, the Jacobian is
\begin{equation}
\mathcal{J}(F^*_{FH}, aF^*_{FH}+c) = \begin{bmatrix}
-r_F \dfrac{F^*_{FH}}{K_F} & -(\FHterme) F^*_{FH} \\
r_H a (\dfrac{aF^*_{FH}+c}{\beta} - 1) & -r_H(\dfrac{aF^*_{FH}+c}{\beta} - 1)
\end{bmatrix}
\end{equation}
Eigenvalues are the roots of 

\begin{multline}
\chi_{FH} = X^2 + \Big(r_F \dfrac{F^*_{FH}}{K_F} + r_H \big(\dfrac{aF^*_{FH}+c}{\beta} - 1\big) \Big)X + \\ r_F  \dfrac{F^*_{FH}}{K_F}r_H \big(\dfrac{aF^*_{FH}+c}{\beta} - 1\big) + r_H a \big(\dfrac{aF^*_{FH}+c}{\beta} - 1\big) (\FHterme) F^*_{FH}
\end{multline}
The two roots of $\chi_{FH}$ have negative real parts if the coefficients are both positive, which is the case if $aF^*_{FH}+c > \beta$.
So, $EE^{FH}$ is AS if
\begin{equation}
1 < \dfrac{a K_F}{\beta} \left( \dfrac{1 - \dfrac{\mu_F}{r_F} - c \dfrac{\omega f + \lf}{r_F}}{1 + a K_F \dfrac{\omega f + \lf}{r_F}} \right) + \dfrac{c}{\beta}
\end{equation}

\end{itemize}

\subsubsection{Limit Cycle}
We can study the existence of limit cycle for this system. We will consider two subset $\Omega_1 = \{(F,H) \in [0, +\infty) \times [\beta, +\infty)\}$ and $\Omega_2 = \{(F,H) \in [0, +\infty) \times [0, \beta])\}$ . 


On $\Omega_1$, we will use the Bendixson-Dulac theorem (see \cite{farkas_1994_periodic}, page 137). Consider the function

\begin{align}
\phi : (0, +\infty) &\times (\beta, +\infty) \rightarrow \mathbf{R} \label{limit cylce:phiFH}
\\
\nonumber
(F,H) & \mapsto \dfrac{1}{F H (H/\beta - 1)}
\end{align}

Multiplying the right hand side of system $(F,H)$ \eqref{model:submodelFH} by this function, and taking the derivative with respect to $F$ or $H$ we obtain :
\begin{subequations}
\begin{align}
&\dfrac{\partial f_{1,FH} \times \phi}{\partial F}(F,H) = - \dfrac{r_F}{K_F} \dfrac{1}{H \big(\dfrac{H}{\beta}-1 \big)} \\
&\dfrac{\partial f_{2,FH} \times \phi}{\partial H}(F,H) = - \dfrac{r_H}{(aF + c) F}
\end{align}
\end{subequations}

Then, for $(F, H) \in \overset{\circ}{\Omega}_1$
\begin{equation}
\Big(\dfrac{\partial f_{1,FH} \times \phi}{\partial F} + \dfrac{\partial f_{2,FH} \times \phi}{\partial H}\Big) (F, H) < 0
\end{equation}
and, according to Bendixson-Dulac theorem, there is no limit cycle on $\Omega_1$.

On $\Omega_2$, we will distinguish two cases. First, we assume that $\beta < c$ . Then, we have on $\Omega_2$:
\begin{equation}
f_{2,FH}(F,H) = r_H \underset{0<}{\underbrace{\Big(1 - \dfrac{H}{aF + c} \Big)}}\underset{<0}{\underbrace{\Big(\dfrac{H}{\beta} -1\Big)}} H < 0
\end{equation}
since $\beta \leq aF + c$. $H(t)$ is thus decreasing on $\Omega_2$, and there is no limit cycle on $\Omega_2$.
Now, we assume that $c < \beta$. By looking the vector fields in all the different configurations, it is possible to show that there is no limit cycle either.

\subsection{Sub-model $V-H$}
Now, we can analyze the sub-model with variables $V-H$. The model is:
\begin{equation}    
\left\{ \begin{array}{l}
\dfrac{dV}{dt}=r_V \dfrac{H}{H + L_V} \left(1-\dfrac{V}{K_{V}}\right)V -\mu_V V -\lv V H\\
\dfrac{dH}{dt}=r_H \left(1-\dfrac{H}{bV + c} \right)  (\dfrac{H}{\beta} - 1) H
\end{array}\right.
\label{model:submodelVH}
\end{equation}

\subsubsection{Equilibrium}
Equilibrium are obtained when the right hand side of the previous system is equal at 0. They are :
\begin{itemize}
\item $TE = (0,0)$
\item A human equilibrium $EE^H_\beta = \Big(0,\beta \Big)$
\item A vegetation-human equilibrium $EE^{VH}_\beta = \Big(V^*_{\beta}, \beta \Big)$ where $V^*_\beta = K_V \Big(1- \dfrac{\beta + L_V}{\beta} \dfrac{\lv \beta + \mu_V}{r_V} \Big)$. This equilibrium exists when $\dfrac{\beta}{\beta + L_V} \dfrac{r_V}{\lv \beta + \mu_V} > 1$
\item A human equilibrium $EE^H = \Big(0,c\Big)$
\item One or two vegetation-human equilibrium $EE^{VH}_i = \Big(0, V^*_{VH,i}, H^*_{VH,i} \Big)$. $H^*_{c, VH,i}$ are equal at $H^*_{VH, i} = bV^*_{VH, i} + c$ and $V^*_{VH,i}$ are solutions of
%\begin{equation}
%V^2 + V \left(\dfrac{\dfrac{c r_V}{K_V} + \lv b (2c+L_V) + b (\mu_V - r_V)}{b^2\lv + \dfrac{b r_V}{K_V}} \right) + \Big(\dfrac{c+L_V}{c} \dfrac{\mu_V + \lv c}{r_V} - 1\Big)\dfrac{r_V c}{b^2 \lv(1 + \dfrac{r_V}{b K_V \lv})} = 0
%\label{equilibreVH:equationV}
%\end{equation}
%
%\begin{equation}
%V^2\left(b^2 \lv(1 + \dfrac{r_V}{b K_V \lv}) \right) + V \left( b r_V(\dfrac{c}{bK_V} -1)(1 + \dfrac{\lv  (2c+L_V) + \mu_V}{b r_V(\dfrac{c}{bK_V} -1)}) \right) + \Big(\dfrac{c+L_V}{c} \dfrac{\mu_V + \lv c}{r_V} - 1\Big)r_V c = 0
%\label{equilibreVH:equationV}
%\end{equation}
%
%\begin{equation}
%V^2\left(b^2 \lv(1 + \dfrac{r_V}{b K_V \lv}) \right) + V \left( b r_V(\dfrac{c}{bK_V} -1) + \lv  (2c+L_V) + \mu_V \right) + \Big(\dfrac{c+L_V}{c} \dfrac{\mu_V + \lv c}{r_V} - 1\Big)r_V c = 0
%\label{equilibreVH:equationV}
%\end{equation}

\begin{equation}
V^2 + V \left(\dfrac{1}{b} \dfrac{ bK_V\Big(\dfrac{\lv(2c+L_V) + \mu_V}{r_V} - 1\Big) + c}{1 + \dfrac{b K_V \lv}{r_V}}  \right) + \dfrac{K_Vc}{b} \dfrac{\Big(\dfrac{c+L_V}{c} \dfrac{\lv c + \mu_V}{r_V} - 1\Big)}{1 + \dfrac{b K_V \lv}{r_V}} = 0
\label{equilibreVH:equationV}
\end{equation}
The solutions of this equation are real if the discriminant $\Delta_{VH}$ is positive. We have

\begin{multline}
\Delta_{VH} \geq 0 \Leftrightarrow \\
\left( bK_V \Big(\dfrac{\lv(2c+L_V) + \mu_V}{r_V} - 1\Big) + c \right)^2 \geq  4 K_V b \Big(1 + \dfrac{b K_V \lv}{r_V}\Big) \Big((c+L_V) \dfrac{\mu_V + \lv c}{r_V} - c\Big)
\label{equilibreVH:discriminant}
\end{multline}

If the right hand side is negative, \textit{ie} if $1< \dfrac{c}{c+L_V} \dfrac{r_V}{\mu_V + \lv c}$, then the discriminant $\Delta_{VH}$ is positive, and the constant coefficient of \eqref{equilibreVH:equationV} is negative. Thus, there is always one positive roots (and the other is negative).

Otherwise, if $\dfrac{c}{c+L_V} \dfrac{r_V}{\mu_V + \lv c} < 1$ and if the discriminant is positive, the solutions are real and have the same sign, which is positive if the coefficient in $V$ is negative. Thus, in addition of inequality \eqref{equilibreVH:discriminant}, the parameters must verify
\begin{subequations}
\begin{align}
& bK_V \Big(\dfrac{\lv(2c+L_V) + \mu_V}{r_V}-1\Big) + c < 0 \\
& \Leftrightarrow c \Big(1 + \dfrac{2 \lv bK_V}{r_V} \Big) < bK_V \Big(1 - \dfrac{\mu_V + \lv L_V}{r_V} \Big) \\
& \Leftrightarrow 1 < \dfrac{bK_V}{c} \dfrac{1 - \dfrac{\mu_V + \lv L_V}{r_V}}{1 + \dfrac{2 \lv bK_V}{r_V}}
\end{align}
\end{subequations}

in order to have positive solutions.
\end{itemize}

Finally, if
\begin{equation}
1 < \dfrac{c}{c+L_V} \dfrac{r_V}{\mu_V + \lv c}
\label{equilibreVH:conditionExistence1}
\end{equation}
there is one equilibrium, noted $EE^{VH}_2$ ; and if 
{\small
\begin{subequations}
    \begin{empheq}[left={\empheqlbrace\,}]{align}
&\dfrac{c}{c+L_V} \dfrac{r_V}{\mu_V + \lv c} < 1 \\
&1 \leq \dfrac{1}{4K_Vb}\Big(1 + \dfrac{b K_V \lv}{r_V}\Big)^{-1}\Big((c+L_V) \dfrac{\mu_V + \lv c}{r_V} - c\Big)^{-1} \left( bK_V \Big(\dfrac{\lv(2c+L_V) + \mu_V}{r_V} - 1\Big)-1 \right)^2 \\
& 1 < \dfrac{bK_V}{c} \dfrac{1 - \dfrac{\mu_V + \lv L_V}{r_V}}{1 + \dfrac{2 \lv bK_V}{r_V}}
    \end{empheq}
    \label{equilibreVH:conditionExistence2}
\end{subequations}
}
there are two equilibrium $EE^{VH}_1 \leq EE^{VH}_2$.

$V^{*}_{VH,i}$ are equal at:
\begin{equation}
V^{*}_{VH,i} = -\dfrac{1}{2b} \dfrac{ bK_V\Big(\dfrac{\lv(2c+L_V) + \mu_V}{r_V} - 1\Big) + c}{1 + \dfrac{b K_V \lv}{r_V}}  \pm \dfrac{1}{2} \sqrt{\Delta_{VH}}
\label{equilibreVH:V}
\end{equation}
with $\Delta_{VH} = \left(\dfrac{1}{b} \dfrac{ bK_V\Big(\dfrac{\lv(2c+L_V) + \mu_V}{r_V} - 1\Big) + c}{1 + \dfrac{b K_V \lv}{r_V}}\right)^2 - 4 \dfrac{K_Vc}{b} \dfrac{\Big(\dfrac{c+L_V}{c} \dfrac{\lv c + \mu_V}{r_V} - 1\Big)}{1 + \dfrac{b K_V \lv}{r_V}} $

and $H^*_{VH,i}$ are equal at
\begin{equation}
H^*_{VH,i} = b V^*_{VH,i} + c
\label{equilibreVH:H}
\end{equation}

\subsubsection{Stability}
The Jacobian matrix of the system is defined by :
\begin{equation}
\mathcal{J}(V,H) =  \begin{bmatrix}
r_V \dfrac{H}{H+L_V}(1-\dfrac{2V}{K_V}) - \lv H - \mu_V & r_V \dfrac{L_V}{(H+L_V)^2}(1-\dfrac{V}{K_V})V  - \lv V\\
r_H \dfrac{bH^2}{(bV+c)^2} (\dfrac{H}{\beta}-1) & r_H(1-\dfrac{H}{bV+c})(\dfrac{2H}{\beta}-1) - \dfrac{r_H H}{bV+c}(\dfrac{H}{\beta}-1)
\end{bmatrix}
\label{stabilityVH:jacobian}
\end{equation}

To assess the stability of the different equilibrium, we look at the eigenvalues of the Jacobian.

\begin{itemize}
\item At point $TE = (0, 0)$, the Jacobian is
\begin{equation}
\mathcal{J}(0,0) = \begin{bmatrix}
-\mu_V & 0 \\
0 & -r_H
\end{bmatrix}
\end{equation}
and thus the eigenvalues are $-\mu_V$ and $-r_H$. $TE$ is asymptotically stable (AS).

\item At point $EE^H_\beta = \Big(0,\beta \Big)$, the Jacobian is
\begin{equation}
\mathcal{J}(0, \beta) = \begin{bmatrix}
r_V\dfrac{\beta}{\beta+L_V} - \lv \beta - \mu_V & 0 \\
0 & r_H (1 - \dfrac{\beta}{c})
\end{bmatrix}
\end{equation}
$EE^H_\beta$ is AS if 

\begin{subequations}
    \begin{empheq}[left={\empheqlbrace\,}]{align}
    & \dfrac{r_V}{\mu_V + \lv \beta} \dfrac{\beta}{\beta + L_V} < 1  \\
    & \dfrac{c}{\beta} < 1
    \end{empheq}
\end{subequations}

\item At point $EE^{VH}_\beta = \Big(V^*_\beta,\beta \Big)$, the Jacobian is
\begin{equation}
\mathcal{J}(V^*_\beta, \beta) = \begin{bmatrix}
- r_V\dfrac{\beta}{\beta+L_V} V^*_\beta & \_ \\
0 & r_H (1 - \dfrac{\beta}{c})
\end{bmatrix}
\end{equation}
Eigenvalues appear on the diagonal. $EE^{VH}_\beta$ is AS if
\begin{equation}
\dfrac{c}{\beta} <1
\end{equation}

\item At point $EE^{H} = \Big(0,c \Big)$, the Jacobian is
\begin{equation}
\mathcal{J}(0,c) = \begin{bmatrix}
r_V \dfrac{c}{L_V +c} - \lv c - \mu_V & 0 \\
r_H b (\dfrac{c}{\beta} - 1) & -r_H(\dfrac{c}{\beta} - 1)
\end{bmatrix}
\end{equation}
$EE^{H}$ is AS if
\begin{subequations}
    \begin{empheq}[left={\empheqlbrace\,}]{align}
    & \dfrac{r_V}{\mu_V + c \lv} \dfrac{c}{c + L_V} < 1  \\
    & 1< \dfrac{c}{\beta} 
    \end{empheq}
\end{subequations}

\item At point(s) $EE^{VH} = \Big(V^*_{H, c}, b V^*_{H, c} + c)$, the Jacobian is
\begin{equation}
\mathcal{J}(V^*_{H, c}, H^*_{V, c}) = \begin{bmatrix}
-r_V\dfrac{H^*_{V, c}}{H^*_{V, c} + L_V} \dfrac{V^*_{H, c}}{K_V} & \Big(\dfrac{L_V\mu_V}{H^*_{V, c}} - \lv H^*_{V, c}) \dfrac{V^*_{H, c}}{H^*_{V, c} + L_V} \\
r_H b (\dfrac{H^*_{V, c}}{\beta} - 1) & -r_H(\dfrac{H^*_{V, c}}{\beta} - 1)
\end{bmatrix}
\label{stabilityVH:jacobianVH}
\end{equation}
We used $r_V \dfrac{H^*_{V, c}}{H^*_{V, c} + L_V} \Big(1 - \dfrac{V^*_{H, c}}{K_V} \Big)V^*_{H, c} = \lv H^*_{V, c} V^*_{H, c} + \mu_VV^*_{H, c}$ to re-write the second coefficient.
Eigenvalues are the roots of 
\begin{multline}
\chi = X^2 + \Big(r_V  \dfrac{H^*_{V, c}}{H^*_{V, c}+L_V}\dfrac{V^*_{H, c}}{K_V} + r_H(\dfrac{H^*_{V, c}}{\beta} - 1)\Big)X + \\
 (\dfrac{H^*_{V, c}}{\beta} - 1) \left(r_V  \dfrac{H^*_{V, c}}{K_V}  + b \Big(\lv H^*_{V, c} - \dfrac{L_V \mu_V}{H^*_{V, c}} \Big) \right) r_H\dfrac{V^*_{H, c}}{H^*_{V, c} + L_V}
 \label{stabilityVH:chi VH}
\end{multline}

The eigenvalues of \eqref{stabilityVH:jacobianVH} have negative real parts if the coefficients of \eqref{stabilityVH:chi VH} are both positive.

The constant coefficient of $\chi$ can be re-written as :
\begin{subequations}
\begin{align}
\chi_0 &= (\dfrac{H^*_{V, c}}{\beta} - 1) \left(r_V  \dfrac{H^*_{V, c}}{K_V}  + b \Big(\lv H^*_{V, c} - \dfrac{L_V \mu_V}{H^*_{V, c}} \Big) \right) r_H\dfrac{V^*_{H, c}}{H^*_{V, c} + L_V} \\
&= (\dfrac{H^*_{V, c}}{\beta} - 1) \left( \Big(\dfrac{r_V}{K_V} + b\lv\Big) (H^*_{V, c})^2 - b L_V \mu_V \right) r_H\dfrac{V^*_{H, c}}{H^*_{V, c}(H^*_{V, c} + L_V)}
\end{align}
\end{subequations}
and using equation \eqref{equilibreVH:equationV}, we can show that $H^*_{V, c}$ are solutions of
\begin{equation}
H^2 - H \left( \dfrac{r_V(1 + \dfrac{c}{bK_V}) - \lv L_V - \mu_V}{\lv(1 + \dfrac{r_V}{\lv bK_V})} \right) + \dfrac{\mu_V L_V}{\lv(1 + \dfrac{r_V}{bK_V\lv})} = 0
\label{equilibreVH:equation H}
\end{equation}

Since $H^*_{V, c}$ are positive solutions of \eqref{equilibreVH:equation H}, the term in $H$ is negative. Then, by using appendix \ref{appendice:ineq2nd}, we know that the term $\left( \Big(\dfrac{r_V}{K_V} + b\lv\Big) (H^*_{V, c})^2 - b L_V \mu_V \right)$ is positive only for $H^*_{V, c, 2}$ 

So, the constant coefficient of $\chi$ is positive for $H^*_{V, c, 2}$ if $H^*_{V, c, 2} > \beta$, and this condition also ensures that the other coefficient of $\chi$ is positive.

On the other hand, the constant coefficient of $\chi$ is positive for $H^*_{V, c, ii_n}$ if $H^*_{V, c, 1} < \beta$.
Moreover, equilibrium $EE^{VH}$ is characterized by the fact that 
\begin{equation}
f_3(H^*_{V, c, 1},V^*_{H, c, 1}) =  H^*_{V, c, 1} \times \Big(1 - \dfrac{H^*_{V, c, 1}}{bV^*_{H, c, 1} + c} \Big) \times \Big(\dfrac{H^*_{V, c, 1}}{\beta} - 1 \Big) = 0
\end{equation}
Let $H^*_{V, c, 1} < \beta$, and $V = V^*_{H, c, 1}$ being constant. Starting from $H = H^*_{V, c, 1}$ a small variation of $H$  will induce a variation in the same sense for $f_3(H, V^*_{H, c, 1})$.
This lead to the assumption that $EE^{VH}_1$ is not stable.

Finally, we assumed that $EE^{VH}_1$ is never stable, while $EE^{VH}_2$ is AS if

\begin{equation}
1 < \dfrac{bV^*_{H,c,i_p} + c}{\beta}
\end{equation}
\end{itemize}

\subsubsection{Limit cycle}
We will study the existence of limit cycle for system $V-H$ in the same way we did for system $F-H$. Considering again $\Omega_1 = \{(V,H) \in [0, +\infty) \times [\beta, +\infty)\}$ and $\Omega_2 = \{(V,H) \in [0, +\infty) \times [0, \beta])\}$, we can use the function \eqref{limit cylce:phiFH} and the Bendixson-Dulac theorem to show that there is no limit cycle on $\Omega_1$.



\subsection{Overall model}
\subsubsection{Equilibrium}
The equilibrium of the overall model are obtained from the equilibrium of the sub-models, to which we add the endemic equilibrium. We obtain the following list (see above for the existence and values)

\begin{itemize}
\item $TE = (0,0,0)$
\item A forest equilibrium $EE^F = \Big(F^*, 0, 0 \Big) $.
\item A human equilibrium $EE^H_\beta = \Big(0,0,\beta \Big)$
\item A forest-human equilibrium $EE^{FH}_\beta = \Big(F^*_\beta, 0, \beta\Big)$ where $F^*_\beta = K_F \Big(1-\dfrac{\mu_F + \beta(\lf + \omega f)}{r_F} \Big)$.
\item A vegetation-human equilibrium $EE^{VH}_\beta = \Big(0, V^*_{\beta}, \beta \Big)$.
\item An endemic equilibrium $EE_\beta^{FVH} = \Big(F^*_\beta, V^*_{F^*_\beta, \beta}, \beta \Big)$ where $\VbetaF = K_V \Big(1- \dfrac{\beta +L_V}{\beta} \dfrac{\alpha \Fbeta + \lv \beta + \mu_V}{r_V} \Big)$. This equilibrium exists  when $\dfrac{r_F}{\mu_F + \beta(\lf + \omega f)} > 1$ and $\dfrac{\beta}{\beta + L_V} \dfrac{r_V}{\alpha \Fbeta + \lv \beta + \mu_V} > 1$.


\item A human equilibrium $EE^H = \Big(0,0,c\Big)$.
\item A forest-human equilibrium $EE^{FH} = \Big(F^*_{FH}, 0, a F^*_{FH} + c \Big)$ where $F^*_{FH} = K_F \left( \dfrac{1 - \dfrac{\mu_F}{r_F} - c \dfrac{\omega f + \lf}{r_F}}{1 + a K_F \dfrac{\omega f + \lf}{r_F}} \right)$.
\item One or two vegetation-human equilibrium $EE^{VH}_i = \Big(0, V^*_{VH, i}, bV^*_{VH, i}+c \Big)$.
\item A endemic equilibrium $EE^{FVH} = \Big(F^*_{FH} - c_1^FV^*_{FVH}, V^*_{FVH}, aF^*_{FH}+(b-ac_1^F)V^*_{FVH}+c \Big)$ where $c^F_1 = \dfrac{bK_F \dfrac{\FHterme}{r_F} }{1 + aK_F \dfrac{\FHterme}{r_F}}$
and $V^*_{FH, c}$ are solutions of 
\end{itemize}

\newpage
\begin{landscape}

\begin{multline}
V^2 + V \dfrac{ \dfrac{(b-ac_1^F)K_V - (aF^*_{FH} + c)}{(b-ac_1^F)}}{1  +\dfrac{K_V(\lv(b-ac_1^F) - \alpha c_1^F)}{r_V}} \left(\dfrac{(b-ac_1^F)K_V}{(b-ac_1^F)K_V - (aF^*_{FH} + c)}\dfrac{\mu_V + \lv(2(aF^*_{FH} + c) + L_V) + \alpha \Big(F^*_{FH} - \dfrac{c_1^F (aF^*_{FH} + c + L_V)}{b-ac_1^F}\Big)}{r_V} -1 \right)  + \\
\dfrac{K_V(aF_c^* + c)}{b-ac_1^F} \dfrac{\Big(\dfrac{aF^*_{FH} + c + L_V}{aF^*_{FH} + c} \dfrac{\mu_V + \alpha F^*_{FH} + \lv (aF^*_{FH} + c)}{r_V} - 1\Big)}{1 + \dfrac{K_V(\lv(b-ac_1^F) - \alpha c_1^F)}{r_V}} = 0
\label{equilibreFVH:equationV}
\end{multline}

The determinant of this equation, $\Delta_{FVH}$, is positive if
\begin{multline}
\Delta_{FVH} \geq 0 \Leftrightarrow 
\dfrac{\Big((b-ac_1^F)K_V - (aF^*_{FH} + c)\Big)^2}{K_V(aF_c^* + c)(b-ac_1^F)}\left(\dfrac{(b-ac_1^F)K_V}{(b-ac_1^F)K_V - (aF^*_{FH} + c)}\dfrac{\mu_V + \lv(2(aF^*_{FH} + c) + L_V) + \alpha \Big(F^*_{FH} - \dfrac{c_1^F (aF^*_{FH} + c + L_V)}{b-ac_1^F}\Big)}{r_V} -1 \right)^2 \geq \\
4 (1  +\dfrac{K_V(\lv(b-ac_1^F) - \alpha c_1^F)}{r_V})
\Big(\dfrac{aF^*_{FH} + c + L_V}{aF^*_{FH} + c} \dfrac{\mu_V + \alpha F^*_{FH} + \lv (aF^*_{FH} + c)}{r_V} - 1\Big)
\label{equilibreFVH:discriminant}
\end{multline}
\end{landscape}

We can detail some of the quantities which appear on the equation:
\begin{subequations}
\begin{align}
& b - ac_1^F = \dfrac{b}{1 + aK_F \dfrac{\FHterme}{r_F}} > 0 \\
&(1  +\dfrac{K_V(\lv(b-ac_1^F) - \alpha c_1^F)}{r_V}) = 1 + \dfrac{bK_V/r_V}{1 + aK_F \dfrac{\FHterme}{r_F}} \Big(\lv - \alpha K_F \dfrac{\FHterme}{r_F}\Big)
\end{align}
\end{subequations}


If $(1  +\dfrac{K_V(\lv(b-ac_1^F) - \alpha c_1^F)}{r_V})
\Big(\dfrac{aF^*_{FH} + c + L_V}{aF^*_{FH} + c} \dfrac{\mu_V + \alpha F^*_{FH} + \lv (aF^*_{FH} + c)}{r_V} - 1\Big)$ is negative, then \eqref{equilibreFVH:discriminant} is true and the constant coefficient in equation \eqref{equilibreFVH:equationV} is negative. This implies the existence of a unique positive roots, noted $V^*_{FVH, 2}$.

On the other hand, if this quantity is positive, in addition of \eqref{equilibreFVH:discriminant}, the parameters should also verified
\begin{equation}
 1 < \dfrac{(b-ac_1^F)K_V - (aF^*_{FH} + c)}{(b-ac_1^F)K_V}\dfrac{r_V}{\mu_V + \lv(2(aF^*_{FH} + c) + L_V) + \alpha \Big(F^*_{FH} - \dfrac{c_1^F (aF^*_{FH} + c + L_V)}{b-ac_1^F}\Big)}
\end{equation}
for the equation \eqref{equilibreFVH:equationV} has positive roots, noted $V^*_{FVH, 1}$ and $V^*_{FVH, 2}$

A last condition of existence is imposed by $F^*_{FVH}$ which must be positive. It is the case if
\begin{equation}
1 < \dfrac{r_F}{\mu_F + (\FHterme)(bV^*_{FVH} + c)}
\end{equation}

So, if:
\begin{subequations}
    \begin{empheq}[left={\empheqlbrace\,}]{align}
    &(1  +\dfrac{K_V(\lv(b-ac_1^F) - \alpha c_1^F)}{r_V})
\Big(\dfrac{aF^*_{FH} + c + L_V}{aF^*_{FH} + c} \dfrac{\mu_V + \alpha F^*_{FH} + \lv (aF^*_{FH} + c)}{r_V} - 1\Big) <0 \\
&1 < \dfrac{r_F}{\mu_F + (\FHterme)(bV^*_{FVH} + c)} 
    \end{empheq}
\end{subequations}
there is one endemic equilibria. And if
\begin{subequations}
    \begin{empheq}[left={\empheqlbrace\,}]{align}
& 0 < (1  +\dfrac{K_V(\lv(b-ac_1^F) - \alpha c_1^F)}{r_V})
\Big(\dfrac{aF^*_{FH} + c + L_V}{aF^*_{FH} + c} \dfrac{\mu_V + \alpha F^*_{FH} + \lv (aF^*_{FH} + c)}{r_V} - 1\Big) \\
& 0 \leq \Delta_{FVH} \text{ (see equation \eqref{equilibreFVH:discriminant})} \\
& 1 < \dfrac{\dfrac{(b-ac_1^F)K_V - (aF^*_{FH} + c)}{(b-ac_1^F)K_V} r_V}{\mu_V + \lv(2(aF^*_{FH} + c) + L_V) + \alpha \Big(F^*_{FH} - \dfrac{c_1^F (aF^*_{FH} + c + L_V)}{b-ac_1^F}\Big)} \\
&1 < \dfrac{r_F}{\mu_F + (\FHterme)(bV^*_{FVH} + c)} 
    \end{empheq}
\end{subequations}
there is two endemic equilibrium.

\subsubsection{Stability}

The Jacobian matrix of the system is defined by :
{\footnotesize
\begin{multline}
\mathcal{J}(F,V,H) = \\ \begin{bmatrix}
r_F \Big(1-\dfrac{2F}{K_F} \Big) - \mu_F - (\lf + \omega f)H & 0 & -(\lf + \omega f)F \\
-\alpha V & r_V \dfrac{H}{H+L_V}(1-\dfrac{2V}{K_V}) - \alpha F - \lv H - \mu_V & r_V \dfrac{L_V}{(H+L_V)^2}(1-\dfrac{V}{K_V})V  - \lv V\\
r_H \dfrac{aH^2}{(aF+bV+c)^2} (\dfrac{H}{\beta}-1) &r_H \dfrac{bH^2}{(aF+bV+c)^2} (\dfrac{H}{\beta}-1) & r_H(1-\dfrac{H}{aF+bV+c})(\dfrac{2H}{\beta}-1) - \dfrac{r_H H (\dfrac{H}{\beta}-1)}{aF+bV+c}
\end{bmatrix}
\label{stab:jacobian}
\end{multline}
}

To assess the stability of the equilibrium we look at the eigenvalues of the Jacobian. Eigenvalues for all the equilibrium, except for the endemic ones, can be easily computed using previous parts.

\begin{itemize}
\item At point $TE = (0,0, 0)$, the eigenvalues are $r_F - \mu_F$, $-\mu_V$ and $-r_H$. $TE$ is asymptotically stable (AS) if $\NF < 1$.

\item At point $EE^{F} = (F^*, 0, 0)$, 
%the Jacobian is 
%\begin{equation}
%\mathcal{J}(F^F, 0, 0)
%\begin{bmatrix}
%-r_F \dfrac{F^F}{K_F}  & 0 & -(\FHterme)F^F \\
%0 & -\alpha F^F - \mu_V & 0 \\
%0 & 0 & -r_H 
%\end{bmatrix}
%\end{equation}
%and thus 
the eigenvalues are all negative. $EE^F$ is asymptotically stable.

\item $EE^H_\beta = \Big(0,0,\beta \Big)$, 
%the Jacobian is
%\begin{equation}
%\mathcal{J}(0, 0, H^*_\beta) = \begin{bmatrix}
%r_F - \mu_F - (\FHterme) H^*_\beta & 0 & 0 \\
%0 & r_V\dfrac{H^*_\beta}{H^*_\beta+L_V} - \lv H^*_\beta - \mu_V & 0 \\
%0 & 0 & r_H (1 - \dfrac{H^*_\beta}{c})
%\end{bmatrix}
%\end{equation}
%We used $H^*_\beta = \beta$ several times. $EE^H_\beta$ is AS if
is AS if

\begin{subequations}
    \begin{empheq}[left={\empheqlbrace\,}]{align}
    & \dfrac{r_F}{\mu_F + \beta (\FHterme)} < 1 \\
    & \dfrac{r_V}{\mu_V + \lv \beta} \dfrac{\beta}{\beta + L_V} < 1  \\
    & \dfrac{c}{\beta} < 1
    \end{empheq}
\end{subequations}

\item Point $EE^{FH}_\beta = \Big(F^*_\beta,0,\beta \Big)$,
% the Jacobian is
%\begin{equation}
%\mathcal{J}(F^*_\beta, 0, H^*_\beta) = \begin{bmatrix}
%-r_F \dfrac{F^*_\beta}{K_F} & 0 & -(\FHterme) F^*_\beta \\
%0 & r_V\dfrac{H^*_\beta}{H^*_\beta+L_V} - \lv H^*_\beta - \alpha F^*_\beta - \mu_V & 0 \\
%0 & 0 & r_H (1 - \dfrac{H^*_\beta}{c})
%\end{bmatrix}
%\end{equation}
%We used $H^*_\beta = \beta$ several times.  $EE^{FH}_\beta$ is AS if
is AS if
\begin{subequations}
    \begin{empheq}[left={\empheqlbrace\,}]{align}
    &\dfrac{\beta}{\beta + L_V} \dfrac{r_V}{\mu_V + \lv \beta + \alpha F^*_\beta}  < 1 \\
    & \dfrac{c}{\beta} < 1
    \end{empheq}
\end{subequations}


\item Point $EE^{VH}_\beta = \Big(0,V^*_\beta,\beta \Big)$,
% the Jacobian is
%\begin{equation}
%\mathcal{J}(0, V^*_\beta, H^*_\beta) = \begin{bmatrix}
%r_F - \mu_F - (\FHterme) H^*_\beta & 0 & 0 \\
%-\alpha V^*_\beta & - r_V\dfrac{H^*_\beta}{H^*_\beta+L_V} V^*_\beta & \_ \\
%0 & 0 & r_H (1 - \dfrac{H^*_\beta}{c})
%\end{bmatrix}
%\end{equation}
%We used $H^*_\beta = \beta$ several times. Eigenvalues appear on the diagonal. $EE^{VH}_\beta$
 is AS if
\begin{subequations}
    \begin{empheq}[left={\empheqlbrace\,}]{align}
    & \dfrac{r_F}{\mu_F + \beta (\FHterme)} < 1 \\
    & \dfrac{c}{\beta} <1 
    \end{empheq}
\end{subequations}

\item At point $EE^{FVH}_\beta = \Big(\Fbeta,\VbetaF, \beta \Big)$, the Jacobian is
\begin{equation}
\mathcal{J}(\Fbeta, \VbetaF, \beta) = \begin{bmatrix}
-r_F \dfrac{\Fbeta}{K_F} & 0 & -(\FHterme)\Fbeta \\
-\alpha \VbetaF & - r_V\dfrac{\beta}{\beta+L_V} \VbetaF & \_ \\
0 & 0 & r_H (1 - \dfrac{\beta}{c})
\end{bmatrix}
\end{equation}
Eigenvalues appear on the diagonal. $EE^{FVH}_\beta$ is AS if
\begin{equation}
\dfrac{c}{\beta} < 1
\end{equation}

\item Point $EE^{H} = \Big(0,0,c \Big)$
%the Jacobian is
%\begin{equation}
%\mathcal{J}(0,0,H^*_c) = \begin{bmatrix}
%r_F - \mu_F - H^*_c(\FHterme) & 0 & 0 \\
%0 & r_V \dfrac{H^*_c}{L_V + H^*_c} - \lv H^*_c - \mu_V & 0 \\
%r_H a (\dfrac{H^*_c}{\beta} - 1) & r_H b (\dfrac{H^*_c}{\beta} - 1) & -r_H(\dfrac{H^*_c}{\beta} - 1)
%\end{bmatrix}
%\end{equation}
%We used $H^*_c = c$ several times. $EE^{H}$
 is AS if
\begin{subequations}
    \begin{empheq}[left={\empheqlbrace\,}]{align}
    &\dfrac{r_F}{\mu_F + c (\FHterme)} < 1 \\
    & \dfrac{r_V}{\mu_V + c \lv} \dfrac{c}{c + L_V} < 1  \\
    & 1< \dfrac{c}{\beta} 
    \end{empheq}
\end{subequations}

\item Point $EE^{FH} = \Big(F^*_{FH}, 0, aF^*_{FH} + c)$
%, the Jacobian is
%\begin{equation}
%\mathcal{J}(F^*_{H, c}, 0, H^*_{F, c}) = \begin{bmatrix}
%-r_F \dfrac{F^*_{H, c}}{K_F} & 0 & -(\FHterme) F^*_{H, c} \\
%0 & r_V \dfrac{H^*_{F, c}}{L_V + H^*_{F, c}} - \lv H^*_{F, c} -\alpha F^*_{H, c} - \mu_V & 0 \\
%r_H a (\dfrac{H^*_{F, c}}{\beta} - 1) & r_H b (\dfrac{H^*_{F, c}}{\beta} - 1) & -r_H(\dfrac{H^*_{F, c}}{\beta} - 1)
%\end{bmatrix}
%\end{equation}
%We used $H^*_{F, c} = a F^*_{H, c} c$ several times. Eigenvalues are the roots of 
%
%\begin{multline}
%\chi = \Big(X - r_V \dfrac{H^*_{F, c}}{L_V + H^*_{F, c}} + \lv H^*_{F, c} +\alpha F^*_{H, c} + \mu_V \Big) \times \\
%\left(X^2 + \Big(r_F \dfrac{F^*_{H, c}}{K_F} + r_H \big(\dfrac{H^*_{F, c}}{\beta} - 1\big) \Big)X + r_f \dfrac{F^*_{H, c}}{K_F}r_H \big(\dfrac{H^*_{F, c}}{\beta} - 1)\big) + r_H a \big(\dfrac{H^*_{F, c}}{\beta} - 1\big) (\FHterme) F^*_{H, c} \right)
%\end{multline}
%
%One root is $r_V \dfrac{H^*_{F, c}}{L_V + H^*_{F, c}} - \lv H^*_{F, c} -\alpha F^*_{H, c} -  \mu_V$. The two others have negative real parts if the coefficients of the second degree polynomial are both positive, which is the case if $H^*_{F, c} > \beta$.
%Finally, $EE^{FH}$ 
is AS if
\begin{subequations}
    \begin{empheq}[left={\empheqlbrace\,}]{align}
    & \dfrac{r_V}{\mu_V + H^*_{F} \lv +\alpha F^*_{FH}} \dfrac{H^*_{F}}{H^*_{F} + L_V} < 1 \\
    &1 < \dfrac{a F^*_{FH} + c}{\beta}
    \end{empheq}
\end{subequations}
with $F^*_{FH} = K_F \left( \dfrac{1 - \dfrac{\mu_F}{r_F} - c \dfrac{\omega f + \lf}{r_F}}{1 + a K_F \dfrac{\omega f + \lf}{r_F}} \right)$ and $H^*_{FH} = a F^*_{FH} + c$.


\item Point $EE^{VH}_1 = \Big(0, V^*_{VH, 1}, H^*_{VH, 1})$ is never stable.
\item Point $EE^{VH}_2 = \Big(0, V^*_{VH, 2}, H^*_{VH, 2})$
%the Jacobian is
%\begin{equation}
%\mathcal{J}(0, V^*_{H, c}, H^*_{V, c}) = \begin{bmatrix}
%r_F - \mu_F - (\FHterme)H^*_{V, c}  & 0 & 0 \\
%-\alpha V^*_{H, c} & -r_V\dfrac{H^*_{V, c}}{H^*_{V, c} + L_V} \dfrac{V^*_{H, c}}{K_V} & \Big(\dfrac{L_V\mu_V}{H^*_{V, c}} - \lv H^*_{V, c}) \dfrac{V^*_{H, c}}{H^*_{V, c} + L_V} \\
%r_H a (\dfrac{H^*_{V, c}}{\beta} - 1) & r_H b (\dfrac{H^*_{V, c}}{\beta} - 1) & -r_H(\dfrac{H^*_{V, c}}{\beta} - 1)
%\end{bmatrix}
%\end{equation}
%We used $H^*_{V, c} = b V^*_{H, c} + c$ several times. Eigenvalues are the roots of 
%\begin{multline}
%\chi = \Big(X - r_F + \mu_F + (\FHterme)H^*_{V, c} \Big) \times \\
%\left(X^2 + \Big(r_V  \dfrac{H^*_{V, c}}{H^*_{V, c}+L_V}\dfrac{V^*_{H, c}}{K_V} + r_H(\dfrac{H^*_{V, c}}{\beta} - 1)\Big)X + \right.\\ \left.
% r_V  \dfrac{H^*_{V, c}}{H^*_{V, c}+L_V}\dfrac{V^*_{H, c}}{K_V} r_H(\dfrac{H^*_{V, c}}{\beta} - 1) + r_Hb(\dfrac{H^*_{V, c}}{\beta} - 1) \Big(\lv H^*_{V, c} - \dfrac{L_V \mu_V}{H^*_{V, c}} \Big) \dfrac{V^*_{H, c}}{H^*_{V, c} + L_V} \right)
% \label{stab:chi VH}
%\end{multline}
%
is AS if
\begin{subequations}
    \begin{empheq}[left={\empheqlbrace\,}]{align}
&\dfrac{r_F}{ \mu_F + (\FHterme)H^*_{VH, 2}} < 1 \\
&1 < \dfrac{H^*_{VH,2}}{\beta}
    \end{empheq}
\end{subequations}
where $V^*_{VH, 2}$ is given by equation \eqref{equilibreVH:V} and $H^*_{VH, 2} =  bV^*_{VH, 2} + c$.


\item At point $EE^{FVH} = (F^*_{FH} - c_1^F V^*_{FVH}, V^*_{FVH}, a F^*_{FH} + c + (b-ac_1^F)V^*_{FVH})$, the Jacobian is 
\begin{multline}
\mathcal{J}(F^*_{FVH}, V^*_{FVH}, H^*_{FVH}) = \\
\begin{bmatrix}
-r_F \dfrac{F^*_{FVH}}{K_F} &0 &-(\FHterme) F^*_{FVH} \\
-\alpha V^*_{FVH} & -r_V \dfrac{H^*_{FVH}}{H^*_{FVH} + L_V}\dfrac{V^*_{FVH}}{K_V} & \Big(\dfrac{(\alpha F^*_{FVH} + \mu_V)L_V}{H^*_{FVH}} - \lv H^*_{FVH} \Big)\dfrac{V^*_{FVH}}{H^*_{FVH}+L_V} \\
a r_H \Big(\dfrac{H^*_{FVH}}{\beta} - 1\Big) & b r_H \Big(\dfrac{H^*_{FVH}}{\beta} - 1\Big) & -r_H\Big(\dfrac{H^*_{FVH}}{\beta} - 1\Big)
\end{bmatrix}
\end{multline}

We used $r_V \dfrac{H^*_{FVH}}{H^*_{FVH} + L_V} \Big(1 - \dfrac{V^*_{FVH}}{K_V} \Big)V^*_{FVH} = \lv H^*_{FVH} V^*_{FVH} + \mu_VV^*_{FVH} + \alpha F^*_{FVH}V^*_{FVH} $.
The characteristic polynomial of $\mathcal{J}(F^*_{FVH}, V^*_{FVH}, H^*_{FVH})$ is 
\begin{equation}
\chi_{FVH} = X^3 + q_2X^2 + q_1 X + q_0
\end{equation}
with 
\begin{subequations}
\begin{align}
&q_2 = \left(r_F \dfrac{F^*_{FVH}}{K_F} + r_V \dfrac{H^*_{FVH}}{H^*_{FVH} + L_V}\dfrac{V^*_{FVH}}{K_V} + r_H\Big(\dfrac{H^*_{FVH}}{\beta} - 1\Big) \right)\\
&q_1 = \left(r_F \dfrac{F^*_{FVH}}{K_F}r_V \dfrac{H^*_{FVH}}{H^*_{FVH} + L_V}\dfrac{V^*_{FVH}}{K_V} +r_F \dfrac{F^*_{FVH}}{K_F} r_H\Big(\dfrac{H^*_{FVH}}{\beta} - 1\Big) + \right. \\ \nonumber &\left. r_H\Big(\dfrac{H^*_{FVH}}{\beta} - 1\Big)r_V \dfrac{H^*_{FVH}}{H^*_{FVH} + L_V}\dfrac{V^*_{FVH}}{K_V} + (\FHterme) F^*_{FVH}a r_H \Big(\dfrac{H^*_{FVH}}{\beta} - 1\Big) - \right. \\ \nonumber &\left. b r_H \Big(\dfrac{H^*_{FVH}}{\beta} - 1\Big)\Big(\dfrac{(\alpha F^*_{FVH} + \mu_V)L_V}{H^*_{FVH}} - \lv H^*_{FVH} \Big)\dfrac{V^*_{FVH}}{H^*_{FVH}+L_V} \right) \\
&q_0 = r_F r_H \dfrac{F^*_{FVH}}{K_F} \dfrac{bV^*_{FVH}}{H^*_{FVH}(H^*_{FVH}+L_V)}\Big(\dfrac{H^*_{FVH}}{\beta}-1\Big) \\ \nonumber
&\left( (H^*_{FVH})^2 \Big(\dfrac{r_V}{bK_V}\Big( 1 + aK_F \dfrac{\FHterme}{r_F} \Big)+ \lv - \alpha K_F \dfrac{\FHterme}{r_F} \Big) - 
\right. \\ \nonumber & \left.
H^*_{FVH} L_V \alpha K_F \dfrac{\FHterme}{r_F} - \alpha F^*_{FVH} L_V - \mu_V L_V \right)
\end{align}
\end{subequations}

According to the Ruth-Hurwitz criterion, the equilibrium is AS if all the coefficients of $\chi_{FVH}$ are positive. We will look for their sign.

First, studying the variation of the function $f_3$ induced by $H$ around the equilibrium point, we notice that if $H^*_{FVH} < \beta$, the equilibrium is unstable. Thus, one stability condition is $H^*_{FVH} > \beta$. This condition also ensures that $q_2 > 0$.


Using $F^*_{FVH} = K_F\Big(1 - \dfrac{\mu_F}{r_F} \Big) - K_F\dfrac{\FHterme}{r_F} H^*_{FVH}$, we can simplify the expression of $q_0$:
\begin{multline}
q_0 = r_F r_H \dfrac{F^*_{FVH}}{K_F} \dfrac{bV^*_{FVH}}{H^*_{FVH}(H^*_{FVH}+L_V)}\Big(\dfrac{H^*_{FVH}}{\beta}-1\Big) \\
\left( (H^*_{FVH})^2 \dfrac{r_V}{bK_V}\Big(\Big( 1 + aK_F \dfrac{\FHterme}{r_F}\Big) + \lv - \alpha K_F \dfrac{\FHterme}{r_F} \Big)
 - \Big(\alpha (K_F\big(1 - \dfrac{\mu_F}{r_F} \big) + \mu_V\Big) L_V \right)
\end{multline}

Using equation \eqref{equilibreFVH:equationV} and $V^*_{FVH} = \dfrac{H^*_{FVH} - aF^*_{FVH} - c}{b}$, we find that $H^*_{FVH}$ is solution of
\begin{multline}
H^2 \Big(\dfrac{r_V}{bK_V}(1 + a K_F \dfrac{\FHterme}{r_F}) + \lv - \alpha K_F\dfrac{\FHterme}{r_F} \Big) - \\ H \Big(r_V + \dfrac{r_V}{b K_V} (c + a K_F\big(1-\dfrac{\mu_F}{r_F}\big) ) - \mu_V - \alpha K_F \big(1 - \dfrac{\mu_F}{r_F}\big) - L_V \lv + L_V \alpha K_F \dfrac{\FHterme}{r_F} \Big) + \\ L_V \Big(\mu_V + \alpha K_F \big(1 - \dfrac{\mu_F}{r_F} \big)\Big) = 0
\label{equilibreFVH:equationH}
\end{multline}

Using this equation and appendix \ref{appendice:ineq2nd} allow to say that $q_0$ is positive only for one value of the equilibria $EE^{FVH}$ among the two possibles. The value in itself will depend on the sign of the $H^2$ and $H$ coefficient in equation \eqref{equilibreFVH:equationH}.


sign of $q_1$ ??
\begin{scriptsize}
\begin{multline}
[\mathcal{J}(F^*_{FVH}, V^*_{FVH}, H^*_{FVH})]^{second compund} = \\
\begin{bmatrix}
-r_F\dfrac{F^*_{FVH}}{K_F}-r_V \dfrac{H^*_{FVH}}{H^*_{FVH} + L_V}\dfrac{V^*_{FVH}}{K_V}  &\Big(\dfrac{(\alpha F^*_{FVH} + \mu_V)L_V}{H^*_{FVH}} - \lv H^*_{FVH} \Big)\dfrac{V^*_{FVH}}{H^*_{FVH}+L_V} & (\FHterme) F^*_{FVH} \\
b r_H \Big(\dfrac{H^*_{FVH}}{\beta} - 1\Big) & -r_F\dfrac{F^*_{FVH}}{K_F} -r_H\Big(\dfrac{H^*_{FVH}}{\beta} - 1\Big)& 0 \\
- a r_H \Big(\dfrac{H^*_{FVH}}{\beta} - 1\Big) & -\alpha V^*_{FVH} & -r_H\Big(\dfrac{H^*_{FVH}}{\beta} - 1\Big) -r_V \dfrac{H^*_{FVH}}{H^*_{FVH} + L_V}\dfrac{V^*_{FVH}}{K_V}
\end{bmatrix}
\end{multline}
\end{scriptsize}


\end{itemize}


\section{Summary of the long term behavior}
We define 
\begin{subequations}
\begin{align}
& T_H(F, V) = \dfrac{aF + bV + c}{\beta} \\
& T_F(H) = \dfrac{r_F}{\mu_F + H (\omega f + \lf)} \\
& T_V(F, H) = \dfrac{r_V}{\mu_V + \alpha F + H \lv} \dfrac{H}{H + L_V}
\end{align}
\end{subequations}

$T_H$ is increasing with respect to both $F$ and $V$, $T_F$ is decreasing wrt $H$ and $T_V$ is decreasing wrt $F$, and increasing on $[0, \sqrt{L_V (\mu_V + \alpha F) /\lv}]$, decreasing on $[\sqrt{L_V (\mu_V + \alpha F) /\lv}, +\infty]$.

We also define
\begin{subequations}
{\footnotesize
\begin{align}
&  T_{\Delta_{VH}} = \Big(1 + \dfrac{b K_V \lv}{r_V}\Big)^{-1}\Big(\dfrac{c+L_V}{c} \dfrac{\mu_V + \lv c}{r_V} - 1\Big)^{-1} \dfrac{(bK_V - c)^2}{4bcK_V}\left( \dfrac{bK_V}{bK_V - c}\dfrac{\lv(2c+L_V) + \mu_V}{r_V} - 1 \right)^2  \\
& T_{V?} = \dfrac{bK_V - c}{bK_V}\dfrac{r_V}{\lv  (2c+L_V) + \mu_V} 
\end{align}
}
\end{subequations}

Some equilibrium values are also recall:
\begin{subequations}
\begin{align*}
& F^*_{FH} = K_F \left( \dfrac{1 - \dfrac{\mu_F}{r_F} - c \dfrac{\omega f + \lf}{r_F}}{1 + a K_F \dfrac{\omega f + \lf}{r_F}} \right) \\
& H^*_{VH,2} = b V^*_{VH, 2} + c
\end{align*}
\end{subequations}

The existence and stability condition are summarized in the following table:

\newpage
\begin{landscape}
\begin{table}
\centering
\begin{tabular}{c|c|c|c|c|c|c|c|c|c|c|c|c|c}
Equilibrium & $T_H(0)$ & $T_H(F^*_{FH})$ & $T_H^*(V^*_{VH,2})$ & $T_F(0)$ & $T_F(\beta)$ & $T_F(c)$& $T_F(H^*_{VH, 2})$ & $T_V(\beta)$ & $T_V(\beta, F^*_\beta)$ & $T_V(c, 0)$ & $T_V(c, F^*_{FH})$ & $T_{\Delta_{VH}}$ & $T_{V?}$ \\ \hline
$TE$ & & & & $<1$ & & & & & & & & \\ \hline
$EE^H_\beta$& $<1$ & & & & $<1$ & & & $<1$ & & & & \\ \hline
$EE^F$ & & & & $1<$ & & & & & & & & \\ \hline
$EE^{FH}_\beta$ & $<1$ & & & & $1<$ & & & & $<1$ & & & \\ \hline
$EE^{VH}_\beta$ & $<1$ & & & & $<1$ & & & $1<$ & & & & \\ \hline
$EE^{FVH}_\beta$ & $<1$ & & & & $1<$ & & & & $1<$ & & & \\ \hline
$EE^H $ & $1<$  & & & & & $<1$ & & & & $<1$ & & \\ \hline
$EE^{FH} $ & & $1<$  & & & & $1<$ & & & & & $<1$ & \\ \hline
$EE^{VH}_2 $ & & & $1<$ & & & & $<1$ & & & $1<$ & & & \\ \hline
$EE^{VH}_2 $ & & & $1<$ & & & & $<1$ & & & $<1$ & &$1<$& $1<$ \\ \hline
\end{tabular}
\end{table}
\end{landscape}


\section{Model with human preferences for forest or vegetation}
Let introduce $\tau \in [0,1]$ which represents the fraction of the human population interested in forest resources. The model becomes
\begin{equation}    
\left\{ \begin{array}{l}
\dfrac{dF}{dt}=r_{F}\left(1-\dfrac{F}{K_{F}}\right)F-\omega f (1-\tau) H F - \mu_F F -\lambda_{FH} F (\tau H) \\
\dfrac{dV}{dt}=r_V \dfrac{(1-\tau)H}{(1-\tau)H + L_V} s(F) \left(1-\dfrac{V}{K_{V}}\right)V -\alpha FV-\mu_V V -\lambda_{VH} (1-\tau) V H\\
\dfrac{dH}{dt}=r_H \left(1-\dfrac{H}{a\tau F + b(1-\tau)V + c} \right) (\dfrac{H}{\beta} - 1) H
\end{array}\right.
\label{model}
\end{equation}

\begin{appendices}
\section{Inequality on second degree polynomial \label{appendice:ineq2nd}}

Consider the equation
\begin{equation}
X^2 - B X + C = 0
\label{appendice:eq}
\end{equation}
with positive discriminant $\Delta = B^2 - 4C \geq 0$. Let note $X_{1} = \dfrac{1}{2}B - \dfrac{1}{2} \sqrt{B^2 - 4 C}$ and $X_2 = \dfrac{1}{2}B + \dfrac{1}{2} \sqrt{B^2 - 4 C}$  the roots of \eqref{appendice:eq}.

When $B > 0$ and $C >0$, the following inequality hold
\begin{equation}
(X_1) ^ 2 - C < 0 \:\:\text{ and } (X_2)^2 - C > 0
\end{equation}

Indeed, we have :
\begin{subequations}
\begin{align}
&(X_i) ^2 = \dfrac{1}{4} \Big(B^2 \pm 2 B\sqrt{B^2 - 4C} + B^2 - 4C \Big) \\
&(X_i)^2 - C = \dfrac{1}{4} \Big(2(B^2 - 4C) \pm 2B \sqrt{B^2 - 4C} \Big) \\
&(X_i)^2 - C = \dfrac{1}{2} \sqrt{B^2 - 4C} \Big( \sqrt{B^2 - 4C} \pm B \Big)
\end{align}
\end{subequations}
and when $B > 0$ and $C>0$, we have $B = \sqrt{B^2} > \sqrt{B^2 - 4C}$.




\end{appendices}

\bibliographystyle{plain}
\bibliography{bibSuivi2}

\end{document}
