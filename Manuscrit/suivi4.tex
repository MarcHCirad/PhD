\documentclass{article}
\usepackage{graphicx} 
\usepackage{color}
\usepackage{amsfonts,amsmath}
\usepackage{amsthm}
\usepackage{empheq}
\usepackage{mathtools}
\usepackage{multirow}
\usepackage{tikz}
\usepackage{titlesec}
\usepackage{caption}
\usepackage{lscape}
\captionsetup{justification=justified}
\usepackage[toc,page]{appendix}

\textheight240mm \voffset-23mm \textwidth160mm \hoffset-20mm

\setcounter{secnumdepth}{4}
\titleformat{\paragraph}
{\normalfont\normalsize\bfseries}{\theparagraph}{1em}{}
\titlespacing*{\paragraph}
{0pt}{3.25ex plus 1ex minus .2ex}{1.5ex plus .2ex}

\newtheorem*{theo}{Theorem}
\newcommand{\yves}{\textcolor{magenta}}
\newcommand{\gf}{(r_F - \omega f - \mu_F)}
\newcommand{\gv}{(r_V- \mu_V)}
\newcommand{\lf}{\lambda_{FH}}
\newcommand{\lv}{\lambda_{VH}}
\newcommand{\RV}{R_0^V}
\newcommand{\RF}{R_0^F}
\newcommand{\NF}{\mathcal{N}_0^F}
\newcommand{\NV}{\mathcal{N}_0^V}
\newcommand{\NH}{\mathcal{N}_0^H}
\newcommand{\Fbeta}{F^*_\beta}
\newcommand{\Hbeta}{H^*_\beta}
\newcommand{\Vbeta}{V^*_\beta}
\newcommand{\VbetaF}{V^*_{\Fbeta, \beta}}
\newcommand{\FHterme}{\omega f + \lf}
\newcommand*\phantomrel[1]{\mathrel{\phantom{#1}}}

\title{Suivi Thèse Marc}
\author{Marc Hétier, Yves Dumont  and Valaire Yatat-Djeumen}

\begin{document}

\maketitle

This model tries to take into account the interactions between human population ($H$), wild fauna and flora ($F$), and human-driven vegetation ($V$). 
We assume that human pick up resources from wild and human-driven vegetation. The rate per person of this collect is assumed to be decreasing with the quantity of resources.
Moreover, human-driven vegetation growth is depended on both services render by human (culture, watering...) and wild vegetation (pollination, moisture retention, wind protection). 
Deforestation by fire happen with intensity $\omega$ and frequency $f(H)$. Assuming that the combustible (dry grass) is available in enough quantity for a good fire's spatial dispersion, we can take $f(H) = fH$. Also, note that in the context of this study, fire is mainly used for hunting activities or to create a agricultural field. In both cases, the frequency and the intensity of those fires are low.
Conversely, wild biomass is a competitor for human driven vegetation.

Human maximum capacity is assumed to depend on resources (wild and human driven). This assumption is justified in \cite{fanta_equilibrium_2018}, used in \cite{bengochea_paz_agricultural_2020}.
Moreover, the human population is assumed have a reduced growth rate for low population size. This corresponds to a weak Allee effect. Some studies have already propose such phenomenon on human population ; see \cite{hamilton_human_2012} and \cite{vaesen_inbreeding_2019}.

All thus assumption lead to the following model :

\begin{equation}    
\left\{ \begin{array}{l}
\dfrac{dF}{dt}=r_{F}\left(1-\dfrac{F}{K_{F}}\right)F-\omega f H F - \mu_F F -\lambda_{FH}g_F(F,V)F H \\
\dfrac{dV}{dt}=r_V \dfrac{H}{H + L_V} s(F) \left(1-\dfrac{V}{K_{V}}\right)V -\alpha FV-\mu_V V -\lambda_{VH} g_V(F,V) V H\\
\dfrac{dH}{dt}=r_H \left(1-\dfrac{H}{aF + bV + c} \right) (\dfrac{H}{\beta} - 1) H
\end{array}\right.
\label{model}
\end{equation}
where both $g_F$ and $g_V$ are decreasing in both $F$ and $V$.

\section{Simplified model}
First we can assume that $g_F = g_V = 1$ and that $s(F) = 1$. The model becomes :
\begin{equation}    
\left\{ \begin{array}{l}
\dfrac{dF}{dt}=r_{F}\left(1-\dfrac{F}{K_{F}}\right)F-\omega f H F - \mu_F F -\lf F H \\
\dfrac{dV}{dt}=r_V \dfrac{H}{H + L_V} \left(1-\dfrac{V}{K_{V}}\right)V -\alpha FV-\mu_V V -\lv V H\\
\dfrac{dH}{dt}=r_H \left(1-\dfrac{H}{aF + bV + c} \right)  (\dfrac{H}{\beta} - 1) H
\end{array}\right.
\label{model:simplified}
\end{equation}

\subsection{Sub-model $F-H$}
First, we can analyze the sub-model with variables $F$ and $H$ only. The model is:

\begin{equation}    
\left\{ \begin{array}{l}
\dfrac{dF}{dt}=r_{F}\left(1-\dfrac{F}{K_{F}}\right)F-\omega f H F - \mu_F F -\lf F H = f_{1, FH} (F,H) \\
\dfrac{dH}{dt}=r_H \left(1-\dfrac{H}{aF + c} \right)  (\dfrac{H}{\beta} - 1) H = f_{2,FH}(F,H)
\end{array}\right.
\label{model:submodelFH}
\end{equation}

\subsubsection{Equilibrium}
Equilibrium are obtained when the right hand side of the previous system is equal at 0. They are :
\begin{itemize}
\item $TE = (0,0)$
\item A forest equilibrium $EE^F = \Big(F^*, 0 \Big) $ where $F^* = K_F \Big(1 - \dfrac{\mu_F}{r_F}\Big)$. It exists when $\NF = \dfrac{r_F}{\mu_F} > 1$.
\item A human equilibrium $EE^H_\beta = \Big(0,\beta \Big)$
\item A forest-human equilibrium $EE^{FH}_\beta = \Big(F^*_\beta, \beta\Big)$ where $F^*_\beta = K_F \Big(1-\dfrac{\mu_F + \beta(\lf + \omega f)}{r_F} \Big)$.
This equilibrium exists when $\dfrac{r_F}{\mu_F + \beta(\lf + \omega f)} > 1$.
\item A human equilibrium $EE^H = \Big(0,c \Big)$
\item A forest-human equilibrium $EE^{FH} = \Big(F^*_{c}, aF^*_{FH} + c \Big)$ where $F^*_{FH} = K_F \left( \dfrac{1 - \dfrac{\mu_F}{r_F} - c \dfrac{\omega f + \lf}{r_F}}{1 + a K_F \dfrac{\omega f + \lf}{r_F}} \right)$. This equilibrium exists when $\dfrac{r_F}{\mu_F + c(\lf + \omega f)} > 1$.
\end{itemize}

\subsubsection{Stability}
The Jacobian matrix of the system is defined by :
\begin{equation}
\mathcal{J}(F,H) =  \begin{bmatrix}
r_F \Big(1-\dfrac{2F}{K_F} \Big) - \mu_F - (\lf + \omega f)H & -(\lf + \omega f)F \\
r_H \dfrac{aH^2}{(aF+c)^2} (\dfrac{H}{\beta}-1) & r_H(1-\dfrac{H}{aF+c})(\dfrac{2H}{\beta}-1) - \dfrac{r_H H}{aF+c}(\dfrac{H}{\beta}-1)
\end{bmatrix}
\label{stab:jacobianFH}
\end{equation}

To assess the stability of the different equilibrium, we look at the eigenvalues of the Jacobian.

\begin{itemize}
\item At point $TE = (0,0, 0)$, the Jacobian is
\begin{equation}
\mathcal{J}(0,0) = \begin{bmatrix}
r_F - \mu_F & 0 \\
0 & -r_H
\end{bmatrix}
\end{equation}
and thus the eigenvalues are $r_F - \mu_F$ and $-r_H$. $TE$ is asymptotically stable (AS) if $\NF < 1$.

\item At point $EE^{F} = (F^*, 0)$, the Jacobian is 
\begin{equation}
\mathcal{J}(F^*, 0, 0)
\begin{bmatrix}
-r_F \dfrac{F^F}{K_F} & -(\FHterme)F^F \\
0 & -r_H 
\end{bmatrix}
\end{equation}
and thus the eigenvalues are all negative. $EE^F$ is asymptotically stable.

\item At point $EE^H_\beta = \Big(0,\beta \Big)$, the Jacobian is
\begin{equation}
\mathcal{J}(0, \beta) = \begin{bmatrix}
r_F - \mu_F - (\FHterme) \beta & 0 \\
0  & r_H (1 - \dfrac{\beta}{c})
\end{bmatrix}
\end{equation}
$EE^H_\beta$ is AS if 

\begin{subequations}
    \begin{empheq}[left={\empheqlbrace\,}]{align}
    & \dfrac{r_F}{\mu_F + \beta (\FHterme)} < 1 \\
    & \dfrac{c}{\beta} < 1
    \end{empheq}
\end{subequations}

\item At point $EE^{FH}_\beta = \Big(F^*_\beta,\beta \Big)$, the Jacobian is
\begin{equation}
\mathcal{J}(F^*_\beta, \beta) = \begin{bmatrix}
-r_F \dfrac{F^*_\beta}{K_F} &  -(\FHterme) F^*_\beta \\
0 & r_H (1 - \dfrac{H^*_\beta}{c})
\end{bmatrix}
\end{equation}
$EE^{FH}_\beta$ is AS if $\dfrac{c}{\beta} < 1$

\item At point $EE^{H} = \Big(0,c\Big)$, the Jacobian is
\begin{equation}
\mathcal{J}(0,c) = \begin{bmatrix}
r_F - \mu_F - c(\FHterme) & 0 \\
r_H a (\dfrac{c}{\beta} - 1) & -r_H(\dfrac{c}{\beta} - 1)
\end{bmatrix}
\end{equation}
$EE^{H}$ is AS if
\begin{subequations}
    \begin{empheq}[left={\empheqlbrace\,}]{align}
    &\dfrac{r_F}{\mu_F + c (\FHterme)} < 1 \\
    & 1< \dfrac{c}{\beta} 
    \end{empheq}
\end{subequations}

\item At point $EE^{FH} = \Big(F^*_{FH}, aF^*_{FH}+c\Big)$, the Jacobian is
\begin{equation}
\mathcal{J}(F^*_{FH}, aF^*_{FH}+c) = \begin{bmatrix}
-r_F \dfrac{F^*_{FH}}{K_F} & -(\FHterme) F^*_{FH} \\
r_H a (\dfrac{aF^*_{FH}+c}{\beta} - 1) & -r_H(\dfrac{aF^*_{FH}+c}{\beta} - 1)
\end{bmatrix}
\end{equation}
Eigenvalues are the roots of 

\begin{multline}
\chi_{FH} = X^2 + \Big(r_F \dfrac{F^*_{FH}}{K_F} + r_H \big(\dfrac{aF^*_{FH}+c}{\beta} - 1\big) \Big)X + \\ r_F  \dfrac{F^*_{FH}}{K_F}r_H \big(\dfrac{aF^*_{FH}+c}{\beta} - 1\big) + r_H a \big(\dfrac{aF^*_{FH}+c}{\beta} - 1\big) (\FHterme) F^*_{FH}
\end{multline}
The two roots of $\chi_{FH}$ have negative real parts if the coefficients are both positive, which is the case if $aF^*_{FH}+c > \beta$.
So, $EE^{FH}$ is AS if
\begin{equation}
1 < \dfrac{a K_F}{\beta} \left( \dfrac{1 - \dfrac{\mu_F}{r_F} - c \dfrac{\omega f + \lf}{r_F}}{1 + a K_F \dfrac{\omega f + \lf}{r_F}} \right) + \dfrac{c}{\beta}
\end{equation}

\end{itemize}

\subsubsection{Longterm dynamic}
The following table indicates the conditions required 
\begin{table}[ht]
\centering
\caption{\centering Stability and existence for subsystem $F-H$ : condition in brackets are required conditions but implied by the other row's elements}
\begin{tabular}{c|c|c|c|c|c}
$T_F(0)$ & $T_H(0)$ & $T_F(c)$ & $T_H(F^*_{FH})$ & $T_F(\beta)$ & Equilibrium \\
\hline
\multirow{2}{*}{$<1$} & $<1$ & & & & $TE$, $EE^H_\beta$ \\
 & $1<$ & & & & $TE$, $EE^H$ \\
 \hline
 \multirow{6}{*}{$1<$}& \multirow{4}{*}{$<1$} & $1<$ & $1<$ & $<1$ & $EE^F$, $EE^H_\beta$, $EE^{FH}$ \\
 & & & & $<1$ & $EE^F$, $EE^H_\beta$ \\
 \cline{3-6}
  & & $(1<)$ & $1<$ & $1<$& $EE^F$, $EE^{FH}_\beta$, $EE^{FH}$ \\
 & & & & $1<$ & $EE^F$, $EE^{FH}_\beta$ \\
 \cline{2-6}
 & \multirow{2}{*}{$1<$} &$<1$ & & & $EE^F$, $EE^H$ \\ 
 & &  $1<$ & & &  $EE^{F}$, $EE^{FH}$ \\\hline
\end{tabular}
\end{table}

\subsubsection{Limit Cycle}
We can study the existence of limit cycle for this system. We will consider two subset $\Omega_1 = \{(F,H) \in [0, +\infty) \times [\beta, +\infty)\}$ and $\Omega_2 = \{(F,H) \in [0, +\infty) \times [0, \beta])\}$ . 


On $\Omega_1$, we will use the Bendixson-Dulac theorem (see \cite{farkas_1994_periodic}, page 137). Consider the function

\begin{align}
\phi : (0, +\infty) &\times (\beta, +\infty) \rightarrow \mathbf{R} \label{limit cylce:phiFH}
\\
\nonumber
(F,H) & \mapsto \dfrac{1}{F H (H/\beta - 1)}
\end{align}

Multiplying the right hand side of system $(F,H)$ \eqref{model:submodelFH} by this function, and taking the derivative with respect to $F$ or $H$ we obtain :
\begin{subequations}
\begin{align}
&\dfrac{\partial f_{1,FH} \times \phi}{\partial F}(F,H) = - \dfrac{r_F}{K_F} \dfrac{1}{H \big(\dfrac{H}{\beta}-1 \big)} \\
&\dfrac{\partial f_{2,FH} \times \phi}{\partial H}(F,H) = - \dfrac{r_H}{(aF + c) F}
\end{align}
\end{subequations}

Then, for $(F, H) \in \overset{\circ}{\Omega}_1$
\begin{equation}
\Big(\dfrac{\partial f_{1,FH} \times \phi}{\partial F} + \dfrac{\partial f_{2,FH} \times \phi}{\partial H}\Big) (F, H) < 0
\end{equation}
and, according to Bendixson-Dulac theorem, there is no limit cycle on $\Omega_1$.

On $\Omega_2$, we will distinguish two cases. First, we assume that $\beta < c$ . Then, we have on $\Omega_2$:
\begin{equation}
f_{2,FH}(F,H) = r_H \underset{0<}{\underbrace{\Big(1 - \dfrac{H}{aF + c} \Big)}}\underset{<0}{\underbrace{\Big(\dfrac{H}{\beta} -1\Big)}} H < 0
\end{equation}
since $\beta \leq aF + c$. $H(t)$ is thus decreasing on $\Omega_2$, and there is no limit cycle on $\Omega_2$.
Now, we assume that $c < \beta$. By looking the vector fields in all the different configurations, it is possible to show that there is no limit cycle either.

\subsection{Sub-model $V-H$}
Now, we can analyze the sub-model with variables $V-H$. The model is:
\begin{equation}    
\left\{ \begin{array}{l}
\dfrac{dV}{dt}=r_V \dfrac{H}{H + L_V} \left(1-\dfrac{V}{K_{V}}\right)V -\mu_V V -\lv V H\\
\dfrac{dH}{dt}=r_H \left(1-\dfrac{H}{bV + c} \right)  (\dfrac{H}{\beta} - 1) H
\end{array}\right.
\label{model:submodelVH}
\end{equation}

\subsubsection{Equilibrium}
Equilibrium are obtained when the right hand side of the previous system is equal at 0. They are :
\begin{itemize}
\item $TE = (0,0)$
\item A human equilibrium $EE^H_\beta = \Big(0,\beta \Big)$
\item A vegetation-human equilibrium $EE^{VH}_\beta = \Big(V^*_{\beta}, \beta \Big)$ where $V^*_\beta = K_V \Big(1- \dfrac{\beta + L_V}{\beta} \dfrac{\lv \beta + \mu_V}{r_V} \Big)$. This equilibrium exists when $\dfrac{\beta}{\beta + L_V} \dfrac{r_V}{\lv \beta + \mu_V} > 1$
\item A human equilibrium $EE^H = \Big(0,c\Big)$
\item One or two vegetation-human equilibrium $EE^{VH}_i = \Big(0, V^*_{VH,i}, H^*_{VH,i} \Big)$. $H^*_{c, VH,i}$ are equal at $H^*_{VH, i} = bV^*_{VH, i} + c$ and $V^*_{VH,i}$ are solutions of
%\begin{equation}
%V^2 + V \left(\dfrac{\dfrac{c r_V}{K_V} + \lv b (2c+L_V) + b (\mu_V - r_V)}{b^2\lv + \dfrac{b r_V}{K_V}} \right) + \Big(\dfrac{c+L_V}{c} \dfrac{\mu_V + \lv c}{r_V} - 1\Big)\dfrac{r_V c}{b^2 \lv(1 + \dfrac{r_V}{b K_V \lv})} = 0
%\label{equilibreVH:equationV}
%\end{equation}
%
%\begin{equation}
%V^2\left(b^2 \lv(1 + \dfrac{r_V}{b K_V \lv}) \right) + V \left( b r_V(\dfrac{c}{bK_V} -1)(1 + \dfrac{\lv  (2c+L_V) + \mu_V}{b r_V(\dfrac{c}{bK_V} -1)}) \right) + \Big(\dfrac{c+L_V}{c} \dfrac{\mu_V + \lv c}{r_V} - 1\Big)r_V c = 0
%\label{equilibreVH:equationV}
%\end{equation}
%
%\begin{equation}
%V^2\left(b^2 \lv(1 + \dfrac{r_V}{b K_V \lv}) \right) + V \left( b r_V(\dfrac{c}{bK_V} -1) + \lv  (2c+L_V) + \mu_V \right) + \Big(\dfrac{c+L_V}{c} \dfrac{\mu_V + \lv c}{r_V} - 1\Big)r_V c = 0
%\label{equilibreVH:equationV}
%\end{equation}

\begin{equation}
V^2 + V \left(\dfrac{K_V\Big(\dfrac{\lv(2c+L_V) + \mu_V}{r_V} - 1\Big) + \dfrac{c}{b}}{1 + \dfrac{b K_V \lv}{r_V}}  \right) + \dfrac{K_Vc}{b} \dfrac{\Big(\dfrac{c+L_V}{c} \dfrac{\lv c + \mu_V}{r_V} - 1\Big)}{1 + \dfrac{b K_V \lv}{r_V}} = 0
\label{equilibreVH:equationV}
\end{equation}
The solutions of this equation are real if the discriminant $\Delta_{VH}$ is positive. We have

\begin{multline}
\Delta_{VH} \geq 0 \Leftrightarrow \\
\left(K_V \Big(\dfrac{\lv(2c+L_V) + \mu_V}{r_V} - 1\Big) + \dfrac{c}{b} \right)^2 \geq  4 \dfrac{K_V}{b}  \Big(1 + \dfrac{b K_V \lv}{r_V}\Big) \Big((c+L_V) \dfrac{\mu_V + \lv c}{r_V} - c\Big)
\label{equilibreVH:discriminant}
\end{multline}

If the right hand side is negative, \textit{ie} if $1< \dfrac{c}{c+L_V} \dfrac{r_V}{\mu_V + \lv c}$, then the discriminant $\Delta_{VH}$ is positive, and the constant coefficient of \eqref{equilibreVH:equationV} is negative. Thus, there is always one positive roots (and the other is negative).

Otherwise, if $\dfrac{c}{c+L_V} \dfrac{r_V}{\mu_V + \lv c} < 1$ and if the discriminant is positive, the solutions are real and have the same sign, which is positive if the coefficient in $V$ is negative. Thus, in addition of inequality \eqref{equilibreVH:discriminant}, the parameters must verify
\begin{subequations}
\begin{align}
& bK_V \Big(\dfrac{\lv(2c+L_V) + \mu_V}{r_V}-1\Big) + c < 0 \\
& \Leftrightarrow c \Big(1 + \dfrac{2 \lv bK_V}{r_V} \Big) < bK_V \Big(1 - \dfrac{\mu_V + \lv L_V}{r_V} \Big) \\
& \Leftrightarrow 1 < \dfrac{bK_V}{c} \dfrac{1 - \dfrac{\mu_V + \lv L_V}{r_V}}{1 + \dfrac{2 \lv bK_V}{r_V}}
\end{align}
\end{subequations}

in order to have positive solutions.
\end{itemize}

Finally, if
\begin{equation}
1 < \dfrac{c}{c+L_V} \dfrac{r_V}{\mu_V + \lv c}
\label{equilibreVH:conditionExistence1}
\end{equation}
there is one equilibrium, noted $EE^{VH}_2$ ; and if 
{\small
\begin{subequations}
    \begin{empheq}[left={\empheqlbrace\,}]{align}
&\dfrac{c}{c+L_V} \dfrac{r_V}{\mu_V + \lv c} < 1 \\
&1 \leq \dfrac{b}{4K_V}\Big(1 + \dfrac{b K_V \lv}{r_V}\Big)^{-1}\Big((c+L_V) \dfrac{\mu_V + \lv c}{r_V} - c\Big)^{-1} \left(K_V \Big(\dfrac{\lv(2c+L_V) + \mu_V}{r_V} - 1\Big)+\dfrac{c}{b} \right)^2 \\
& 1 < \dfrac{bK_V}{c} \dfrac{1 - \dfrac{\mu_V + \lv L_V}{r_V}}{1 + \dfrac{2 \lv bK_V}{r_V}}
    \end{empheq}
    \label{equilibreVH:conditionExistence2}
\end{subequations}
}
there are two equilibrium $EE^{VH}_1 \leq EE^{VH}_2$.

$V^{*}_{VH,i}$ are equal at:
\begin{equation}
V^{*}_{VH,i} = -\dfrac{1}{2} \dfrac{ K_V\Big(\dfrac{\lv(2c+L_V) + \mu_V}{r_V} - 1\Big) + \dfrac{c}{b}}{1 + \dfrac{b K_V \lv}{r_V}}  \pm \dfrac{1}{2} \sqrt{\Delta_{VH}}
\label{equilibreVH:V}
\end{equation}
with $\Delta_{VH} = \left( \dfrac{ K_V\Big(\dfrac{\lv(2c+L_V) + \mu_V}{r_V} - 1\Big) + \dfrac{c}{b}}{1 + \dfrac{b K_V \lv}{r_V}}\right)^2 - 4 \dfrac{K_Vc}{b} \dfrac{\Big(\dfrac{c+L_V}{c} \dfrac{\lv c + \mu_V}{r_V} - 1\Big)}{1 + \dfrac{b K_V \lv}{r_V}} $

and $H^*_{VH,i}$ are equal at
\begin{equation}
H^*_{VH,i} = b V^*_{VH,i} + c
\label{equilibreVH:H}
\end{equation}

\subsubsection{Stability}
The Jacobian matrix of the system is defined by :
\begin{equation}
\mathcal{J}(V,H) =  \begin{bmatrix}
r_V \dfrac{H}{H+L_V}(1-\dfrac{2V}{K_V}) - \lv H - \mu_V & r_V \dfrac{L_V}{(H+L_V)^2}(1-\dfrac{V}{K_V})V  - \lv V\\
r_H \dfrac{bH^2}{(bV+c)^2} (\dfrac{H}{\beta}-1) & r_H(1-\dfrac{H}{bV+c})(\dfrac{2H}{\beta}-1) - \dfrac{r_H H}{bV+c}(\dfrac{H}{\beta}-1)
\end{bmatrix}
\label{stabilityVH:jacobian}
\end{equation}

To assess the stability of the different equilibrium, we look at the eigenvalues of the Jacobian.

\begin{itemize}
\item At point $TE = (0, 0)$, the Jacobian is
\begin{equation}
\mathcal{J}(0,0) = \begin{bmatrix}
-\mu_V & 0 \\
0 & -r_H
\end{bmatrix}
\end{equation}
and thus the eigenvalues are $-\mu_V$ and $-r_H$. $TE$ is asymptotically stable (AS).

\item At point $EE^H_\beta = \Big(0,\beta \Big)$, the Jacobian is
\begin{equation}
\mathcal{J}(0, \beta) = \begin{bmatrix}
r_V\dfrac{\beta}{\beta+L_V} - \lv \beta - \mu_V & 0 \\
0 & r_H (1 - \dfrac{\beta}{c})
\end{bmatrix}
\end{equation}
$EE^H_\beta$ is AS if 

\begin{subequations}
    \begin{empheq}[left={\empheqlbrace\,}]{align}
    & \dfrac{r_V}{\mu_V + \lv \beta} \dfrac{\beta}{\beta + L_V} < 1  \\
    & \dfrac{c}{\beta} < 1
    \end{empheq}
\end{subequations}

\item At point $EE^{VH}_\beta = \Big(V^*_\beta,\beta \Big)$, the Jacobian is
\begin{equation}
\mathcal{J}(V^*_\beta, \beta) = \begin{bmatrix}
- r_V\dfrac{\beta}{\beta+L_V} V^*_\beta & \_ \\
0 & r_H (1 - \dfrac{\beta}{c})
\end{bmatrix}
\end{equation}
Eigenvalues appear on the diagonal. $EE^{VH}_\beta$ is AS if
\begin{equation}
\dfrac{c}{\beta} <1
\end{equation}

\item At point $EE^{H} = \Big(0,c \Big)$, the Jacobian is
\begin{equation}
\mathcal{J}(0,c) = \begin{bmatrix}
r_V \dfrac{c}{L_V +c} - \lv c - \mu_V & 0 \\
r_H b (\dfrac{c}{\beta} - 1) & -r_H(\dfrac{c}{\beta} - 1)
\end{bmatrix}
\end{equation}
$EE^{H}$ is AS if
\begin{subequations}
    \begin{empheq}[left={\empheqlbrace\,}]{align}
    & \dfrac{r_V}{\mu_V + c \lv} \dfrac{c}{c + L_V} < 1  \\
    & 1< \dfrac{c}{\beta} 
    \end{empheq}
\end{subequations}

\item At point(s) $EE^{VH} = \Big(V^*_{VH}, b V^*_{VH} + c)$, the Jacobian is
\begin{equation}
\mathcal{J}(V^*_{VH}, H^*_{VH}) = \begin{bmatrix}
-r_V\dfrac{H^*_{VH}}{H^*_{VH} + L_V} \dfrac{V^*_{VH}}{K_V} & \Big(\dfrac{L_V\mu_V}{H^*_{VH}} - \lv H^*_{VH}) \dfrac{V^*_{VH}}{H^*_{VH} + L_V} \\
r_H b (\dfrac{H^*_{VH}}{\beta} - 1) & -r_H(\dfrac{H^*_{VH}}{\beta} - 1)
\end{bmatrix}
\label{stabilityVH:jacobianVH}
\end{equation}
We used $r_V \dfrac{H^*_{VH}}{H^*_{VH} + L_V} \Big(1 - \dfrac{V^*_{VH}}{K_V} \Big)V^*_{VH} = \lv H^*_{VH} V^*_{VH} + \mu_VV^*_{VH}$ to re-write the second coefficient.

$EE^{VH}$ is asymptotically stable if the Jacobian's trace is negative and its determinant is positive. We have:
\begin{equation}
det(\mathcal{J}(V^*_{VH}, H^*_{VH})) = (\dfrac{H^*_{VH}}{\beta} - 1) \left( \Big(\dfrac{r_V}{K_V} + b\lv\Big) (H^*_{VH})^2 - b L_V \mu_V \right) \dfrac{r_H V^*_{VH}}{H^*_{VH}(H^*_{VH} + L_V)}
\end{equation}

\begin{equation}
Tr(\mathcal{J}(V^*_{VH}, H^*_{VH})) = -r_V  \dfrac{H^*_{VH}}{H^*_{VH}+L_V}\dfrac{V^*_{VH}}{K_V} - r_H(\dfrac{H^*_{VH}}{\beta} - 1)
\end{equation}

and using equation \eqref{equilibreVH:equationV}, we can show that $H^*_{VH}$ are solutions of
\begin{equation}
H^2 + H \left( \dfrac{-r_V(1 + \dfrac{c}{bK_V}) + \lv L_V + \mu_V}{\lv(1 + \dfrac{r_V}{\lv bK_V})} \right) + \dfrac{\mu_V L_V}{\lv(1 + \dfrac{r_V}{bK_V\lv})} = 0
\label{equilibreVH:equation H}
\end{equation}

Since $H^*_{VH}$ are positive solutions of \eqref{equilibreVH:equation H}, the term in $H$ is negative. Then, by using appendix \ref{appendice:ineq2nd}, we know that the term $\left( \Big(\dfrac{r_V}{K_V} + b\lv\Big) (H^*_{VH})^2 - b L_V \mu_V \right)$ is positive for $H^*_{VH, 2}$ and negative for $H^*_{VH, 1}$.
\\

So, on the first hand, the determinant of $\mathcal{J}(V^*_{VH}, H^*_{VH})$ is positive for $H^*_{V, c, 2}$ if $H^*_{VH, 2} > \beta$, and this condition also ensures that the trace is negative. In conclusion, $EE^{VH}_2$ is AS if $H^*_{VH, 2} > \beta$.
\\

On the other hand, the determinant of $\mathcal{J}(V^*_{VH}, H^*_{VH})$ is positive for $H^*_{VH, 1}$ if $H^*_{VH, 1} < \beta$.
The trace is negative at $EE^{VH}_1$  if 
\begin{subequations}
\begin{align}
&-r_V  \dfrac{H^*_{VH, 1}}{H^*_{VH}+L_V}\dfrac{V^*_{VH, 1}}{K_V} - r_H(\dfrac{H^*_{VH, 1}}{\beta} - 1) <0 \\
& -r_V  H^*_{VH, 1} \dfrac{H^*_{VH, 1} - c}{b K_V} - r_H \Big(\dfrac{H^*_{VH, 1}}{\beta} - 1\Big)\Big(H^*_{VH, 1}+L_V \Big) <0 \\
&(H^*_{VH, 1})^2 \Big(\dfrac{r_V}{K_Vb} + \dfrac{r_H}{\beta} \Big) + H^*_{VH, 1} \Big(-\dfrac{r_V c}{K_V	b} + \dfrac{r_H L_V}{\beta} - r_H \Big) - r_H L_V > 0 \\
&\tilde{Tr}(H^*_{VH, 1}) := (H^*_{VH, 1})^2\Big(\dfrac{r_V}{r_H}\dfrac{\beta}{K_Vb} + 1 \Big) + H^*_{VH, 1} \Big(L_V - \beta - c \dfrac{r_V}{r_H}\dfrac{\beta}{K_V	b}\Big) - \beta L_V > 0
\end{align}
\end{subequations}

We have $\tilde{Tr}(c) = r_H (c+ L_V) (\dfrac{c}{\beta}-1)$ and $\tilde{Tr}(\beta) = \beta^2 \dfrac{r_V}{K_V}(1 - \dfrac{c}{\beta})$

Since $c < b V^*_{VH, 1} + c = H^*_{VH, 1}$ and assuming $H^*_{VH, 1} < \beta$, we have $c < \beta$ and the polynomial $\tilde{Tr}$ has the following shape (figure \ref{fig:Tr tilde}).
\begin{figure}[ht]
\centering
\begin{tikzpicture}
\draw (0,-1.5)[->] -- (0,2.5)node[left]{$\tilde{Tr}(H)$};
\draw (-0.25, 0)[->] -- (5,0)node[below]{$H$};
\draw plot [domain=-0.25:4.5](\x, 0.1*\x^2 + 0.1*\x - 1);
\draw (0.25, -0.1) -- (0.25, 0.1) node[above] {$c$};
\draw (4.25, -0.1) -- (4.25, 0.1) node[above] {$\beta$};
\draw (2.7015, -0.1)node[below] {$H_{min}$} -- (2.7015, 0.1) ;
\end{tikzpicture}
\caption{\centering Curve of $\tilde{Tr}$, which has two real roots, one positive and one negative. We are looking for conditions such that $\tilde{Tr}(H_{VH,1}^*) > 0$, with $H_{VH,1}^*\in (c, \beta)$. }
\label{fig:Tr tilde}
\end{figure}

We can deduce that the trace of $\mathcal{J}(V^*_{VH}, H^*_{VH})$ is negative at $H^*_{VH, 1} $ if $H^*_{VH, 1} $ is higher than the largest root of $\tilde{Tr}$, named $H_{min}$.
\\

Finally, $EE^{VH}_1$ is AS if 
\begin{equation}
H_{min} < H^*_{VH, 1} < \beta
\end{equation}
where 
\begin{equation}
H_{min} := -\dfrac{L_V - \beta - \dfrac{c r_V}{r_H}\dfrac{\beta}{K_V	b}}{2\dfrac{r_V}{r_H}\dfrac{\beta}{K_Vb} + 2} + \dfrac{\sqrt{\Big(L_V - \beta - \dfrac{r_V}{r_H}\dfrac{c \beta}{K_V	b}\Big)^2 + 4 \beta L_V \Big(\dfrac{r_V\beta}{r_H K_V b} + 1\Big)}}{2\dfrac{r_V}{r_H}\dfrac{\beta}{K_Vb} + 2} 
\label{equilibreVH:Hmin}
\end{equation}
\end{itemize}

\subsubsection{Longterm dynamic}

\begin{table}[ht]
\centering
\caption{\centering Stability and existence for subsystem $V-H$ : condition in brackets are required conditions but implied by the other row's elements}
\begin{tabular}{c|c|c|c|c|c|c|c}
$T_H(0)$ & $T_V(\beta)$ & $T_V(c)$ & $\Delta_{VH}$ and $T_{V,i}$ & $T_H(V^*_{VH,1})$ & $T_{H,i}$ & $T_H(V^*_{VH,2})$ & Equilibrium \\
\hline
\multirow{3}{*}{$1<$} & & \multirow{2}{*}{$<1$} &$<1$& & &$(1<)$ &$TE, EE^H, EE^{VH}_2$ \\
 & & & & & & &$TE, EE^H$ \\
 \cline{2-8}
 & &$1<$ & & & &$(1<)$ &$TE, EE^{VH}_2$ \\
\hline
\multirow{10}{*}{$<1$} & \multirow{5}{*}{$<1$} & \multirow{3}{*}{$<1$} &\multirow{3}{*}{$<1$} & \multirow{2}{*}{$<1$} & \multirow{2}{*}{$<1$} & $1<$ & $TE, EE^H_\beta, EE^{VH}_1, EE^{VH}_2$ \\
 & & & & & & $<1$ & $TE, EE^H_\beta, EE^{VH}_1$ \\
 \cline{5-8}
 & & & & & & $1<$ & $TE, EE^H_\beta, EE^{VH}_2$ \\
 \cline{3-8}
 & & $1<$ & & & & $1<$ & $TE, EE^H_\beta, EE^{VH}_2$ \\
 \cline{3-8}
 & & & & & & & $TE, EE^H_\beta$ \\
 \cline{2-8}
& \multirow{5}{*}{$1<$} & \multirow{3}{*}{$<1$} &\multirow{3}{*}{$<1$} & \multirow{2}{*}{$<1$} & \multirow{2}{*}{$<1$} & $1<$ & $TE, EE^{VH}_\beta, EE^{VH}_1, EE^{VH}_2$ \\
 & & & & & & $<1$ & $TE, EE^{VH}_\beta, EE^{VH}_1$ \\
 \cline{5-8}
 & & & & & & $1<$ & $TE, EE^{VH}_\beta, EE^{VH}_2$ \\
 \cline{3-8}
 & & $1<$ & & & & $1<$ & $TE, EE^{VH}_\beta, EE^{VH}_2$ \\
 \cline{3-8}
 & & & & & & & $TE, EE^{VH}_\beta$ \\
 \hline
\end{tabular}
\end{table}

\subsubsection{Limit cycle}
The same study than the one done for the $F-H$ system shows that no limit cycle exists for $V-H$ system.


\subsection{Overall model}
\subsubsection{Equilibrium}
The equilibrium of the overall model are obtained from the equilibrium of the sub-models, to which we add the endemic equilibrium. We obtain the following list (see above for the existence and values)

\begin{itemize}
\item $TE = (0,0,0)$
\item A forest equilibrium $EE^F = \Big(F^*, 0, 0 \Big) $.
\item A human equilibrium $EE^H_\beta = \Big(0,0,\beta \Big)$
\item A forest-human equilibrium $EE^{FH}_\beta = \Big(F^*_\beta, 0, \beta\Big)$ where $F^*_\beta = K_F \Big(1-\dfrac{\mu_F + \beta(\lf + \omega f)}{r_F} \Big)$.
\item A vegetation-human equilibrium $EE^{VH}_\beta = \Big(0, V^*_{\beta}, \beta \Big)$.
\item An endemic equilibrium $EE_\beta^{FVH} = \Big(F^*_\beta, V^*_{F^*_\beta, \beta}, \beta \Big)$ where $\VbetaF = K_V \Big(1- \dfrac{\beta +L_V}{\beta} \dfrac{\alpha \Fbeta + \lv \beta + \mu_V}{r_V} \Big)$. This equilibrium exists  when $\dfrac{r_F}{\mu_F + \beta(\lf + \omega f)} > 1$ and $\dfrac{\beta}{\beta + L_V} \dfrac{r_V}{\alpha \Fbeta + \lv \beta + \mu_V} > 1$.


\item A human equilibrium $EE^H = \Big(0,0,c\Big)$.
\item A forest-human equilibrium $EE^{FH} = \Big(F^*_{FH}, 0, a F^*_{FH} + c \Big)$ where $F^*_{FH} = K_F \left( \dfrac{1 - \dfrac{\mu_F}{r_F} - c \dfrac{\omega f + \lf}{r_F}}{1 + a K_F \dfrac{\omega f + \lf}{r_F}} \right)$.
\item One or two vegetation-human equilibrium $EE^{VH}_i = \Big(0, V^*_{VH, i}, bV^*_{VH, i}+c \Big)$.
\item A endemic equilibrium $EE^{FVH} = \Big(F^*_{FH} - \dfrac{b K_F (\FHterme) / r_F}{1 + aK_F \dfrac{\FHterme}{r_F}} V^*_{FVH}, V^*_{FVH}, H^*_{FH}+ \dfrac{b}{1 + aK_F \dfrac{\FHterme}{r_F}}V^*_{FVH} \Big)$. $V^*_{FH, c}$ are solutions of 
\end{itemize}

\newpage
\begin{landscape}

%\begin{multline}
%V^2 + V \dfrac{ \dfrac{(b-ac_1^F)K_V - (aF^*_{FH} + c)}{(b-ac_1^F)}}{1  +\dfrac{K_V(\lv(b-ac_1^F) - \alpha c_1^F)}{r_V}} \left(\dfrac{(b-ac_1^F)K_V}{(b-ac_1^F)K_V - (aF^*_{FH} + c)}\dfrac{\mu_V + \lv(2(aF^*_{FH} + c) + L_V) + \alpha \Big(F^*_{FH} - \dfrac{c_1^F (aF^*_{FH} + c + L_V)}{b-ac_1^F}\Big)}{r_V} -1 \right)  + \\
%\dfrac{K_V(aF_c^* + c)}{b-ac_1^F} \dfrac{\Big(\dfrac{aF^*_{FH} + c + L_V}{aF^*_{FH} + c} \dfrac{\mu_V + \alpha F^*_{FH} + \lv (aF^*_{FH} + c)}{r_V} - 1\Big)}{1 + \dfrac{K_V(\lv(b-ac_1^F) - \alpha c_1^F)}{r_V}} = 0
%\label{equilibreFVH:equationV0}
%\end{multline}

%\begin{multline}
%V^2 + V \dfrac{\left(K_V \left(
%\dfrac{\lv(2H^*_{FH} + L_V) + \mu_V + \alpha \Big(F^*_{FH} - K_F \dfrac{\FHterme}{r_F}(H^*_{FH} + L_V)\Big)}{r_V} -1 \right)
%+ \dfrac{a}{b}K_F(1-\dfrac{\mu_F}{r_F}) + \dfrac{c}{b}
% \right)}{1  +\dfrac{K_Vb\lv}{r_V}\dfrac{1- \dfrac{\alpha}{\lv} K_F \dfrac{\FHterme}{r_F}}{1 + a K_F\dfrac{\FHterme}{r_F}}}
%  + \\
%\dfrac{K_VH^*_{FH}}{\dfrac{b}{1 + aK_F \dfrac{\FHterme}{r_F}}} \dfrac{\Big(\dfrac{H^*_{FH} + L_V}{H^*_{FH}} \dfrac{\mu_V + \alpha F^*_{FH} + \lv H^*_{FH}}{r_V} - 1\Big)}{1  +\dfrac{K_Vb\lv}{r_V}\dfrac{1- \dfrac{\alpha}{\lv} K_F \dfrac{\FHterme}{r_F}}{1 + a K_F\dfrac{\FHterme}{r_F}}} = 0
%\label{equilibreFVH:equationV}
%\end{multline}

\begin{multline}
V^2 + V \dfrac{K_V \Big(
\dfrac{\big(\lv - \alpha K_F \dfrac{\FHterme}{r_F}\big)\big(2H^*_{FH} + L_V\big) + \mu_V + \alpha K_F \big(1-\dfrac{\mu_F}{r_F}\big)}{r_V} -1 \Big)
+ \dfrac{a K_F(1-\dfrac{\mu_F}{r_F}) + c}{b}
 }{1  +\dfrac{K_Vb\lv}{r_V}\dfrac{1- \dfrac{\alpha}{\lv} K_F \dfrac{\FHterme}{r_F}}{1 + a K_F\dfrac{\FHterme}{r_F}}}
  + \\
\dfrac{K_V\Big(a K_F(1-\dfrac{\mu_F}{r_F}) + c\Big)}{b} \dfrac{\Big(\dfrac{H^*_{FH} + L_V}{H^*_{FH}} \dfrac{\mu_V + \alpha F^*_{FH} + \lv H^*_{FH}}{r_V} - 1\Big)}{1  +\dfrac{K_Vb\lv}{r_V}\dfrac{1- \dfrac{\alpha}{\lv} K_F \dfrac{\FHterme}{r_F}}{1 + a K_F\dfrac{\FHterme}{r_F}}} = 0
\label{equilibreFVH:equationV}
\end{multline}


The discriminant of this equation, $\Delta_{FVH}$, is positive if
%\begin{multline}
%\Delta_{FVH} \geq 0 \Leftrightarrow 
%\dfrac{\Big((b-ac_1^F)K_V - (aF^*_{FH} + c)\Big)^2}{K_V(aF_c^* + c)(b-ac_1^F)}\left(\dfrac{(b-ac_1^F)K_V}{(b-ac_1^F)K_V - (aF^*_{FH} + c)}\dfrac{\mu_V + \lv(2(aF^*_{FH} + c) + L_V) + \alpha \Big(F^*_{FH} - \dfrac{c_1^F (aF^*_{FH} + c + L_V)}{b-ac_1^F}\Big)}{r_V} -1 \right)^2 \geq \\
%4 (1  +\dfrac{K_V(\lv(b-ac_1^F) - \alpha c_1^F)}{r_V})
%\Big(\dfrac{aF^*_{FH} + c + L_V}{aF^*_{FH} + c} \dfrac{\mu_V + \alpha F^*_{FH} + \lv (aF^*_{FH} + c)}{r_V} - 1\Big)
%\label{equilibreFVH:discriminant0}
%\end{multline}

\begin{multline}
\Delta_{FVH} \geq 0 \Leftrightarrow 
\left(K_V \Big(
\dfrac{\big(\lv - \alpha K_F \dfrac{\FHterme}{r_F}\big)\big(2H^*_{FH} + L_V\big) + \mu_V + \alpha K_F \big(1-\dfrac{\mu_F}{r_F}\big)}{r_V} -1 \Big)
+ \dfrac{a K_F(1-\dfrac{\mu_F}{r_F}) + c}{b}\right) ^ 2 \geq 4 \dfrac{K_V\Big(aK_F(1-\dfrac{\mu_F}{r_F})+ c\Big)}{b} \\
\left( 1  +\dfrac{K_Vb\lv}{r_V}\dfrac{1- \dfrac{\alpha}{\lv} K_F \dfrac{\FHterme}{r_F}}{1 + a K_F\dfrac{\FHterme}{r_F}} \right) 
\left(\dfrac{H^*_{FH} + L_V}{H^*_{FH}} \dfrac{\mu_V + \alpha F^*_{FH} + \lv H^*_{FH}}{r_V} - 1\right)
\label{equilibreFVH:discriminant}
\end{multline}
\end{landscape}

First, \textbf{we assume}
\begin{subequations}
\begin{align}
& 0 < 1  +\dfrac{K_Vb\lv}{r_V}\dfrac{1- \dfrac{\alpha}{\lv} K_F \dfrac{\FHterme}{r_F}}{1 + a K_F\dfrac{\FHterme}{r_F}} \\
&\Leftrightarrow \alpha < \dfrac{r_F r_V}{K_V b K_F (\FHterme)} + \dfrac{a r_V}{bK_V} + \dfrac{\lv r_F}{K_F (\FHterme)}
\end{align}
\end{subequations}


If the right hand side of \eqref{equilibreFVH:discriminant} is negative, \textit{ie} if $1 < \dfrac{H^*_{FH}}{H^*_{FH} + L_V} \dfrac{r_V}{\mu_V + \alpha F^*_{FH} + \lv H^*_{FH}}$ then $\Delta_{FVH}$ is positive and the constant coefficient of equation \eqref{equilibreFVH:equationV} is negative. This implies the existence of a unique positive root, noted $V^*_{FVH, 2}$.

Otherwise, if $\dfrac{H^*_{FH}}{H^*_{FH} + L_V} \dfrac{r_V}{\mu_V + \alpha F^*_{FH} + \lv H^*_{FH}} < 1$, in addition of \eqref{equilibreFVH:discriminant}, the parameters must also verify
%(we will use $H^*_{FH} (1 + aK_F\dfrac{\FHterme}{r_F}) = aK_F(1-\mu_F/r_F) + c$)
%\begin{subequations}
%{\footnotesize
%\begin{align}
%& b K_V \left(\dfrac{\big(\lv - \alpha K_F \dfrac{\FHterme}{r_F}\big)\big(2H^*_{FH} + L_V\big) + \mu_V + \alpha K_F \big(1-\dfrac{\mu_F}{r_F}\big)}{r_V} -1 \right)
%+ aK_F(1-\dfrac{\mu_F}{r_F}) + c < 0 \\
%&  \Big(\dfrac{2b K_V \lv}{r_V}(1 - \dfrac{\alpha}{\lv} K_F\dfrac{\FHterme}{r_F}) + (1 + a K_F \dfrac{\FHterme}{r_F})\Big)H^*_{FH} < bK_V \Big(1 - \dfrac{\mu_V + \alpha K_F (1-\dfrac{\mu_F}{r_F}) + L_V \lv \big( 1- \dfrac{\alpha K_F}{\lv} \dfrac{\FHterme}{r_F}\big)}{r_V} \Big) \\
%&  \Big(1+\dfrac{2b K_V \lv}{r_V}\dfrac{1 - \dfrac{\alpha}{\lv} K_F\dfrac{\FHterme}{r_F}}{1 + a K_F \dfrac{\FHterme}{r_F}}\Big) < \dfrac{bK_V}{aK_F (1 - \dfrac{\mu_F}{r_F}) + c} \Big(1 - \dfrac{\mu_V + \alpha K_F (1-\dfrac{\mu_F}{r_F}) + L_V \lv \big( 1- \dfrac{\alpha K_F}{\lv} \dfrac{\FHterme}{r_F}\big)}{r_V} \Big)
%\end{align}
%}
%\end{subequations}
\begin{subequations}
{\footnotesize
\begin{align}
& b K_V \left(\dfrac{\big(\lv - \alpha K_F \dfrac{\FHterme}{r_F}\big)\big(2H^*_{FH} + L_V\big) + \mu_V + \alpha K_F \big(1-\dfrac{\mu_F}{r_F}\big)}{r_V} -1 \right)
+ aK_F(1-\dfrac{\mu_F}{r_F}) + c < 0 \\
& 1 < \dfrac{bK_V}{aK_F(1-\dfrac{\mu_F}{r_F}) + c} \left( 1 - \dfrac{\big(\lv - \alpha K_F \dfrac{\FHterme}{r_F}\big)\big(2H^*_{FH} + L_V\big) + \mu_V + \alpha K_F \big(1-\dfrac{\mu_F}{r_F}\big)}{r_V}\right)
\end{align}}
\end{subequations}

for the equation \eqref{equilibreFVH:equationV} has two positive roots, noted $V^*_{FVH, 1}$ and $V^*_{FVH, 2}$.

A last condition of existence is given by $F^*_{FVH}$ which must be positive. It is the case if
\begin{equation}
1 < \dfrac{r_F}{\mu_F + (\FHterme)(bV^*_{FVH} + c)}
\end{equation}

So, if:
\begin{subequations}
    \begin{empheq}[left={\empheqlbrace\,}]{align}
&\alpha < \dfrac{r_F r_V}{K_V b K_F (\FHterme)} + \dfrac{a r_V}{bK_V} + \dfrac{\lv r_F}{K_F (\FHterme)} \\
&1 < \dfrac{H^*_{FH}}{H^*_{FH} + L_V} \dfrac{r_V}{\mu_V + \alpha F^*_{FH} + \lv H^*_{FH}} \\
&1 < \dfrac{r_F}{\mu_F + (\FHterme)(bV^*_{FVH} + c)} 
    \end{empheq}
    \label{equilibreFVH:existence11}
\end{subequations}
there is one endemic equilibria. And if
\begin{subequations}
    \begin{empheq}[left={\empheqlbrace\,}]{align}
&\alpha < \dfrac{r_F r_V}{K_V b K_F (\FHterme)} + \dfrac{a r_V}{bK_V} + \dfrac{\lv r_F}{K_F (\FHterme)} \\
&\dfrac{H^*_{FH}}{H^*_{FH} + L_V} \dfrac{r_V}{\mu_V + \alpha F^*_{FH} + \lv H^*_{FH}} < 1 \\
& 0 \leq \Delta_{FVH} \text{ see equation \eqref{equilibreFVH:discriminant}}\\
&  1 < \dfrac{bK_V \Big( 1 - \dfrac{\big(\lv - \alpha K_F \dfrac{\FHterme}{r_F}\big)\big(2H^*_{FH} + L_V\big) + \mu_V + \alpha K_F \big(1-\dfrac{\mu_F}{r_F}\big)}{r_V}\Big)}{aK_F(1-\dfrac{\mu_F}{r_F}) + c}  \\
&1 < \dfrac{r_F}{\mu_F + (\FHterme)(bV^*_{FVH} + c)} 
    \end{empheq}
    \label{equilibreFVH:existence12}
\end{subequations}
there are two endemic equilibrium.

Now, \textbf{we assume}:
\begin{equation}
 \dfrac{r_F r_V}{K_V b K_F (\FHterme)} + \dfrac{a r_V}{bK_V} + \dfrac{\lv r_F}{K_F (\FHterme)} < \alpha
\end{equation}

The same study than above leads to the following conditions. If

\begin{subequations}
    \begin{empheq}[left={\empheqlbrace\,}]{align}
&\dfrac{r_F r_V}{K_V b K_F (\FHterme)} + \dfrac{a r_V}{bK_V} + \dfrac{\lv r_F}{K_F (\FHterme)} < \alpha \\
&\dfrac{H^*_{FH}}{H^*_{FH} + L_V} \dfrac{r_V}{\mu_V + \alpha F^*_{FH} + \lv H^*_{FH}} < 1\\
&1 < \dfrac{r_F}{\mu_F + (\FHterme)(bV^*_{FVH} + c)} 
    \end{empheq}
    \label{equilibreFVH:existence21}
\end{subequations}
there is one endemic equilibria. And if
\begin{subequations}
    \begin{empheq}[left={\empheqlbrace\,}]{align}
&\dfrac{r_F r_V}{K_V b K_F (\FHterme)} + \dfrac{a r_V}{bK_V} + \dfrac{\lv r_F}{K_F (\FHterme)} < \alpha \\
&1 < \dfrac{H^*_{FH}}{H^*_{FH} + L_V} \dfrac{r_V}{\mu_V + \alpha F^*_{FH} + \lv H^*_{FH}}\\
& 0 \leq \Delta_{FVH} \text{ see equation \eqref{equilibreFVH:discriminant}}\\
&\dfrac{bK_V \Big( 1 - \dfrac{\big(\lv - \alpha K_F \dfrac{\FHterme}{r_F}\big)\big(2H^*_{FH} + L_V\big) + \mu_V + \alpha K_F \big(1-\dfrac{\mu_F}{r_F}\big)}{r_V}\Big)}{aK_F(1-\dfrac{\mu_F}{r_F}) + c} < 1  \\
&1 < \dfrac{r_F}{\mu_F + (\FHterme)(bV^*_{FVH} + c)} 
    \end{empheq}
    \label{equilibreFVH:existence22}
\end{subequations}
there are two endemic equilibrium.





\subsubsection{Stability}

The Jacobian matrix of the system is defined by :
{\footnotesize
\begin{multline}
\mathcal{J}(F,V,H) = \\ \begin{bmatrix}
r_F \Big(1-\dfrac{2F}{K_F} \Big) - \mu_F - (\lf + \omega f)H & 0 & -(\lf + \omega f)F \\
-\alpha V & r_V \dfrac{H}{H+L_V}(1-\dfrac{2V}{K_V}) - \alpha F - \lv H - \mu_V & r_V \dfrac{L_V}{(H+L_V)^2}(1-\dfrac{V}{K_V})V  - \lv V\\
r_H \dfrac{aH^2}{(aF+bV+c)^2} (\dfrac{H}{\beta}-1) &r_H \dfrac{bH^2}{(aF+bV+c)^2} (\dfrac{H}{\beta}-1) & r_H(1-\dfrac{H}{aF+bV+c})(\dfrac{2H}{\beta}-1) - \dfrac{r_H H (\dfrac{H}{\beta}-1)}{aF+bV+c}
\end{bmatrix}
\label{stab:jacobian}
\end{multline}
}

To assess the stability of the equilibrium we look at the eigenvalues of the Jacobian. Eigenvalues for all the equilibrium, except for the endemic ones, can be easily computed using previous parts.

\begin{itemize}
\item At point $TE = (0,0, 0)$, the eigenvalues are $r_F - \mu_F$, $-\mu_V$ and $-r_H$. $TE$ is asymptotically stable (AS) if $\NF < 1$.

\item At point $EE^{F} = (F^*, 0, 0)$, 
%the Jacobian is 
%\begin{equation}
%\mathcal{J}(F^F, 0, 0)
%\begin{bmatrix}
%-r_F \dfrac{F^F}{K_F}  & 0 & -(\FHterme)F^F \\
%0 & -\alpha F^F - \mu_V & 0 \\
%0 & 0 & -r_H 
%\end{bmatrix}
%\end{equation}
%and thus 
the eigenvalues are all negative. $EE^F$ is asymptotically stable.

\item $EE^H_\beta = \Big(0,0,\beta \Big)$, 
%the Jacobian is
%\begin{equation}
%\mathcal{J}(0, 0, H^*_\beta) = \begin{bmatrix}
%r_F - \mu_F - (\FHterme) H^*_\beta & 0 & 0 \\
%0 & r_V\dfrac{H^*_\beta}{H^*_\beta+L_V} - \lv H^*_\beta - \mu_V & 0 \\
%0 & 0 & r_H (1 - \dfrac{H^*_\beta}{c})
%\end{bmatrix}
%\end{equation}
%We used $H^*_\beta = \beta$ several times. $EE^H_\beta$ is AS if
is AS if

\begin{subequations}
    \begin{empheq}[left={\empheqlbrace\,}]{align}
    & \dfrac{r_F}{\mu_F + \beta (\FHterme)} < 1 \\
    & \dfrac{r_V}{\mu_V + \lv \beta} \dfrac{\beta}{\beta + L_V} < 1  \\
    & \dfrac{c}{\beta} < 1
    \end{empheq}
\end{subequations}

\item Point $EE^{FH}_\beta = \Big(F^*_\beta,0,\beta \Big)$,
% the Jacobian is
%\begin{equation}
%\mathcal{J}(F^*_\beta, 0, H^*_\beta) = \begin{bmatrix}
%-r_F \dfrac{F^*_\beta}{K_F} & 0 & -(\FHterme) F^*_\beta \\
%0 & r_V\dfrac{H^*_\beta}{H^*_\beta+L_V} - \lv H^*_\beta - \alpha F^*_\beta - \mu_V & 0 \\
%0 & 0 & r_H (1 - \dfrac{H^*_\beta}{c})
%\end{bmatrix}
%\end{equation}
%We used $H^*_\beta = \beta$ several times.  $EE^{FH}_\beta$ is AS if
is AS if
\begin{subequations}
    \begin{empheq}[left={\empheqlbrace\,}]{align}
    &\dfrac{\beta}{\beta + L_V} \dfrac{r_V}{\mu_V + \lv \beta + \alpha F^*_\beta}  < 1 \\
    & \dfrac{c}{\beta} < 1
    \end{empheq}
\end{subequations}


\item Point $EE^{VH}_\beta = \Big(0,V^*_\beta,\beta \Big)$,
% the Jacobian is
%\begin{equation}
%\mathcal{J}(0, V^*_\beta, H^*_\beta) = \begin{bmatrix}
%r_F - \mu_F - (\FHterme) H^*_\beta & 0 & 0 \\
%-\alpha V^*_\beta & - r_V\dfrac{H^*_\beta}{H^*_\beta+L_V} V^*_\beta & \_ \\
%0 & 0 & r_H (1 - \dfrac{H^*_\beta}{c})
%\end{bmatrix}
%\end{equation}
%We used $H^*_\beta = \beta$ several times. Eigenvalues appear on the diagonal. $EE^{VH}_\beta$
 is AS if
\begin{subequations}
    \begin{empheq}[left={\empheqlbrace\,}]{align}
    & \dfrac{r_F}{\mu_F + \beta (\FHterme)} < 1 \\
    & \dfrac{c}{\beta} <1 
    \end{empheq}
\end{subequations}

\item At point $EE^{FVH}_\beta = \Big(\Fbeta,\VbetaF, \beta \Big)$, the Jacobian is
\begin{equation}
\mathcal{J}(\Fbeta, \VbetaF, \beta) = \begin{bmatrix}
-r_F \dfrac{\Fbeta}{K_F} & 0 & -(\FHterme)\Fbeta \\
-\alpha \VbetaF & - r_V\dfrac{\beta}{\beta+L_V} \VbetaF & \_ \\
0 & 0 & r_H (1 - \dfrac{\beta}{c})
\end{bmatrix}
\end{equation}
Eigenvalues appear on the diagonal. $EE^{FVH}_\beta$ is AS if
\begin{equation}
\dfrac{c}{\beta} < 1
\end{equation}

\item Point $EE^{H} = \Big(0,0,c \Big)$
%the Jacobian is
%\begin{equation}
%\mathcal{J}(0,0,H^*_c) = \begin{bmatrix}
%r_F - \mu_F - H^*_c(\FHterme) & 0 & 0 \\
%0 & r_V \dfrac{H^*_c}{L_V + H^*_c} - \lv H^*_c - \mu_V & 0 \\
%r_H a (\dfrac{H^*_c}{\beta} - 1) & r_H b (\dfrac{H^*_c}{\beta} - 1) & -r_H(\dfrac{H^*_c}{\beta} - 1)
%\end{bmatrix}
%\end{equation}
%We used $H^*_c = c$ several times. $EE^{H}$
 is AS if
\begin{subequations}
    \begin{empheq}[left={\empheqlbrace\,}]{align}
    &\dfrac{r_F}{\mu_F + c (\FHterme)} < 1 \\
    & \dfrac{r_V}{\mu_V + c \lv} \dfrac{c}{c + L_V} < 1  \\
    & 1< \dfrac{c}{\beta} 
    \end{empheq}
\end{subequations}

\item Point $EE^{FH} = \Big(F^*_{FH}, 0, aF^*_{FH} + c)$
%, the Jacobian is
%\begin{equation}
%\mathcal{J}(F^*_{VH}, 0, H^*_{F, c}) = \begin{bmatrix}
%-r_F \dfrac{F^*_{H, c}}{K_F} & 0 & -(\FHterme) F^*_{H, c} \\
%0 & r_V \dfrac{H^*_{F, c}}{L_V + H^*_{F, c}} - \lv H^*_{F, c} -\alpha F^*_{H, c} - \mu_V & 0 \\
%r_H a (\dfrac{H^*_{F, c}}{\beta} - 1) & r_H b (\dfrac{H^*_{F, c}}{\beta} - 1) & -r_H(\dfrac{H^*_{F, c}}{\beta} - 1)
%\end{bmatrix}
%\end{equation}
%We used $H^*_{F, c} = a F^*_{H, c} c$ several times. Eigenvalues are the roots of 
%
%\begin{multline}
%\chi = \Big(X - r_V \dfrac{H^*_{F, c}}{L_V + H^*_{F, c}} + \lv H^*_{F, c} +\alpha F^*_{H, c} + \mu_V \Big) \times \\
%\left(X^2 + \Big(r_F \dfrac{F^*_{H, c}}{K_F} + r_H \big(\dfrac{H^*_{F, c}}{\beta} - 1\big) \Big)X + r_f \dfrac{F^*_{H, c}}{K_F}r_H \big(\dfrac{H^*_{F, c}}{\beta} - 1)\big) + r_H a \big(\dfrac{H^*_{F, c}}{\beta} - 1\big) (\FHterme) F^*_{H, c} \right)
%\end{multline}
%
%One root is $r_V \dfrac{H^*_{F, c}}{L_V + H^*_{F, c}} - \lv H^*_{F, c} -\alpha F^*_{H, c} -  \mu_V$. The two others have negative real parts if the coefficients of the second degree polynomial are both positive, which is the case if $H^*_{F, c} > \beta$.
%Finally, $EE^{FH}$ 
is AS if
\begin{subequations}
    \begin{empheq}[left={\empheqlbrace\,}]{align}
    & \dfrac{r_V}{\mu_V + H^*_{F} \lv +\alpha F^*_{FH}} \dfrac{H^*_{F}}{H^*_{F} + L_V} < 1 \\
    &1 < \dfrac{a F^*_{FH} + c}{\beta}
    \end{empheq}
\end{subequations}
with $F^*_{FH} = K_F \left( \dfrac{1 - \dfrac{\mu_F}{r_F} - c \dfrac{\omega f + \lf}{r_F}}{1 + a K_F \dfrac{\omega f + \lf}{r_F}} \right)$ and $H^*_{FH} = a F^*_{FH} + c$.


\item Point $EE^{VH}_1 = \Big(0, V^*_{VH, 1}, H^*_{VH, 1})$ is AS if 
\begin{subequations}
    \begin{empheq}[left={\empheqlbrace\,}]{align}
&\dfrac{r_F}{ \mu_F + (\FHterme)H^*_{VH, 1}} < 1 \\
&H_{min} < H^*_{VH, 1} < \beta
    \end{empheq}
\end{subequations}
where where $V^*_{VH, 1}$ is given by equation \eqref{equilibreVH:V}, $H^*_{VH, 1} =  bV^*_{VH, 1} + c$ and $H_{min}$ is given by equation \eqref{equilibreVH:Hmin}.
\item Point $EE^{VH}_2 = \Big(0, V^*_{VH, 2}, H^*_{VH, 2})$
%the Jacobian is
%\begin{equation}
%\mathcal{J}(0, V^*_{VH}, H^*_{VH}) = \begin{bmatrix}
%r_F - \mu_F - (\FHterme)H^*_{VH}  & 0 & 0 \\
%-\alpha V^*_{VH} & -r_V\dfrac{H^*_{VH}}{H^*_{VH} + L_V} \dfrac{V^*_{VH}}{K_V} & \Big(\dfrac{L_V\mu_V}{H^*_{VH}} - \lv H^*_{VH}) \dfrac{V^*_{VH}}{H^*_{VH} + L_V} \\
%r_H a (\dfrac{H^*_{VH}}{\beta} - 1) & r_H b (\dfrac{H^*_{VH}}{\beta} - 1) & -r_H(\dfrac{H^*_{VH}}{\beta} - 1)
%\end{bmatrix}
%\end{equation}
%We used $H^*_{VH} = b V^*_{VH} + c$ several times. Eigenvalues are the roots of 
%\begin{multline}
%\chi = \Big(X - r_F + \mu_F + (\FHterme)H^*_{VH} \Big) \times \\
%\left(X^2 + \Big(r_V  \dfrac{H^*_{VH}}{H^*_{VH}+L_V}\dfrac{V^*_{VH}}{K_V} + r_H(\dfrac{H^*_{VH}}{\beta} - 1)\Big)X + \right.\\ \left.
% r_V  \dfrac{H^*_{VH}}{H^*_{VH}+L_V}\dfrac{V^*_{VH}}{K_V} r_H(\dfrac{H^*_{VH}}{\beta} - 1) + r_Hb(\dfrac{H^*_{VH}}{\beta} - 1) \Big(\lv H^*_{VH} - \dfrac{L_V \mu_V}{H^*_{VH}} \Big) \dfrac{V^*_{VH}}{H^*_{VH} + L_V} \right)
% \label{stab:chi VH}
%\end{multline}
%
is AS if
\begin{subequations}
    \begin{empheq}[left={\empheqlbrace\,}]{align}
&\dfrac{r_F}{ \mu_F + (\FHterme)H^*_{VH, 2}} < 1 \\
&1 < \dfrac{H^*_{VH,2}}{\beta}
    \end{empheq}
\end{subequations}
where $V^*_{VH, 2}$ is given by equation \eqref{equilibreVH:V} and $H^*_{VH, 2} =  bV^*_{VH, 2} + c$.


\item At point $EE^{FVH} = (F^*_{FVH}, V^*_{FVH},H^*_{FVH})$, the Jacobian is 
\begin{multline}
\mathcal{J}(F^*_{FVH}, V^*_{FVH}, H^*_{FVH}) = \\
\begin{bmatrix}
-r_F \dfrac{F^*_{FVH}}{K_F} &0 &-(\FHterme) F^*_{FVH} \\
-\alpha V^*_{FVH} & -r_V \dfrac{H^*_{FVH}}{H^*_{FVH} + L_V}\dfrac{V^*_{FVH}}{K_V} & \Big(\dfrac{(\alpha F^*_{FVH} + \mu_V)L_V}{H^*_{FVH}} - \lv H^*_{FVH} \Big)\dfrac{V^*_{FVH}}{H^*_{FVH}+L_V} \\
a r_H \Big(\dfrac{H^*_{FVH}}{\beta} - 1\Big) & b r_H \Big(\dfrac{H^*_{FVH}}{\beta} - 1\Big) & -r_H\Big(\dfrac{H^*_{FVH}}{\beta} - 1\Big)
\end{bmatrix}
\end{multline}

We used $r_V \dfrac{H^*_{FVH}}{H^*_{FVH} + L_V} \Big(1 - \dfrac{V^*_{FVH}}{K_V} \Big)V^*_{FVH} = \lv H^*_{FVH} V^*_{FVH} + \mu_VV^*_{FVH} + \alpha F^*_{FVH}V^*_{FVH} $.
The characteristic polynomial of $\mathcal{J}(F^*_{FVH}, V^*_{FVH}, H^*_{FVH})$ is 
\begin{equation}
\chi_{FVH} = X^3 + q_2X^2 + q_1 X + q_0
\end{equation}
with 
\begin{subequations}
\begin{align}
&q_2 = \left(r_F \dfrac{F^*_{FVH}}{K_F} + r_V \dfrac{H^*_{FVH}}{H^*_{FVH} + L_V}\dfrac{V^*_{FVH}}{K_V} + r_H\Big(\dfrac{H^*_{FVH}}{\beta} - 1\Big) \right)\\
&q_1 = \left(r_F \dfrac{F^*_{FVH}}{K_F}r_V \dfrac{H^*_{FVH}}{H^*_{FVH} + L_V}\dfrac{V^*_{FVH}}{K_V} +r_F \dfrac{F^*_{FVH}}{K_F} r_H\Big(\dfrac{H^*_{FVH}}{\beta} - 1\Big) + \right. \\ \nonumber &\left. r_H\Big(\dfrac{H^*_{FVH}}{\beta} - 1\Big)r_V \dfrac{H^*_{FVH}}{H^*_{FVH} + L_V}\dfrac{V^*_{FVH}}{K_V} + (\FHterme) F^*_{FVH}a r_H \Big(\dfrac{H^*_{FVH}}{\beta} - 1\Big) - \right. \\ \nonumber &\left. b r_H \Big(\dfrac{H^*_{FVH}}{\beta} - 1\Big)\Big(\dfrac{(\alpha F^*_{FVH} + \mu_V)L_V}{H^*_{FVH}} - \lv H^*_{FVH} \Big)\dfrac{V^*_{FVH}}{H^*_{FVH}+L_V} \right) \\
&q_0 = r_F r_H \dfrac{F^*_{FVH}}{K_F} \dfrac{bV^*_{FVH}}{H^*_{FVH}(H^*_{FVH}+L_V)}\Big(\dfrac{H^*_{FVH}}{\beta}-1\Big) \times \\ \nonumber
&\left( (H^*_{FVH})^2 \Big(\dfrac{r_V}{bK_V}\Big( 1 + aK_F \dfrac{\FHterme}{r_F} \Big)+ \lv - \alpha K_F \dfrac{\FHterme}{r_F} \Big) - 
\right. \\ \nonumber & \left.
H^*_{FVH} L_V \alpha K_F \dfrac{\FHterme}{r_F} - \alpha (F^*_{FVH} +\mu_V)L_V \right)
\end{align}
\end{subequations}

According to the Ruth-Hurwitz criterion, the equilibrium is AS if $q_2 > 0$, $q_0 > 0$ and $q_2 q_1 - q_0 > 0$. We will look for the sign of those quantity.

Using $F^*_{FVH} = K_F\Big(1 - \dfrac{\mu_F}{r_F} \Big) - K_F\dfrac{\FHterme}{r_F} H^*_{FVH}$, we can simplify the expression of $q_0$:
\begin{multline}
q_0 = r_F r_H \dfrac{F^*_{FVH}}{K_F} \dfrac{bV^*_{FVH}}{H^*_{FVH}(H^*_{FVH}+L_V)}\Big(\dfrac{H^*_{FVH}}{\beta}-1\Big) \times \\
\left( \Big(H^*_{FVH}\Big)^2 \Big(\dfrac{r_V}{bK_V}\Big(1 + aK_F \dfrac{\FHterme}{r_F}\Big) + \lv - \alpha K_F \dfrac{\FHterme}{r_F} \Big)
 - L_V \Big(\alpha K_F\big(1 - \dfrac{\mu_F}{r_F} \big) + \mu_V\Big)  \right)
 \label{equilibreFVH:q0}
\end{multline}

Using equation \eqref{equilibreFVH:equationV} and $V^*_{FVH} = \dfrac{H^*_{FVH} - aF^*_{FVH} - c}{b}$, we find that $H^*_{FVH}$ is solution of
%\begin{multline}
%H^2 \Big(\dfrac{r_V}{bK_V}(1 + a K_F \dfrac{\FHterme}{r_F}) + \lv - \alpha K_F\dfrac{\FHterme}{r_F} \Big) - \\ H \Big(r_V + \dfrac{r_V}{b K_V} (c + a K_F\big(1-\dfrac{\mu_F}{r_F}\big) ) - \mu_V - \alpha K_F \big(1 - \dfrac{\mu_F}{r_F}\big) - L_V \lv + L_V \alpha K_F \dfrac{\FHterme}{r_F} \Big) + \\ L_V \Big(\mu_V + \alpha K_F \big(1 - \dfrac{\mu_F}{r_F} \big)\Big) = 0
%\label{equilibreFVH:equationH0}
%\end{multline}

\begin{multline}
H^2 \left(1 + \dfrac{bK_V\lv}{r_V}\dfrac{1- \dfrac{\alpha}{\lv} K_F \dfrac{\FHterme}{r_F}}{1 + a K_F\dfrac{\FHterme}{r_F}} \right) - \\ 
H \left( \dfrac{c + a K_F \Big(1 - \dfrac{\mu_F}{r_F}\Big)}{1 + a K_F\dfrac{\FHterme}{r_F}} +  \dfrac{bK_V \Big( 1 - \dfrac{\lv L_V  + \mu_V + \alpha K_F \Big(1 - \dfrac{\mu_F}{r_F} - \dfrac{\FHterme}{r_F}L_V \Big)}{r_V}\Big)}{1 + a K_F\dfrac{\FHterme}{r_F}} \right) + \\ 
\dfrac{L_V bK_V}{r_V (1 + aK_F \dfrac{\FHterme}{r_F})} \Big(\mu_V + \alpha K_F \big(1 - \dfrac{\mu_F}{r_F} \big)\Big) = 0
\label{equilibreFVH:equationH}
\end{multline}

Once again, we can distinguish two cases according to the sign of $1 + \dfrac{bK_V\lv}{r_V}\dfrac{1- \dfrac{\alpha}{\lv} K_F \dfrac{\FHterme}{r_F}}{1 + a K_F\dfrac{\FHterme}{r_F}}$. 

First, we assume that
\begin{subequations}
\begin{align}
& 0 < 1  +\dfrac{K_Vb\lv}{r_V}\dfrac{1- \dfrac{\alpha}{\lv} K_F \dfrac{\FHterme}{r_F}}{1 + a K_F\dfrac{\FHterme}{r_F}} 
%\\
%&\Leftrightarrow \alpha < \dfrac{r_F r_V}{K_V b K_F (\FHterme)} + \dfrac{a r_V}{bK_V} + \dfrac{\lv r_F}{K_F (\FHterme)}
\end{align}
\end{subequations} as we did in equation \eqref{equilibreFVH:existence11} and \eqref{equilibreFVH:existence12}. In this case, the term 
$$\left( \Big(H^*_{FVH}\Big)^2 \Big(\dfrac{r_V}{bK_V}\Big(1 + aK_F \dfrac{\FHterme}{r_F}\Big) + \lv - \alpha K_F \dfrac{\FHterme}{r_F} \Big)
 - L_V \Big(\alpha K_F\big(1 - \dfrac{\mu_F}{r_F} \big) + \mu_V\Big)  \right)$$ which appears on the expression \eqref{equilibreFVH:q0} of $q_0$ is positive for $H_{FVH, 2}^*$ and negative for $H_{FVH, 1}^*$.
 Therefore, in order to get $q_0 > 0$, we must have 
\begin{equation}
H_{FVH, 1}^* < \beta \text{   and   } \beta < H_{FVH, 2}
\end{equation}
Note that $\beta < H_{FVH, 2}$ also ensures $q_2 > 0$.


sign of $q_2 q_1 - q_0$ ?? or of the determinant of the second compound ?
The second compound of $\mathcal{J}(F^*_{FVH}, V^*_{FVH}, H^*_{FVH})$ is:
\begin{scriptsize}
\begin{multline}
\mathcal{J}(F^*_{FVH}, V^*_{FVH}, H^*_{FVH})^{[2]} = \\
\begin{bmatrix}
-r_F\dfrac{F^*_{FVH}}{K_F}-r_V \dfrac{H^*_{FVH}}{H^*_{FVH} + L_V}\dfrac{V^*_{FVH}}{K_V}  &\Big(\dfrac{(\alpha F^*_{FVH} + \mu_V)L_V}{H^*_{FVH}} - \lv H^*_{FVH} \Big)\dfrac{V^*_{FVH}}{H^*_{FVH}+L_V} & (\FHterme) F^*_{FVH} \\
b r_H \Big(\dfrac{H^*_{FVH}}{\beta} - 1\Big) & -r_F\dfrac{F^*_{FVH}}{K_F} -r_H\Big(\dfrac{H^*_{FVH}}{\beta} - 1\Big)& 0 \\
- a r_H \Big(\dfrac{H^*_{FVH}}{\beta} - 1\Big) & -\alpha V^*_{FVH} & -r_H\Big(\dfrac{H^*_{FVH}}{\beta} - 1\Big) -r_V \dfrac{H^*_{FVH}}{H^*_{FVH} + L_V}\dfrac{V^*_{FVH}}{K_V}
\end{bmatrix}
\end{multline}
\end{scriptsize}

We have
\begin{multline}
det(\mathcal{J}(F^*_{FVH}, V^*_{FVH}, H^*_{FVH})^{[2]}) = \\ -\Big(r_F \dfrac{F^*}{K_F} + r_V \dfrac{H^*}{H^*+L_V}\dfrac{V^*}{K_V}\Big)\Big(r_F \dfrac{F^*}{K_F} + r_H(\dfrac{H^*}{\beta}-1)\Big)\Big(r_H(\dfrac{H^*}{\beta}-1) + r_V \dfrac{H^*}{H^*+L_V}\dfrac{V^*}{K_V}\Big) \\
-\Big(\alpha(\FHterme)br_H (\dfrac{H^*}{\beta}-1) F^* V^* \Big) \\
- (\FHterme)  ar_H \Big(r_F \dfrac{F^*}{K_F} + r_H(\dfrac{H^*}{\beta}-1)\Big) F^* (\dfrac{H^*}{\beta}-1) \\
+ br_H \Big(r_H(\dfrac{H^*}{\beta}-1) + r_V \dfrac{H^*}{H^*+L_V}\dfrac{V^*}{K_V}\Big) (\dfrac{H^*}{\beta}-1) \Big((\alpha F^* + \mu_V) L_V - \lv (H^*)^2\Big) \dfrac{V^*}{H^*(H^*+L_V)}
\end{multline}

Look up for the sign of $\Big((\alpha F^* + \mu_V) L_V - \lv (H^*)^2\Big)$ ? if negative the determinant is negative, otherwise, we don't know..

...

Second, we assume that
\begin{subequations}
\begin{align}
& 1  +\dfrac{K_Vb\lv}{r_V}\dfrac{1- \dfrac{\alpha}{\lv} K_F \dfrac{\FHterme}{r_F}}{1 + a K_F\dfrac{\FHterme}{r_F}}  < 0
\end{align}
\end{subequations} as we did in equation \eqref{equilibreFVH:existence21} and \eqref{equilibreFVH:existence22}.

Then...

\end{itemize}

\subsection{Summary of the long term behavior}
We define 
\begin{subequations}
\begin{align}
& T_H(F, V) = \dfrac{aF + bV + c}{\beta} \\
& T_F(H) = \dfrac{r_F}{\mu_F + H (\omega f + \lf)} \\
& T_V(F, H) = \dfrac{r_V}{\mu_V + \alpha F + H \lv} \dfrac{H}{H + L_V}
\end{align}
\end{subequations}

$T_H$ is increasing with respect to both $F$ and $V$, $T_F$ is decreasing wrt $H$ and $T_V$ is decreasing wrt $F$, and increasing on $[0, \sqrt{L_V (\mu_V + \alpha F) /\lv}]$, decreasing on $[\sqrt{L_V (\mu_V + \alpha F) /\lv}, +\infty]$.

We also define
\begin{subequations}
{\footnotesize
\begin{align}
&  T_{\Delta_{VH}} = \dfrac{b}{4K_V}\Big(1 + \dfrac{b K_V \lv}{r_V}\Big)^{-1}\Big((c+L_V) \dfrac{\mu_V + \lv c}{r_V} - c\Big)^{-1} \left(K_V \Big(\dfrac{\lv(2c+L_V) + \mu_V}{r_V} - 1\Big)+\dfrac{c}{b} \right)^2 \\
& T_{V?} = \dfrac{bK_V}{c} \dfrac{1 - \dfrac{\mu_V + \lv L_V}{r_V}}{1 + \dfrac{2 \lv bK_V}{r_V}}
\end{align}
}
\end{subequations}


Some equilibrium values are also recall:
\begin{subequations}
\begin{align*}
& F^*_{FH} = K_F \left( \dfrac{1 - \dfrac{\mu_F}{r_F} - c \dfrac{\omega f + \lf}{r_F}}{1 + a K_F \dfrac{\omega f + \lf}{r_F}} \right) \\
& H^*_{VH,2} = b V^*_{VH, 2} + c
\end{align*}
\end{subequations}

The existence and stability condition are summarized in the following table:

\newpage
\begin{landscape}
\begin{table}
\centering
\caption{Summary of conditions of existence and stability for all the possible equilibrium excepting $EE^{FVH}$}
{\small
\begin{tabular}{c|c|c|c|c|c|c|c|c|c|c|c|c|c|c}
Equilibrium & $T_H(0, 0)$ & $T_H(F^*_{FH},0)$ & $T_H(0,V^*_{VH,i})$ & $T_F(0)$ & $T_F(\beta)$ & $T_F(c)$& $T_F(H^*_{VH, i})$ & $T_V(\beta, 0)$ & $T_V(\beta, F^*_\beta)$ & $T_V(c, 0)$ & $T_V(c, F^*_{FH})$ & $T_{\Delta_{VH}}$ & $T_{V?}$ & $\dfrac{ H^*_{VH, 1}}{H_{min}} $ \\ \hline
$TE$ & & & & $<1$ & & & & & & & & & \\ \hline
$EE^H_\beta$& $<1$ & & & & $<1$ & & & $<1$ & & & & & \\ \hline
$EE^F$ & & & & $1<$ & & & & & & & & & \\ \hline
$EE^{FH}_\beta$ & $<1$ & & & & $1<$ & & & & $<1$ & & & & \\ \hline
$EE^{VH}_\beta$ & $<1$ & & & & $<1$ & & & $1<$ & & & & & \\ \hline
$EE^{FVH}_\beta$ & $<1$ & & & & $1<$ & & & & $1<$ & & & & \\ \hline
$EE^H $ & $1<$  & & & & & $<1$ & & & & $<1$ & & & \\ \hline
$EE^{FH} $ & & $1<$  & & & & $1<$ & & & & & $<1$ & & \\ \hline
$EE^{VH}_2 $ & & & $1<$ & & & & $<1$ & & & $1<$ & & & & \\ \hline
$EE^{VH}_2 $ & & & $1<$ & & & & $<1$ & & & $<1$ & &$1\leq$& $1<$ & \\ \hline
$EE^{VH}_1 $ & & & $<1$ & & & & $<1$ & & & $<1$ & &$1\leq$& $1<$ & $1<$\\ \hline
\end{tabular}
}
\end{table}
\end{landscape}
\subsection{Long term dynamic with biological restrictions}

Among the possible equilibrium, $EE^H_\beta = (0,0,\beta)$ gives cause for concern. Indeed, stability of this equilibria means that the human population has no resources available by wild ($F^*_{H_\beta} = 0$) or from cultivation ($V^*_{H_\beta} = 0$), and that the number of people is highest than the number allowed by importation (parameter $c$). In other word, this equilibrium is AS if the human population is highest than the maximum capacity.

To avoid this situation, we impose a limitation on $\beta$. We define $\beta_{max}$ such that:
\begin{equation}
T_F(\beta_{max}) = 1 \Leftrightarrow \beta_{max} = \dfrac{r_F - \mu_F}{\omega f + \lf}
\end{equation}
and we restrict $\beta \in (0, \beta_{max})$. Since $T_F(H)$ is decreasing, we have, for all $\beta \in (0, \beta_{max})$, $T_F(\beta) > 1$. This restriction has several consequences.

First, as required, $EE^H_\beta$ will never be stable. It is also the case for $EE_\beta^{VH}$, which is understandable by saying that a low human population can not, in the same time, destroy the forest and cultivate land.

Second, since we have $T_F(0) > T_F(\beta) > 1$, $TE$ will also never be stable while $EE^F$ will always be stable.

Third, $EE^{VH}_1$ will also never be stable. Indeed, the stability conditions for the equilibria are
\begin{subequations}
    \begin{empheq}[left={\empheqlbrace\,}]{align}
&H_{min} < H^*_{VH,1} < \beta \\
&T_F(H^*_{VH,1}) < 1
    \end{empheq}
\end{subequations}
and, once again using monotony of $T_F$, we see that first condition implies $T_F(H^*_{VH,1}) > 1$, which prevents the second condition to be satisfied.

All those consideration leads to the following simplified table:
\begin{table}[ht]
\centering
\caption{\centering Summary of conditions of existence and stability for all the possible equilibrium excepting $EE^{FVH}$, considering biological limitations ($\beta < \beta_{max}$)}
{\small
\begin{tabular}{c|c|c|c|c|c|c|c|c|c|c}
Eq & $T_H(0, 0)$ & $T_H(F^*_{FH}, 0)$ & $T_H(0, V^*_{VH,2})$ & $T_F(c)$& $T_F(H^*_{VH, 2})$ & $T_V(\beta, F^*_\beta)$ & $T_V(c, 0)$ & $T_V(c, F^*_{FH})$ & $T_{\Delta_{VH}}$ & $T_{V?}$  \\ \hline
$EE^F$ & & & & & & & & & & \\ \hline
$EE^{FH}_\beta$ & $<1$ & & & & & $<1$ & & & & \\ \hline
$EE^{FVH}_\beta$ & $<1$ & & & & & $1<$ & & & & \\ \hline
$EE^H $ & $1<$  & & & $<1$& & & $<1$ & & & \\ \hline
$EE^{FH} $ & & $1<$  & & $1<$ & & & & $<1$ & & \\ \hline
$EE^{VH}_2 $ & & & $1<$ & & $<1$ & & $1<$ & & & \\ \hline
$EE^{VH}_2 $ & & & $1<$ & & $<1$ & & $<1$ & &$1\leq$& $1<$ \\ \hline
\end{tabular}
}
\end{table}



\section{Model with human preferences for forest or vegetation}
Let introduce $\tau \in [0,1]$ which represents the fraction of the human population interested in forest resources. The model becomes
\begin{equation}    
\left\{ \begin{array}{l}
\dfrac{dF}{dt}=r_{F}\left(1-\dfrac{F}{K_{F}}\right)F-\omega f (1-\tau) H F - \mu_F F -\lambda_{FH} \tau F H \\
\dfrac{dV}{dt}=r_V \dfrac{(1-\tau)H}{(1-\tau)H + L_V} s(F) \left(1-\dfrac{V}{K_{V}}\right)V -\alpha FV-\mu_V V -\lambda_{VH} (1-\tau) V H\\
\dfrac{dH}{dt}=r_H \left(1-\dfrac{H}{aF + bV + c} \right) (\dfrac{H}{\beta} - 1) H
\end{array}\right.
\label{modelPreferences}
\end{equation}

\begin{appendices}
\section{Inequality on second degree polynomial \label{appendice:ineq2nd}}

Consider the equation
\begin{equation}
X^2 + B X + C = 0
\label{appendice:eq}
\end{equation}
with positive discriminant $\Delta = B^2 - 4C \geq 0$. Let note $X_{1} = \dfrac{-B}{2} - \dfrac{1}{2} \sqrt{B^2 - 4 C}$ and $X_2 = \dfrac{-B}{2} + \dfrac{1}{2} \sqrt{B^2 - 4 C}$  the roots of \eqref{appendice:eq}.

When $B < 0$ and $0 < C$, the following inequality hold
\begin{equation}
(X_1) ^ 2 - C < 0 \:\:\text{ and } (X_2)^2 - C > 0
\end{equation}

Indeed, we have :
\begin{subequations}
\begin{align}
&(X_i) ^2 = \dfrac{1}{4} \Big(B^2 \mp 2 B\sqrt{B^2 - 4C} + B^2 - 4C \Big) \\
&(X_i)^2 - C = \dfrac{1}{4} \Big(2(B^2 - 4C) \mp 2B \sqrt{B^2 - 4C} \Big) \\
&(X_i)^2 - C = \dfrac{1}{2} \sqrt{B^2 - 4C} \Big( \sqrt{B^2 - 4C} \mp B \Big)
\end{align}
\end{subequations}
and when $B < 0$ and $0 < C$, we have $ -B = \sqrt{B^2} > \sqrt{B^2 - 4C}$.




\end{appendices}

\bibliographystyle{plain}
\bibliography{bibSuivi2}

\end{document}
