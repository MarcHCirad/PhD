\documentclass{article}
\usepackage{graphicx} 
\usepackage{color}
\usepackage{amsfonts,amsmath}
\usepackage{amsthm}
\usepackage{empheq}
\usepackage{mathtools}
\usepackage{multirow}
\usepackage{tikz}
\usepackage{titlesec}
\usepackage{caption}
\usepackage{lscape}
\captionsetup{justification=justified}
\usepackage[toc,page]{appendix}

\textheight240mm \voffset-23mm \textwidth160mm \hoffset-20mm

\setcounter{secnumdepth}{4}
\titleformat{\paragraph}
{\normalfont\normalsize\bfseries}{\theparagraph}{1em}{}
\titlespacing*{\paragraph}
{0pt}{3.25ex plus 1ex minus .2ex}{1.5ex plus .2ex}

\newtheorem*{theo}{Theorem}
\newcommand{\yves}{\textcolor{magenta}}
\newcommand{\gf}{(r_F - \omega f - \mu_F)}
\newcommand{\gv}{(r_V- \mu_V)}
\newcommand{\lf}{\lambda_{FH}}
\newcommand{\lv}{\lambda_{VH}}
\newcommand{\RV}{R_0^V}
\newcommand{\RF}{R_0^F}
\newcommand{\NF}{\mathcal{N}_0^F}
\newcommand{\NV}{\mathcal{N}_0^V}
\newcommand{\NH}{\mathcal{N}_0^H}
\newcommand{\Fbeta}{F^*_\beta}
\newcommand{\Hbeta}{H^*_\beta}
\newcommand{\Vbeta}{V^*_\beta}
\newcommand{\VbetaF}{V^*_{\Fbeta, \beta}}
\newcommand{\FHterme}{\omega f + \lf}
\newcommand*\phantomrel[1]{\mathrel{\phantom{#1}}}

\title{Suivi Thèse Marc}
\author{Marc Hétier, Yves Dumont  and Valaire Yatat-Djeumen}

\begin{document}

\maketitle

This model tries to take into account the interactions between human population ($H$), wild fauna and flora ($F$), and human-driven vegetation ($V$). 
We assume that human pick up resources from wild and human-driven vegetation. The rate per person of this collect is assumed to be decreasing with the quantity of resources.
Moreover, human-driven vegetation growth is depended on both services render by human (culture, watering...) and wild vegetation (pollination, moisture retention, wind protection). 
Deforestation by fire happen with intensity $\omega$ and frequency $f(H)$. Assuming that the combustible (dry grass) is available in enough quantity for a good fire's spatial dispersion, we can take $f(H) = fH$. Also, note that in the context of this study, fire is mainly used for hunting activities or to create a agricultural field. In both cases, the frequency and the intensity of those fires are low.
Conversely, wild biomass is a competitor for human driven vegetation.

Human maximum capacity is assumed to depend on resources (wild and human driven). This assumption is justified in \cite{fanta_equilibrium_2018}, used in \cite{bengochea_paz_agricultural_2020}.
Moreover, the human population is assumed have a reduced growth rate for low population size. This corresponds to a weak Allee effect. Some studies have already propose such phenomenon on human population ; see \cite{hamilton_human_2012} and \cite{vaesen_inbreeding_2019}.

All thus assumption lead to the following model :

\begin{equation}    
\left\{ \begin{array}{l}
\dfrac{dF}{dt}=r_{F}\left(1-\dfrac{F}{K_{F}}\right)F-\omega f H F - \mu_F F -\lambda_{FH}g_F(F,V)F H \\
\dfrac{dV}{dt}=r_V \dfrac{H}{H + L_V} s(F) \left(1-\dfrac{V}{K_{V}}\right)V -\alpha FV-\mu_V V -\lambda_{VH} g_V(F,V) V H\\
\dfrac{dH}{dt}=r_H \left(1-\dfrac{H}{aF + bV + c} \right) (\dfrac{H}{\beta} - 1) H
\end{array}\right.
\label{model}
\end{equation}
where both $g_F$ and $g_V$ are decreasing in both $F$ and $V$.

\section{Simplified model}
First we can assume that $g_F = g_V = 1$ and that $s(F) = 1$. The model becomes :
\begin{equation}    
\left\{ \begin{array}{l}
\dfrac{dF}{dt}=r_{F}\left(1-\dfrac{F}{K_{F}}\right)F-\omega f H F - \mu_F F -\lf F H \\
\dfrac{dV}{dt}=r_V \dfrac{H}{H + L_V} \left(1-\dfrac{V}{K_{V}}\right)V -\alpha FV-\mu_V V -\lv V H\\
\dfrac{dH}{dt}=r_H \left(1-\dfrac{H}{aF + bV + c} \right)  (\dfrac{H}{\beta} - 1) H
\end{array}\right.
\label{model:simplified}
\end{equation}

\subsection{Equilibrium}
Equilibrium are obtained by solving the following system :
\begin{equation}    
\left\{ \begin{array}{l}
F\Big(r_{F}\big(1-\dfrac{F}{K_{F}}\big) - \mu_F -\omega f H -\lf H \Big) = 0\\
V \Big( r_V \dfrac{H}{H + L_V} \big(1-\dfrac{V}{K_{V}}\big) -\alpha F-\mu_V -\lv H\Big) = 0\\
\left(1-\dfrac{H}{aF + bV + c} \right)  (\dfrac{H}{\beta} - 1) H = 0
\end{array}\right.
\label{model:equilibrium}
\end{equation}

We can first notice that ($H=0, F\geq 0$) implies $V=0$. Thus, an equilibria with vegetation and without human does not exist. All the existing equilibrium are listed below :

\begin{itemize}
\item $TE = (0,0,0)$
\item A forest equilibrium $EE^F = \Big( K_F(1-1/\NF), 0, 0 \Big) $ exists when $\NF = \dfrac{r_F}{\mu_F} > 1$.

\item A human equilibrium $EE^H_\beta = \Big(0,0,H^*_\beta \Big)$ where $ H^*_\beta = \beta$
\item A forest-human equilibrium $EE^{FH}_\beta = \Big(F^*_\beta, 0, H^*_\beta\Big)$ where $F^*_\beta = K_F \Big(1-\dfrac{\mu_F + \beta(\lf + \omega f)}{r_F} \Big)$.
This equilibrium exists when $\dfrac{r_F}{\mu_F + \beta(\lf + \omega f)} > 1$.
\item A vegetation-human equilibrium $EE^{VH}_\beta = \Big(0, V^*_{\beta}, H^*_\beta \Big)$ where $V^*_\beta = K_V \Big(1- \dfrac{\beta + L_V}{\beta} \dfrac{\lv \beta + \mu_V}{r_V} \Big)$. This equilibrium exists when $\dfrac{\beta}{\beta + L_V} \dfrac{r_V}{\lv \beta + \mu_V} > 1$
\item An endemic equilibrium $EE_\beta^{FVH} = \Big(F^*_\beta, V^*_{F^*_\beta, \beta}, H^*_\beta \Big)$ where $\VbetaF = K_V \Big(1- \dfrac{\beta +L_V}{\beta} \dfrac{\alpha \Fbeta + \lv \beta + \mu_V}{r_V} \Big)$. This equilibrium exists  when $\dfrac{r_F}{\mu_F + \beta(\lf + \omega f)} > 1$ and $\dfrac{\beta}{\beta + L_V} \dfrac{r_V}{\alpha \Fbeta + \lv \beta + \mu_V} > 1$


\item A human equilibrium $EE^H = \Big(0,0,H^*_c)$ where $H^*_c = c$.
\item A forest-human equilibrium $EE^{FH} = \Big(F^*_{H, c}, 0, H^*_{F, c} \Big)$ where $F^*_{H,c} = K_F \left( \dfrac{1 - \dfrac{\mu_F}{r_F} - c \dfrac{\omega f + \lf}{r_F}}{1 + a K_F \dfrac{\omega f + \lf}{r_F}} \right)$ and $H^*_{F,c} = a F^*_{H, c} + c$. This equilibrium exists when $\dfrac{r_F}{\mu_F + c(\lf + \omega f)} > 1$.

\item A vegetation-human equilibrium $EE^{VH} = \Big(0, V^*_{H, c}, H^*_{V, c} \Big)$. $V^*_{H, c}$ are given by $V^*_{H, c} = \dfrac{H^*_{V, c} - c}{b}$ and $H^*_{V, c}$ are solutions of

\begin{equation}
H^2 - H \left( \dfrac{r_V(1 + \dfrac{c}{bK_V}) - \lv L_V - \mu_V}{\lv + \dfrac{r_V}{bK_V}} \right) + \dfrac{\mu_V L_V}{\lv + \dfrac{r_V}{bK_V}} = 0
\label{eqVH:equation H}
\end{equation}

Equation \eqref{eqVH:equation H} has real and positive solutions if 
\begin{subequations}
    \begin{empheq}[left={\empheqlbrace\,}]{align}
    &\left(r_V \Big(1 + \dfrac{c}{bK_V}\Big) - \lv L_V - \mu_V \right)^2 \geq 4 \mu_V L_V \Big(\lv + \dfrac{r_V}{bK_V} \Big) \\
    &\dfrac{r_V(1 + \dfrac{c}{bK_V})}{\mu_V + \lv L_V} > 1
    \end{empheq}
    \label{eqVH:existence1}
\end{subequations}

In this case, the solutions are given by
\begin{subequations}
\begin{align}
H^*_{V, c, i} = \dfrac{1}{2}\left( \dfrac{r_V(1 + \dfrac{c}{bK_V}) - \lv L_V - \mu_V}{\lv + \dfrac{r_V}{bK_V}} \right) \pm \dfrac{1}{2}\sqrt{\left( \dfrac{r_V(1 + \dfrac{c}{bK_V}) - \lv L_V - \mu_V}{\lv + \dfrac{r_V}{bK_V}} \right)^2 - 4\dfrac{\mu_V L_V}{\lv + \dfrac{r_V}{bK_V}} } \\
H^*_{V, c, i} = \dfrac{1}{2}\left( \dfrac{r_V(1 + \dfrac{c}{bK_V}) - \lv L_V - \mu_V}{\lv + \dfrac{r_V}{bK_V}} \right)\left( 1  \pm \sqrt{1 - \dfrac{4\mu_V L_V \Big(\lv + \dfrac{r_V}{bK_V} \Big)}{\left(r_V \Big(1 + \dfrac{c}{bK_V}\Big) - \lv L_V - \mu_V \right)^2}} \right)
\end{align}
\end{subequations}

In order to have $V^*_{H, c} > 0$, those solutions should also be higher than $c$. So we have a last condition of existence :
\begin{equation}
H^*_{V, c, i} > c
\label{eqVH:existence2}
\end{equation}

\item A endemic equilibrium $EE^{FVH} = \Big(F^*_{VH, c}, V^*_{FH, c}, H^*_{FV, c} \Big)$. The variables' values are given by 

\begin{subequations}
    \begin{empheq}[left={\empheqlbrace\,}]{align}
    &F^*_{VH, c} = F^*_{H, c} -  \dfrac{bK_F \dfrac{\FHterme}{r_F} }{1 + aK_F \dfrac{\FHterme}{r_F}} V^*_{FH, c} :=  F^*_{H, c} - c_1^F V^*_{FH, c}\\
    &H^*_{FV,c} = a F^*_{H, c} + c + (b-ac^F_1) V^*_{FH, c}
    \end{empheq}
    \label{eqFVH:values}
\end{subequations}

NB : $c_1^F> 0$, $b-ac^F_1 > 0$

and $V^*_{FH, c}$ are solutions of 

\newpage
\begin{landscape}

\begin{subequations}
\begin{align}
& r_V \Big(1 - \dfrac{V}{K_V}) (a F^*_{H, c} + c + (b-ac^F_1) V) - \alpha (F^*_{H, c} - c_1^F V)(a F^*_{H, c} + c +L_V + (b-ac^F_1) V) - \mu_V (a F^*_{H, c} + c +L_V + (b-ac^F_1) V) - \lv (a F^*_{H, c} + c + (b-ac^F_1) V) (a F^*_{H, c} + c + L_V + (b-ac^F_1) V) = 0\\
&\begin{multlined}[t][8cm]
V^2 \Big(-\dfrac{r_V}{K_V} (b-ac^F_1) + \alpha c_1(b-ac^F_1) - \lv (b-ac^F_1)^2 \Big) + \\
V \Big(r_V(b-ac^F_1) - \dfrac{r_V}{K_V} (a F^*_{H, c} + c) + \alpha c_1^F (a F^*_{H, c} + c + L_V) - \alpha (b-ac^F_1)F^*_{H, c} - \mu_V(b-ac^F_1) - \lv (b-ac^F_1)(a F^*_{H, c} + c) - \lv (a F^*_{H, c} + c + L_V)(b-ac^F_1) \Big)+ \\
r_V (a F^*_{H, c} + c) - \alpha F^*_{H, c}(a F^*_{H, c} + c+L_V) - \mu_V (a F^*_{H, c} + c+L_V) - \lv (a F^*_{H, c} + c)(a F^*_{H, c} + c + L_V) = 0
\end{multlined} \\
&\begin{multlined}[t][1cm]
V^2 \Big(-\dfrac{r_V}{K_V} + \alpha c_1^F - \lv (b-ac^F_1) \Big) + \\
V \Big(r_V - \mu_V - \lv (2 a F^*_{H, c} + 2c + L_V) - \alpha F^*_{H, c} - \dfrac{r_V}{K_V} \dfrac{a F^*_{H, c} + c}{b-ac_1^F} - \alpha \dfrac{c1^F(a F^*_{H, c} + c + L_V)}{b-ac_1^F} \Big) +  \\
\dfrac{(a F^*_{H, c} + c + L_V)}{(b-ac^F_1)} \Big(- \alpha F^*_{H, c} - \mu_V - \lv (a F^*_{H, c} + c)\Big) + r_V \dfrac{(a F^*_{H, c} + c)}{(b-ac^F_1)} = 0
\end{multlined}
\end{align}
\end{subequations}

\end{landscape}
\end{itemize}

\section{Stability}

The Jacobian matrix of the system is defined by :
{\footnotesize
\begin{multline}
\mathcal{J}(F,V,H) = \\ \begin{bmatrix}
r_F \Big(1-\dfrac{2F}{K_F} \Big) - \mu_F - (\lf + \omega f)H & 0 & -(\lf + \omega f)F \\
-\alpha V & r_V \dfrac{H}{H+L_V}(1-\dfrac{2V}{K_V}) - \alpha F - \lv H - \mu_V & r_V \dfrac{L_V}{(H+L_V)^2}(1-\dfrac{V}{K_V})V  - \lv V\\
r_H \dfrac{aH^2}{(aF+bV+c)^2} (\dfrac{H}{\beta}-1) &r_H \dfrac{bH^2}{(aF+bV+c)^2} (\dfrac{H}{\beta}-1) & r_H(1-\dfrac{H}{aF+bV+c})(\dfrac{2H}{\beta}-1) - \dfrac{r_H H}{aF+bV+c}(\dfrac{H}{\beta}-1)
\end{bmatrix}
\label{stab:jacobian}
\end{multline}
}

To assess the stability of the equilibrium we look at the eigenvalues of the Jacobian.

\begin{itemize}
\item At point $TE = (0,0, 0)$, the Jacobian is
\begin{equation}
\mathcal{J}(0,0,0) = \begin{bmatrix}
r_F - \mu_F & 0 \\
0 & -\mu_V & 0 \\
0 & 0 & -r_H
\end{bmatrix}
\end{equation}
and thus the eigenvalues are $r_F - \mu_F$, $-\mu_V$ and $-r_H$. $TE$ is asymptotically stable (AS) if $\NF < 1$.

\item At point $EE^{F} = (F^F, 0, 0)$, the Jacobian is 
\begin{equation}
\mathcal{J}(F^F, 0, 0)
\begin{bmatrix}
-r_F \dfrac{F^F}{K_F}  & 0 & -(\FHterme)F^F \\
0 & -\alpha F^F - \mu_V & 0 \\
0 & 0 & -r_H 
\end{bmatrix}
\end{equation}
and thus the eigenvalues are all negative. $EE^F$ is asymptotically stable.

\item At point $EE^H_\beta = \Big(0,0,H^*_\beta \Big)$, the Jacobian is
\begin{equation}
\mathcal{J}(0, 0, H^*_\beta) = \begin{bmatrix}
r_F - \mu_F - (\FHterme) H^*_\beta & 0 & 0 \\
0 & r_V\dfrac{H^*_\beta}{H^*_\beta+L_V} - \lv H^*_\beta - \mu_V & 0 \\
0 & 0 & r_H (1 - \dfrac{H^*_\beta}{c})
\end{bmatrix}
\end{equation}
We used $H^*_\beta = \beta$ several times. $EE^H_\beta$ is AS if 

\begin{subequations}
    \begin{empheq}[left={\empheqlbrace\,}]{align}
    & \dfrac{r_F}{\mu_F + \beta (\FHterme)} < 1 \\
    & \dfrac{r_V}{\mu_V + \lv \beta} \dfrac{\beta}{\beta + L_V} < 1  \\
    & \dfrac{c}{\beta} < 1
    \end{empheq}
\end{subequations}

\item At point $EE^{FH}_\beta = \Big(F^*_\beta,0,H^*_\beta \Big)$, the Jacobian is
\begin{equation}
\mathcal{J}(F^*_\beta, 0, H^*_\beta) = \begin{bmatrix}
-r_F \dfrac{F^*_\beta}{K_F} & 0 & -(\FHterme) F^*_\beta \\
0 & r_V\dfrac{H^*_\beta}{H^*_\beta+L_V} - \lv H^*_\beta - \alpha F^*_\beta - \mu_V & 0 \\
0 & 0 & r_H (1 - \dfrac{H^*_\beta}{c})
\end{bmatrix}
\end{equation}
We used $H^*_\beta = \beta$ several times.  $EE^{FH}_\beta$ is AS if 
\begin{subequations}
    \begin{empheq}[left={\empheqlbrace\,}]{align}
    &r_V \dfrac{\beta}{\beta + L_V} < \mu_V + \lv \beta -\alpha K_F\Big(1- \dfrac{\mu_F + \beta(\FHterme)}{r_F} \Big) \\
    & c < \beta
    \end{empheq}
\end{subequations}


\item At point $EE^{VH}_\beta = \Big(0,V^*_\beta,H^*_\beta \Big)$, the Jacobian is
\begin{equation}
\mathcal{J}(0, V^*_\beta, H^*_\beta) = \begin{bmatrix}
r_F - \mu_F - (\FHterme) H^*_\beta & 0 & 0 \\
-\alpha V^*_\beta & - r_V\dfrac{H^*_\beta}{H^*_\beta+L_V} V^*_\beta & \_ \\
0 & 0 & r_H (1 - \dfrac{H^*_\beta}{c})
\end{bmatrix}
\end{equation}
We used $H^*_\beta = \beta$ several times. Eigenvalues appear on the diagonal. $EE^{VH}_\beta$ is AS if
\begin{subequations}
    \begin{empheq}[left={\empheqlbrace\,}]{align}
    & \dfrac{r_F}{\mu_F + \beta (\FHterme)} < 1 \\
    & \dfrac{c}{\beta} <1 
    \end{empheq}
\end{subequations}

\item At point $EE^{FVH}_\beta = \Big(\Fbeta,\VbetaF, \Hbeta \Big)$, the Jacobian is
\begin{equation}
\mathcal{J}(\Fbeta, \VbetaF, \Hbeta) = \begin{bmatrix}
-r_F \dfrac{\Fbeta}{K_F} & 0 & -(\FHterme)\Fbeta \\
-\alpha \VbetaF & - r_V\dfrac{H^*_\beta}{H^*_\beta+L_V} \VbetaF & \_ \\
0 & 0 & r_H (1 - \dfrac{H^*_\beta}{c})
\end{bmatrix}
\end{equation}
We used $H^*_\beta = \beta$ several times. Eigenvalues appear on the diagonal. $EE^{FVH}_\beta$ is AS if
\begin{equation}
\dfrac{c}{\beta} < 1
\end{equation}

\item At point $EE^{H} = \Big(0,0,H^*_c \Big)$, the Jacobian is
\begin{equation}
\mathcal{J}(0,0,H^*_c) = \begin{bmatrix}
r_F - \mu_F - H^*_c(\FHterme) & 0 & 0 \\
0 & r_V \dfrac{H^*_c}{L_V + H^*_c} - \lv H^*_c - \mu_V & 0 \\
r_H a (\dfrac{H^*_c}{\beta} - 1) & r_H b (\dfrac{H^*_c}{\beta} - 1) & -r_H(\dfrac{H^*_c}{\beta} - 1)
\end{bmatrix}
\end{equation}
We used $H^*_c = c$ several times. $EE^{H}$ is AS if
\begin{subequations}
    \begin{empheq}[left={\empheqlbrace\,}]{align}
    &\dfrac{r_F}{\mu_F + c (\FHterme)} < 1 \\
    & \dfrac{r_V}{\mu_V + c \lv} \dfrac{c}{c + L_V} < 1  \\
    & 1< \dfrac{c}{\beta} 
    \end{empheq}
\end{subequations}

\item At point $EE^{FH} = \Big(F^*_{H, c}, 0, H^*_{F, c})$, the Jacobian is
\begin{equation}
\mathcal{J}(F^*_{H, c}, 0, H^*_{F, c}) = \begin{bmatrix}
-r_F \dfrac{F^*_{H, c}}{K_F} & 0 & -(\FHterme) F^*_{H, c} \\
0 & r_V \dfrac{H^*_{F, c}}{L_V + H^*_{F, c}} - \lv H^*_{F, c} -\alpha F^*_{H, c} - \mu_V & 0 \\
r_H a (\dfrac{H^*_{F, c}}{\beta} - 1) & r_H b (\dfrac{H^*_{F, c}}{\beta} - 1) & -r_H(\dfrac{H^*_{F, c}}{\beta} - 1)
\end{bmatrix}
\end{equation}
We used $H^*_{F, c} = a F^*_{H, c} c$ several times. Eigenvalues are the roots of 

\begin{multline}
\chi = \Big(X - r_V \dfrac{H^*_{F, c}}{L_V + H^*_{F, c}} + \lv H^*_{F, c} +\alpha F^*_{H, c} + \mu_V \Big) \times \\
\left(X^2 + \Big(r_F \dfrac{F^*_{H, c}}{K_F} + r_H \big(\dfrac{H^*_{F, c}}{\beta} - 1\big) \Big)X + r_f \dfrac{F^*_{H, c}}{K_F}r_H \big(\dfrac{H^*_{F, c}}{\beta} - 1)\big) + r_H a \big(\dfrac{H^*_{F, c}}{\beta} - 1\big) (\FHterme) F^*_{H, c} \right)
\end{multline}

One root is $r_V \dfrac{H^*_{F, c}}{L_V + H^*_{F, c}} - \lv H^*_{F, c} -\alpha F^*_{H, c} -  \mu_V$. The two others have negative real parts if the coefficients of the second degree polynomial are both positive, which is the case if $H^*_{F, c} > \beta$.
Finally, $EE^{FH}$ is AS if
\begin{subequations}
    \begin{empheq}[left={\empheqlbrace\,}]{align}
    & \dfrac{r_V}{\mu_V + H^*_{F, c} \lv +\alpha F^*_{H, c}} \dfrac{H^*_{F, c}}{H^*_{F, c} + L_V} < 1 \\
    &1 < \dfrac{H^*_{F, c}}{\beta}
    \end{empheq}
\end{subequations}
with $F^*_{H, c} = K_F \left( \dfrac{1 - \dfrac{\mu_F}{r_F} - c \dfrac{\omega f + \lf}{r_F}}{1 + a K_F \dfrac{\omega f + \lf}{r_F}} \right)$ and $H^*_{F, c} = a F^*_{H, c} + c$.


\item At point $EE^{VH} = \Big(0, V^*_{H, c}, H^*_{V, c})$, the Jacobian is
\begin{equation}
\mathcal{J}(0, V^*_{H, c}, H^*_{V, c}) = \begin{bmatrix}
r_F - \mu_F - (\FHterme)H^*_{V, c}  & 0 & 0 \\
-\alpha V^*_{H, c} & -r_V\dfrac{H^*_{V, c}}{H^*_{V, c} + L_V} \dfrac{V^*_{H, c}}{K_V} & \Big(\dfrac{L_V\mu_V}{H^*_{V, c}} - \lv H^*_{V, c}) \dfrac{V^*_{H, c}}{H^*_{V, c} + L_V} \\
r_H a (\dfrac{H^*_{V, c}}{\beta} - 1) & r_H b (\dfrac{H^*_{V, c}}{\beta} - 1) & -r_H(\dfrac{H^*_{V, c}}{\beta} - 1)
\end{bmatrix}
\end{equation}
We used $H^*_{V, c} = b V^*_{H, c} + c$ several times. Eigenvalues are the roots of 
\begin{multline}
\chi = \Big(X - r_F + \mu_F + (\FHterme)H^*_{V, c} \Big) \times \\
\left(X^2 + \Big(r_V  \dfrac{H^*_{V, c}}{H^*_{V, c}+L_V}\dfrac{V^*_{H, c}}{K_V} + r_H(\dfrac{H^*_{V, c}}{\beta} - 1)\Big)X + \right.\\ \left.
 r_V  \dfrac{H^*_{V, c}}{H^*_{V, c}+L_V}\dfrac{V^*_{H, c}}{K_V} r_H(\dfrac{H^*_{V, c}}{\beta} - 1) + r_Hb(\dfrac{H^*_{V, c}}{\beta} - 1) \Big(\lv H^*_{V, c} - \dfrac{L_V \mu_V}{H^*_{V, c}} \Big) \dfrac{V^*_{H, c}}{H^*_{V, c} + L_V} \right)
 \label{stab:chi VH}
\end{multline}

One root is $r_F - \mu_F - (\FHterme)H^*_{V, c}$. The two others have negative real parts if the coefficients of the second degree polynomial are both positive. The constant coefficient is positive if 
\begin{subequations}
\begin{align}
&r_V  \dfrac{H^*_{V, c}}{H^*_{V, c}+L_V}\dfrac{V^*_{H, c}}{K_V} r_H(\dfrac{H^*_{V, c}}{\beta} - 1) + r_Hb(\dfrac{H^*_{V, c}}{\beta} - 1) \Big(\lv H^*_{V, c} - \dfrac{L_V \mu_V}{H^*_{V, c}} \Big) \dfrac{V^*_{H, c}}{H^*_{V, c} + L_V} > 0 \\
& r_V  \dfrac{H^*_{V, c}}{K_V}(\dfrac{H^*_{V, c}}{\beta} - 1) + b (\dfrac{H^*_{V, c}}{\beta} - 1) \Big(\lv H^*_{V, c} - \dfrac{L_V \mu_V}{H^*_{V, c}} \Big) > 0 \\
&(\dfrac{H^*_{V, c}}{\beta} - 1) \Big(r_V  \dfrac{H^*_{V, c}}{K_V} + b\Big(\lv H^*_{V, c} - \dfrac{L_V \mu_V}{H^*_{V, c}} \Big)\Big) > 0 \\
&\Big(\dfrac{H^*_{V, c}}{\beta} - 1 \Big) \times \Big( (H^*_{V, c})^2 \big(\dfrac{r_V }{K_V} + b\lv\big) -  b L_V \mu_V \Big) > 0
\end{align}
\end{subequations}
Using appendix \ref{appendice:ineq2nd} for $H^*_{V, c, i}$, we obtain that the second factor is positive for $H^*_{V, c, 2}$ and negative for $H^*_{V, c, 1}$. We can distinguish those two cases.
\\
Concerning  $H^*_{V, c, 2}$, the constant coefficient is positive when $H^*_{V, c, 2} > \beta$ and this condition also ensures that the coefficient in $X$ in \eqref{stab:chi VH} is also positive. 
\\
Concerning  $H^*_{V, c, 1}$, the constant coefficient is positive if $H^*_{V, c, 1} < \beta$. Equilibrium $EE^{VH}_1$ is characterized by the fact that 
\begin{equation}
f_3(H^*_{V, c, 1},V^*_{H, c, 1}) =  H^*_{V, c, 1} \times \Big(1 - \dfrac{H^*_{V, c, 1}}{bV^*_{H, c, 1} + c} \Big) \times \Big(\dfrac{H^*_{V, c, 1}}{\beta} - 1 \Big) = 0
\end{equation}
Let $H^*_{V, c, 1} < \beta$, and $V = V^*_{H, c, 1}$ being constant. Starting from $H = H^*_{V, c, 1}$ a small variation of $H$  will induce a variation in the same sense for $f_3(H, V^*_{H, c, 1})$.
This lead to the assumption that $EE^{VH}_1$ is not stable.

Finally, $EE^{VH}_2$ is AS if

\begin{subequations}
    \begin{empheq}[left={\empheqlbrace\,}]{align}
&r_F < \mu_F + (\FHterme)H^*_{V, c, 2} \\
&\beta < H^*_{V,C,2}
    \end{empheq}
\end{subequations}

and $EE^{VH}_1$ is assumed to be never stable.

\end{itemize}


\section{Summary of the long term behavior}

\begin{subequations}
\begin{align}
& T_H(F) = \dfrac{aF + c}{\beta} \\
& T_F(H) = \dfrac{r_F}{\mu_F + H (\omega f + \lf)} \\
& T_V(F, H) = \dfrac{r_V}{\mu_V + \alpha F + H \lv} \dfrac{H}{H + L_V}
\end{align}
\end{subequations}

\newpage
\begin{landscape}

\begin{table}
\centering
\caption{•}
\begin{tabular}{c|c|c|c|c|c|c|c|c|c|c|c}
\hline
\multicolumn{11}{c|}{Thresholds} & \multirow{2}{*}{Equilibrium} \\
$T_H(0)$ & $T_F(0)$ &$T_F(\beta) $ & $T_V(\beta, 0)$ &$T_V(\beta, F^*_\beta)$& $T_H(F^*_{H, c})$ & $T_F(c)$ & $T_V(c, 0)$ & $T_F(H^*_{V, c, 2})$ &$T_V(H^*_{F,c}, F^*_{H,c})$ &Pour VH \\
\hline
\multirow{12}{*}{$< 1 $} & \multirow{3}{*}{$<1$} & $(<1)$ & & & &$(<1)$ & &$(<1)$ & & &$TE$ \\ 
 & & $(<1)$ & $<1$ & & &$(<1)$ & &$(<1)$ & & & $TE, EE^H_\beta$ \\ 
 & & $(<1)$ & $1<$ & & &$(<1)$ & &$(<1)$ & & & $TE, EE^{VH}_\beta$ \\
\cline{2-11}
 & \multirow{9}{*}{$1 < $} & & & & & & & & & & $EE^F$ \\
 \cline{3-11}
 & & \multirow{4}{*}{$<1$} & \multirow{2}{*}{$<1$} & & & & & & & & $EE^F$, $EE^{H}_\beta$ \\
  & & & & &$1<$ &$1<$ & & & $<1$ & & $EE^F$, $EE^{H}_\beta$, $EE^{FH}$ \\
\cline{4-11}
 & & & \multirow{2}{*}{$1<$} & & & & & & & & $EE^F$, $EE^{VH}_\beta$ \\
   &  & & & &$1<$ &$1<$ & & &$<1$ & & $EE^F$, $EE^{VH}_\beta$, $EE^{FH}$ \\
\cline{3-11}
 & & \multirow{4}{*}{$1<$} & & \multirow{2}{*}{$<1$} & & $(1<)$& & & & & $EE^F$, $EE^{FH}_\beta$ \\
  & &  & &  & $1<$ & $(1<)$& & &$<1$ & & $EE^F$, $EE^{FH}_\beta$, $EE^{FH}$ \\
  \cline{4-11}
   & & & & \multirow{2}{*}{$1<$} & & $(1<)$& & & & & $EE^F$, $EE^{FVH}_\beta$ \\
  & &  & &  & $1<$ & $(1<)$& & &$<1$ & & $EE^F$, $EE^{FVH}_\beta$, $EE^{FH}$ \\
  \hline
\multirow{8}{*}{$1 < $} & \multirow{3}{*}{$<1$} & $(<1)$ & & &$(1<)$ &$(<1)$ & & $(<1)$ & & & $TE$ \\
 &  & $(<1)$ & & &$(1<)$ &$(<1)$ &$<1$ &$(<1)$ & & & $TE$, $EE^H$ \\ 
 &  & $(<1)$ & & &$(1<)$ &$(<1)$ &  &$(<1)$ & $1<$ & & $TE$, $EE^{VH}$ \\
 \cline{2-11}
 & \multirow{3}{*}{$1<$} & & & &$(1<)$ & & & & & & $EE^F$ \\
 & & & & &$(1<)$ &$<1$ & $1<$ & & & & $EE^F$, $EE^H$ \\
 & & & & &$(1<)$ &$1<$ & & & $<1$ & & $EE^F$, $EE^{FH}$ \\
% & \multirow{2}{*}{$1 < $} & & & & & & $< 1 $ & & $TE$, $EE^H$ \\
%\cline{3-10}
% & &  & & &  $1<$&$1<$& & & $TE$, $EE^{VH}$ \\
% \hline
% \multirow{8}{*}{$1 <$} & \multirow{8}{*}{$<1$} &\multirow{4}{*}{$<1$} &\multirow{2}{*}{$<1$} & & $1 < $ & $1<$ & $1<$ & & $EE^F$, $EE^H_\beta$ \\
% & & & & & $1<$& & & & $EE^F$, $EE^{VH}_\beta$ \\ 
% \cline{4-10}
% & & &\multirow{2}{*}{$1<$} & & & $<1$ & & &$EE^F$, $EE^{FH}_\beta$ \\
% & & & & & & $1<$ & & &$EE^F$, $EE^{FVH}_\beta$ \\
% \cline{3-10}
% & & $1< $& &
\end{tabular}
\end{table}

\end{landscape}

\section{Sub-models}
Let consider the sub-models of \eqref{model:simplified} with variables $(F,H)$ 

\begin{equation}    
\left\{ \begin{array}{l}
\dfrac{dF}{dt}=r_{F}\left(1-\dfrac{F}{K_{F}}\right)F-\omega f H F - \mu_F F -\lf F H = f_{1, FH} (F,H) \\
\dfrac{dH}{dt}=r_H \left(1-\dfrac{H}{aF + c} \right)  (\dfrac{H}{\beta} - 1) H = f_{2,FH}(F,H)
\end{array}\right.
\label{model:submodelFH}
\end{equation}

and $(V,H)$ :

\begin{equation}    
\left\{ \begin{array}{l}
\dfrac{dV}{dt}=r_V \dfrac{H}{H + L_V} \left(1-\dfrac{V}{K_{V}}\right)V -\mu_V V -\lv V H\\
\dfrac{dH}{dt}=r_H \left(1-\dfrac{H}{bV + c} \right)  (\dfrac{H}{\beta} - 1) H
\end{array}\right.
\label{model:submodelVH}
\end{equation}

We can study the existence of limit cycle for those system, using the Bendixson-Dulac theorem. Consider the function

\begin{align}
\phi : (0, +\infty) &\times (\beta, +\infty) \rightarrow \mathbf{R} \\
\nonumber
(F,H) & \mapsto \dfrac{1}{F H (H/\beta - 1)}
\end{align}

Multiplying the right hand side of system $(F,H)$ \eqref{model:submodelFH} by this function, and taking the derivative with respect to $F$ or $H$ we obtain :
\begin{subequations}
\begin{align}
&\dfrac{\partial f_{1,FH} \times \phi}{\partial F}(F,H) = - \dfrac{r_F}{K_F} \dfrac{1}{H \big(\dfrac{H}{\beta}-1 \big)} \\
&\dfrac{\partial f_{2,FH} \times \phi}{\partial H}(F,H) = - \dfrac{r_H}{(aF + c) F}
\end{align}
\end{subequations}

Then, for $(F, H) \in (0, +\infty) \times (\beta, +\infty)$
\begin{equation}
\Big(\dfrac{\partial f_{1,FH} \times \phi}{\partial F} + \dfrac{\partial f_{2,FH} \times \phi}{\partial H}\Big) (F, H) < 0
\end{equation}
and, according to Bendixson-Dulac theorem, there is no limit cycle on this interval.

The same function can be used to show that the system $(V,H)$ \eqref{model:submodelVH} does not admit a limit cycle on $(0, +\infty) \times (\beta, +\infty)$.

Now, let consider the case $\beta < c$, $ H \in (0,\beta)$ and $F \in (0, +\infty)$. In this case, function $f_{2,FH}$  is decreasing and there is also no limit cycle.


\begin{appendices}
\section{Inequality on second degree polynomial \label{appendice:ineq2nd}}

Consider the equation
\begin{equation}
X^2 - B X + C = 0
\label{appendice:eq}
\end{equation}
with positive discriminant $\Delta = B^2 - 4C \geq 0$. Let note $X_{1} = \dfrac{1}{2}B - \dfrac{1}{2} \sqrt{B^2 - 4 C}$ and $X_2 = \dfrac{1}{2}B + \dfrac{1}{2} \sqrt{B^2 - 4 C}$  the roots of \eqref{appendice:eq}.

When $B > 0$ and $C >0$, the following inequality hold
\begin{equation}
(X_1) ^ 2 - C < 0 \:\:\text{ and } (X_2)^2 - C > 0
\end{equation}

Indeed, we have :
\begin{subequations}
\begin{align}
&(X_i) ^2 = \dfrac{1}{4} \Big(B^2 \pm 2 B\sqrt{B^2 - 4C} + B^2 - 4C \Big) \\
&(X_i)^2 - C = \dfrac{1}{4} \Big(2(B^2 - 4C) \pm 2B \sqrt{B^2 - 4C} \Big) \\
&(X_i)^2 - C = \dfrac{1}{2} \sqrt{B^2 - 4C} \Big( \sqrt{B^2 - 4C} \pm B \Big)
\end{align}
\end{subequations}
and when $B > 0$ and $C>0$, we have $B = \sqrt{B^2} > \sqrt{B^2 - 4C}$.




\end{appendices}

\bibliographystyle{plain}
\bibliography{bibSuivi2}

\end{document}
