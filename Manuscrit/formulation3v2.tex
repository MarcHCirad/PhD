\documentclass{article}
\usepackage{graphicx,ulem} 
\usepackage{color}
\usepackage{amsfonts,amsmath}
\usepackage{amsthm}
\usepackage{empheq}
\usepackage{mathtools}
\usepackage{multirow}
%\usepackage{tikz}
\usepackage{titlesec}
\usepackage{caption}
%\usepackage{lscape}
\usepackage{graphicx}
\captionsetup{justification=justified}
\usepackage[toc,page]{appendix}
\usepackage{hyperref}
\usepackage{subcaption}
\usepackage{pdftricks}
\usepackage{xcolor}
\begin{psinputs}
\usepackage{amsfonts,amsmath}
	\usepackage{pstricks-add}
   \usepackage{pstricks, pst-node}
   \usepackage{multido}
   \newcommand{\lfw}{\lambda_{F}}
\end{psinputs}

\textheight240mm \voffset-23mm \textwidth160mm \hoffset-20mm

   \graphicspath{{./Images/}{../Schema}}
%\graphicspath{{Figures}}
\setcounter{secnumdepth}{4}
\titleformat{\paragraph}
{\normalfont\normalsize\bfseries}{\theparagraph}{1em}{}
\titlespacing*{\paragraph}
{0pt}{3.25ex plus 1ex minus .2ex}{1.5ex plus .2ex}

\newcommand{\lfd}{\lambda_{F, D}}
\newcommand{\lfw}{\lambda_{F}}
\newcommand{\Kfa}{K_{F,\alpha}}
\newcommand{\cI}{c \mathcal{I}}

\newcommand{\marc}[1]{\textcolor{teal}{#1}}
\newcommand{\YD}[1]{\textcolor{magenta}{#1}}
\newcommand{\VY}[1]{\textcolor{blue}{#1}}

\DeclareMathOperator{\Tr}{Tr}
\newtheorem{theorem}{Theorem}
\newtheorem{prop}{Proposition}
\newtheorem{definition}{Definition}
\newtheorem{remark}{Remark}
\newtheorem{cor}{Corollary}
\newcommand*\phantomrel[1]{\mathrel{\phantom{#1}}}

\title{Modèle Chasseur}
\author{Marc Hétier, Yves Dumont  and Valaire Yatat-Djeumen}

\begin{document}

\maketitle
%{\hypersetup{hidelinks}
%\tableofcontents}
%\newpage


\section{Hunter Model}

We model the human-wild interactions by a consumer-resource model. We consider two areas, one corresponding to a domestic area (typically a village, etc), the other to a wild area (Savana, Forest, etc).


On the wild area, are present wild fauna $F_W$. The dynamic of $F_W$ follows a logistic equation, with a carrying capacity, $K_F$, dependent on the surrounding vegetation. To take into account the level of anthropization of the habitat, we introduce the non-negative parameter $\alpha \in [0, 1)$. When $\alpha > 0$, the carrying capacity of the habitat is reduced of $\alpha \%$ from its original value.

Wild fauna is hunted by humans present in the wild area. This is take into account by the functional response $\lfw H_W$, where $\lfw$ is the hunting rate and $H_W$ the number of hunter. The functional response is unbounded, to take into account the possibility of over-hunt.

Humans present in the domestic area, $H_D$, follow a consumer equation. Its growth will depend on the available resources. The dynamic followed by $H_D$ can be separated in four groups of term. 
First there is a constant growth due to \marc{à discuter}
\begin{itemize}
 \item food importation. We note $\mathcal{I}$ the quantity of raw products imported, and $c$ the conversion parameter between these raw products and the number of human it can nourish.  
 \item OR
 \item immigration from other inhabited area. This immigration recoup installation of new workers to develop industrial complex for example.
 \end{itemize} 

Prey hunted in the wild area are used to feed the villagers. Therefore, we consider a growth term $e \lfw H_W F_W$, where $e$ is a conversion rate between these hunted prey and the number of human it can nourish.

Villagers are also able to produce a certain amount of food at the rate $f_D$. Note that $f_D$ can be understand as $f_D = e \lfd F_D$ where $F_D$ is a constant amount of domestic animals.
Moreover, we assume that $H_D$ has a natural death-rate, $\mu_D$.

Fourth, villagers and hunters come back and force between the domestic and wild area. This migration is modeled by linear functions: $-m_D H_D + m_W H_W$.

The dynamic of human present in the wild area, $H_W$, corresponds simply to the migration terms. Note that we assume that the migration rate from the domestic area to the wild area is such that $m = \dfrac{m_D}{m_W} < 1$. This assumption makes sense, because humans stay a short amount of time in the Wild area than in the Domestic area.

Finally, the model is given by the following equations:
\begin{subequations}
\begin{equation}
\left\{ \begin{array}{l}
\dfrac{dH_D}{dt}= \cI + e\lfw H_W F_W + (f_D - \mu_D) H_D - m_D H_D + m_W H_W.
\end{array}\right.
\end{equation}
\begin{equation}
\left\lbrace \begin{array}{l}
\dfrac{dF_W}{dt} = r_F \left(1 - \dfrac{F_W}{K_F(1-\alpha)} \right) F_W - \lfw F_W H_W \\
\dfrac{dH_W}{dt}= m_D H_D - m_W H_W 
\end{array} \right.
\end{equation}
\label{equationsHDFWHW}
\end{subequations}


\begin{table}
\centering
\begin{tabular}{|c|c|c|c|}
\hline 
Parameter & Description & Unit & Value \\ 
\hline 
$t$ & Time & Year \\
$e$ & Prey-food conversion & Human Kg $^{-1}$ & \\
$f_D$ & Food produced by human population & Year$^{-1}$\\
$\mu_D$ & Human mortality rate  & Year$^{-1}$ & $1/60$\\
$m_D$ & Migration from domestic area to wild area & Year$^{-1}$ & \multirow{2}{*}{$\dfrac{m_D}{m_W} = 0.23$}\\
$m_W$ & Migration from wild area to domestic area & Year$^{-1}$\\
$r_F$ & Wild animal growth rate & Year$^{-1}$ & $0.46 \pm 0.37$\\
$K_F$ & Carrying capacity for wild animal, fixed by the environment& Kg\\
$\alpha$ & Proportion of anthropized environment & - & $[0, 1)$\\
$\lfw$ & Hunting rate & Humans Year $^{-1}$ \\
\hline
$\mathcal{I}$ & Imported product OR Incoming migration & Year$^{-1}$ OR Human Year$^{-1}$ \\
$c$ & Conversion rate between raw products and food OR nothing & Human OR -
\end{tabular}
\caption{List of the parameters used}
\end{table}

\section{Existence and uniqueness of global solutions}
In this section, we state general results on system \eqref{equationsHDFWHW}:  existence of an invariant region, existence and uniqueness of global solutions.

We begin by proving the local existence and uniqueness of solutions of system \eqref{equationsHDFWHW}. The right hand side of equations \eqref{equationsHDFWHW} defines a function $f(y)$ (with $y = (H_D, F_W, H_W)$) which is of class $\mathcal{C}^1$ on $\mathbf{R}^3$. The theorem of Cauchy-Lipschitz ensures that model \eqref{equationsHDFWHW} admits a unique solution, at least locally, for any given initial condition, see \cite{walter_ordinary_1998}.

Second, one can notice that having $f_D - \mu_D > 0$ implies $\dfrac{dH_D}{dt} > 0$ for all values of $t$, $F_W$ and $H_W$. This means that the human population will infinitely grow, which is not realistic. Then, on the following, we assume $f_D - \mu_D < 0$. This is only a necessary condition to avoid infinite growth. The following proposition \ref{Invariant region} indicates a compact and invariant subset of $\mathbf{R}_+^3$, on which the solutions are bounded.

\begin{prop}\label{Invariant region}
The region
$$\Omega = \Big\{\Big(H_D, F_W, H_W \Big) \in (\mathbb{R}_+)^3  \Big|H_D + H_W + eF_W \leq S^{max}, F_W \leq F_W^{max}, H_W \leq H_W^{max} \Big\},$$
where
$$
S^{max} = \Big(1 + \dfrac{m_D}{m_W} \Big) \dfrac{cI + e K_F (1-\alpha) (r_F + \mu_D - f_D)}{\mu_D - f_D},
\quad
F_W^{max} = K_F(1-\alpha),
\quad
H_W^{max} = \dfrac{m_D}{m_D + m_W} S^{max}
$$
is a compact and invariant set for system \eqref{equationsHDFWHW}. In particular, this means that any solutions of equations \eqref{equationsHDFWHW} with initial condition in $\Omega$ remains bounded.
\end{prop}

\begin{proof}
To prove this, we will use the notion of invariant region, see \cite{smoller_shock_1994}. Before, we introduce the variable $S = H_D + H_W + e F_W$. We have:

\begin{equation}
\dfrac{dS}{dt} = \cI + (f_D - \mu_D) \Big(S - H_W - eF_W \Big) + e r_F \left(1 - \dfrac{F_W}{K_F(1-\alpha)} \right) F_W.
\end{equation}

With this new variable, the model writes:
\begin{subequations}
\begin{equation}
\left\{ \begin{array}{l}
\dfrac{dS}{dt} = \cI + (f_D - \mu_D) \Big(S - H_W - eF_W \Big) + e r_F \left(1 - \dfrac{F_W}{K_F(1-\alpha)} \right) F_W.
\end{array}\right.
\end{equation}
\begin{equation}
\left\lbrace \begin{array}{l}
\dfrac{dF_W}{dt} = r_F \left(1 - \dfrac{F_W}{K_F(1-\alpha)} \right) F_W - \lfw F_W H_W \\
\dfrac{dH_W}{dt}= m_D (S - eF_W) - (m_W + m_D) H_W 
\end{array} \right.
\end{equation}
\label{equationsSFWHW}
\end{subequations}

We define the function $g(z)$ as the right hand side of this equations. We also introduce the following functions:
$$
G_1(z) = S - S^{max},
\quad
G_2(z) = F_W - F_W^{max},
\quad
G_3(z) = H_W - H_W^{max}
$$

Following \cite{smoller_shock_1994}, we will show that quantities $(\nabla G_1 \cdot g)|_{S = S^{max}}$, $(\nabla G_2 \cdot g)|_{F_W = F_W^{max}}$ and $(\nabla G_3 \cdot g)|_{H_W = H_W^{max}}$ are non-positive for $z \in \Omega_S = \Big\{ \Big(S, F_W, H_W \Big) \in (\mathbb{R})^3  \Big|S \leq S^{max}, F_W \leq F_W^{max}, H_W \leq H_W^{max} \Big\}$.

Using the fact that $\mu_D - f_D >0$ and $z\in \Omega_S$, we have:

\begin{align*}
(\nabla G_1 \cdot g)|_{S = S^{max}} &= \cI + (f_D - \mu_D) \Big(S^{max} - H_W - eF_W \Big) + e r_F \left(1 - \dfrac{F_W}{K_F(1-\alpha)} \right) F_W, \\
&= \cI + (f_D - \mu_D) S^{max} + (\mu_D - f_D) H_W + (\mu_D - f_D + r_F)eF_W - e r_F \dfrac{F_W^2}{K_F(1-\alpha)}, \\
&\leq  \cI + (f_D - \mu_D) S^{max} +(\mu_D - f_D) \dfrac{m_D}{m_D + m_W} S^{max} +e (\mu_D - f_D + r_F) K_F(1-\alpha), \\
&\leq  \cI + (\mu_D - f_D)\big(\dfrac{m_D}{m_D + m_W} - 1\Big)S^{max} +e (\mu_D - f_D + r_F) K_F(1-\alpha), \\
&\leq  - \dfrac{m_W}{m_D + m_W}(\mu_D - f_D)S^{max} +\cI +e (\mu_D - f_D + r_F) K_F(1-\alpha), \\
&\leq \left(-\dfrac{m_W}{m_D + m_W} \Big(1+ \dfrac{m_D}{m_W}\Big) + 1 \right) \Big(\cI +e (\mu_D - f_D + r_F) K_F(1-\alpha)\Big),\\
&\leq \left(-\dfrac{m_D + m_W}{m_D + m_W} + 1 \right) \Big(\cI +e (\mu_D - f_D + r_F) K_F(1-\alpha)\Big),
\end{align*}

that is $(\nabla G_1 \cdot g)|_{S = S^{max}} \leq 0$. The two others inequalities are straightforward to obtain. We have:
\begin{align*}
(\nabla G_2 \cdot g)|_{F_W = F_W^{max}} &= r_F  \left(1 - \dfrac{K_F (1-\alpha)}{K_F (1-\alpha)}\right)K_F (1-\alpha)  - \lfw H_W K_F (1-\alpha), \\
(\nabla G_2 \cdot g)|_{F_W = F_W^{max}} & = - \lfw H_W K_F (1-\alpha), \\
(\nabla G_2 \cdot g)|_{F_W = F_W^{max}} & \leq 0.
\end{align*}

The computations for $(\nabla G_3 \cdot g)|_{H_W = H_W^{max}}$ give:

\begin{align*}
(\nabla G_3 \cdot g)|_{H_W = H_W^{max}} &= m_D (S - eF_W) - (m_W + m_D) H_W^{max} \\
(\nabla G_3 \cdot g)|_{H_W = H_W^{max}} &= m_D (S - eF_W) - m_D S^{max} \\
(\nabla G_3 \cdot g)|_{H_W = H_W^{max}} & \leq m_D (S - S^{max} -  eF_W) \\
(\nabla G_3 \cdot g)|_{H_W = H_W^{max}} & \leq 0
\end{align*}

We have shown that $(\nabla G_1 \cdot g)|_{S = S^{max}} \leq 0$, $(\nabla G_2 \cdot g)|_{F_W = F_W^{max}} \leq 0$ and $(\nabla G_3 \cdot g)|_{H_W = H_W^{max}} \leq 0$ in  $\Omega_S$.  According to \cite{smoller_shock_1994}, this prove that $\Omega_S$ is an invariant region for system \eqref{equationsSFWHW}.

This also shows that the set  $\Big\{\Big(H_D, F_W, H_W \Big) \in \mathbb{R}^3  \Big|H_D + H_W + eF_W \leq S^{max}, F_W \leq F_W^{max}, H_W \leq H_W^{max} \Big\}$ is invariant for system \eqref{equationsHDFWHW}. 

Moreover, for any point $y \in \partial (\mathbb{R}_+)^3$, the vector field defined by $f(y)$ is either tangent or directed inward. Then, $\Omega$ is an invariant region for equations \eqref{equationsHDFWHW}. 
\end{proof}

We finally conclude this section by proving that equations \eqref{equationsHDFWHW} define a dynamical system on $\Omega$.

\begin{prop}
Equations \eqref{equationsHDFWHW} define a dynamical system on $\Omega$, that is, for any initial condition $(t_0, y)$ with $t_0 \in \mathbf{R}$ and $y \in \Omega$, it exists a unique solution of equations \eqref{equationsHDFWHW}, and this solution is defined for all $t \geq t_0$.
\end{prop}

\begin{proof}
We already prove that equations \eqref{equationsHDFWHW} admit, at least locally, a unique solution for every initial condition. Moreover, since $\Omega$ is an invariant region, the solutions with initial condition on $\Omega$ are bounded. Based on uniform boundedness, we deduce that solutions of system $\eqref{equationsHDFWHW}$ with initial condition on $\Omega$ exists globally, for all $t\geq t_0$. Therefore, $\eqref{equationsHDFWHW}$ defines a dynamical system on $\Omega$.
\end{proof}

\section{Model analysis in the case $\cI = 0$}
In this section, we study the specific case where $\cI = 0$. This mean that there is no importation of resources and the human population mainly dependents on hunt to subsist. The system rewrites:
\begin{subequations}
\begin{equation}
\left\{ \begin{array}{l}
\dfrac{dH_D}{dt}= e\lfw H_W F_W + (f_D - \mu_D) H_D - m_D H_D + m_W H_W.
\end{array}\right.
\end{equation}
\begin{equation}
\left\lbrace \begin{array}{l}
\dfrac{dF_W}{dt} = r_F \left(1 - \dfrac{F_W}{K_F(1-\alpha)} \right) F_W - \lfw F_W H_W \\
\dfrac{dH_W}{dt}= m_D H_D - m_W H_W 
\end{array} \right.
\end{equation}
\label{equationsHDFWHW, cI=0}
\end{subequations}

\subsection{Theoretical analysis}
On the following, we state and prove results dealing with existence of equilibrium points of this system as well as with their stability. We start by a proposition concerning their existence.


\begin{prop}
\label{theoremEquilibre, cI=0}
The following results hold:
\begin{itemize}
\item System \eqref{equationsHDFWHW, cI=0} admits a trivial equilibrium $TE = \Big(0,0,0\Big)$ that always exists.
\item System \eqref{equationsHDFWHW, cI=0} admits a fauna-only equilibrium $EE^{F_W} = \Big(0, K_F(1-\alpha), 0 \Big)$ that always exists.
\item When
$$
\dfrac{m e \lfw K_F(1-\alpha)}{\mu_D - f_D} >1,
$$ 
then system \eqref{equationsHDFWHW, cI=0} admits a unique coexistence equilibrium $EE^{HF_W} = \Big(H^*_{D, \cI = 0}, F^*_{W, \cI = 0}, H^*_{W, \cI = 0} \Big)$ \\ 
where 
$$F^*_{W, \cI = 0} = \dfrac{\mu_D - f_D}{\lfw m e},
\quad 
H^*_{D, \cI = 0} = \dfrac{r_F}{\lfw m} \Big(1 - \dfrac{F^*_{W, \cI = 0}}{K_F(1-\alpha)} \Big),
\quad 
H^*_{W, \cI = 0} = m H^*_{D, \cI = 0}.$$
\end{itemize}
\end{prop}

\begin{proof}
An equilibrium of system \eqref{equationsHDFWHW, cI=0} satisfies the system of equations:
\begin{equation}\label{system-equilibre, cI=0}
\left\lbrace \begin{array}{cll}
 e \lfw m F_W^* + f_D - \mu_D = 0& \mbox{or} & H_D^* = 0,\\
F_W^* - K_F(1-\alpha) \Big(1 - \dfrac{\lfw m H^*_D}{r_F} \Big) = 0& \mbox{or} & F^*_W = 0,\\
H_W^* = \dfrac{m_D}{m_W} H_D^* = m H_D^*.&&
\end{array} \right.
\end{equation}
When $H_D^*=0$ and $F_W^*=0$, we recover the trivial equilibrium $TE = \Big(0,0,0\Big)$. When $H_D^*=0$ and $F_W^*\neq0$, we obtain the fauna-only equilibrium $EE^{F_W} = \Big(0, K_F(1-\alpha), 0 \Big)$. Finally, when $H_D^*\neq0$ and $F_W^*\neq0$, direct computations lead to a unique coexistence equilibrium 
$EE^{HF_W}_{\cI = 0} = \Big(H^*_{D, \cI = 0}, F^*_{W, \cI = 0}, H^*_{W, \cI = 0} \Big)$ \\ 
where 
$$
F^*_{W, \cI = 0} = \dfrac{\mu_D - f_D}{\lfw m e},
\quad
H^*_{D, \cI = 0} = \dfrac{r_F}{\lfw m} \Big(1 - \dfrac{F^*_{W, \cI = 0}}{K_F(1-\alpha)} \Big),
\quad
H^*_{W, \cI = 0} = m H^*_{D, \cI = 0}
$$
which is biologically meaningful whenever $\dfrac{\mu_D - f_D}{\lfw m e} < K_F(1-\alpha).$
\end{proof}

Now, we look for the local asymptotic stability of the equilibrium.

\begin{prop}\label{propLAS, cI=0} The following results are valid.
\begin{itemize}
\item The trivial equilibrium $TE$ is unstable.
\item When $\dfrac{m e \lfw K_F(1-\alpha)}{\mu_D - f_D} < 1$, the fauna equilibrium, $EE^{F_W}$, is Locally Asymptotically Stable (LAS).
\item When $\dfrac{m e \lfw K_F(1-\alpha)}{\mu_D - f_D} > 1$, the coexistence equilibrium, $EE^{HF_W}_{\cI =0}$, exists. It is LAS if 
$$
\lfw K_F(1-\alpha) < \Big(\Kfa \lfw \Big)^*
$$
where \begin{multline*}
\Big(\Kfa \lfw \Big)^* = \\
 \dfrac{\left[m_{W}(\mu_{D}-f_{D})+\big(\mu_{D}-f_{D}+m_{D}+m_{W})^{2}\right]\left(1+\sqrt{1+4\dfrac{m_{W}r_{F}\left(\mu_{D}-f_{D}\right)\big(\mu_{D}-f_{D}+m_{D}+m_{W})}{\left[m_{W}\dfrac{\mu_{D}-f_{D}}{e}+\big(\mu_{D}-f_{D}+m_{D}+m_{W})^{2}\right]^{2}}}\right)}{2em_D}
\end{multline*}
\end{itemize}
\end{prop}

\begin{proof}
To prove this theorem, we look at the Jacobian of system \eqref{equationsHDFWHW, cI=0}. It is given by:

\begin{equation*}
\mathcal{J}(H_D, F_W, H_W) = \begin{bmatrix}
f_D-\mu_D - m_D & e \lfw H_W & e\lfw F_W + m_W \\
0 & r_F \left( 1 - \dfrac{2F_W}{K_F(1-\alpha)} \right) - \lfw H_W & - \lfw F_W \\
m_D & 0 & -m_W
\end{bmatrix}.
\end{equation*}

\begin{itemize}
\item At equilibrium $TE$, we have:
\begin{equation*}
\mathcal{J}(TE) = \begin{bmatrix}
f_D-\mu_D - m_D & 0 &  m_W \\
0 & r_F  &  0\\
m_D & 0 & -m_W
\end{bmatrix}.
\end{equation*}
and $r_F > 0$ is an eigenvalue of $\mathcal{J}(TE)$. So, $TE$ is unstable.
\item At equilibrium $EE^{F_W}$, we have
\begin{equation*}
\mathcal{J}(EE^{F_W}) = \begin{bmatrix}
f_D-\mu_D - m_D & 0 & e\lfw K_F(1-\alpha) + m_W \\
0 & -r_F  & -\lfw K_F(1-\alpha)  \\
m_D & 0 & -m_W
\end{bmatrix}.
\end{equation*}

The characteristic polynomial of $\mathcal{J}(EE^{F_W})$ is given by:
\begin{equation*}
\chi(X) = (X +r_F) \times \left(X^2 - X\Big(f_D - \mu_D - m_D - m_W \Big) + m_W(\mu_D - f_D) - m_D e \lfw K_F(1-\alpha) \right).
\end{equation*}

We need to determine the sign of the roots' real part of the second factor. Since the coefficient in $X$ is positive, the sign of their real part is determined by the sign of the constant coefficient.
The roots have a negative real part if the constant coefficient is positive \textit{ie} if $\dfrac{m e \lfw K_F(1-\alpha)}{\mu_D - f_D} < 1 $, and a positive real part if $\dfrac{m e \lfw K_F(1-\alpha)}{\mu_D - f_D} > 1 $. Stability of $\mathcal{J}(EE^{F_W})$ follows.

\item Now, we look for the asymptotic stability of the equilibrium of coexistence $EE^{HF_W}_{\cI=0}$. The first part of the computations are common with the ones for proving the LAS of equilibrium $EE^{HF_W}_{\cI >0}$.   To keep some generality, we use notation $EE^{HF_W}$ for both $EE^{HF_W}_{\cI =0}$ and $EE^{HF_W}_{\cI >0}$. We have 
\begin{equation*}
\mathcal{J}(EE^{H F_W}) = \begin{bmatrix}
f_D -\mu_D - m_D & e \lfw m H_D^* & e \lfw F^*_W +m_W \\
0 & -r_F \dfrac{F_W^*}{K_F(1-\alpha)} & - \lfw F_W^* \\
m_D & 0 & -m_W
\end{bmatrix}.
\end{equation*} and its characteristic polynomial is given by: $\chi = X^3 + a_2 X^2 + a_1 X + a_0$. According to the Routh-Hurwitz criterion \marc{ref}, $EE^{H F_W}$ are LAS if $a_i > 0$ for $i=1,2,3$ and $a_2 a_1 - a_0 > 0$. Note that $a_2 = - \Tr(\mathcal{J}(EE^{H F_W}))$ and $a_0 = - \det (\mathcal{J}(EE^{H F_W}))$.

We have:
\begin{equation}\label{expressionA2}
a_2 = - \Tr = -(f_D - \mu_D - m_D - r_F \dfrac{F_W^*}{K_{F, \alpha}} - m_W)
\end{equation}
that is $a_2>0$. Coefficient $a_0$ is given by:

\begin{subequations}
\begin{align}
a_0 &= -\det\Big(\mathcal{J}(EE^{H F_W})\Big), \\
a_0 &= \Big(\mu_D + m_D -f_D \Big) m_W r_F \dfrac{F^*_W}{K_{F, \alpha}} - m_D\Big(-e\lfw ^2 m F_W^* H_D^* + r_F \big(e \lfw F^*_W + m_W \big) \dfrac{F^*_W}{K_{F, \alpha}} \Big), \\
a_0 &= (\mu_D-f_D) m_W r_F \dfrac{F^*_W}{K_{F, \alpha}} + m_D e \lfw \Big( m \lfw H_D^* - r_F \dfrac{F^*_W}{K_{F, \alpha}} \Big) F^*_W, \\
a_0 &= (\mu_D-f_D) m_W r_F \dfrac{F^*_W}{K_{F, \alpha}} + m_D e \lfw \Big( r_F - 2 r_F \dfrac{F^*_W}{K_{F, \alpha}} \Big) F^*_W, \\
a_0 &= - m_D \lfw \left( F_W^* \Big(2 \dfrac{e r_F}{K_{F, \alpha}} \Big) - \Big(\dfrac{\mu_D - f_D}{m} \dfrac{r_F}{K_{F, \alpha} \lfw} + er_F \Big) \right) F_W^*. \label{expressionA0}
\end{align}
\end{subequations}

When $\cI = 0$, we have:

\begin{equation*}
F_W^* = \dfrac{\mu_D - f_D}{\lfw m e}.
\end{equation*} 
Injecting this expression into \eqref{expressionA0}, we obtain:

\begin{equation*}
a_{0, \cI=0} = m_D \lfw r_F e\left(1 - \dfrac{\mu_D - f_D}{\lfw m \Kfa e } \right) F_W^*
\end{equation*}
that is $a_{0, \cI=0}>0$ given the assumption. The coefficient $a_1$ is given by:
\begin{subequations}
\begin{align}
a_1 &= \big( \mu_D  -f_D + m_D) r_F \dfrac{F^*_W}{K_{F, \alpha}} + (\mu_D -f_D + m_D) m_W + r_F \dfrac{F_W^*}{K_{F, \alpha}} m_W - m_D (e\lfw F^*_W + m_W), \\
a_1 &= \big( \mu_D -f_D + m_D) r_F \dfrac{F^*_W}{K_{F, \alpha}}  + r_F \dfrac{F_W^*}{K_{F, \alpha}} m_W + m_D e\lfw   \left(\dfrac{\mu_D - f_D}{m e\lfw} - F^*_W \right),  \\
a_1 &= \big( \mu_D -f_D + m_D + m_W) r_F \dfrac{F^*_W}{K_{F, \alpha}}   + m_D e\lfw   \left(\dfrac{\mu_D - f_D}{m e\lfw} - F^*_W \right). \label{expressionA1}
\end{align}
\end{subequations}

Again, using the expression of $F^*_W$ in the case where $\cI = 0$, we have:

\begin{equation*}
a_{1, \cI =0} = \big( \mu_D -f_D + m_D + m_W) r_F \dfrac{F^*_W}{K_{F, \alpha}}.
\end{equation*}
and we do have $a_{1, \cI =0} > 0$.

The first assumption of Rough-Hurwitz is verified, $a_{i, \cI =0} > 0$ for $i=1,2,3$. Therefore, the asymptotic stability of $EE^{HF_W,  \cI =0}$ only depends on the sign of $\Delta_{Stab}= a_2 a_1 - a_0$, which has to be positive. We have:

\begin{multline*}
\Delta_{Stab, \cI = 0} > 0 \\
\Leftrightarrow \left(\mu_D - f_D + m_D + m_W + \dfrac{r_F(\mu_D-f_D)}{\lfw \Kfa m e} \right) \times  \left(\big( \mu_D -f_D + m_D + m_W)\dfrac{r_F(\mu_D-f_D)}{\lfw \Kfa m e} \right) > \\ m_D\left(er_F - \dfrac{(\mu_D - f_D) r_F}{\lfw m \Kfa} \right) \dfrac{\mu_D - f_D}{m e}, \\
\Leftrightarrow \left(\mu_D - f_D + m_D + m_W \right)^2 \dfrac{r_F(\mu_D-f_D)}{\lfw \Kfa m e} + \left(\mu_D - f_D + m_D + m_W \right)\left(\dfrac{r_F(\mu_D-f_D)}{\lfw \Kfa m e} \right)^2 > \\ \left(er_F - \dfrac{(\mu_D - f_D) r_F}{\lfw m \Kfa} \right) \dfrac{m_W(\mu_D - f_D)}{e}, \\
\Leftrightarrow \lfw \Kfa \left(\mu_D - f_D + m_D + m_W \right)^2 \dfrac{r_F(\mu_D-f_D)}{m e} + \left(\mu_D - f_D + m_D + m_W \right)\left(\dfrac{r_F(\mu_D-f_D)}{m e} \right)^2 > \\m_D \left((\lfw \Kfa)^2 er_F - \dfrac{(\mu_D - f_D) r_F}{m}\lfw \Kfa  \right) \dfrac{(\mu_D - f_D)}{me}, \\
\Leftrightarrow \lfw \Kfa \left(\mu_D - f_D + m_D + m_W \right)^2 + \left(\mu_D - f_D + m_D + m_W \right)\dfrac{r_F(\mu_D-f_D)}{m e} > \\ m_D\left((\lfw \Kfa)^2 e - \dfrac{(\mu_D - f_D) }{m}\lfw \Kfa  \right), \\
\Leftrightarrow 0 > \Big(\lfw \Kfa \Big)^2 em_D -  \Big(\lfw \Kfa \Big) \left(m_W (\mu_D - f_D) + \big(\mu_D - f_D + m_D + m_W\big)^2 \right) \\ - \big(\mu_D - f_D + m_D + m_W\big) \dfrac{r_F (\mu_D - f_D)m_W}{m_D e}.
\end{multline*}

We define 
\begin{multline*}
P_{\Delta_{Stab, \cI = 0}}(X) := X^2 em_D -  X \left(m_W (\mu_D - f_D) + \big(\mu_D - f_D + m_D + m_W\big)^2 \right) \\ - \big(\mu_D - f_D + m_D + m_W\big) \dfrac{r_F (\mu_D - f_D)m_W}{m_D e},
\end{multline*} 

such that we have 
\begin{equation}
\Delta_{Stab, \cI = 0} > 0 \Leftrightarrow P_{\Delta_{Stab, \cI = 0}}(K_F(1-\alpha)\lfw) < 0.
\label{equivalenceDeltaStabP}
\end{equation}

$P_{\Delta_{Stab, \cI = 0}}$ has a positive dominant coefficient, and its other coefficients are negative. So,  $P_{\Delta_{Stab, \cI = 0}}$ admits a unique positive root, noted $\Big(\Kfa \lfw \Big)^*$, given by:
\begin{multline}
\Big(\Kfa \lfw \Big)^* = \\
 \dfrac{\left[m_{W}(\mu_{D}-f_{D})+\big(\mu_{D}-f_{D}+m_{D}+m_{W})^{2}\right]\left(1+\sqrt{1+4\dfrac{m_{W}r_{F}\left(\mu_{D}-f_{D}\right)\big(\mu_{D}-f_{D}+m_{D}+m_{W})}{\left[m_{W}\dfrac{\mu_{D}-f_{D}}{e}+\big(\mu_{D}-f_{D}+m_{D}+m_{W})^{2}\right]^{2}}}\right)}{2em_D}
\end{multline}

Moreover, $P_{\Delta_{Stab}}$ is negative on $\left[0, \Big(\Kfa \lfw \Big)^* \right)$ and positive on $\left(\Big(\Kfa \lfw \Big)^*, +\infty \right)$. Using \eqref{equivalenceDeltaStabP}, we obtain that $EE^{HF_W}_{\cI = 0}$ is locally asymptotically stable if $\lfw \Kfa < \Big(\Kfa \lfw \Big)^*$.

\end{itemize}
\end{proof}


The following intermediate table gives an overview of the long term behavior:

\begin{table}[!ht]
\centering
\def\arraystretch{2}
\begin{tabular}{c|c|c|c}
$\cI$ & $\dfrac{\mu_D - f_D}{\lfw m e K_F(1-\alpha)}$ &  $\dfrac{\Big(K_F(1-\alpha) \lfw \Big)^*}{K_F(1-\alpha) \lfw}$ & \\
\hline
\multirow{3}{*}{$=0$} & $ > 1$ & &$EE^{F_W}$ exists and is LAS.  \\
\cline{2-4}
 & \multirow{2}{*}{$< 1$} & $>1$ &$EE^{HF_W}_{\cI=0}$ exists and is LAS.\\
 \cline{3-4}
 & & $ <1$ &$EE^{HF_W}_{\cI=0}$ exists and is unstable.
\end{tabular}
\caption{\centering Intermediate table giving the known conditions of existence and asymptotic stability of equilibrium for system \eqref{equationsHDFWHW, cI=0}}
\end{table}

We can complete this table by looking for global asymptotic stability and existence of limit cycle. We will use the following theorem:

\begin{theorem}\label{theoremVidyasagar} \cite{vidyasagar_decomposition_1980, dumont_mathematical_2012}
Consider the following $\mathcal{C}^1$ system
\begin{equation}
\def\arraystretch{2}
\left\{ \begin{array}{l}
\dfrac{dx}{dt} = f(x), \\
\dfrac{dy}{dt} = g(x, y) 
\end{array} \right.
\label{equationVidyasagar}
\end{equation}

with $(x, y) \in \mathbf{R}^n \times\mathbf{R}^m$. Let $(x^*, y^*)$ be an equilibrium point.
If $x^*$ is GAS in $\mathbf{R}^n$ for the system $\dfrac{dx}{dt} = f(x)$, and if $y^*$ is GAS in $\mathbf{R}^m$ for the system $\dfrac{dy}{dt} = g(x^*, y)$, then $(x^*, y^*)$ is (locally) asymptotically stable for system \eqref{equationVidyasagar}. Moreover, if all trajectories of \eqref{equationVidyasagar} are forward bounded, then $(x^*, y^*)$ is GAS for \eqref{equationVidyasagar}.
\end{theorem}

The following proposition holds true:

\begin{prop}\label{propEEFGAS}If 
$$
\dfrac{\mu_D - f_D}{\lfw m e K_F(1-\alpha)} >1,
$$
that is if equilibrium $EE^{F_W}$ is LAS, then it is globally asymptotically stable (GAS) on $\Omega$ for system \eqref{equationsHDFWHW, cI=0}.
\end{prop}

\begin{proof}
In the following, we assume $\dfrac{\mu_D - f_D}{\lfw m e K_F(1-\alpha)} >1$. We consider a solution $(H_D^s, F_W^s, H_W^s)$ of equations \eqref{equationsHDFWHW, cI=0} with initial conditions in $\Omega$. Using the fact that $\Omega$ is an invariant region, we have:

\begin{equation}
\def\arraystretch{2}
\left\{ \begin{array}{l}
\dfrac{dH^s_D}{dt} \leq e\lfw H^s_W K_F(1-\alpha) + (f_D - \mu_D) H^s_D - m_D H^s_D + m_W H^s_W , \\
\dfrac{dF^s_W}{dt} = r_F \left(1 - \dfrac{F^s_W}{K_F(1-\alpha)} \right) F^s_W - \lfw F^s_W H^s_W \\
\dfrac{dH^s_W}{dt}= m_D H^s_D - m_W H^s_W 
\end{array} \right.
\end{equation}

We can study the limit system, given by:
\begin{equation}
\def\arraystretch{2}
\left\{ \begin{array}{l}
\dfrac{dH_D}{dt} = \Big(e\lfw K_F(1-\alpha) + m_W\Big)H_W + (f_D - \mu_D - m_D) H_D \\
\dfrac{dF_W}{dt} = r_F \left(1 - \dfrac{F_W}{K_F(1-\alpha)} \right) F_W - \lfw F_W H_W \\
\dfrac{dH_W}{dt}= m_D H_D - m_W H_W 
\end{array} \right.
\label{limitSystem}
\end{equation}

We will apply theorem \ref{theoremVidyasagar} on this system, with $x = (H_D, H_W)$, $y = F_W$, $x^* = (0,0)$ and $y^* = K_F(1- \alpha)$.

We have that $x^*$ is GAS for system $\dfrac{dx}{dt} = f_{[1,3]}(x)$. Indeed, $x^*$ is the unique equilibrium of this system, and it is LAS since $\dfrac{\mu_D - f_D}{\lfw m e K_F(1-\alpha)} >1$. By applying the Bendixson-Dulac theorem \marc{ref}, we show that $\dfrac{dx}{dt} = f_{[1,3]}(x)$ does not admit any limit cycle. Therefore, the Poincarré theorem \marc{ref} shows that $(0, 0)$ is GAS for $\dfrac{dx}{dt} = f_{[1,3]}(x)$.

It is quite immediate to show that $y^*$ is GAS for system $\dfrac{dy}{dt} = f_{[2]}(x^*, y)$. 

Moreover, the trajectories of the solution of limit-system \eqref{limitSystem} with initial condition in $\Omega$ are bounded (theorem \ref{Invariant region}). So, we can apply theorem \ref{theoremVidyasagar}, and we obtain that equilibrium $\Big(0, K_F(1-\alpha), 0 \Big)$ is GAS on $\Omega$ for the limit system, and therefore for the original system \eqref{equationsHDFWHW}
\end{proof}


On the following, we will show the existence of limit cycle. The following theorem will be useful:

\begin{theorem}\cite{zhu_stable_1994}\label{periodicASOrbit}
We consider the system of differential equations
$$
\dfrac{dx}{dt} = g(x), \quad x \in \mathcal{D}.
$$
If
\begin{itemize}
\item $\mathcal{D}$ is an open, $p$-convex subset of $\mathbf{R}^3$,
\item $g$ is analytic in $\mathcal{D}$,
\item $\mathcal{D}$ contains a unique equilibrium point $x^*$ and $\det(\mathcal{J}_g(x^*)) < 0$,
\item the system is competitive and irreducible in $\mathcal{D}$,
\item the system is dissipative: For each $x_0 \in \mathcal{D}$, the positive semi-orbit through $x_0$, $\phi^+(x_0)$ has a compact closure in $\mathcal{D}$ . Moreover, there exists a compact subset $\mathcal{B}$ of $\mathcal{D}$ with the property that for each $x_0 \in \mathcal{D}$, there exists $T(x_0) > 0$ such that $x(t, x_0) \in \mathcal{B}$ for $t \geq T(x_0)$.
\end{itemize}

then either $x^*$ is stable, or there exists at least one non-trivial orbitally asymptotically stable  periodic orbit in $\mathcal{D}$.
\end{theorem}

The following proposition holds true:

\begin{prop}\label{LimitCycle, cI=0}
If $\dfrac{\mu_D - f_D}{\lfw m e K_F(1-\alpha)} < 1$ and if:
\begin{itemize}
\item $\lfw K_F(1-\alpha) < \Big(K_F(1-\alpha) \lfw \Big)^*$, that is if equilibrium $EE^{HF_W}$ is LAS, then it is GAS on $\Omega$ for system \eqref{equationsHDFWHW, cI=0}.
\item $\lfw K_F(1-\alpha) > \Big(K_F(1-\alpha) \lfw \Big)^*$, system \eqref{equationsHDFWHW, cI=0} admits an orbitally asymptotically stable periodic solution.
\end{itemize}
\end{prop}

\begin{proof}
Following \cite{wang_predator-prey_1997}, we do the following change of variables: $h_D =  H_D$, $f_W = -F_W$ and $h_W = -H_W$.  The system \eqref{equationsHDFWHW, cI=0} is transformed into:
\begin{subequations}
\begin{equation}
\left\{ \begin{array}{l}
\dfrac{dh_D}{dt}= e\lfw h_W f_W + (f_D - \mu_D) h_D - m_D h_D - m_W h_W.
\end{array}\right.
\end{equation}
\begin{equation}
\left\lbrace \begin{array}{l}
\dfrac{df_W}{dt} = r_F \left(1 + \dfrac{f_W}{K_F(1-\alpha)} \right) f_W + \lfw f_W h_W \\
\dfrac{dh_W}{dt}= -m_D h_D - m_W h_W 
\end{array} \right.
\end{equation}
\label{equationshDfWhW, cI=0}
\end{subequations}

We note $\mathcal{D} = \Big\{z = (h_D, f_W, h_W) | 0 < h_D, f_W < 0, h_W < 0 \Big\}$, and $g(z)$ the right hand side of the system. It is clear that $\mathcal{D}$ is a $p$-convex set, in which $g$ is analytic, and irreducible.

The Jacobian of $g$ is given by:

\begin{equation*}
\mathcal{J}_g(z) = \begin{bmatrix}
f_D -\mu_D - m_D & e \lfw h_W & e \lfw f_W - m_W \\
0 & r_F \Big(1 + \dfrac{2 f_W}{K_F(1-\alpha)} + \lfw h_W & \lfw f_W \\
-m_D & 0 & -m_W
\end{bmatrix}.
\end{equation*}
The non diagonal term are non positive for $z \in \mathcal{D}$. Thus, system \eqref{equationshDfWhW, cI=0} is competitive in $\mathcal{D}$.

We note $h_D^* = H_D^*$, $f_W^* = -F_W^*$ and $h_W^* = -H_W^*$. According to propositions \ref{theoremEquilibre, cI=0} and \ref{propLAS, cI=0}  $EE^{hf_W} = \Big(h_D^*, f_W^*, h_W^* \Big)$, is the unique equilibrium of system \eqref{equationshDfWhW, cI=0}, and $\det(\mathcal{J}_g((h_D^*, f_W^*,h_W^*)) < 0$. 

When $\lfw K_F(1-\alpha) > \Big(K_F(1-\alpha) \lfw \Big)^*$, $EE^{hf_W}$ is unstable. Since it exists a compact and invariant region for system \eqref{equationsHDFWHW}, it is also the case for system \eqref{equationshDfWhW, cI=0}. Thus, system \eqref{equationshDfWhW, cI=0} is also dissipative, and according to theorem \ref{periodicASOrbit}, system \eqref{equationshDfWhW, cI=0} admits a non-trivial orbitally asymptotically stable  periodic orbit in $\mathcal{D}$. 

On the other hand, if $\lfw K_F(1-\alpha) < \Big(K_F(1-\alpha) \lfw \Big)^*$,  $EE^{hf_W}$ is locally asymptotically stable, and since the system is competitive, it is globally asymptotically stable.

Since the change of variable we have done is an automorphism, this result is also true for system \eqref{equationsHDFWHW, cI=0}.
\end{proof}

The results we obtained are summarized in the following table:
\begin{table}[!ht]
\centering
\def\arraystretch{2}
\begin{tabular}{c|c|c|c}
$\cI$ & $\dfrac{\mu_D - f_D}{\lfw m e K_F(1-\alpha)}$ &  $\dfrac{\Big(K_F(1-\alpha) \lfw \Big)^*}{K_F(1-\alpha) \lfw}$ & \\
\hline
\multirow{3}{*}{$=0$} & $ > 1$ & &$EE^{F_W}$ exists and is GAS.  \\
\cline{2-4}
 & \multirow{3}{*}{$< 1$} & $>1$ &$EE^{HF_W}_{\cI=0}$ exists and is GAS.\\
 \cline{3-4}
 & & \multirow{2}{*}{$ < 1$} & $EE^{HF_W}_{\cI=0}$ exists and is unstable ; there is an asymptotically \\
 & & &  stable periodic solution.
\end{tabular}
\caption{\centering Conditions of existence and asymptotic stability of equilibrium for system \eqref{equationsHDFWHW, cI=0}}
\end{table}


\subsection{Interpretation}
Inside this model, the parameters $\alpha$ and $\lfw$ represents the impact of human activities on their environment. It is interesting to interpret the previous results as a function of this parameters, in order to understand the consequences of an increase (or decrease) of hunt activities or environment destruction.

First, we start by condition $\dfrac{\mu_D - f_D}{\lfw m e K_F(1-\alpha)} < 1$, which is required for $EE^{HF_W}$ to exist. We propose the following definition:

\begin{definition}\label{defLambdaMin, cI=0} We define 
$$\lambda_{F, \cI=0}^{Min} := \dfrac{\mu_D - f_D}{m e K_F(1-\alpha)}$$
such that 
$$
\text{$EE^{HF_W}_{c\mathcal{I} = 0}$ exists} \Leftrightarrow \lambda_{F, \cI=0}^{Min}  <  \lfw.
$$
\end{definition}
\begin{proof}
The proof is straightforward: $EE^{HF_W}$ exists if $\dfrac{\mu_D - f_D}{\lfw m e K_F(1-\alpha)} < 1 \Leftrightarrow \dfrac{\mu_D - f_D}{ m e K_F(1-\alpha)} < \lfw$.
\end{proof}

This means that if the hunting rate is not sufficient, there is no co-existence possible, and even no steady state with human population. This is due to the fact that when $\cI = 0$, only hunting activities ensure food intake. Consequently, if there is not enough hunt, the human population can not subsist.

We can note that $\lambda_{F, \cI=0}^{Min}$ is a increasing function of the antrhopization parameter $\alpha$ : the more anthropized the environment, the fewer wild animals there are, and the greater the hunting rate required. 

\begin{definition}
We define
\begin{multline*}
\lambda_{F, \cI =0}^{Max}  = \\
 \dfrac{\left[m_{W}(\mu_{D}-f_{D})+\big(\mu_{D}-f_{D}+m_{D}+m_{W})^{2}\right]\left(1+\sqrt{1+4\dfrac{m_{W}r_{F}\left(\mu_{D}-f_{D}\right)\big(\mu_{D}-f_{D}+m_{D}+m_{W})}{\left[m_{W}\dfrac{\mu_{D}-f_{D}}{e}+\big(\mu_{D}-f_{D}+m_{D}+m_{W})^{2}\right]^{2}}}\right)}{2em_D K_F(1- \alpha)}
\end{multline*}

such that $$
\text{$EE^{HF_W}_{c\mathcal{I} = 0}$ is GAS} \Leftrightarrow \lfw < \lambda_{F, \cI =0}^{Max}
.$$
\end{definition}

This result can be interpreted as follow: if the hunting rate is too high, the system's dynamic tends toward a limit cycle, which correspond to a classical predator-prey system. We can also note that $\lambda_{F, \cI =0}^{Max} $ is decreasing as a function of $K_F(1-\alpha)$. 



\section{Model analysis in the case $c\mathcal{I} > 0$}
Now, we consider the case were food import occurs, \textit{ie} we assume $c\mathcal{I} > 0$. Since the human subsistence does not depend only on hunt, the system's dynamic and its interpretation change.


\subsection{Theoretical analysis}
\begin{prop}
The following results hold:
\begin{itemize}
\item System \eqref{equationsHDFWHW} has a Human-only equilibrium $EE^{H} = \Big(\dfrac{c\mathcal{I}}{\mu_D - f_D}, 0, \dfrac{m \ c\mathcal{I}}{\mu_D - f_D} \Big)$ that always exists.
\item If $$ \lfw < \dfrac{r_F (\mu_D -f_D)}{m c\mathcal{I}},$$ then system \eqref{equationsHDFWHW} has a unique coexistence equilibrium $EE^{HF_W}_{c\mathcal{I} = 0} = \Big(H^*_{D}, F^*_{W}, m H^*_{D} \Big)$
where
$$F^*_{W} = \dfrac{K_F(1-\alpha)}{2}\left(1 - \dfrac{\sqrt{\Delta_F}}{er_F}\right) + \dfrac{\mu_D - f_D}{2\lfw m e},\quad
H^*_{D} = \dfrac{r_F}{\lfw m} \Big(1 - \dfrac{F^*_{W}}{K_F(1-\alpha)} \Big),
\quad 
H^*_{W} = m H^*_{D}$$
and
$$
\Delta_F = \left(er_F - \dfrac{(\mu_D - f_D) r_F}{\lfw m K_F(1-\alpha)} \right)^2 + 4\dfrac{er_F}{K_F(1-\alpha)}  c\mathcal{I}.
$$
\end{itemize} 
\end{prop}

\begin{proof}
An equilibrium of system \eqref{equationsHDFWHW} satisfies the system of equations:
\begin{equation}\label{system-equilibre}
\left\lbrace \begin{array}{cll}
c\mathcal{I} + e \lfw m F_W^* H_D^* + (f_D - \mu_D) H_D^* = 0,&&\\
F_W^* - K_F(1-\alpha) \Big(1 - \dfrac{\lfw m H^*_D}{r_F} \Big) = 0& \mbox{or} & F^*_W = 0,\\
H_W^* = \dfrac{m_D}{m_W} H_D^* = m H_D^*.&&
\end{array} \right.
\end{equation}

The solution of system \eqref{system-equilibre} when $F_W^* = 0$ is the Human-only equilibrium $EE^{H} = \Big(\dfrac{c\mathcal{I}}{\mu_D - f_D}, 0, \dfrac{m \ c\mathcal{I}}{\mu_D - f_D} \Big)$.
In the sequel, we assume that $F_W^* > 0$. In this case, $F^*_W$ is solution of the quadratic equation
\begin{equation}
P_F(X) := X^2 \left(\dfrac{er_F}{K_F(1-\alpha)} \right) - X \left(er_F + \dfrac{(\mu_D - f_D) r_F}{\lfw m K_F(1-\alpha)} \right) + \left(\dfrac{(\mu_D - f_D) r_F}{\lfw m} - c\mathcal{I} \right) = 0.
\label{polynome-Feq}
\end{equation}

We note $\Delta_F$ the discriminant of this equation, which is positive. Indeed, we have:
\begin{align*}
\Delta_F &= \left(er_F + \dfrac{(\mu_D - f_D) r_F}{\lfw m K_F(1-\alpha)} \right)^2 - 4\dfrac{er_F}{K_F(1-\alpha)}  \left(\dfrac{(\mu_D -f_D) r_F}{\lfw m} - c\mathcal{I} \right), \\
\Delta_F &= \left(er_F - \dfrac{(\mu_D - f_D) r_F}{\lfw m K_F(1-\alpha)} \right)^2 + 4\dfrac{er_F}{K_F(1-\alpha)}  c\mathcal{I}, \\
\Delta_F & > 0.
\end{align*}

Therefore, $P_F$ admits two real roots. Their sign depends on the sign of the constant coefficient. $P_F$ admits:
\begin{itemize}
\item One non positive root $F^*_1$ and one positive root $F^*_2$ if $\dfrac{(\mu_D - f_D) r_F}{\lfw m} - c\mathcal{I} \leq 0 \Leftrightarrow \dfrac{(\mu_D - f_D) r_F}{\lfw m } \leq c\mathcal{I}.$
\item Two positive roots $F^*_1\leq  F^*_2$ if $\dfrac{(\mu_D - f_D) r_F}{\lfw m } > c\mathcal{I}$.
\end{itemize}
They are given by:

\begin{equation*}
F_i^* = \dfrac{K_F(1-\alpha)}{2}\left(1 \pm \dfrac{\sqrt{\Delta_F}}{er_F}\right) + \dfrac{\mu_D - f_D}{2\lfw m e}, \quad i=1,2.
\end{equation*}

Moreover, $P_F$ is positive on $(-\infty, F_1^*)$, negative on $(F^*_1, F^*_2)$ and positive on $(F^*_2, +\infty)$. Let us compute $P_F(K_{F,\alpha})$ (where $ K_{F,\alpha} = K_F(1-\alpha)$).
\begin{align*}
P_F(K_{F,\alpha}) &= K_{F,\alpha}^2 \left(\dfrac{er_F}{K_{F,\alpha}} \right) - K_{F,\alpha} \left(er_F + \dfrac{(\mu_D - f_D) r_F}{\lfw m K_{F,\alpha}} \right) + \left(\dfrac{(\mu_D - f_D) r_F}{\lfw m} - c\mathcal{I} \right), \\
&= K_{F,\alpha}er_F - K_{F,\alpha}er_F - \dfrac{(\mu_D - f_D) r_F}{\lfw m} + \dfrac{(\mu_D - f_D) r_F}{\lfw m} - c\mathcal{I}, \\
&= -c\mathcal{I}< 0.
\end{align*}

Therefore, $F_1^* < K_{F, \alpha} < F_2^*$. This means that $F_2^*$ is not biologically meaningful.

Therefore, if $\dfrac{(\mu_D - f_D) r_F}{\lfw m} > c\mathcal{I}$, we have a unique meaningful coexistence equilibrium $EE^{HF_W} = \Big(H^*_{D}, F^*_{W}, H^*_{W} \Big)$  
where $$F^*_{W} = \dfrac{K_F(1-\alpha)}{2}\left(1 - \dfrac{\sqrt{\Delta_F}}{er_F}\right) + \dfrac{\mu_D - f_D}{2\lfw m e},\quad
H^*_{D} = \dfrac{r_F}{\lfw m} \Big(1 - \dfrac{F^*_{W, c\mathcal{I} = 0}}{K_F(1-\alpha)} \Big),
\quad 
H^*_{W} = m H^*_{D}.$$
\end{proof}

\begin{remark} \label{RemarqueSimilariteEqCoexistence}
We can note that when $c\mathcal{I}=0$, $P_F$ rewrites:
$$
P_F(X, c\mathcal{I}=0) = \Big(X-K_F(1-\alpha) \Big) \Big(X - \dfrac{\mu_D - f_D}{\lfw m e} \Big).
$$
Therefore, $F^*_{W,c\mathcal{I} =0}$ still satisfies equation \eqref{polynome-Feq}, and in particular, when $\dfrac{\mu_D - f_D}{\lfw m e} < K_F(1-\alpha)$ we still have
$$
F^*_{W, c\mathcal{I} =0} = \dfrac{K_F(1-\alpha)}{2}\left(1 - \dfrac{\sqrt{\Delta_{F, c\mathcal{I} = 0}}}{er_F}\right) + \dfrac{\mu_D - f_D}{2\lfw m e} = \dfrac{\mu_D - f_D}{\lfw m e}.
$$ 
\end{remark}

%
%\begin{remark}
%In the case where $c\mathcal{I} = 0$, the equilibrium of coexistence $EE^{HF_W}_{c\mathcal{I}=0}$ exists if $\lambda_{F, c\mathcal{I}=0}^{Min} < \lfw$, see remark \marc{ref}. When $c\mathcal{I} > 0$, the equilibrium of coexistence exists, according to previous proposition, only if $\lfw < \dfrac{r_F(\mu_D - f_D)}{m c \mathcal{I}} := \lambda_{F, c\mathcal{I}>0}^{Max}$. This will be interpret in the section \marc{ref}.
%\end{remark}

Now, we look for the asymptotic stability of the equilibriums. We have the following results:

\begin{prop}\label{propLAS} The following results are valid.
\begin{itemize}
\item When $r_F(\mu_D - f_D) < m \lfw c\mathcal{I}$, the human equilibrium $EE^{H}$ is LAS.
\item When $\lfw < \dfrac{r_F (\mu_D -f_D)}{m c\mathcal{I}}$, equilibrium of coexistence $EE^{HF_W}$  exists. it is LAS if 
$$\Delta_{Stab} > 0,$$  where 

\begin{multline*}
\Delta_{Stab} = \left(\mu_D -f_D + m_D + r_F \dfrac{F_W^*}{K_{F, \alpha}} + m_W\right) \times \\
\left(\big( \mu_D -f_D + m_D + m_W) r_F \dfrac{F^*_W}{K_{F, \alpha}}   + m_D e\lfw   \left(\dfrac{\mu_D - f_D}{m e\lfw} - F^*_W \right)\right) - m_D \lfw \sqrt{\Delta_F}  F^*_{W}.
\end{multline*}
\end{itemize}
\end{prop}

\begin{proof}
To assess the local stability or instability of the different equilibria, we look at the Jacobian matrix. The Jacobian of system \eqref{equationsHDFWHW} is given by
\begin{equation*}
\mathcal{J}(H_D, F_W, H_W) = \begin{bmatrix}
f_D-\mu_D - m_D & e \lfw H_W & e\lfw F_W + m_W \\
0 & r_F \left( 1 - \dfrac{2F_W}{K_F(1-\alpha)} \right) - \lfw H_W & - \lfw F_W \\
m_D & 0 & -m_W
\end{bmatrix}.
\end{equation*}
\begin{itemize}
\item At equilibrium $EE^{H}$, we have
\begin{equation*}
\mathcal{J}(EE^{H}) = \begin{bmatrix}
f_D-\mu_D - m_D & e \lfw m \dfrac{c\mathcal{I}}{\mu_D - f_D} & m_W \\
0 & r_F - \lfw m \dfrac{c\mathcal{I}}{\mu_D - f_D} & 0 \\
m_D & 0 & -m_W
\end{bmatrix}.
\end{equation*}


The characteristic polynomial of $\mathcal{J}(EE^{H})$ is given by:
\begin{equation*}
\chi(X) = (X - r_F + \dfrac{\lfw m c\mathcal{I}}{\mu_D - f_D}) \times \left(X^2 - X\Big(f_D - \mu_D - m_D - m_W \Big) + m_W(\mu_D - f_D)\right).
\end{equation*}

The constant coefficient of the second factor and its coefficient in $X$ are positive. So, the roots of the second factor have a negative real part. Therefore, only the sign of $-r_F + \dfrac{\lfw m c\mathcal{I}}{\mu_D - f_D}$ determines the stability of $EE^{H}$.

\item Now, we look for the asymptotic stability of the equilibrium of coexistence $EE^{HF_W}_{\cI > 0}$. Some computations have already been done when we prove that equilibrium $EE^{HF_W}_{\cI = 0}$ is LAS, and we use these results here. 

We still note the characteristic polynomial of $\mathcal{J}(EE^{H F_W})$ is given by: $\chi = X^3 + a_2 X^2 + a_1 X + a_0$. According to the Routh-Hurwitz criterion \marc{ref}, $EE^{H F_W}$ are LAS if $a_i > 0$ for $i=1,2,3$ and $a_2 a_1 - a_0 > 0$. Note that $a_2 = - \Tr(\mathcal{J}(EE^{H F_W}))$ and $a_0 = - \det (\mathcal{J}(EE^{H F_W}))$.

According to \eqref{expressionA2}, we have:
\begin{equation*}
a_2 = - \Tr = -(f_D - \mu_D - m_D - r_F \dfrac{F_W^*}{K_{F, \alpha}} - m_W)
\end{equation*}
that is $a_2>0$. According to \eqref{expressionA0}, we have:
\begin{equation*}
a_0 = - m_D \lfw \left( F_W^* \Big(2 \dfrac{e r_F}{K_{F, \alpha}} \Big) - \Big(\dfrac{\mu_D - f_D}{m} \dfrac{r_F}{K_{F, \alpha} \lfw} + er_F \Big) \right) F_W^*
\end{equation*}

Using
\begin{equation*}
F_W^* = \dfrac{K_F(1-\alpha)}{2}\left(1 - \dfrac{\sqrt{\Delta_F}}{er_F}\right) + \dfrac{\mu_D - f_D}{2\lfw m e},
\end{equation*}
we obtain
\begin{equation*}
a_0 = m_D \lfw \sqrt{\Delta_F}  F^*_{W},
\end{equation*}
that is $a_0>0$. According to \eqref{expressionA1}, coefficient $a_1$ is given by:
\begin{equation*}
a_1 = \big( \mu_D -f_D + m_D + m_W) r_F \dfrac{F^*_W}{K_{F, \alpha}}   + m_D e\lfw   \left(\dfrac{\mu_D - f_D}{m e\lfw} - F^*_W \right).
\end{equation*}

We can look for the sign of $\left(\dfrac{\mu_D -f_D}{m e\lfw} - F^*_{W}\right)$. To do so, we will use the sign of $P_F(X)$ (negative on $(F^*_W, F^*_2)$, positive otherwise). We have:

\begin{align*}
P_F \left(\dfrac{\mu_D-f_D}{m e\lfw} \right) &= \left(\dfrac{\mu_D-f_D}{m e\lfw} \right)^2 \left(\dfrac{er_F}{K_{F, \alpha}} \right) - \dfrac{\mu_D-f_D}{m e\lfw} \left(er_F + \dfrac{(\mu_D -f_D) r_F}{\lfw m K_{F, \alpha}} \right) + \left(\dfrac{(\mu_D-f_D) r_F}{\lfw m} - c\mathcal{I}\right), \\
&= \dfrac{(\mu_D-f_D)^2 r_F}{m^2 \lfw^2 K_{F, \alpha} e} - \dfrac{(\mu_D-f_D)^2 r_F}{m^2 \lfw^2 K_{F, \alpha} e} - \dfrac{(\mu_D-f_D) r_F}{m \lfw} + \dfrac{(\mu_D-f_D) r_F}{m \lfw} - c\mathcal{I}, \\
&= - c\mathcal{I},\\
& < 0.
\end{align*}

Therefore, $\dfrac{\mu_D - f_D}{m e\lfw} > F^*_{W}$. This implies $a_1 > 0$.

The first assumption of Rough-Hurwitz is verified, $a_i > 0$ for $i=1,2,3$. Therefore, the local asymptotic stability of $EE^{HF_W}$ only depends on the sign of $\Delta_{Stab}= a_2 a_1 - a_0$, which has to be positive.

\end{itemize}





\end{proof}

For now, we have the following table:
\begin{table}[ht!]
\def\arraystretch{2}
\centering
\begin{tabular}{c|c|c|c}
$c\mathcal{I}$ & $\dfrac{r_F(\mu_D-f_D)}{m\lfw c\mathcal{I}} $ & $\Delta_{Stab}$ & \\
\hline
\multirow{3}{*}{$>0$} & $<1$ & &$EE^{H}$ exists and is LAS \\
\cline{2-4}
 & \multirow{2}{*}{$> 1$}  & $>0$ &$EE^{HF_W}$ exists and is LAS\\
 \cline{3-4}
 & & $ < 0$ & $EE^{HF_W}$ is unstable \\
\end{tabular}
\caption{Intermediate table summarizing the long term behavior}
\end{table}

As before, we can complete it with information about global asymptotic stability and existence of limit cycles. The following proposition is obtained using the same proof than for proposition \ref{LimitCycle, cI=0}.

\begin{prop}
If $\dfrac{r_F (\mu_D - f_D)}{m \lfw c \mathcal{I}} > 1$ and if 

\begin{itemize}
\item $\Delta_{Stab} > 0$, that is if equilibrium $EE^{HF_W}_{\cI >0}$ is LAS, then it is GAS on $\Omega$ for system \eqref{equationsHDFWHW}.
\item $\Delta_{Stab} < 0$, system \eqref{equationsHDFWHW} admits an orbitally asymptotically stable periodic solution.
\end{itemize}

\end{prop}

The following theorem will be useful to prove the global asymptotic stability of equilibrium $EE^{H}$:

\begin{prop}
If $$\dfrac{r_F(\mu_D-f_D)}{m\lfw c\mathcal{I}} < 1$$
that is if equilibrium $EE^{H}$ is LAS, then it is GAS on $\Omega$ for system \eqref{equationsHDFWHW}.
\end{prop}

\begin{proof}
On the following, we assume $\dfrac{r_F(\mu_D-f_D)}{m\lfw c\mathcal{I}} < 1$. For any solutions $(H_D^s, F_W^s, H_W^s)$ of equations \eqref{equationsHDFWHW} with initial condition in $\Omega$, we have:

\begin{align*}
\dfrac{dH_D^s}{dt} &= c\mathcal{I} + e\lfw H_W^s F_W^s + (f_D - \mu_D) H_D^s - m_D H_D^s + m_W H_W^s \\
&\geq c\mathcal{I} + (f_D - \mu_D) H_D^s - m_D H_D^s + m_W H_W^s
\end{align*}

We consider the sub-system for $\Big(H_D, H_W\Big)$, given by
\begin{equation}
\def\arraystretch{2}
\left\{ \begin{array}{l}
\dfrac{dH_D}{dt}= c\mathcal{I} + (f_D - \mu_D) H_D- m_D H_D + m_W H_W \\
\dfrac{dH_W}{dt}= m_D H_D - m_W H_W 
\end{array}\right.
\label{limitSystem - eqH}
\end{equation}

On the following, we note $H = \Big(H_D, H_W\Big)$, and $g$ the right hand side of previous system. The system $\dfrac{dH}{dt} = g(H)$ is cooperative, and admits a unique equilibrium $\Big(H_D^*, mH_D^*\Big)$ where $H_D^* = \dfrac{cI}{\mu_D - f_D}$. This equilibrium is GAS on $\Omega_H := \Big\{\Big(H_D, H_W\Big) \Big | H_D + H_W \leq S^{max}, H_W \leq H^{max}_W \Big\}$, and therefore, any solutions with initial condition in $\Omega_H$ will converge to it.

Moreover, we have the following inequality: $$f_{[1,3]}(H) \geq g(H).$$ We can apply the Kamke's inequality, see for example \cite{kirkilionis_comparison_2004}: for any initial conditions $H_0$, and for any time $t$, we have:
\begin{equation}
H_f^s(t, H_0) \geq H_g^s(t, H_0),
\label{inequalitySolution - eqH}
\end{equation}

where $H_g^s(t, H_0)$ is the solution of $\dfrac{dH}{dt} = g(H)$ with initial condition $H_0$.

\medskip

We assumed $\dfrac{r_F(\mu_D-f_D)}{m\lfw c\mathcal{I}} = \dfrac{r_F}{\lfw}\dfrac{1}{H_W^*} < 1$. Since $H_W^s(t, H_0)$ converges toward $H_W^*$, it exists $T > 0$ such that for all $t \geq T$, we have $\dfrac{r_F}{\lfw}\dfrac{1}{H_{W, f}^s(t, H_0)} < 1$. Using inequality \eqref{inequalitySolution - eqH}, we have:

\begin{equation}
\forall t \geq T, \quad \dfrac{r_F}{\lfw}\dfrac{1}{H_{W, f}^s(t, H_0)} < 1.
\end{equation}

\medskip
Therefore, for all $t \geq T$, we have
\begin{align*}
\dfrac{dF_W^s}{dt} &= r_F \left(1 - \dfrac{F_W^s}{K_F(1-\alpha)} \right) F_W^s - \lfw F_W^s H_W^s \\
&= \left(r_F- \lfw H_W^s\right) F_W^s - r_F \dfrac{F_W^s}{K_F(1-\alpha)}F_W^s \\
& \leq 0
\end{align*}

This implies that $F_W^s$ converges to 0. The limit system of equations \eqref{equationsHDFWHW} is given by:

\begin{equation}
\def\arraystretch{2}
\left\{ \begin{array}{l}
\dfrac{dH_D}{dt}= c\mathcal{I} + (f_D - \mu_D) H_D - m_D H_D + m_W H_W  \\
\dfrac{dH_W}{dt}= m_D H_D - m_W H_W
\end{array}\right.
\end{equation}

As we saw before, equilibrium $\Big(H_D^*, mH_D^*\Big)$ where $H_D^* = \dfrac{cI}{\mu_D - f_D}$ is GAS $\Omega_H$ for this system. Therefore, equilibrium $EE^{H} = \Big(H_D^*, 0, mH_D^*\Big)$ is GAS on $\Omega$ for system \eqref{equationsHDFWHW}.
\end{proof}


\begin{table}[!ht]
\def\arraystretch{2}
\centering
\begin{tabular}{c|c|c|c}
$c\mathcal{I}$ & $\dfrac{r_F(\mu_D-f_D)}{m\lfw c\mathcal{I}} $ & $\Delta_{Stab}$ & \\
\hline
\multirow{3}{*}{$>0$} & $<1$ & &$EE^{H}$ exists and is GAS \\
\cline{2-4}
 & \multirow{3}{*}{$> 1$}  & $>0$ &$EE^{HF_W}_{\cI>0}$ exists and is GAS\\
 \cline{3-4}
 & & \multirow{2}{*}{$ < 0$} & $EE^{HF_W}_{\cI>0}$ exists and is unstable ; there is an asymptotically \\
 & & &  stable periodic solution. \\
\end{tabular}
\caption{\centering Conditions of existence and asymptotic stability of equilibrium for system \eqref{equationsHDFWHW}}
\end{table}

\subsection{Interpretation}

\begin{definition}
We define 
$$\lambda_{F, \cI>0}^{Max} := \dfrac{r_F(\mu_D - f_D)}{m c \mathcal{I}}$$
such that 
$$
\text{$EE^{HF_W}$ exists} \Leftrightarrow  \lfw < \lambda_{F, \cI>0}^{Max}.
$$
\end{definition}

When $\cI = 0$, the existence of the equilibrium of coexistence is implied by the condition $\lambda_{F, cI = 0}^{Min} < \lfw$, see definition \ref{defLambdaMin, cI=0}. When $\cI > 0$, we instead obtain a maximum bound for $\lfw$. This is due to the fact that when $\cI > 0$, a human population is always present (equilibrium $EE^{H}$ exists), and has to not over-hunt in order to preserve the wild fauna (we could rewrite the condition has $1 < \dfrac{r_F}{\lfw H^*_{W, H}}$ : comparison between growth and intake).


We can note that when $\cI > 0$, the existence of the equilibrium of coexistence does not depend on the level of anthropization, $K_F(1-\alpha)$ but the value of $F^*_W$ does. When $\cI = 0$, it is the opposite. 
\marc{donc quand $\cI =0$, l'existence de la coexistence dépend la capacité du milieu, et la valeur de l'équilibre (=centre du cyle limite ?) uniquement de la chasse ; si on reprend les calculs, on doit vérifier ce seuil pour éviter que $H^*$ devienne négatif = on n'est pas sûr que la pop humaine puisse exister. Dans le cas $\cI > 0$, c'est l'existence de $F^*$ qui n'est pas sûr, et qui dépend de la pression de chasse.}

\section{Numerical scheme}
For a given $\Delta t>0$, we set $X^n=\Big(F_W^n,H_D^n,H_W^n \Big)$ as an approximation of $X(t)=\Big(F_W(t),H_D(t),H_W(t)\Big)$ at $t=n\Delta t$, for $n=0,...,N$, where $T=N\Delta$. We consider the following nonstandard scheme

\begin{equation}
\def\arraystretch{2}
\left\{ \begin{array}{l}
F_{W}^{n+1}=\dfrac{\left(1+\phi(\Delta t)r_{F}\right)}{1+\phi(\Delta t)\left(\dfrac{r_{F}}{K_{F}(1-\alpha)}F_{W}^{n}+\lambda_{F}H_{W}^{n}\right)}F_{W}^{n}\\ 
H_{D}^{n+1}=\Big(1-\phi(\Delta t) \left(\mu_{D}+m_{D}-f_{D}\right)\Big)H_{D}^{n}+ \phi(\Delta t)\Big(c\mathcal{I}+ \big(e\lambda_{F}F_{W}^{n+1} + m_{W}\big)H_{W}^{n}\Big)\\ 
H_{W}^{n+1}=\Big(1-\phi(\Delta t)m_{W}\Big)H_{W}^n+\phi(\Delta t)m_{D}H_{D}^n
\end{array}\right.
\end{equation}

where $\phi(\Delta t)=\dfrac{1-e^{-Q\Delta t}}{Q}$, with $Q=\max\{\mu_D+m_D-f_D,m_W\}$. 
It is straightforward to check that
$$
X^n \geq 0 \Rightarrow X^{n+1}\geq 0,\qquad \forall n\in \mathbb{N}.
$$
\YD{
\begin{definition}
A numerical scheme is called elementary stable whenever it has no other fixed points than those of the continuous system it approximates, the local stability of these fixed points is the same for both the discrete and the continuous dynamical systems for each value of $\Delta t$.
\end{definition}
Il faut donc vérifier cela....
}
\section{Results}

\bibliographystyle{plain}
\bibliography{Biblio/Math}

\end{document}

