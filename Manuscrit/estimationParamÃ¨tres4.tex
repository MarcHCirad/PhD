\documentclass{article}
\usepackage{graphicx} 
\usepackage{color}
\usepackage{amsfonts,amsmath}
\usepackage{amsthm}
\usepackage{empheq}
\usepackage{mathtools}
\usepackage{multirow}
\usepackage{tikz}
\usepackage{titlesec}
\usepackage{caption}
\usepackage{lscape}
\usepackage{graphicx}
\captionsetup{justification=justified}
\usepackage[toc,page]{appendix}
\usepackage{hyperref}
\usepackage{subcaption}

\textheight240mm \voffset-23mm \textwidth160mm \hoffset-20mm

\graphicspath{{./Images/}{./Images/ComparisonBifurcationFAHA}}

\setcounter{secnumdepth}{4}
\titleformat{\paragraph}
{\normalfont\normalsize\bfseries}{\theparagraph}{1em}{}
\titlespacing*{\paragraph}
{0pt}{3.25ex plus 1ex minus .2ex}{1.5ex plus .2ex}


\newcommand{\marc}[1]{\textcolor{red}{#1}}

\DeclareMathOperator{\Tr}{Tr}
\newcommand{\lfd}{\lambda_{F, D}}
\newcommand{\lfw}{\lambda_{F}}
\newcommand{\Kfa}{K_{F,\alpha}}
\newcommand{\cI}{\mathcal{I}}
\newcommand*\phantomrel[1]{\mathrel{\phantom{#1}}}

\title{Paramètres Thèse Marc}
\author{Marc Hétier, Yves Dumont  and Valaire Yatat-Djeumen}

\begin{document}

\maketitle
\section{Model}

\begin{equation}
\def\arraystretch{2}
\left\{ 
\begin{array}{l}
\dfrac{dH_D}{dt}= \cI + e\lfw H_W F_W + (f_D - \mu_D) H_D - m_D H_D + m_W H_W. \\
\dfrac{dF_W}{dt} = r_F(1- \alpha) (1+ \beta H_W) \left(1 - \dfrac{F_W}{K_F(1-\alpha)} \right) F_W - \lfw F_W H_W \\
\dfrac{dH_W}{dt}= m_D H_D - m_W H_W 
\end{array} \right.
\label{equationsHDFWHW}
\end{equation}



\section{Parameters values}
\begin{table}[ht]
\centering

\begin{tabular}{|c|c|c|}
\hline 
Parameter & Description & Unit \\ 
\hline 
$t$ & Time & Year \\
$e$ & Prey-food conversion & Human Ind Kg $^{-1}$ \\
$f_D$ & Food produced by human population & Year$^{-1}$ \\
$\mu_D$ & Human mortality rate  & Year$^{-1}$ \\
$m_D$ & Migration from domestic area to wild area & Year$^{-1}$ \\
$m_W$ & Migration from wild area to domestic area & Year$^{-1}$ \\
$r_F$ & Wild animal growth rate & Year$^{-1}$ \\
$K_F$ & Carrying capacity for wild animal, fixed by the environment& Kg \\
$\alpha$ & Proportion of anthropized environment & - \\
$\beta$ & Positive impact from human activities to animal growth rate & Human Ind$^{-1}$  \\
$\lfw$ & Hunting rate & Human Ind Year $^{-1}$ \\
$\mathcal{I}$ & Immigration rate & Human Ind $\times$ Year$^{-1}$\\
\hline
\end{tabular}

\begin{tabular}{|c|c|c|}
\hline 
Parameter & Value & Reference \\ 
\hline 
$t$ & - & -\\
$e$ & &\\
$f_D$ & $0.011$ &\\
$\mu_D$ & $1/60 \simeq 0.017$ & \cite{ins_demographie}\\
$m_D$ & \multirow{2}{*}{$\dfrac{m_D}{m_W} \in [0.006, 0.016]$} & \multirow{2}{*}{\cite{avila_interpreting_2019}}\\
$m_W$ & &\\
$r_F$ & $0.46 \pm 0.37$ &\\
$K_F$ & &\\
$\alpha$ & $[0, 1)$ & - \\
$\beta$ & &  \\
$\lfw$ & & \\
$\mathcal{I}$ & &\\
\hline
\end{tabular}

\caption{List of the parameters used}
\end{table}

\subsection{Complementary explication}
\begin{itemize}

\item Parameter $\mu_D$ : Life expectancy in Cameroon is around 60 years, so we just take $\mu_D = \dfrac{1}{60}$

\item Parameter $m$ is estimated using data from \cite{avila_interpreting_2019, jones_incentives_2019, jones_consequences_2020}. Using the fact that the total population is equal at the population in the domestic area plus the population in the wild area, and the equilibrium relations ($H_D^* = \dfrac{1}{m}H_W^*$), we have:
$$
H^* = H_D^* + H_W^* = (1 + \dfrac{1}{m}) H_W^*
$$

and this gives
$$
m = \dfrac{H_W^*}{H^* - H_W^*}
$$
where $H^*$ is the total population, and $H_W^*$ is the population in the wild area, at a given time.

Article \cite{avila_interpreting_2019} registers the total populations $H^*$ of villages in the Dja area (Southeast Cameroon), as well as the number of hunters $H_{hunter}$ and estimate that hunters spend 4.60h per week (10 days per year) hunting. Thus, a hunter as a probability of $p = \dfrac{4.60}{7 \times 24} = 0.027$ to be hunting. Therefore, we can estimate that $H_W^* = 0.027 \times H_{hunter}$.

Taking the mean of the different villages population, we obtain that $m = 0.006$. However, other articles (\cite{jones_consequences_2020, jones_incentives_2019}) estimates different hunting durations : 25 days per year for \cite{jones_incentives_2019} and 5 days per year for \cite{jones_consequences_2020}. This gives the following values for $m = 0.016$ or $m = 0.0032$.

We will retain $m \in [0.006, 0.016]$.

\item Parameter $f_D$ can be estimated using data from \cite{koppert_consommation_1996} and the equilibrium expressions. When no immigration is considered, the ratio $r$ between the food locally produced (parameter $f_D$) and the resources coming from the hunt ($e \lfw F_W^*$) writes:

\begin{align*}
r :=&  \dfrac{f_D}{e \lfw F_W^*} \\
=& \dfrac{f_D}{e \lfw \times \dfrac{\mu_D - f_D}{e m \lfw}}\\
=& \dfrac{f_D m }{\mu_D - f_D}
\end{align*}

Therefore, $f_D = \dfrac{r}{m + r} \mu_D$ and we do have $f_D < \mu_D$, which is conformed to the model hypothesis.

We consider that the resources coming from the hunt correspond to the bush meat consumed and the imported food (obtained by selling bushmeat). According to \cite{koppert_consommation_1996}, this represents 90\% (resp 73\% ; 80\%) of the consumed proteins by the Yassa (resp Mvae; Bakola). The local production furnishes 10\% (resp 27\% ; 18\%) of the consumed proteins. 

However, if we reason using calories and not proteins intakes, the figures are inversed: the calories coming from the hunt represent 18\% (resp 17\%; 17\%) of the diet and the ones coming from local production represents 82\% (resp 81\%; 81\%).

Therefore, we have $r = 0.11$ (resp $=0.37$ ; $=0.225$) if we use proteins data, and $r = 4.56$ (resp $=4.76$; $=4.76$) if we use the calories data.

Using previously found values of $m$ and $\mu_D$, we obtain $f_D = 0.009$ (resp. $=0.013$; $=0.011$) (average $f_D = 0.011$) or $f_D = 0.016$ (resp $=0.016$, $= 0.016$) (average $f_D = 0.016$). Let recall that $\mu_D \simeq 0.017$.

 \begin{figure}
 \centering
 \includegraphics[width=0.8\textwidth]{DiagConsommation.png}
 \end{figure}

\end{itemize}



\newpage
\bibliographystyle{plain}
\bibliography{Biblio/Math, Biblio/Context}


\end{document}