\documentclass{article}
\usepackage{graphicx,ulem} 
\usepackage{color}
\usepackage{amsfonts,amsmath}
\usepackage{amsthm}
\usepackage{empheq}
\usepackage{mathtools}
\usepackage{multirow}
%\usepackage{tikz}
\usepackage{titlesec}
\usepackage{caption}
%\usepackage{lscape}
\usepackage{graphicx}
\captionsetup{justification=justified}
\usepackage[toc,page]{appendix}
\usepackage{hyperref}
\usepackage{subcaption}
\usepackage{pdftricks}
\usepackage{xcolor}
\begin{psinputs}
\usepackage{amsfonts,amsmath}
	\usepackage{pstricks-add}
   \usepackage{pstricks, pst-node}
   \usepackage{multido}
   \newcommand{\lfw}{\lambda_{F}}
\end{psinputs}

\textheight240mm \voffset-23mm \textwidth160mm \hoffset-20mm

   \graphicspath{{./Images/}{../Schema}}
%\graphicspath{{Figures}}
\setcounter{secnumdepth}{4}
\titleformat{\paragraph}
{\normalfont\normalsize\bfseries}{\theparagraph}{1em}{}
\titlespacing*{\paragraph}
{0pt}{3.25ex plus 1ex minus .2ex}{1.5ex plus .2ex}

\newcommand{\lfd}{\lambda_{F, D}}
\newcommand{\lfw}{\lambda_{F}}
\newcommand{\Kfa}{K_{F,\alpha}}
\newcommand{\cI}{\mathcal{I}}

\newcommand{\marc}[1]{\textcolor{teal}{#1}}
\newcommand{\YD}[1]{\textcolor{magenta}{#1}}
\newcommand{\VY}[1]{\textcolor{blue}{#1}}

\DeclareMathOperator{\Tr}{Tr}
\newtheorem{theorem}{Theorem}
\newtheorem{prop}{Proposition}
\newtheorem{definition}{Definition}
\newtheorem{remark}{Remark}
\newtheorem{cor}{Corollary}
\newcommand*\phantomrel[1]{\mathrel{\phantom{#1}}}

\title{Modèle Chasseur}
\author{Marc Hétier, Yves Dumont  and Valaire Yatat-Djeumen}

\begin{document}

\maketitle
%{\hypersetup{hidelinks}
%\tableofcontents}
%\newpage


\section{Hunter Model}

We model the human-wild interactions by a consumer-resource model. We consider two areas, one corresponding to a domestic area (typically a village, etc), the other to a wild area (Savana, Forest, etc).


On the wild area, are present wild fauna $F_W$. The dynamic of $F_W$ follows a logistic equation, with a carrying capacity, $K_F$, dependent on the surrounding vegetation. To take into account the level of anthropization of the habitat, we introduce the non-negative parameter $\alpha \in [0, 1)$. When $\alpha > 0$, the carrying capacity of the habitat is reduced of $\alpha \%$ from its original value. Anthropization may also have a negative impact on the animal's growth rate $r_F$. For the sake of simplicity, we model this impact in the same way, by multiplying the growth rate by $(1-\alpha)$.

We consider two different interactions between wild fauna and the population located in the wild area. On the first hand, wild fauna is hunted by humans present in the wild area. This is take into account by the functional response $\lfw H_W$, where $\lfw$ is the hunting rate and $H_W$ the number of hunter. The functional response is unbounded, to take into account the possibility of over-hunt. On the second hand, it is known that housing, culture and food supply may attract some mammals, specially rodents (\marc{donner des noms?}) and favor their reproduction. We take this into account by multiplying the wild animal's growth rate, $r_F$, by the functional $(1 +  \beta H_W)$.

Humans present in the domestic area, $H_D$, follow a consumer equation. Its growth will depend on the available resources. The dynamic followed by $H_D$ can be separated in four groups of term. 
First there is a constant growth due to immigration from other inhabited area. This immigration recoup installation of new workers to develop industrial complex for example.

Prey hunted in the wild area are used to feed the villagers. Therefore, we consider a growth term $e \lfw H_W F_W$, where $e$ is a conversion rate between these hunted prey and the number of human it can nourish.

Villagers are also able to produce a certain amount of food at the rate $f_D$. Note that $f_D$ can be understand as $f_D = e \lfd F_D$ where $F_D$ is a constant amount of domestic animals.
Moreover, we assume that $H_D$ has a natural death-rate, $\mu_D$.

Fourth, villagers and hunters come back and force between the domestic and wild area. This migration is modeled by linear functions: $-m_D H_D + m_W H_W$.

The dynamic of human present in the wild area, $H_W$, corresponds simply to the migration terms. Note that we assume that the migration rate from the domestic area to the wild area is such that $m = \dfrac{m_D}{m_W} < 1$. This assumption makes sense, because humans stay a short amount of time in the Wild area than in the Domestic area.

Finally, the model is given by the following equations:
\begin{subequations}
\begin{equation}
\left\{ \begin{array}{l}
\dfrac{dH_D}{dt}= \cI + e\lfw H_W F_W + (f_D - \mu_D) H_D - m_D H_D + m_W H_W.
\end{array}\right.
\end{equation}
\begin{equation}
\left\lbrace \begin{array}{l}
\dfrac{dF_W}{dt} = r_F(1- \alpha) (1+ \beta H_W) \left(1 - \dfrac{F_W}{K_F(1-\alpha)} \right) F_W - \lfw F_W H_W \\
\dfrac{dH_W}{dt}= m_D H_D - m_W H_W 
\end{array} \right.
\end{equation}
\label{equationsHDFWHW}
\end{subequations}

Second equations rewrites:
$$
\dfrac{dF_W}{dt} = (1- \alpha)r_F \left(1 - \dfrac{F_W}{(1-\alpha)K_F} \right) F_W - \Big(\lfw - (1-\alpha) \beta r_F\Big) F_W H_W - \dfrac{r_F}{K_F} \beta H_W F_W^2
$$
OR
$$
\dfrac{dF_W}{dt} = (1- \alpha)r_F \left(1 - \dfrac{1 + \beta H_W}{(1-\alpha)K_F}F_W \right) F_W - \Big(\lfw - (1-\alpha) \beta r_F\Big) F_W H_W
$$

\begin{table}
\centering
\begin{tabular}{|c|c|c|c|}
\hline 
Parameter & Description & Unit & Value \\ 
\hline 
$t$ & Time & Year \\
$e$ & Prey-food conversion & Human Ind Kg $^{-1}$ & \\
$f_D$ & Food produced by human population & Year$^{-1}$\\
$\mu_D$ & Human mortality rate  & Year$^{-1}$ & $1/60$\\
$m_D$ & Migration from domestic area to wild area & Year$^{-1}$ & \multirow{2}{*}{$\dfrac{m_D}{m_W} = 0.23$}\\
$m_W$ & Migration from wild area to domestic area & Year$^{-1}$\\
$r_F$ & Wild animal growth rate & Year$^{-1}$ & $0.46 \pm 0.37$\\
$K_F$ & Carrying capacity for wild animal, fixed by the environment& Kg\\
$\alpha$ & Proportion of anthropized environment & - & $[0, 1)$\\
$\beta$ & Positive impact from human activities to animal growth rate & Kg$^{-1}$ & \\
$\lfw$ & Hunting rate & Humans Ind Year $^{-1}$ \\
$\mathcal{I}$ & Migration rate & Human Ind $\times$ Year$^{-1}$ \\
\hline
%$c$ & Conversion rate between raw products and food OR nothing & Human OR -
\end{tabular}
\caption{List of the parameters used}
\end{table}

\section{Existence and uniqueness of global solutions}
In this section, we state general results on system \eqref{equationsHDFWHW}:  existence of an invariant region, existence and uniqueness of global solutions.

We begin by proving the local existence and uniqueness of solutions of system \eqref{equationsHDFWHW}. The right hand side of equations \eqref{equationsHDFWHW} defines a function $f(y)$ (with $y = (H_D, F_W, H_W)$) which is of class $\mathcal{C}^1$ on $\mathbf{R}^3$. The theorem of Cauchy-Lipschitz ensures that model \eqref{equationsHDFWHW} admits a unique solution, at least locally, for any given initial condition, see \cite{walter_ordinary_1998}.

On the following, we will assume $\lfw - (1-\alpha) \beta r_F > 0$. \marc{A justifier ! mais sans ça, je n'arrive pas à avoir de zone invariante pour le système}.

Second, we need to add a constraint on the sign of $f_D - \mu_D$, to avoid infinite growth of human population. That is the subject of the next proposition:

\begin{prop}
If $\cI \geq 0$ and $f_D - \mu_D > 0$, the following is true:
$$
\forall	y=\Big(H_D, F_W, H_W\Big) \in \mathbf{R}^3_+, \quad \dfrac{d(H_D + H_W)}{dt} \geq 0.
$$
This means that the human population can infinitely growth.
\end{prop}

\begin{proof}
Assume $\cI \geq 0$ and $f_D - \mu_D > 0$ and let $H = H_D + H_W$. For $\Big(H_D, F_W, H_W\Big) \in \mathbf{R}^3_+$, we have:
\begin{align*}
\dfrac{dH}{dt} &= \cI + e \lfw F_W H_W + (f_D - \mu_D) \Big(H - H_W \Big) \\
\dfrac{dH}{dt} & \geq (f_D - \mu_D) \Big(H - H_W \Big) \\
\dfrac{dH}{dt} & \geq 0
\end{align*}
\end{proof}

To avoid this infinite growth, we will assume on the following that if $\cI \geq 0$ then $\mu_D > f_D$. This means that the food produced by the human population living in the residential area is not sufficient to ensure their livelihood. Hunt, or immigration, is necessary. 

The following proposition indicates a compact and invariant subset of $\mathbf{R}_+^3$, on which the solutions are bounded.

\begin{prop}\label{Invariant region, cI>=0} 
Assume $\cI \geq 0$ and $\mu_D - f_D > 0$. Then, the region
$$\Omega = \Big\{\Big(H_D, F_W, H_W \Big) \in \mathbb{R}_+^3  \Big|H_D + H_W + eF_W \leq S^{max}, F_W \leq F_W^{max}, H_W \leq H_W^{max} \Big\},$$
is a compact and invariant set for system \eqref{equationsHDFWHW}, 
where
$$
S^{max} = \Big(1 + \dfrac{m_D}{m_W} \Big) \dfrac{\cI + \dfrac{e \lfw (1-\alpha) K_F}{\lfw - \beta (1-\alpha) r_F} \Big((1-\alpha)r_F + \mu_D - f_D\Big)}{\mu_D - f_D},
\quad
F_W^{max} = (1-\alpha)K_F,
\quad
H_W^{max} = \dfrac{m_D}{m_D + m_W} S^{max}
$$
In particular, this means that any solutions of equations \eqref{equationsHDFWHW} with initial condition in $\Omega$ remains bounded.
\end{prop}
%
\begin{proof}
To prove this proposition we will use the notion of invariant region, see \cite{smoller_shock_1994}. Before, we introduce the variable $S = H_D + H_W + \dfrac{e \lfw}{\lfw - (1-\alpha) \beta r_F } F_W$. We have:

\begin{equation}
\dfrac{dS}{dt} = \cI + (f_D - \mu_D) \Big(S - H_W - \dfrac{e \lfw}{\lfw - (1-\alpha) \beta r_F }F_W \Big) + \dfrac{e \lfw(1-\alpha)r_F}{\lfw - (1-\alpha) \beta r_F }  \left(1 - \dfrac{(1+\beta H_W)F_W}{(1-\alpha)K_F} \right) F_W.
\end{equation}

With this new variable, the model writes:
\begin{subequations}
\begin{equation}
\left\{ \begin{array}{l}
\dfrac{dS}{dt} = \cI + (f_D - \mu_D) \Big(S - H_W - \dfrac{e \lfw}{\lfw - (1-\alpha) \beta r_F }F_W \Big) + \dfrac{e \lfw(1-\alpha)r_F}{\lfw - (1-\alpha) \beta r_F } \left(1 - \dfrac{(1+\beta H_W)F_W}{(1-\alpha)K_F} \right) F_W.
\end{array}\right.
\end{equation}
\begin{equation}
\left\lbrace \begin{array}{l}
\dfrac{dF_W}{dt} = (1-\alpha)(1+\beta H_W) r_F \left(1 - \dfrac{F_W}{K_F(1-\alpha)} \right) F_W - \lfw F_W H_W \\
\dfrac{dH_W}{dt}= m_D \left(S - \dfrac{e \lfw}{\lfw - (1-\alpha) \beta r_F }F_W\right) - (m_W + m_D) H_W 
\end{array} \right.
\end{equation}
\label{equationsSFWHW}
\end{subequations}

We define the function $g(z)$ as the right hand side of this equations. We also introduce the following functions:
$$
G_1(z) = S - S^{max},
\quad
G_2(z) = F_W - F_W^{max},
\quad
G_3(z) = H_W - H_W^{max}
$$

Following \cite{smoller_shock_1994}, we will show that quantities $(\nabla G_1 \cdot g)|_{S = S^{max}}$, $(\nabla G_2 \cdot g)|_{F_W = F_W^{max}}$ and $(\nabla G_3 \cdot g)|_{H_W = H_W^{max}}$ are non-positive for $z \in \Omega_S = \Big\{ \Big(S, F_W, H_W \Big) \in (\mathbb{R})^3  \Big|S \leq S^{max}, F_W \leq F_W^{max}, H_W \leq H_W^{max} \Big\}$.

Using the fact that $\mu_D - f_D >0$ and $z\in \Omega_S$, we have:

\begin{align*}
(\nabla G_1 \cdot g)|_{S = S^{max}} &= \cI + (f_D - \mu_D) \Big(S^{max} - H_W - \dfrac{e \lfw}{\lfw - (1-\alpha) \beta r_F }F_W \Big) + \dfrac{e \lfw(1-\alpha)r_F}{\lfw - (1-\alpha) \beta r_F }  \left(1 - \dfrac{(1+\beta H_W)F_W}{(1-\alpha)K_F} \right) F_W, \\
&= \cI + (f_D - \mu_D) S^{max} + (\mu_D - f_D) H_W + \Big(\mu_D - f_D + (1-\alpha)r_F \Big)\dfrac{e \lfw}{\lfw - (1-\alpha) \beta r_F } F_W -\\& \dfrac{e\lfw}{\lfw - (1-\alpha)\beta r_F} \dfrac{(1+\beta H_W) F_W^2}{K_F}, \\
&  \leq \cI + (f_D - \mu_D) S^{max} + (\mu_D - f_D) \dfrac{m_D}{m_D + m_W} S^{max} + \Big(\mu_D - f_D + (1-\alpha)r_F \Big)\dfrac{e \lfw (1-\alpha)K_F}{\lfw - (1-\alpha) \beta r_F }, \\
&  \leq \cI + (\mu_D - f_D)\Big( \dfrac{m_D}{m_D + m_W}-1\Big) S^{max} + \Big(\mu_D - f_D + (1-\alpha)r_F \Big)\dfrac{e \lfw (1-\alpha)K_F}{\lfw - (1-\alpha) \beta r_F }, \\
&  \leq - \dfrac{m_W}{m_D + m_W}(\mu_D - f_D) S^{max} + \cI+ \Big(\mu_D - f_D + (1-\alpha)r_F \Big)\dfrac{e \lfw (1-\alpha)K_F}{\lfw - (1-\alpha) \beta r_F }, \\
&  \leq - \dfrac{m_W}{m_D + m_W}\Big(1 + \dfrac{m_D}{m_W} \Big) \left(  \cI + \dfrac{e \lfw (1-\alpha)K_F}{\lfw - \beta (1-\alpha) r_F}\Big((1-\alpha)r_F + \mu_D - f_D\Big)  \right) + \\ &\cI+ \Big(\mu_D - f_D + (1-\alpha)r_F \Big)\dfrac{e \lfw (1-\alpha)K_F}{\lfw - (1-\alpha) \beta r_F }, \\
&  \leq \left(- \dfrac{m_W}{m_D + m_W}\Big(1 + \dfrac{m_D}{m_W} \Big)+ 1 \right)\left(\cI+ \Big(\mu_D - f_D + (1-\alpha)r_F \Big)\dfrac{e \lfw (1-\alpha)K_F}{\lfw - (1-\alpha) \beta r_F } \right), \\
&  \leq \left(- 1 + 1 \right)\left(\cI+ \Big(\mu_D - f_D + (1-\alpha)r_F \Big)\dfrac{e \lfw (1-\alpha)K_F}{\lfw - (1-\alpha) \beta r_F } \right), \\
\end{align*}

that is $(\nabla G_1 \cdot g)|_{S = S^{max}} \leq 0$. The two others inequalities are straightforward to obtain. We have:
\begin{align*}
(\nabla G_2 \cdot g)|_{F_W = F_W^{max}} &= r_F  \left(1 - \dfrac{K_F (1-\alpha)}{K_F (1-\alpha)}\right)K_F (1-\alpha)  - \lfw H_W K_F (1-\alpha), \\
(\nabla G_2 \cdot g)|_{F_W = F_W^{max}} & = - \lfw H_W K_F (1-\alpha), \\
(\nabla G_2 \cdot g)|_{F_W = F_W^{max}} & \leq 0.
\end{align*}

The computations for $(\nabla G_3 \cdot g)|_{H_W = H_W^{max}}$ give:

\begin{align*}
(\nabla G_3 \cdot g)|_{H_W = H_W^{max}} &= m_D (S - eF_W) - (m_W + m_D) H_W^{max}, \\
(\nabla G_3 \cdot g)|_{H_W = H_W^{max}} &= m_D (S - eF_W) - m_D S^{max}, \\
(\nabla G_3 \cdot g)|_{H_W = H_W^{max}} & \leq m_D (S - S^{max} -  eF_W), \\
(\nabla G_3 \cdot g)|_{H_W = H_W^{max}} & \leq 0.
\end{align*}

We have shown that $(\nabla G_1 \cdot g)|_{S = S^{max}} \leq 0$, $(\nabla G_2 \cdot g)|_{F_W = F_W^{max}} \leq 0$ and $(\nabla G_3 \cdot g)|_{H_W = H_W^{max}} \leq 0$ in  $\Omega_S$.  According to \cite{smoller_shock_1994}, this prove that $\Omega_S$ is an invariant region for system \eqref{equationsSFWHW}.

This also shows that the set  $\Big\{\Big(H_D, F_W, H_W \Big) \in \mathbb{R}^3  \Big|H_D + H_W + eF_W \leq S^{max}, F_W \leq F_W^{max}, H_W \leq H_W^{max} \Big\}$ is invariant for system \eqref{equationsHDFWHW}. 


Moreover, for any point $y \in \partial (\mathbb{R}_+)^3$, the vector field defined by $f(y)$ is either tangent or directed inward. Then, $\Omega$ is an invariant region for equations \eqref{equationsHDFWHW}. 

\end{proof}


We finally conclude this section by proving that equations \eqref{equationsHDFWHW} define a dynamical system on $\Omega$.

\begin{prop}
Equations \eqref{equationsHDFWHW} define a dynamical system on $\Omega$, that is, for any initial condition $(t_0, y)$ with $t_0 \in \mathbf{R}$ and $y \in \Omega$, it exists a unique solution of equations \eqref{equationsHDFWHW}, and this solution is defined for all $t \geq t_0$.
\end{prop}

\begin{proof}
We already prove that equations \eqref{equationsHDFWHW} admit, at least locally, a unique solution for every initial condition. Moreover, since $\Omega$ is an invariant region, the solutions with initial condition on $\Omega$ are bounded. Based on uniform boundedness, we deduce that solutions of system \eqref{equationsHDFWHW} with initial condition on $\Omega$ exists globally, for all $t\geq t_0$. Therefore, $\eqref{equationsHDFWHW}$ defines a dynamical system on $\Omega$.
\end{proof}

The following definitions are necessary for theorem \ref{theorem: periodicASOrbit}, which will be used all along the theoretical study.

\begin{definition}\cite{kaszkurewicz_matrix_2012}
A square matrix $A \in Mn (\mathbf{R})$ is said reducible if there exists a permutation matrix $P$ such that $P^ T AP$ is block triangular. Otherwise, $A$ is called irreducible.

\marc{OR a square matrix $A \in Mn (\mathbf{R})$ is said irreducible if for each nonempty proper subset $I$ of $N = \{1, ..., n\}$, there exists $i \in I$ and $j \in N\backslash I$ such that $A_{i,j} \neq 0$}
\end{definition}

\begin{definition}
An open set $\mathcal{D} \in \mathbf{R}^n$ is said to be p-convex provided that for every $x, y \in \mathcal{D}$, with $x \leq y$, the line segment joining $x$ and $y$ belongs to $\mathcal{D}$.
\end{definition}

\begin{definition}\cite{smith_monotone_1995}
A system of differential equations
$$ \dfrac{d x}{dt} = g(x), \quad x \in \mathcal{D}$$
where $\mathcal{D}$ is an open subset of $\mathbf{R}^n$ and $g$ is continuously differentiable in $\mathcal{D}$ is said 

\begin{itemize}
\item irreducible if the Jacobian matrix of $g$ at $x$, $\mathcal{J}_g(x)$ is irreducible.
\item competitive if $\mathcal{J}_g(x)$ has non positive off-diagonal elements:
$$ \dfrac{\partial g_i}{\partial x_j}(x) \leq 0, \quad i \neq j.
$$
\end{itemize}

\end{definition}



\begin{theorem}\cite{zhu_stable_1994}\label{theorem: periodicASOrbit}
We consider the system of differential equations
$$
\dfrac{dx}{dt} = g(x), \quad x \in \mathcal{D}.
$$
If
\begin{itemize}
\item $\mathcal{D}$ is an open, $p$-convex subset of $\mathbf{R}^3$,
\item $\mathcal{D}$ contains a unique equilibrium point $x^*$ and $\det(\mathcal{J}_g(x^*)) < 0$,
\item $g$ is analytic in $\mathcal{D}$,
\item the system is competitive and irreducible in $\mathcal{D}$,
\item the system is dissipative: For each $x_0 \in \mathcal{D}$, the positive semi-orbit through $x_0$, $\phi^+(x_0)$ has a compact closure in $\mathcal{D}$ . Moreover, there exists a compact subset $\mathcal{B}$ of $\mathcal{D}$ with the property that for each $x_0 \in \mathcal{D}$, there exists $T(x_0) > 0$ such that $x(t, x_0) \in \mathcal{B}$ for $t \geq T(x_0)$.
\end{itemize}

then either $x^*$ is stable, or there exists at least one non-trivial orbitally asymptotically stable  periodic orbit in $\mathcal{D}$.
\end{theorem}

One of the assumptions of theorem \eqref{theorem: periodicASOrbit} is that the system of equations is competitive, and that is not the case of system \eqref{equationsHDFWHW}. However, we will see that this system is equivalent to a competitive system, on which we can apply the theorem, and thus get back the conclusions.

\begin{prop}
System \eqref{equationsHDFWHW} is equivalent to a competitive irreducible and dissipative system.
\end{prop}

\begin{proof}
Following \cite{wang_predator-prey_1997}, we do the following change of variable: $h_D =  H_D$, $f_W = -F_W$ and $h_W = -H_W$.  The system \eqref{equationsHDFWHW, cI=0} is transformed into:
\begin{subequations}
\begin{equation}
\left\{ \begin{array}{l}
\dfrac{dh_D}{dt}= \cI + e\lfw h_W f_W + (f_D - \mu_D) h_D - m_D h_D - m_W h_W.
\end{array}\right.
\end{equation}
\begin{equation}
\left\lbrace \begin{array}{l}
\dfrac{df_W}{dt} = (1-\alpha)r_F \left(1 + \dfrac{1 - \beta h_W}{K_F(1-\alpha)}f_W \right) f_W + (\lfw - (1-\alpha)\beta r_F) f_W h_W \\
\dfrac{dh_W}{dt}= -m_D h_D - m_W h_W 
\end{array} \right.
\end{equation}
\label{equationshDfWhW}
\end{subequations}

We note $\mathcal{D} = \Big\{z = (h_D, f_W, h_W) | 0 < h_D, f_W < 0, h_W < 0 \Big\}$, and $g(z)$ the right hand side of the system. It is clear that $\mathcal{D}$ is a $p$-convex set, in which $g$ is analytic.

The Jacobian of $g$ is given by:

\begin{equation*}
\mathcal{J}_g(z) = \begin{bmatrix}
f_D -\mu_D - m_D & e \lfw h_W & e \lfw f_W - m_W \\
0 & r_F \Big(1 + \dfrac{2(1-\beta h_W) f_W}{K_F(1-\alpha)}\Big) + (\lfw - (1-\alpha)\beta r_F) h_W & (\lfw - (1-\alpha)\beta r_F) f_W - \beta r_F \dfrac{f_W^2}{K_F}\\
-m_D & 0 & -m_W
\end{bmatrix}.
\end{equation*}
Therefore, it is clear that system \eqref{equationshDfWhW} is irreducible. Moreover, the non diagonal term are non positive for $z \in \mathcal{D}$. Thus, system \eqref{equationshDfWhW} is competitive in $\mathcal{D}$.

\marc{montrer que le système est dissipatif !}


\end{proof}


\section{Model analysis in the case $\cI = 0$}
In this section, we study the specific case where $\cI = 0$. This mean that there is no importation of resources and the human population mainly dependents on hunt to subsist. The system rewrites:
\begin{subequations}
\begin{equation}
\left\{ \begin{array}{l}
\dfrac{dH_D}{dt}= e\lfw H_W F_W + (f_D - \mu_D) H_D - m_D H_D + m_W H_W.
\end{array}\right.
\end{equation}
\begin{equation}
\left\lbrace \begin{array}{l}
\dfrac{dF_W}{dt} = r_F(1- \alpha) (1+ \beta H_W) \left(1 - \dfrac{F_W}{(1-\alpha)K_F} \right) F_W - \lfw F_W H_W \\
\dfrac{dH_W}{dt}= m_D H_D - m_W H_W 
\end{array} \right.
\end{equation}
\label{equationsHDFWHW, cI=0}
\end{subequations}

\subsection{Theoretical analysis}
On the following, we state and prove results dealing with existence of equilibrium points of this system as well as with their stability. We start by a proposition concerning their existence.


\begin{prop}
\label{theoremEquilibre, cI=0}
The following results hold:
\begin{itemize}
\item System \eqref{equationsHDFWHW, cI=0} admits a trivial equilibrium $TE = \Big(0,0,0\Big)$ that always exists.
\item System \eqref{equationsHDFWHW, cI=0} admits a fauna-only equilibrium $EE^{F_W} = \Big(0, (1-\alpha)K_F, 0 \Big)$ that always exists.
\item When
$$
\dfrac{m e \lfw K_F(1-\alpha)}{\mu_D - f_D} >1,
$$ 
then system \eqref{equationsHDFWHW, cI=0} admits a unique coexistence equilibrium $EE^{HF_W} = \Big(H^*_{D, \cI = 0}, F^*_{W, \cI = 0}, H^*_{W, \cI = 0} \Big)$ \\ 
where 
$$F^*_{W, \cI = 0} = \dfrac{\mu_D - f_D}{\lfw m e},
\quad 
H^*_{D, \cI = 0} = \dfrac{(1-\alpha)r_F\Big(1 - \dfrac{F^*_{W, \cI = 0}}{K_F(1-\alpha)} \Big)}{m\left(\lfw - \beta (1-\alpha) r_F + \beta r_F  \dfrac{F^*_{W, \cI = 0}}{K_F}\right)} ,
\quad 
H^*_{W, \cI = 0} = m H^*_{D, \cI = 0}.$$
\end{itemize}
\end{prop}

\begin{proof}
An equilibrium of system \eqref{equationsHDFWHW, cI=0} satisfies the system of equations:
\begin{equation}\label{system-equilibre, cI=0}
\left\lbrace \begin{array}{cll}
 e \lfw m F_W^* + f_D - \mu_D = 0& \mbox{or} & H_D^* = 0,\\
m H_D^*\Big(\lfw - (1-\alpha)r_F \beta - \dfrac{r_F \beta}{K_F}F_W^* \Big) + r_F \dfrac{F_W^*}{K_F} - (1-\alpha)r_F= 0& \mbox{or} & F^*_W = 0,\\
H_W^* = \dfrac{m_D}{m_W} H_D^* = m H_D^*.&&
\end{array} \right.
\end{equation}
When $H_D^*=0$ and $F_W^*=0$, we recover the trivial equilibrium $TE = \Big(0,0,0\Big)$. When $H_D^*=0$ and $F_W^*\neq0$, we obtain the fauna-only equilibrium $EE^{F_W} = \Big(0, K_F(1-\alpha), 0 \Big)$. Finally, when $H_D^*\neq0$ and $F_W^*\neq0$, direct computations lead to a unique coexistence equilibrium 
$$F^*_{W, \cI = 0} = \dfrac{\mu_D - f_D}{\lfw m e},
\quad 
H^*_{D, \cI = 0} = \dfrac{r_F\Big(1 - \dfrac{F^*_{W, \cI = 0}}{K_F(1-\alpha)} \Big)}{m\left(\lfw - \beta (1-\alpha) r_F + \beta r_F  \dfrac{F^*_{W, \cI = 0}}{K_F}\right)} ,
\quad 
H^*_{W, \cI = 0} = m H^*_{D, \cI = 0}.$$
which is biologically meaningful whenever $\dfrac{\mu_D - f_D}{\lfw m e} < K_F(1-\alpha)$ since we assumed $\lfw - \beta (1-\alpha) r_F > 0$.
\end{proof}

Now, we look for the local asymptotic stability of the equilibrium.

\begin{prop}\label{propLAS, cI=0} The following results are valid.
\begin{itemize}
\item The trivial equilibrium $TE$ is unstable.
\item When $\dfrac{m e \lfw K_F(1-\alpha)}{\mu_D - f_D} < 1$, the fauna equilibrium, $EE^{F_W}$, is Locally Asymptotically Stable (LAS).
\item When $\dfrac{m e \lfw K_F(1-\alpha)}{\mu_D - f_D} > 1$, the coexistence equilibrium, $EE^{HF_W}_{\cI =0}$, exists. It is LAS if ...
%$$
%\lfw K_F(1-\alpha) < \Big(\Kfa \lfw \Big)^*
%$$
%where \begin{multline*}
%\Big(\Kfa \lfw \Big)^* = \\
% \dfrac{\left[m_{W}(\mu_{D}-f_{D})+\big(\mu_{D}-f_{D}+m_{D}+m_{W})^{2}\right]\left(1+\sqrt{1+4\dfrac{m_{W}r_{F}\left(\mu_{D}-f_{D}\right)\big(\mu_{D}-f_{D}+m_{D}+m_{W})}{\left[m_{W}\dfrac{\mu_{D}-f_{D}}{e}+\big(\mu_{D}-f_{D}+m_{D}+m_{W})^{2}\right]^{2}}}\right)}{2em_D}
%\end{multline*}
\end{itemize}
\end{prop}

\begin{proof}
To prove this theorem, we look at the Jacobian of system \eqref{equationsHDFWHW, cI=0}. It is given by:

\begin{multline*}
\mathcal{J}(H_D, F_W, H_W) = \\
\begin{bmatrix}
f_D-\mu_D - m_D & e \lfw H_W & e\lfw F_W + m_W \\
0 & r_F(1-\alpha)(1+\beta H_W) \left( 1 - \dfrac{2F_W}{K_F(1-\alpha)} \right) - \lfw H_W & - (\lfw - (1-\alpha)\beta r_F) F_W -  \dfrac{r_F\beta}{K_F} F_W^2\\
m_D & 0 & -m_W
\end{bmatrix}.
\end{multline*}

\begin{itemize}
\item At equilibrium $TE$, we have:
\begin{equation*}
\mathcal{J}(TE) = \begin{bmatrix}
f_D-\mu_D - m_D & 0 &  m_W \\
0 & r_F(1-\alpha)  &  0\\
m_D & 0 & -m_W
\end{bmatrix}.
\end{equation*}
and $r_F > 0$ is an eigenvalue of $\mathcal{J}(TE)$. So, $TE$ is unstable.
\item At equilibrium $EE^{F_W}$, we have
\begin{equation*}
\mathcal{J}(EE^{F_W}) = \begin{bmatrix}
f_D-\mu_D - m_D & 0 & e\lfw K_F(1-\alpha) + m_W \\
0 & -(1-\alpha)r_F  & -\Big(\lfw - (1-\alpha)\beta r_F + \beta (1-\alpha)r_F \Big)(1-\alpha)K_F  \\
m_D & 0 & -m_W
\end{bmatrix}.
\end{equation*}

The characteristic polynomial of $\mathcal{J}(EE^{F_W})$ is given by:
\begin{equation*}
\chi(X) = (X +r_F) \times \left(X^2 - X\Big(f_D - \mu_D - m_D - m_W \Big) + m_W(\mu_D - f_D) - m_D e \lfw K_F(1-\alpha) \right).
\end{equation*}

We need to determine the sign of the roots' real part of the second factor. Since the coefficient in $X$ is positive, the sign of their real part is determined by the sign of the constant coefficient.
The roots have a negative real part if the constant coefficient is positive \textit{ie} if $\dfrac{m e \lfw K_F(1-\alpha)}{\mu_D - f_D} < 1 $, and a positive real part if $\dfrac{m e \lfw K_F(1-\alpha)}{\mu_D - f_D} > 1 $. Stability of $\mathcal{J}(EE^{F_W})$ follows.

\item Now, we look for the asymptotic stability of the equilibrium of coexistence $EE^{HF_W}_{\cI=0}$. The first part of the computations are common with the ones for proving the LAS of equilibrium $EE^{HF_W}_{\cI >0}$.   To keep some generality, we use notation $EE^{HF_W}$ for both $EE^{HF_W}_{\cI =0}$ and $EE^{HF_W}_{\cI >0}$. We have


\begin{equation*}
\mathcal{J}(EE^{H F_W}) = \begin{bmatrix}
f_D -\mu_D - m_D & e \lfw H_W^* & e \lfw F^*_W +m_W \\
0 & -(1 + \beta H_W^*)r_F \dfrac{F_W^*}{K_F} & - (\lfw - (1-\alpha)\beta r_F) F_W^* -  \dfrac{r_F\beta}{K_F} (F_W^*)^2 \\
m_D & 0 & -m_W
\end{bmatrix}.
\end{equation*} 

and its characteristic polynomial is given by: $\chi = X^3 + a_2 X^2 + a_1 X + a_0$. Note that $a_2 = - \Tr(\mathcal{J}(EE^{H F_W}))$ and $a_0 = - \det (\mathcal{J}(EE^{H F_W}))$.

According to the Routh-Hurwitz criterion \marc{ref}, $EE^{H F_W}$ are LAS if $a_i > 0$ for $i=1,2,3$ and $a_2 a_1 - a_0 > 0$.

We have:
\begin{equation}\label{expressionA2}
a_2 = - \Tr = -(f_D - \mu_D - m_D - (1+\beta H_W^*)r_F \dfrac{F_W^*}{K_F} - m_W)
\end{equation}
that is $a_2>0$. Coefficient $a_0$ is given by:

\begin{subequations}
\begin{align}
a_0 &= -\det\Big(\mathcal{J}(EE^{H F_W})\Big), \\
a_0 &= \Big(\mu_D + m_D -f_D \Big) m_W (1+\beta H_W^*) r_F \dfrac{F^*_W}{K_F}  - m_D (1 + \beta H_W^*) r_F \dfrac{F_W^*}{K_F}(e\lfw F_W^* + m_W) + \\
\nonumber
&  m_D e \lfw  \left((\lfw - (1-\alpha)\beta r_F)  + \dfrac{r_F\beta}{K_F} F_W^* \right)H_W^* F_W^* \\
a_0 &= \Big(\mu_D -f_D \Big) m_W (1+\beta H_W^*) r_F \dfrac{F^*_W}{K_F}  - m_D e\lfw (1 + \beta H_W^*) r_F \dfrac{(F_W^*)^2}{K_F} + \\
\nonumber
&  m_D e \lfw \left((\lfw - (1-\alpha)\beta r_F)  + \dfrac{r_F\beta}{K_F} F_W^* \right)H_W^*F_W^* \\
a_0 &= \Big(\mu_D -f_D \Big) m_W (1+\beta H_W^*) r_F \dfrac{F^*_W}{K_F}  - m_D e\lfw (1 + \beta H_W^*) r_F \dfrac{(F_W^*)^2}{K_F} + \\
\nonumber
&  m_D e \lfw (1- \alpha) r_F \left(1 - \dfrac{F_W^*}{(1- \alpha)K_F}\right) F_W^* \\
a_0 &= e \lfw m_D r_F (1 + \beta H_W^*) \left(\dfrac{\mu_D -f_D }{e \lfw m} - F_W^*\right) \dfrac{F_W^*}{K_F} + m_D e \lfw (1- \alpha) r_F \left(1 - \dfrac{F_W^*}{(1- \alpha)K_F}\right) F_W^*  \\
a_0 &= e \lfw m_D r_F \left(\dfrac{\mu_D -f_D }{e \lfw m K_F} - 2\dfrac{F_W^*}{K_F} + (1-\alpha) + \dfrac{\beta H_W^*}{K_F} \left(\dfrac{\mu_D -f_D }{e \lfw m} - F_W^*\right) \right) F_W^*  \label{expressionA0}
\end{align}
\end{subequations}

When $\cI = 0$, we have:

\begin{equation*}
F_W^* = \dfrac{\mu_D - f_D}{\lfw m e}.
\end{equation*} 
Injecting this expression into \eqref{expressionA0}, we obtain:

\begin{equation*}
a_{0, \cI=0} = m_D e \lfw (1- \alpha) r_F \left(1 - \dfrac{F_W^*}{(1- \alpha)K_F}\right) F_W^* 
\end{equation*}
that is $a_{0, \cI=0}>0$ given the assumption. The coefficient $a_1$ is given by:
\begin{subequations}
\begin{align}
a_1 &= \big( \mu_D  -f_D + m_D) r_F(1+ \beta H_W^*) \dfrac{F^*_W}{K_F} + (\mu_D -f_D + m_D) m_W + r_F(1+ \beta H_W^*) \dfrac{F_W^*}{K_F} m_W - \\ \nonumber &m_D (e\lfw F^*_W + m_W), \\
a_1 &= \big( \mu_D  -f_D + m_D + m_W) r_F(1+ \beta H_W^*) \dfrac{F^*_W}{K_F} + (\mu_D -f_D) m_W  - m_D e\lfw F^*_W, \\
a_1 &= \big( \mu_D  -f_D + m_D + m_W) r_F(1+ \beta H_W^*) \dfrac{F^*_W}{K_F} + \left(\dfrac{\mu_D -f_D}{e\lfw m} - F_W^*\right) e \lfw m_D . \label{expressionA1}
\end{align}
\end{subequations}

Again, using the expression of $F^*_W$ in the case where $\cI = 0$, we have:

\begin{equation*}
a_1 = \big( \mu_D  -f_D + m_D + m_W) r_F(1+ \beta H_W^*) \dfrac{F^*_W}{K_F} .
\end{equation*}
and we do have $a_{1, \cI =0} > 0$.

The first assumption of Rough-Hurwitz is verified, $a_{i, \cI =0} > 0$ for $i=1,2,3$. Therefore, the asymptotic stability of $EE^{HF_W,  \cI =0}$ only depends on the sign of $\Delta_{Stab}= a_2 a_1 - a_0$, which has to be positive. 

We have:

\begin{multline} \label{DeltaStab, generalCase}
\Delta_{Stab, \cI = 0} > 0 \\ 
\Leftrightarrow \left(\mu_D - f_D + m_D + m_W + r_F(1+ \beta H_W^*)\dfrac{F_W^*}{K_F} \right) \times  r_F(1+ \beta H_W^*) \left( \mu_D -f_D + m_D + m_W \right)\dfrac{F_W^*}{K_F} > \\ m_D e \lfw (1-\alpha) r_F \left(1- \dfrac{F_W^*}{(1-\alpha)K_F} \right)F_W^*, \\
\Leftrightarrow \left(\mu_D - f_D + m_D + m_W + r_F(1+ \beta H_W^*)\dfrac{F_W^*}{K_F} \right) \times  (1+ \beta H_W^*) \left( \mu_D -f_D + m_D + m_W \right) > \\ m_D e \lfw \Big(K_F(1-\alpha) - F_W^* \Big),
\end{multline}

with $F_W^* = \dfrac{\mu_D - f_D}{\lfw m e}$  and $ 
H^*_{W} = \dfrac{(1-\alpha) r_F - r_F \dfrac{\mu_D - f_D}{e\lfw m K_F}}{\lfw - \beta (1-\alpha) r_F + \beta r_F  \dfrac{\mu_D - f_D}{e m \lfw K_F}} $

We have $1 + \beta H_W^* = \dfrac{\lfw}{\lfw - \beta (1-\alpha) r_F + \beta r_F  \dfrac{\mu_D - f_D}{e m \lfw K_F}}$.

In the particular case of $\beta = 0$, we have:

\begin{multline*}
\Delta_{Stab, \cI = 0} > 0 \\
\Leftrightarrow \left(\mu_D - f_D + m_D + m_W + r_F \dfrac{\mu_D - f_D}{\lfw K_F m e} \right) \times   \left( \mu_D -f_D + m_D + m_W \right) > \\ m_D e \lfw \left(K_F(1-\alpha) - \dfrac{\mu_D - f_D}{\lfw m e} \right), \\
\Leftrightarrow (\mu_D - f_D + m_D + m_W)^2 + r_F \dfrac{\mu_D - f_D}{\lfw K_F m e}  \times   \left( \mu_D -f_D + m_D + m_W \right) > \\ m_D e \lfw K_F(1-\alpha) - (\mu_D - f_D)m_W , \\
\Leftrightarrow \lfw (\mu_D - f_D + m_D + m_W)^2 + r_F \dfrac{\mu_D - f_D}{K_F m e}  \times   \left( \mu_D -f_D + m_D + m_W \right) > \\ m_D e \lfw^2 K_F(1-\alpha) - \lfw (\mu_D - f_D)m_W , \\
\Leftrightarrow 0 > \lfw^2 (1-\alpha) K_F  m_D e - \lfw \Big((\mu_D - f_D + m_D + m_W)^2 +(\mu_D - f_D)m_W \Big) - \\ \dfrac{r_F (\mu_D - f_D) }{K_F m e}  \big( \mu_D -f_D + m_D + m_W \big).\\
\end{multline*}

We define 
\begin{multline*}
P_{\Delta_{Stab, \cI = 0}}(X) := X^2 (1-\alpha) K_F  m_D e - X \Big((\mu_D - f_D + m_D + m_W)^2 +(\mu_D - f_D)m_W \Big) - \\ \dfrac{r_F (\mu_D - f_D) m_W}{K_F m_D e}  \big( \mu_D -f_D + m_D + m_W \big),
\end{multline*} 

such that we have 
\begin{equation}
\Delta_{Stab, \cI = 0} > 0 \Leftrightarrow P_{\Delta_{Stab, \cI = 0}}(\lfw) < 0.
\label{equivalenceDeltaStabP}
\end{equation}

$P_{\Delta_{Stab, \cI = 0}}$ has a positive dominant coefficient, and its other coefficients are negative. So,  $P_{\Delta_{Stab, \cI = 0}}$ admits a unique positive root, noted $\lfw^*$, given by:
\begin{multline}
\lfw^* = \\
 \dfrac{\left[m_{W}(\mu_{D}-f_{D})+\big(\mu_{D}-f_{D}+m_{D}+m_{W})^{2}\right]\left(1+\sqrt{1+4\dfrac{(1-\alpha)m_{W}r_{F}\left(\mu_{D}-f_{D}\right)\big(\mu_{D}-f_{D}+m_{D}+m_{W})}{\left[m_{W}\dfrac{\mu_{D}-f_{D}}{e}+\big(\mu_{D}-f_{D}+m_{D}+m_{W})^{2}\right]^{2}}}\right)}{2em_D (1-\alpha) K_F }
\end{multline}

Moreover, $P_{\Delta_{Stab}}$ is negative on $\left[0, \lfw^* \right)$ and positive on $\left(\lfw ^*, +\infty \right)$. Using \eqref{equivalenceDeltaStabP}, we obtain that $EE^{HF_W}_{\cI = 0}$ is locally asymptotically stable if $\lfw  < \lfw ^*$.

\marc{
When $\beta > 0$, we obtain:
\begin{multline*}
\Delta_{Stab, \cI = 0} > 0 \Leftrightarrow 0 > a_5 \lfw ^5 + a_4 \lfw ^4 + a_3 \lfw ^3 + \beta a_2 \lfw ^2 + \beta a_1 \lfw ^1 + \beta a_0
\end{multline*}
with $a_5, a_1 > 0$ and $a_4, a_3, a_2, a_0 < 0$ and no clear sign for coefficient $a_3$. Therefore, the polynomial may have (if $a_3<0$ 2 or 0 positive roots) or (if $a_3 > 0$, 5, 3 or 1 positive roots).}

\end{itemize}
\end{proof}

%
%The following intermediate table gives an overview of the long term behavior:
%
%\begin{table}[!ht]
%\centering
%\def\arraystretch{2}
%\begin{tabular}{c|c|c|c|c}
%$\cI$ & $\beta$ & $\dfrac{m e\lfw K_F(1-\alpha)}{\mu_D - f_D}$ &  $\dfrac{\lfw}{ \Big( \lfw \Big)^*}$ & \\
%\hline
%  \multirow{3}{*}{$=0$}&\multirow{3}{*}{$=0$}  & $ < 1$ & &$EE^{F_W}$ exists and is LAS. \\
%   \cline{3-5}
%& &\multirow{2}{*}{$ > 1$} & $<1$ &$EE^{HF_W}_{\cI=0}$ exists and is LAS.\\
% \cline{4-5}
% & & &$>1$ &$EE^{HF_W}_{\cI=0}$ exists and is unstable. 
%\end{tabular}
%\caption{\centering Intermediate table giving the known conditions of existence and asymptotic stability of equilibrium for system \eqref{equationsHDFWHW, cI=0}}
%\end{table}
We look now for global stability. The following propositions hold true:

\begin{prop}\label{propEEFGAS}If 
$$
\dfrac{m e \lfw K_F(1-\alpha)}{\mu_D - f_D} < 1,
$$
that is if equilibrium $EE^{F_W}$ is LAS, then it is globally asymptotically stable (GAS) on $\Omega$ for system \eqref{equationsHDFWHW, cI=0}.
\end{prop}

\begin{proof}
When $\dfrac{\mu_D - f_D}{\lfw m e K_F(1-\alpha)} >1,$ the system \eqref{equationsHDFWHW, cI=0} admits $EE^{F_W}$ as a unique equilibrium. Using the monocity of the equivalent system \eqref{equationshDfWhW}, and the fact that $\Omega$ is an invariant compact set, we obtain that $EE^{F_W}$ is GAS on $\Omega$.
\end{proof}


\begin{prop}\label{LimitCycle, cI=0}
If $\dfrac{m e \lfw K_F(1-\alpha)}{\mu_D - f_D} > 1$, and if:
\begin{itemize}
\item $0 < \Delta_{stab, \cI =0}$, that is if equilibrium $EE^{HF_W}$ is LAS, then it is GAS on $\Omega$ for system \eqref{equationsHDFWHW, cI=0}.
\item $0 > \Delta_{stab, \cI =0}$, system \eqref{equationsHDFWHW, cI=0} admits an orbitally asymptotically stable periodic solution.
\end{itemize}

When $\beta = 0$, we have the following:
\begin{itemize}
\item $\lfw < \Big( \lfw \Big)^*$, that is if equilibrium $EE^{HF_W}$ is LAS, then it is GAS on $\Omega$ for system \eqref{equationsHDFWHW, cI=0}.
\item $\lfw  > \Big(\lfw \Big)^*$, system \eqref{equationsHDFWHW, cI=0} admits an orbitally asymptotically stable periodic solution.
\end{itemize}
\end{prop}

\begin{proof}
When $\dfrac{m e \lfw K_F(1-\alpha)}{\mu_D - f_D} > 1$, system \eqref{equationsHDFWHW, cI=0} admits a unique positive equilibrium, $EE^{HF_W}$.  
By applying theorem \ref{theorem: periodicASOrbit} to the equivalent system \eqref{equationshDfWhW}, we know that either $EE^{HF_W}$ is (G)AS, or it exists a asymptotically stable periodic solutions. moreovern Condition for stability is precisely $0 < \Delta_{stab, \cI =0}$, which is equivalent to $\lfw < \lfw^*$ when $\beta = 0$.
\end{proof} 

The results we obtained are summarized in the following table:
\begin{table}[!ht]
\centering
\def\arraystretch{2}
\begin{tabular}{c|c|c|c|c}
$\cI$ &$\beta$ & $\dfrac{m e\lfw K_F(1-\alpha)}{\mu_D - f_D}$ &  $\dfrac{\lfw}{ \Big( \lfw \Big)^*}$ & \\
\hline
\multirow{4}{*}{$=0$}&\multirow{4}{*}{$=0$} & $ < 1$ & &$EE^{F_W}$ exists and is GAS.  \\
\cline{3-5}
 & & \multirow{3}{*}{$> 1$} & $ <1$ &$EE^{HF_W}_{\cI=0}$ exists and is GAS.\\
 \cline{4-5}
 & & &\multirow{2}{*}{$ > 1$} & $EE^{HF_W}_{\cI=0}$ exists and is unstable ; there is an asymptotically \\
& & & &  stable periodic solution.
\end{tabular}
\caption{\centering Conditions of existence and asymptotic stability of equilibrium for system \eqref{equationsHDFWHW, cI=0}, when $\beta = 0$}
\end{table}


\begin{table}[!ht]
\centering
\def\arraystretch{2}
\begin{tabular}{c|c|c|c|c}
$\cI$ &$\beta$ & $\dfrac{m e\lfw K_F(1-\alpha)}{\mu_D - f_D}$ &  $\Delta_{Stab, \cI =0}$ & \\
\hline
\multirow{4}{*}{$=0$}&\multirow{4}{*}{$>0$} & $ < 1$ & &$EE^{F_W}$ exists and is GAS.  \\
\cline{3-5}
 & & \multirow{3}{*}{$> 1$} & $ 0<$ &$EE^{HF_W}_{\cI=0}$ exists and is GAS.\\
 \cline{4-5}
 & & &\multirow{2}{*}{$ 0> $} & $EE^{HF_W}_{\cI=0}$ exists and is unstable ; there is an asymptotically \\
& & & &  stable periodic solution.
\end{tabular}
\caption{\centering Conditions of existence and asymptotic stability of equilibrium for system \eqref{equationsHDFWHW, cI=0}}
\end{table}

\subsection{Interpretation}
Inside this model, the parameters $\alpha$ and $\lfw$ represents the impact of human activities on their environment. It is interesting to interpret the previous results as a function of this parameters, in order to understand the consequences of an increase (or decrease) of hunt activities or environment destruction.

First, we start by condition $\dfrac{m e \lfw  K_F(1-\alpha)}{\mu_D - f_D} > 1$, which is required for $EE^{HF_W}$ to exist. We propose the following definition:

\begin{definition}\label{defLambdaMin, cI=0} We define 
$$\lambda_{F, \cI=0}^{Min} := \dfrac{\mu_D - f_D}{m e K_F(1-\alpha)}$$
such that 
$$
\text{$EE^{HF_W}_{\cI = 0}$ exists} \Leftrightarrow  \lfw > \lambda_{F, \cI=0}^{Min}.
$$
\end{definition}
\begin{proof}
The proof is straightforward: $EE^{HF_W}$ exists if $\dfrac{m e \lfw K_F(1-\alpha)}{\mu_D - f_D} > 1 \Leftrightarrow  \lfw> \dfrac{\mu_D - f_D}{ m e K_F(1-\alpha)} $.
\end{proof}

This means that if the hunting rate is not sufficient, there is no co-existence possible, and even no steady state with human population. This is due to the fact that when $\cI = 0$, only hunting activities ensure food intake. Consequently, if there is not enough hunt, the human population can not subsist.

We can note that $\lambda_{F, \cI=0}^{Min}$ is a increasing function of the antrhopization parameter $\alpha$ : the more anthropized the environment, the fewer wild animals there are, and the greater the hunting rate required. 

\begin{definition}
When $\beta = 0 $, we define
\begin{multline*}
\lambda_{F, \cI =0}^{Max}  = \\
\dfrac{\left[m_{W}(\mu_{D}-f_{D})+\big(\mu_{D}-f_{D}+m_{D}+m_{W})^{2}\right]\left(1+\sqrt{1+4\dfrac{(1-\alpha)m_{W}r_{F}\left(\mu_{D}-f_{D}\right)\big(\mu_{D}-f_{D}+m_{D}+m_{W})}{\left[m_{W}\dfrac{\mu_{D}-f_{D}}{e}+\big(\mu_{D}-f_{D}+m_{D}+m_{W})^{2}\right]^{2}}}\right)}{2em_D (1-\alpha) K_F }
\end{multline*}
such that $$
\text{$EE^{HF_W}_{\cI = 0}$ exists and is GAS} \Leftrightarrow \lambda_{F, \cI=0}^{Min} < \lfw < \lambda_{F, \cI =0}^{Max}
.$$
\end{definition}

This result can be interpreted as follow: if the hunting rate is too high, the system's dynamic tends toward a limit cycle, which correspond to a classical predator-prey system. We can also note that $\lambda_{F, \cI =0}^{Max} $ is decreasing as a function of $K_F(1-\alpha)$. 

\begin{figure}
\centering
\begin{subfigure}{0.49\textwidth}
\centering
\includegraphics[width=\textwidth]{SurfaceLambdaMinAlone.png}
\caption{Only the surface $\lfw(\Kfa, m)^{min}$ is plot.}
\end{subfigure}
\begin{subfigure}{0.49\textwidth}
\centering
\includegraphics[width=\textwidth]{SurfaceLambdaMaxAlone.png}
\caption{Both surfaces $\lfw(\Kfa, m)^{min}$ and $\lfw(\Kfa, m)^{max}$ are plot.}
\end{subfigure}
\hfill
\begin{subfigure}{\textwidth}
\includegraphics[width=1\textwidth]{BifurcationLambdaCurve.png}
\caption{}
\end{subfigure}
\caption{\centering Bifurcation diagram for system \eqref{equationsHDFWHW, cI=0}. Two first figures are plot in the $(K_F(1-\alpha), m, \lfw)$ space. The third one is a cut for $m=0.2$. Other parameters value are $r_F = 0.8$, $K_F=3000$, $e=0.1$, $\cI=0$, $\mu_D = 0.017$, $f_D = 0.001$.}
\end{figure}


\section{Model analysis in the case $\cI > 0$}
Now, we consider the case were food import occurs, \textit{ie} we assume $\cI > 0$. Since the human subsistence does not depend only on hunt, the system's dynamic and its interpretation change.

\subsection{Theoretical analysis}
\begin{prop} \label{equilibrium, I>0}
The following results hold:
\begin{itemize}
\item System \eqref{equationsHDFWHW} has a Human-only equilibrium $EE^{H} = \Big(\dfrac{\cI}{\mu_D - f_D}, 0, \dfrac{m \cI}{\mu_D - f_D} \Big)$ that always exists.
\item If $$ \lfw < r_F(1-\alpha)\Big({\dfrac{\mu_D - f_D}{m\cI}+\beta\Big)} ,$$ then system \eqref{equationsHDFWHW} has a unique coexistence equilibrium $EE^{HF_W}_{\cI = 0} = \Big(H^*_{D}, F^*_{W}, m H^*_{D} \Big)$
where
$$F^*_{W} = \dfrac{(1-\alpha)K_F}{2}\left(1 - \dfrac{\sqrt{\Delta_F}}{e(1-\alpha)r_F}\right) + \dfrac{\mu_D - f_D + \cI \beta m}{2\lfw m e},\quad
H^*_{D} = \dfrac{r_F}{\lfw m} \Big(1 - \dfrac{F^*_{W}}{K_F(1-\alpha)} \Big),
\quad 
H^*_{W} = m H^*_{D}$$
and
$$
\Delta_F = \left(e(1-\alpha)r_F - \dfrac{(\mu_D - f_D) r_F}{\lfw m K_F}\right)^2 + \dfrac{\cI \beta r_F}{\lfw K_F} \left(\dfrac{\cI \beta r_F}{\lfw K_F} + 2\dfrac{(\mu_D - f_D) r_F}{\lfw m K_F} + 2e(1-\alpha)r_F \right) + 4\dfrac{er_F}{K_F}  \cI\Big(1 - \dfrac{(1-\alpha)\beta r_F}{\lfw} \Big)
$$
\end{itemize} 
\end{prop}

\begin{proof}
An equilibrium of system \eqref{equationsHDFWHW} satisfies the system of equations:
\begin{equation}\label{system-equilibre}
\left\lbrace \begin{array}{cll}
\cI + e \lfw m F_W^* H_D^* + (f_D - \mu_D) H_D^* = 0,&&\\
F_W^* - \dfrac{(1-\alpha)K_F}{1 + \beta m H_D^*} \Big(1 - \dfrac{m(\lfw - (1-\alpha)\beta r_F) H^*_D}{(1-\alpha)r_F} \Big) = 0& \mbox{or} & F^*_W = 0,\\
H_W^* = \dfrac{m_D}{m_W} H_D^* = m H_D^*.&&
\end{array} \right.
\end{equation}

The solution of system \eqref{system-equilibre} when $F_W^* = 0$ is the Human-only equilibrium $EE^{H} = \Big(\dfrac{\cI}{\mu_D - f_D}, 0, \dfrac{m \ \cI}{\mu_D - f_D} \Big)$.
In the sequel, we assume that $F_W^* > 0$. In this case, $F^*_W$ is solution of the quadratic equation
\begin{equation}
P_F(X) := X^2 \left(\dfrac{er_F}{K_F} \right) - X \left(e(1-\alpha)r_F + \dfrac{(\mu_D - f_D) r_F}{\lfw m K_F} + \dfrac{\cI \beta r_F}{\lfw K_F} \right) + \left(\dfrac{(\mu_D - f_D)(1-\alpha) r_F}{\lfw m} - \cI\Big(1 - \dfrac{(1-\alpha)\beta r_F}{\lfw} \Big) \right) = 0.
\label{polynome-Feq}
\end{equation}

We note $\Delta_F$ the discriminant of this equation, which is positive. Indeed, we have:
\begin{align*}
\Delta_F &= \left(e(1-\alpha)r_F + \dfrac{(\mu_D - f_D) r_F}{\lfw m K_F} + \dfrac{\cI \beta r_F}{\lfw K_F} \right)^2 - 4\dfrac{er_F}{K_F}  \left(\dfrac{(\mu_D - f_D)(1-\alpha) r_F}{\lfw m} - \cI\Big(1 - \dfrac{(1-\alpha)\beta r_F}{\lfw} \Big) \right), \\
\Delta_F &= \left(e(1-\alpha)r_F - \dfrac{(\mu_D - f_D) r_F}{\lfw m K_F}\right)^2 + \dfrac{\cI \beta r_F}{\lfw K_F} \left(\dfrac{\cI \beta r_F}{\lfw K_F} + 2\dfrac{(\mu_D - f_D) r_F}{\lfw m K_F} + 2e(1-\alpha)r_F \right) + 4\dfrac{er_F}{K_F}  \cI\Big(1 - \dfrac{(1-\alpha)\beta r_F}{\lfw} \Big), \\
\Delta_F & > 0.
\end{align*}

Therefore, $P_F$ admits two real roots. Their sign depends on the sign of the constant coefficient. $P_F$ admits:
\begin{itemize}
\item One non positive root $F^*_1$ and one positive root $F^*_2$ if $$\dfrac{(\mu_D - f_D)(1-\alpha) r_F}{\lfw m} - \cI\Big(1 - \dfrac{(1-\alpha)\beta r_F}{\lfw} \Big) \leq 0 \Leftrightarrow \dfrac{(\mu_D - f_D) r_F}{\lfw m } \leq \cI\Big(1 - \dfrac{(1-\alpha)\beta r_F}{\lfw} \Big).$$
\item Two positive roots $F^*_1\leq  F^*_2$ if $\dfrac{(\mu_D - f_D) r_F}{\lfw m } > \cI\Big(1 - \dfrac{(1-\alpha)\beta r_F}{\lfw} \Big)$.
\end{itemize}
They are given by:

\begin{equation*}
F_i^* = \dfrac{K_F(1-\alpha)}{2}\left(1 \pm \dfrac{\sqrt{\Delta_F}}{e(1-\alpha)r_F}\right) + \dfrac{\mu_D - f_D}{2\lfw m e} + \dfrac{\cI \beta}{2\lfw e}, \quad i=1,2.
\end{equation*}

Moreover, $P_F$ is positive on $(-\infty, F_1^*)$, negative on $(F^*_1, F^*_2)$ and positive on $(F^*_2, +\infty)$. Let us compute $P_F(K_{F,\alpha})$ (where $ K_{F,\alpha} = K_F(1-\alpha)$).
\begin{align*}
P_F(\Kfa) &= \Kfa^2 \left(\dfrac{er_F}{K_F} \right) - \Kfa \left(e(1-\alpha)r_F + \dfrac{(\mu_D - f_D) r_F}{\lfw m K_F} + \dfrac{\cI \beta r_F}{\lfw K_F} \right) + \left(\dfrac{(\mu_D - f_D)(1-\alpha) r_F}{\lfw m} - \cI\Big(1 - \dfrac{(1-\alpha)\beta r_F}{\lfw} \Big) \right), \\
&=(1-\alpha)^2 e K_F r_F - e(1-\alpha)^2 K_F r_F - \dfrac{(\mu_D - f_D) (1-\alpha) r_F}{\lfw m} - \dfrac{\cI \beta (1-\alpha)r_F}{\lfw}  + \\ &\dfrac{(\mu_D - f_D)(1-\alpha) r_F}{\lfw m} - \cI +\cI \dfrac{(1-\alpha)\beta r_F}{\lfw}, \\
&= -\cI< 0.
\end{align*}

Therefore, $F_1^* < K_{F, \alpha} < F_2^*$. This means that $F_2^*$ is not biologically meaningful. The equilibrium of coexistence exists only when $\dfrac{(\mu_D - f_D) r_F}{\lfw m } > \cI\Big(1 - \dfrac{(1-\alpha)\beta r_F}{\lfw} \Big)$. In this case, it is given by: 
$$F^*_{W} = \dfrac{(1-\alpha)K_F}{2}\left(1 - \dfrac{\sqrt{\Delta_F}}{e(1-\alpha)r_F}\right) + \dfrac{\mu_D - f_D + \cI \beta m}{2\lfw m e},\quad
H^*_{D} = \dfrac{r_F}{\lfw m} \Big(1 - \dfrac{F^*_{W}}{K_F(1-\alpha)} \Big),
\quad 
H^*_{W} = m H^*_{D}.$$
\end{proof}

Now, we look for the asymptotic stability of the equilibriums.

\begin{prop}\label{propLAS} The following results are valid.
\begin{itemize}
\item When $\dfrac{r_F(1-\alpha)\Big({\dfrac{\mu_D - f_D}{m\cI}+\beta\Big)}}{\lfw} < 1$, the human equilibrium $EE^{H}$ is LAS.
\item When $\dfrac{r_F(1-\alpha)\Big({\dfrac{\mu_D - f_D}{m\cI}+\beta\Big)}}{\lfw} > 1$, equilibrium of coexistence $EE^{HF_W}$  exists. it is LAS if 
$$\Delta_{Stab} > 0,$$  where 

\begin{multline*}
\Delta_{Stab} = \left(\mu_D -f_D + m_D + r_F \dfrac{F_W^*}{K_{F, \alpha}} + m_W\right) \times \\
\left(\big( \mu_D -f_D + m_D + m_W) r_F \dfrac{F^*_W}{K_{F, \alpha}}   + m_D e\lfw   \left(\dfrac{\mu_D - f_D}{m e\lfw} - F^*_W \right)\right) - m_D \lfw \sqrt{\Delta_F}  F^*_{W}.
\end{multline*}
\end{itemize}
\end{prop}

\begin{proof}
To assess the local stability or instability of the different equilibria, we look at the Jacobian matrix. The Jacobian of system \eqref{equationsHDFWHW} is given by

\begin{multline*}
\mathcal{J}(H_D, F_W, H_W) = \\
\begin{bmatrix}
f_D-\mu_D - m_D & e \lfw H_W & e\lfw F_W + m_W \\
0 & r_F(1-\alpha)(1+\beta H_W) \left( 1 - \dfrac{2F_W}{K_F(1-\alpha)} \right) - \lfw H_W & - (\lfw - (1-\alpha)\beta r_F) F_W -  \dfrac{r_F\beta}{K_F} F_W^2\\
m_D & 0 & -m_W
\end{bmatrix}.
\end{multline*}


\begin{itemize}
\item At equilibrium $EE^{H}$, we have
\begin{equation*}
\mathcal{J}(EE^{H}) = \begin{bmatrix}
f_D-\mu_D - m_D & e \lfw \dfrac{m \cI}{\mu_D - f_D} & m_W \\
0 & r_F(1-\alpha)(1+\beta\dfrac{m\cI}{\mu_D - f_D}) - \lfw\dfrac{m\cI}{\mu_D - f_D} & 0 \\
m_D & 0 & -m_W
\end{bmatrix}.
\end{equation*}


The characteristic polynomial of $\mathcal{J}(EE^{H})$ is given by:
\begin{equation*}
\chi(X) = \left(X - r_F(1-\alpha)(1+\beta\dfrac{m\cI}{\mu_D - f_D}) + \lfw\dfrac{m\cI}{\mu_D - f_D} \right) \times \left(X^2 - X\Big(f_D - \mu_D - m_D - m_W \Big) + m_W(\mu_D - f_D)\right).
\end{equation*}

The constant coefficient of the second factor and its coefficient in $X$ are positive. So, the roots of the second factor have a negative real part. Therefore, only the sign of $r_F(1-\alpha)(1+\beta\dfrac{m\cI}{\mu_D - f_D}) - \lfw\dfrac{m\cI}{\mu_D - f_D}$ determines the stability of $EE^{H}$. If it is negative, $EE^{H}$ is LAS and otherwise it is unstable.

\item Now, we look for the asymptotic stability of the equilibrium of coexistence $EE^{HF_W}_{\cI > 0}$. We will reuse some of the computations done in the case $\cI = 0$, the Jacobian matrix being the same.

The characteristic polynomial of $\mathcal{J}(EE^{H F_W})$ is given by: $\chi = X^3 + a_2 X^2 + a_1 X + a_0$, where expressions for $a_i$ are given by \eqref{expressionA2}, \eqref{expressionA1}, \eqref{expressionA0} respectively. According to the Routh-Hurwitz criterion \marc{ref}, $EE^{H F_W}$ are LAS if $a_i > 0$ for $i=1,2,3$ and $a_2 a_1 - a_0 > 0$.

According to \eqref{expressionA2}, we have:
\begin{equation*}
a_2 = -\left(f_D - \mu_D - m_D - (1+\beta H_W^*)r_F \dfrac{F_W^*}{K_F} - m_W \right)
\end{equation*}
that is $a_2>0$. According to \eqref{expressionA0}, we have:
\begin{equation*}
a_0 = e \lfw m_D r_F \left(\dfrac{\mu_D -f_D }{e \lfw m K_F} - 2\dfrac{F_W^*}{K_F} + (1-\alpha) + \dfrac{\beta H_W^*}{K_F} \left(\dfrac{\mu_D -f_D }{e \lfw m} - F_W^*\right) \right) F_W^*
\end{equation*}

Using
\begin{equation*}
F_W^* = \dfrac{(1-\alpha)K_F}{2}\left(1 - \dfrac{\sqrt{\Delta_F}}{e(1-\alpha)r_F}\right) + \dfrac{\mu_D - f_D + \cI \beta m}{2\lfw m e},
\end{equation*}
we obtain
\begin{equation*}
a_0 = m_D \lfw e r_F \left(\dfrac{\sqrt{\Delta_F}}{er_F} - \dfrac{\cI \beta}{\lfw K_F e} +  \dfrac{\beta H_W^*}{K_F} \left(\dfrac{\mu_D -f_D }{e \lfw m} - F_W^*\right)\right)  F^*_{W},
\end{equation*}

We will show that we have both $\dfrac{\mu_D -f_D }{e \lfw m} - F_W^* > 0$ and $\dfrac{\sqrt{\Delta_F}}{er_F} - \dfrac{\cI \beta}{\lfw K_F e} > 0$. We start by the second inequality. According to the proposition \ref{equilibrium, I>0}, we have:

\begin{multline*}
\Delta_F = \left(e(1-\alpha)r_F - \dfrac{(\mu_D - f_D) r_F}{\lfw m K_F}\right)^2 + \dfrac{\cI \beta r_F}{\lfw K_F} \left(\dfrac{\cI \beta r_F}{\lfw K_F} + 2\dfrac{(\mu_D - f_D) r_F}{\lfw m K_F} + 2e(1-\alpha)r_F \right) + \\ 4\dfrac{er_F}{K_F}  \cI\Big(1 - \dfrac{(1-\alpha)\beta r_F}{\lfw} \Big)
\end{multline*}
 which gives $\Delta_F > \left(\dfrac{\cI \beta r_F}{\lfw K_F}\right)^2$ and therefore,

\begin{equation*}
\dfrac{\sqrt{\Delta_F}}{er_F} - \dfrac{\cI \beta}{\lfw K_F e} > 0.
\end{equation*}

We look for the sign of $\left(\dfrac{\mu_D -f_D}{m e\lfw} - F^*_{W}\right)$ using the sign of $P_F(X)$ (negative on $(F^*_W, F^*_2)$, positive otherwise). We have:

\begin{align*}
P_F\Big(\dfrac{\mu_D - f_D}{\lfw m e}\Big) &= \left(\dfrac{\mu_D - f_D}{\lfw m e}\right)^2 \left(\dfrac{er_F}{K_F} \right) - \dfrac{\mu_D - f_D}{\lfw m e} \left(e(1-\alpha)r_F + \dfrac{(\mu_D - f_D) r_F}{\lfw m K_F} + \dfrac{\cI \beta r_F}{\lfw K_F} \right) + \\ & \left(\dfrac{(\mu_D - f_D)(1-\alpha) r_F}{\lfw m} - \cI\Big(1 - \dfrac{(1-\alpha)\beta r_F}{\lfw} \Big) \right), \\
&= - \dfrac{(\mu_D - f_D) \cI \beta r_F}{e \lfw ^2 K_F} - \cI + \dfrac{\cI (1-\alpha)r_F \beta r_F}{\lfw}, \\
&= -\cI \left( 1 - \dfrac{(1-\alpha)r_F \beta r_F}{\lfw} + \dfrac{(\mu_D - f_D) \cI \beta r_F}{e \lfw ^2 K_F} \right), \\
& < 0
\end{align*}
since we assumed $1 - \dfrac{(1-\alpha)r_F \beta r_F}{\lfw} > 0$. Therefore, $\dfrac{\mu_D - f_D}{m e\lfw} > F^*_{W}$, which was the last step to show that $a_0>0$. 

According to \eqref{expressionA1}, coefficient $a_1$ is given by:
\begin{equation*}
a_1 = \big( \mu_D  -f_D + m_D + m_W) r_F(1+ \beta H_W^*) \dfrac{F^*_W}{K_F} + \left(\dfrac{\mu_D -f_D}{e\lfw m} - F_W^*\right) e \lfw m_D .
\end{equation*}

which is positive, since we just have shown that $\dfrac{\mu_D - f_D}{m e\lfw} > F^*_{W}$.

The first assumption of Rough-Hurwitz is verified, $a_i > 0$ for $i=1,2,3$. Therefore, the local asymptotic stability of $EE^{HF_W}$ only depends on the sign of $\Delta_{Stab}= a_2 a_1 - a_0$, which has to be positive.

\end{itemize}





\end{proof}

For now, we have the following table:
\begin{table}[ht!]
\def\arraystretch{2}
\centering
\begin{tabular}{c|c|c|c}
$\cI$ & $\dfrac{r_F(1-\alpha)\Big({\dfrac{\mu_D - f_D}{m\cI}+\beta\Big)}}{\lfw} $ & $\Delta_{Stab}$ & \\
\hline
\multirow{3}{*}{$>0$} & $<1$ & &$EE^{H}$ exists and is LAS \\
\cline{2-4}
 & \multirow{2}{*}{$> 1$}  & $>0$ &$EE^{HF_W}$ exists and is LAS\\
 \cline{3-4}
 & & $ < 0$ & $EE^{HF_W}$ is unstable \\
\end{tabular}
\caption{Intermediate table summarizing the long term behavior}
\end{table}

As before, we can complete it with information about global asymptotic stability and existence of limit cycles. 

\begin{prop}
If $$\dfrac{r_F(1-\alpha)\Big({\dfrac{\mu_D - f_D}{m\cI}+\beta\Big)}}{\lfw} < 1$$
that is if equilibrium $EE^{H}$ is LAS, then it is GAS on $\Omega$ for system \eqref{equationsHDFWHW}.
\end{prop}


\begin{prop}
If $\dfrac{r_F(1-\alpha)\Big({\dfrac{\mu_D - f_D}{m\cI}+\beta\Big)}}{\lfw} > 1$ and if 

\begin{itemize}
\item $\Delta_{Stab} > 0$, that is if equilibrium $EE^{HF_W}_{\cI >0}$ is LAS, then it is GAS on $\Omega$ for system \eqref{equationsHDFWHW}.
\item $\Delta_{Stab} < 0$, system \eqref{equationsHDFWHW} admits an orbitally asymptotically stable periodic solution.
\end{itemize}
\end{prop}



\begin{table}[!ht]
\def\arraystretch{2}
\centering
\begin{tabular}{c|c|c|c}
$\cI$ & $\dfrac{r_F(1-\alpha)\Big({\dfrac{\mu_D - f_D}{m\cI}+\beta\Big)}}{\lfw} $ & $\Delta_{Stab}$ & \\
\hline
\multirow{3}{*}{$>0$} & $<1$ & &$EE^{H}$ exists and is GAS \\
\cline{2-4}
 & \multirow{3}{*}{$> 1$}  & $>0$ &$EE^{HF_W}_{\cI>0}$ exists and is GAS\\
 \cline{3-4}
 & & \multirow{2}{*}{$ < 0$} & $EE^{HF_W}_{\cI>0}$ exists and is unstable ; there is an asymptotically \\
 & & &  stable periodic solution. \\
\end{tabular}
\caption{\centering Conditions of existence and asymptotic stability of equilibrium for system \eqref{equationsHDFWHW}}
\end{table}

\subsection{Interpretation}

\begin{definition}
We define 
$$\lambda_{F, \cI>0}^{Max} := r_F(1-\alpha)\Big({\dfrac{\mu_D - f_D}{m\cI}+\beta\Big)}$$
such that 
$$
\text{$EE^{HF_W}$ exists} \Leftrightarrow  \lfw < \lambda_{F, \cI>0}^{Max}.
$$
\end{definition}

When $\cI = 0$, the existence of the equilibrium of coexistence is implied by the condition $\lambda_{F, cI = 0}^{Min} < \lfw$, see definition \ref{defLambdaMin, cI=0}. When $\cI > 0$, we instead obtain a maximum bound for $\lfw$. This is due to the fact that when $\cI > 0$, a human population is always present (equilibrium $EE^{H}$ exists), and has to not over-hunt in order to preserve the wild fauna (we could rewrite the condition has $1 < \dfrac{r_F}{\lfw H^*_{W, H}}$ : comparison between growth and intake).


We can note that when $\cI > 0$, the existence of the equilibrium of coexistence does not depend on the level of anthropization, $K_F(1-\alpha)$ but the value of $F^*_W$ does. When $\cI = 0$, it is the opposite. 
\marc{donc quand $\cI =0$, l'existence de la coexistence dépend la capacité du milieu, et la valeur de l'équilibre (=centre du cyle limite ?) uniquement de la chasse ; si on reprend les calculs, on doit vérifier ce seuil pour éviter que $H^*$ devienne négatif = on n'est pas sûr que la pop humaine puisse exister. Dans le cas $\cI > 0$, c'est l'existence de $F^*$ qui n'est pas sûr, et qui dépend de la pression de chasse.}

\section{Numerical scheme \marc{A REPRENDRE AVEC $\beta$ !!}}
\begin{definition} A matrix $A \in \mathcal{M}_n (\mathbb{R})$ is called an $M$-matrix if\begin{itemize}
\item all its off-diagonal term are negative
\item and its real eigenvalues are positive.
\end{itemize}

If $A$ is a $M$-Matrix, then it is inverse positive, that is $A^{-1}$ exists and all its coefficients are positive.


\end{definition}


For a given $\Delta t>0$, we set $Y^n=\Big(H_D^n,F_W^n,H_W^n \Big)$ as an approximation of $y(t)=\Big(H_D(t),F_W(t),H_W(t)\Big)$ at $t=n\Delta t$, for $n=0,...,N$, where $T=N\Delta$.


\subsection{Implicit NS scheme}
We can consider the following implicit non standard scheme:

\begin{equation} \label{NSImplicit scheme}
\Big(I_3 - \phi(\Delta t) M(Y^n) \Big) Y^{n+1} = Y^{n} + \phi(\Delta t)V,
\end{equation}
where $I_3$ is the identity matrix, $V = \begin{bmatrix}
\cI & 0 & 0
\end{bmatrix}^T$ and 
\begin{equation}
M(Y^n) = \begin{bmatrix}
f_D - \mu_D - m_D & e \lfw H_W^n & m_W \\
0 & r_F(1-\alpha)(1+\beta H_W^n)\left(1 - \dfrac{F_W^n}{K_F(1 - \alpha)} \right) - \lfw H_W^n & 0 \\
m_D & 0 & -m_W
\end{bmatrix}.
\end{equation}


If we choose $\phi(\Delta t)$ such that $I_3 - \phi(\Delta t) M(Y) $ is an $M$-matrix for all $Y$, we will obtain the conservation of the system's positivity. We have:

\begin{equation}
I_3 - \phi(\Delta t) M(Y)  = \begin{bmatrix}
1 + \phi(\Delta t) \Big( \mu_D + m_D -f_D \Big) & - \phi(\Delta t) e \lfw H_W^n & -\phi(\Delta t) m_W \\
0 & 1 - \phi(\Delta t) \left(r_F\left(1 - \dfrac{F_W^n}{K_F(1 - \alpha)} \right) - \lfw H_W^n \right)& 0 \\
-\phi(\Delta t) m_D & 0 & 1 + \phi(\Delta t) m_W
\end{bmatrix}
\end{equation}

For all $Y \in \mathbb{R}^3_+$ and $\Delta t \geq 0$, the off-diagonals term are negative. Therefore, $I_3 - \phi(\Delta t) M(Y) $ is an $M$-matrix if its real eigenvalues are positive. Its characteristic polynomial is given by:
\begin{multline}
\chi = \left(X - 1 + \phi(\Delta t) \Big(r_F(1-\alpha)(1+\beta H_W^n)\Big(1 - \dfrac{F_W^n}{K_F(1 - \alpha)} \Big) - \lfw H_W^n \Big)\right) \times \\
\left(X^2 - X \Big(2 + \phi(\Delta t) (\mu_D - f_D + m_D + m_W) \Big) + 1 + \phi(\Delta t) (\mu_D - f_D + m_D + m_W) + \phi(\Delta t)^2 m_W ( \mu_D - f_D) \right)
\end{multline}
The constant and dominant coefficient of the polynomial $$\left(X^2 - X \Big(2 + \phi(\Delta t) (\mu_D - f_D + m_D + m_W) \Big) + 1 + \phi(\Delta t) (\mu_D - f_D + m_D + m_W) + \phi(\Delta t)^2 m_W ( \mu_D - f_D) \right)$$ are positive, while its coefficient in $X$ is negative. Therefore, this polynomial admits either two complex and conjugate roots either two positive real roots.

If we set
\begin{equation}
\phi(\Delta t) := \dfrac{1- e^{-Q \Delta t}}{Q}
\end{equation}

\marc{Si l'on suit la construction du schéma exact implicit pour $x' = a x$ , on devrait prendre
\YD{
\begin{equation}
\phi(\Delta t) := \dfrac{e^{Q \Delta t} - 1}{Q}
\end{equation}}
mais il ne me semble pas y avoir de relation évidente entre les paramrètres ($r_F$, $\mu_D+m_D-f_D$ ou $m_W$) et $Q$ qui permettent d'avoir une M-matrix ; j'ai donc gardé la première formule, comme dans l'article ``Mathematical studies on the sterile insect technique for the Chikungunya disease and Aedes albopictus", Dumont \& Tchuenche)
}

with $Q \geq r_F(1 - \alpha)$, we have $1 - \phi(\Delta t)r_F(1-\alpha) \geq 0$ and the eigenvalue $$1 - \phi(\Delta t) \Big(r_F\Big(1 - \dfrac{F_W^n}{K_F(1 - \alpha)} \Big) - \lfw H_W^n \Big) = 1 - \phi(\Delta t)(1-\alpha)r_F + r_F \dfrac{1 + \beta H_W^n}{K_F}F^n_W + (\lfw - (1-\alpha) r_F  \beta) H^n_W $$ is positive. This show that $F(Y, \Delta t)$ is an $M$-matrix.

\medskip
A fixed point $Y^*$ of equation \eqref{NSImplicit scheme} verifies:
\begin{align*}
\Big(I_3 - \phi(\Delta t) M(Y^*) \Big) Y^* &= Y^* + \phi(\Delta t)V \\
 M(Y^*) Y^* + V&= 0
\end{align*}
which is precisely the system of equations satisfied by the fixed points of the continuous system \eqref{equationsHDFWHW}. Therefore, the discrete and continuous system have the same fixed points.

\YD{
\begin{definition}
A numerical scheme is called elementary stable whenever it has no other fixed points than those of the continuous system it approximates, the local stability of these fixed points is the same for both the discrete and the continuous dynamical systems for each value of $\Delta t$.
\end{definition}
Il faut donc vérifier cela....
}

\subsection{Explicit NSS}
 We consider the following nonstandard scheme

\begin{equation}
\def\arraystretch{2}
\left\{ \begin{array}{l}
H_{D}^{n+1}=\Big(1-\phi(\Delta t) \left(\mu_{D}+m_{D}-f_{D}\right)\Big)H_{D}^{n}+ \phi(\Delta t)\Big(\cI+ \big(e\lambda_{F}F_{W}^{n+1} + m_{W}\big)H_{W}^{n}\Big)\\
F_{W}^{n+1}=\dfrac{\left(1+\phi(\Delta t)r_{F}\right)}{1+\phi(\Delta t)\left(\dfrac{r_{F}}{K_{F}(1-\alpha)}F_{W}^{n}+\lambda_{F}H_{W}^{n}\right)}F_{W}^{n}\\ 
H_{W}^{n+1}=\Big(1-\phi(\Delta t)m_{W}\Big)H_{W}^n+\phi(\Delta t)m_{D}H_{D}^n
\end{array}\right.
\end{equation}
where $\phi(\Delta t)=\dfrac{1-e^{-Q\Delta t}}{Q}$, with $Q=\max\{\mu_D+m_D-f_D,m_W\}$. 
It is straightforward to check that
$$
Y^n \geq 0 \Rightarrow Y^{n+1}\geq 0,\qquad \forall n\in \mathbb{N}.
$$

\section{Results}

\bibliographystyle{plain}
\bibliography{Math,Biblio/Math}

\end{document}

