\documentclass{article}
\usepackage{graphicx,ulem} 
\usepackage{color}
\usepackage{amsfonts,amsmath}
\usepackage{amsthm}
\usepackage{empheq}
\usepackage{mathtools}
\usepackage{multirow}
%\usepackage{tikz}
\usepackage{titlesec}
\usepackage{caption}
%\usepackage{lscape}
\usepackage{graphicx}
\captionsetup{justification=justified}
\usepackage[toc,page]{appendix}
\usepackage{hyperref}
\usepackage{subcaption}
\usepackage{pdftricks}
\usepackage{xcolor}
\begin{psinputs}
\usepackage{amsfonts,amsmath}
	\usepackage{pstricks-add}
   \usepackage{pstricks, pst-node}
   \usepackage{multido}
   \newcommand{\lfw}{\lambda_{F}}
\end{psinputs}

\textheight240mm \voffset-23mm \textwidth160mm \hoffset-20mm

 \graphicspath{{./Images/}{../Schema}{./Images/HT1/}}
%\graphicspath{{Figures}}
\setcounter{secnumdepth}{4}
\titleformat{\paragraph}
{\normalfont\normalsize\bfseries}{\theparagraph}{1em}{}
\titlespacing*{\paragraph}
{0pt}{3.25ex plus 1ex minus .2ex}{1.5ex plus .2ex}

\newcommand{\lfd}{\lambda_{F, D}}
\newcommand{\lfw}{\lambda_{F}}
\newcommand{\Kfa}{K_{F,\alpha}}
\newcommand{\cI}{\mathcal{I}}
\newcommand{\mW}{\tilde{m}_W}
\newcommand{\mD}{\tilde{m}_D}
\newcommand{\R}{\mathbb{R}}
\newcommand{\N}{\mathcal{N}}

\newcommand{\marc}[1]{\textcolor{teal}{#1}}
\newcommand{\YD}[1]{\textcolor{magenta}{#1}}
\newcommand{\VY}[1]{\textcolor{blue}{#1}}
\newcommand{\vdeux}[1]{\textcolor{orange}{#1}}
\newcommand{\vtrois}[1]{\textcolor{gray}{#1}}

\DeclareMathOperator{\Tr}{Tr}
\newtheorem{theorem}{Theorem}
\newtheorem{prop}[theorem]{Proposition}
\theoremstyle{definition}
\newtheorem{definition}[theorem]{Definition}
\theoremstyle{remark}
\newtheorem{remark}[theorem]{Remark}
\newtheorem{cor}[theorem]{Corollary}
\newcommand*\phantomrel[1]{\mathrel{\phantom{#1}}}

\title{Impact of anthropisation and over-hunting on the coexistence between humans and wildlife.}
\author{Yves Dumont, Marc Hétier and Valaire Yatat-Djeumen}
\begin{document}

\maketitle
%{\hypersetup{hidelinks}
%\tableofcontents}
%\newpage
\section*{Changement Globaux}
\begin{itemize}
\item Réorganisation : la partie sur le système compétitif équivalent est remontée, et est devenue une section (et non plus une sous section)
\item Ajout dans l'introduction de l'objectif de l'article et du plan
\item Numérotation proposition/théorème/remarque continue
\item Ajout des référence et de la page pour les théorèmes lorsque j'en parle
\item Introduction d'un système générique pour l'annexe sur les théorèmes généraux (section \ref{sec:litterature theorems}), afin d'éviter des répétitions
\item ...
\end{itemize}

\section*{Questions ?}
\begin{itemize}
\item Est ce que je laisse les tables et diagrammes de résulat théorique (par exemple la table \ref{table:long term dynamic, I = 0}) en variable $H_D, F_W, H_W$ (mais dans ce cas il faudrait les changer de place) ou est ce que je les réecris avec les variables $h_D, h_W, f_W$ ?
\end{itemize}

\section{Introduction}

%\begin{enumerate}
%\item Contexte général : forêt tropical, population, environnement affecté : chasse et industrialisation. Perte de biodiv. Important car ..
%\item  Intérêt de la modèlisation math
%\item Zone d'étude considérée : sud cameroun, population principalement de chasseur, peu de ressources (nourriture) proviennent de la végétation
%\item Nouveauté de l'étude : contrairement aux autres modèles, on considère une pop de chasseur : l'interaction se fait directement entre pop humaine et la faune sauvage
%\item But de l'article : analyser l'impact de la (sur)-chasse et de l'anthropisation du milieu sur les populations (animales et humaines)
%\item Plan de l'article
%
%\begin{enumerate}
%\item Présentation du modèle
%\item Analyse théorique
%\begin{itemize}
%\item Analyse en QSSA
%\item Montrer Orbites + Différence 2D-3D
%\item Analyse 3D : Stab locale
%\item Monotonie
%\item Stab Globale
%\item Qu'est ce qui se passe quand epsilon tend vers 0
%\end{itemize}
%\item Interprétation ecologique
%\item 
%\item Conclusion
%\end{enumerate}
%\end{enumerate}

% \YD{Marc, ton niveau d'anglais.... si tu as des doutes utilise un "translator", ou https://www.deepl.com/fr/translator... mais on ne peut pas continuellement corriger et recorriger des formulations ou des expressions qui n'existent pas en anglais! Je doute que "defaunation" existe, même en français.... Il vaut mieux faire des phrases simples et compréhensibles. On n'a pas besoin de beaucoup de blabla....} \marc{Defaunation existe bien en anglais, ou est du moins utilisé par les auteurs qui parlent du sujet ; voir par exemple \cite{benitez-lopez_intact_2019} dont le titre est "Intact but empty forests? Patterns of hunting induced mammal defaunation in the tropics"...}

Tropical forests are particularly rich ecosystems, in term of plants and animals diversity \YD{une ou plusieurs  références}. They also provide resources for forest-based people, and other human populations living nearby. These resources include food \cite{avila_martin_food_2024}, medicine
%cultural \YD{????} purposes \cite{kumar_marginal_2014}  \YD{(qu'est ce que cela veut dire)
and energy sources \cite{mangula_energy_2019} for example.

However, tropical forests are \vdeux{increasingly} degraded and fragmented by human settlement and activities, as the development of infrastructures (roadways, harbors, dams, ...), agriculture, and industrial complexes (industrial palm groves, industrial logging,...), etc. 
Today, only 20\% of the remaining area are considered as intact, see \cite{benitez-lopez_intact_2019}. It is not only the vegetation that is endangered, but also the wildlife that it shelters. Of course, the destruction of their habitat has indirect consequences on animal populations, but they are also directly threatened by human activities, and especially by over-hunting \cite{benitez-lopez_intact_2019, wilkie_empty_2011}. \vdeux{In 2019, it was estimated in \cite{benitez-lopez_intact_2019}that the abundance of tropical mammal species had declined by an average of 13\%. The authors also predicted a decline of over 70\% for mammals in West Africa. }

Defaunation is not harmless. It impacts both vegetation and human populations. In \cite{ripple_bushmeat_2016}, the authors recall that even partial defaunation has consequences on the environment and on the human populations. For example, extinction of mammals affects the forest regeneration, as they play a role in seed dispersal among other things (see \cite{peres_dispersal_2016, wright_bushmeat_2007}). Defaunation is also associated with the (re)emergence of zoonotic diseases, see for example \cite{dobigny_zoonotic_2022, white_emerging_2020}. 

Moreover, bushmeat remains an important source of food and income for certain populations, \cite{jones_incentives_2019}. 
\vdeux{
On one hand this means that hunting can not simply be prohibited; on the other hand, it also means that over hunting threatens the food security and livelihoods of those populations. Therefore, it is necessary to tackle the subject of sustainable hunting.
}

It can be difficult to determine whether a specific hunting, as practised in a particular place, is sustainable or not. Indeed, acquiring knowledge about wildlife and hunting practices can not be done using remote-sensing \cite{peres_detecting_2006}. It requires specific field studies such as hunter recall interviews or continuous monitoring. These surveys are costly and susceptible to biases, see \cite{jones_consequences_2020}.

However, we believe that mathematical modeling can help to overcome these problems. Indeed, mathematical models not only allow us to synthesize and generalize complex reality, as well as determine parameters and thresholds of interest. This knowledge may, in return, facilitate future field studies, and help decision makers to make justified decisions, see \cite{deangelis_towards_2021}.

%Previous works modeled human-environment interactions in various ways.
\vdeux{
Previous studies have modeled human-environment interactions in a variety of ways.
Some of them consider that the environment - including both vegetation and wildlife- to provide necessary services to agriculture, see for example \cite{bengochea-paz_agricultural_2020, roman_dynamics_2018} and are interested in how and when social-ecological system collapse. Other studies (see for example \cite{bulte_habitat_2003, nlom_bio-economic_2021}), in the bio-economic fields, reason in terms of economic cost and profit seeking to determine the most effective wildlife management strategy. In the article \cite{fanuel_modelling_2023}, the authors review articles modeling the impact of human and human pressure over forest biomass and wildlife. The dynamical models presented rarely focus on overhunting or on coexistence of human population and wildlife. Instead, they focus on the impact of specific types of anthropisation, such as industrialization \cite{agarwal_depletion_2010}, mining activities or human pressure \cite{dubey_modelling_2009}) on the forest biomass. Moreover, the majority of these models involve at least 4 variables, which makes the theoretical analysis and the understanding of the models more complicated.
}

\vdeux{
The originality of the present study is to model direct interactions between a forest based human population and the surrounding wildlife. The focus is on over-hunting rather than on the anthropization of the habitat: vegetation will not be a variable. There are two reasons for this. First, we aim to keep the model simple. A simple model is easier to understand and analyze. It can also be easily supplemented with other models. In the context of this study, it would be interesting to develop a One Health approach in order to take into account the connexions between human-environment interactions and epidemiological risk. The second justification is given by the geographic area studied.
}
Indeed, we focus on a situation representing South-Cameroon. In this tropical region, some of the populations (as the Baka Pygmies) still rely on hunting and agriculture for their livelihood, see \cite{avila_martin_food_2024}. These populations do not threaten the vegetation (consummation of plant for food or medical uses are low compared to forest regeneration, see \cite{koppert_consommation_1996}). On the other hand, the forest is impacted by industrial development: a deep sea harbor was built near the city of Kribi and industrial palm groves are expanded. Moreover, these new constructions are supported by the development of roads, which fragmented the environment. See for example \cite{foonde_change_2018, romain_deforestation_2017} about landscape and biodiversity changes around Kribi.

\vdeux{
This work aims to investigate the impact of overhunting and the anthropisation of habitat on the possibility of coexistence between a forest based human population and wildlife. It also highlights how the presence of a constant human immigration affects this possibility.}

\vdeux{
The manuscript is organized as follows. In Section \ref{sec:model}, a dynamical model describing human-environment interactions is described. In Sections \ref{sec:qssa} and \ref{sec:competitive}, we propose a mathematical analysis of this model. First, we present a simple but inconclusive quasi-steady state approach. Then, we develop a more complicated but complete analysis using an equivalent competitive system. In Section \ref{sec:ecological}, we provide an ecological interpretation of the results we obtained, as well as numerical simulations. Finally, conclusions are given in Section \ref{sec:conclusion}.
}

\section{A Model of Human-Environment Interactions} \label{sec:model}
We model the interactions between a hunter community and wilderness using a consumer-resource model. We consider two areas, one corresponding to a domestic area (a village for example), the second to a wild area (a tropical forest for example). Humans in the domestic area, $H_D$, are modeled using a consumer equation based on available resource. The dynamic of $H_D$ includes several terms. First there is a constant growth term, denoted by $\cI$, representing the immigration from other inhabited areas. For instance, the development of an industrial complex attracts new workers to the site. A second term in the equation is the natural death-rate of $H_D$, noted $\mu_D$. Furthermore, we assume that the villagers are able to produce a certain amount of food at the rate $f_D$. Note that $f_D$ can be understand as $f_D = e \lfd F_D$ where $F_D$ is a constant amount of cattle or agricultural resources, $\lfd$ a harvest rate and $e$ the proportion of these resources in human diet. 

The villagers come and go between the domestic and wild areas, mainly for hunting activities. These displacements are modeled by the function $M(H_D, H_W) = m_W H_W - m_D H_D$, where $H_W$ is the human population presents in the wild area. We note $m_D$ (respectively $m_W$) the migration rate from the domestic (resp. wild) area to the wild (resp. domestic) area. The dynamic of $H_W$ simply corresponds to the migration process. 

Wild animals, $F_W$, hunted in the wild area are partially used to feed the villagers. Therefore, we consider a last growth term $e \Lambda_F(F_W) H_W$, where $\Lambda_F(F_W)$ is the number of prey hunted by one hunter, and $e$ is the proportion of wild prey in diet. This parameter $e$ encompasses both the direct consumption of the meat by the villagers and the fact that a large proportion of bush-meat is sold on city market, allowing the villagers to purchase other supply \cite{wilkie_bushmeat_1998}.


The dynamic of $F_W$ follows a logistic equation, with a carrying capacity $K_F$ dependent on the surrounding vegetation. To take into account the level of anthropization of their habitat, we introduce the non-negative parameter $\alpha \in [0, 1)$. When $\alpha > 0$, the carrying capacity of the habitat is reduced of $\alpha \%$ from its original value. Anthropization may also have a negative impact on the animal's growth rate $r_F$. For the sake of simplicity, we model this impact in the same way, by multiplying the growth rate by $(1-\alpha)$.

We consider two different interactions between wild fauna and the population located in the wild area. Firstly, as mentioned above, the wild fauna is hunted by humans present in the wild area. We model this predation by the functional $\Lambda_F(F_W)H_W =  \lfw F_W H_W$, where $\lfw$ is the hunting rate. We choose an unbounded functional response to take into account the possibility of over-hunt. On the other hand, it is known that housing, culture and food supply may attract some mammals, specially rodents and can favor their reproduction (see for example \cite{dobigny_zoonotic_2022, dounias_foraging_2011}). We take this effect into account by multiplying the wild animal's growth rate, $r_F$, by the functional $(1 +  \beta H_W)$, where $\beta$ is the influence rate that human activities may have on wild animal's growth.

\medskip
Finally, the model is given by the following equations:
\begin{equation}
\def\arraystretch{2}
\left\{ 
\begin{array}{l}
\dfrac{dH_D}{dt}= \cI + e\lfw H_W F_W + (f_D - \mu_D) H_D - m_D H_D + m_W H_W, \\
\dfrac{dF_W}{dt} = r_F(1- \alpha) (1+ \beta H_W) \left(1 - \dfrac{F_W}{K_F(1-\alpha)} \right) F_W - \lfw F_W H_W,\\
\dfrac{dH_W}{dt}= m_D H_D - m_W H_W.
\end{array} \right.
\label{equation:HDFWHW}
\end{equation}

For the rest of the study, we will note $y$ the vector of state variables, that is $y = \Big(H_D, F_W, H_W \Big)$ and $f(y)$ the right hand side of the system.

The whole dynamic is represented through the flow chart in figure \ref{fig:flow chart} and the table \ref{table:symbol} summarizes the meaning of the state variables and parameters used.

\begin{figure}[!ht]
\centering
\includegraphics[width=0.5\textwidth]{flowChartVertical.pdf}
\caption{System flow chart.}
\label{fig:flow chart}
\end{figure}


\begin{table}[ht]
\center
\begin{tabular}{|c|c|c|}
\hline 
Symbol & Description & Unit \\ 
\hline \hline
$H_D$ & Humans in the domestic area & Ind \\
$H_W$ & Humans in the wild area & Ind \\
$F_W$ & Wild fauna & Ind \\
\hline \hline
$t$ & Time & Year \\
$e$ & Proportion of wild meat in the diet & - \\
$f_D$ & Food produced by human population & Year$^{-1}$ \\
$\mu_D$ & Human mortality rate  & Year$^{-1}$ \\
$m_D$ & Migration from domestic area to wild area & Year$^{-1}$ \\
$m_W$ & Migration from wild area to domestic area & Year$^{-1}$ \\
$r_F$ & Wild animal growth rate & Year$^{-1}$ \\
$K_F$ & Carrying capacity for wild fauna, fixed by the environment& Ind \\
$\alpha$ & Proportion of anthropized environment & - \\
$\beta$ & Positive impact from human activities to animal growth rate & Ind$^{-1}$  \\
$\lfw$ & Hunting rate & Ind/Year\\
$\mathcal{I}$ & Immigration rate &Ind/Year\\
\hline
\end{tabular}
\caption{State Variables and Parameters of the model}
\label{table:symbol}
\end{table}

To analyze system \eqref{equation:HDFWHW} and its long term behavior, we will use divers notions and theorems; they are recalled in Appendix \ref{sec:litterature theorems}, page \pageref{sec:litterature theorems}. We start by showing that the problem is well posed.

\subsection{Existence and uniqueness of global solutions}
In this section, we state general results on system \eqref{equation:HDFWHW}:  existence of an invariant region, existence and uniqueness of global solutions.

We begin by proving the local existence and uniqueness of solutions of system \eqref{equation:HDFWHW}. The right hand side of equations \eqref{equation:HDFWHW} defines a function $f(y)$ which is of class $\mathcal{C}^1$ on $\mathbb{R}^3$. The theorem of Cauchy-Lipschitz ensures that model \eqref{equation:HDFWHW} admits a unique solution, at least locally, for any given initial condition, see \cite{walter_ordinary_1998}.

\begin{remark}
To avoid infinite growth of human population, and ensure that the system is well defined, we need to add a constraint on the sign of $f_D - \mu_D$: on the following, we will assume that $f_D - \mu_D < 0$. This means that the food produced by the human population living in the domestic area is not sufficient to ensure the permanence of the population. Hunt, or immigration, is necessary. 
\end{remark}

The following proposition indicates a compact and invariant subset of $\mathbb{R}_+^3$ on which the solutions of system \eqref{equation:HDFWHW} are bounded.

\begin{prop}\label{prop:invariantRegion} 
\vdeux{J'ai retravaillé la proposition et la preuve, en posant $S = H_D + eF_W$ au lieu de $S = H_D + eF_W + H_W$. Je pense que les choses sont plus claires comme ça, et avoir $H_W^{max} = m S^{max}$ (au lieu de $(1 + m)$) facilite aussi d'autre preuve.}

Assume 
\begin{equation*}
\beta < \dfrac{4(\mu_D - f_D)}{m e r_F (1-\alpha)^2 K_F} := \beta^*.
\end{equation*}
Then, the region
$$\Omega = \Big\{\Big(H_D, F_W, H_W \Big) \in \mathbb{R}_+^3  \Big|H_D + eF_W \leq S^{max}, F_W \leq F_W^{max}, H_W \leq H_W^{max} \Big\},$$
is a compact and invariant set for system \eqref{equation:HDFWHW}, 
where
$$
S^{max} = \dfrac{\cI + \left( \mu_D - f_D   +  \dfrac{(1-\alpha)r_F}{4}\right)e(1-\alpha)K_F}{  \Big(\mu_D -f_D - \dfrac{er_F (1-\alpha)^2K_F m}{4} \beta\Big)} ,
\quad
F_W^{max} = (1-\alpha)K_F,
\quad
H_W^{max} = \dfrac{m_D}{m_W} S^{max}.
$$
In particular, this means that any solutions of equations \eqref{equation:HDFWHW} with initial condition in $\Omega$ are bounded and remains in $\Omega$.
\end{prop}
%
\begin{proof} 
To prove this proposition we will use the notion of invariant region, see \cite{smoller_shock_1994}. Before, we introduce the variable $S = H_D + e F_W$. 
%We have:
%
%\begin{equation}
%\dfrac{dS}{dt} = \cI + (f_D - \mu_D - m_D) \Big(S - eF_W \Big) + m_WH_W + e (1-\alpha)(1+\beta H_W)r_F  \left(1 - \dfrac{F_W}{(1-\alpha)K_F} \right) F_W.
%\end{equation}
With this new variable, the model writes:

\begin{equation}
\left\{ \begin{array}{l}
\dfrac{dS}{dt} = \cI + (f_D - \mu_D - m_D) \Big(S - eF_W \Big) + m_WH_W + e (1-\alpha)(1+\beta H_W)r_F  \left(1 - \dfrac{F_W}{(1-\alpha)K_F} \right) F_W, \\
\dfrac{dF_W}{dt} = (1-\alpha)(1+\beta H_W) r_F \left(1 - \dfrac{F_W}{K_F(1-\alpha)} \right) F_W - \lfw F_W H_W, \\
\dfrac{dH_W}{dt}= m_D \left(S - eF_W\right) - m_W H_W.
\end{array} \right.
\label{equation:SFWHW}
\end{equation}


We define the function $g(z)$, with $z=\Big(S, F_W, H_W \Big)$, as the right hand side of system \eqref{equation:SFWHW}. We also introduce the following functions:
$$
G_1(z) = S - S^{max},
\quad
G_2(z) = F_W - F_W^{max},
\quad
G_3(z) = H_W - H_W^{max}.
$$

Following \cite{smoller_shock_1994}, we will show that quantities $(\nabla G_1 \cdot g)|_{S = S^{max}}$, $(\nabla G_2 \cdot g)|_{F_W = F_W^{max}}$ and $(\nabla G_3 \cdot g)|_{H_W = H_W^{max}}$ are non-positive for $z \in \Omega_S = \Big\{ \Big(S, F_W, H_W \Big) \in (\mathbb{R})^3  \Big|S \leq S^{max}, F_W \leq F_W^{max}, H_W \leq H_W^{max} \Big\}$.

We have:
\begin{multline*}
(\nabla G_1 \cdot g)|_{S^{max}} = \cI + (f_D - \mu_D -m_D) \Big(S^{max} - eF_W \Big) +m_WH_W + e r_F(1-\alpha)(1+\beta H_W)  \left(1 - \dfrac{F_W}{(1-\alpha)K_F} \right) F_W.
\end{multline*}
Using that $\left(1-\dfrac{F_W}{(1-\alpha)K_F}\right) F_W \leq \dfrac{(1-\alpha)K_F}{4}$ for $0\leq F_W \leq (1-\alpha)K_F$ and $H_W \leq H_W^{max}$,

\begin{multline*}
(\nabla G_1 \cdot g)|_{S^{max}} \leq \cI + (f_D - \mu_D - m_D) S^{max} + m_W H_W^{max} - \Big(f_D - \mu_D \Big) e(1-\alpha)K_F + \\ er_F (1-\alpha)(1+\beta H_W^{max}) \dfrac{(1-\alpha)K_F}{4}.
\end{multline*}

Using $H_W^{max} = \dfrac{m_D}{m_W}S^{max}$,

\begin{align*}
(\nabla G_1 \cdot g)|_{S^{max}} &\leq \cI + (f_D - \mu_D) S^{max} - \Big(f_D - \mu_D \Big) e(1-\alpha)K_F +  er_F (1-\alpha)(1+\beta m S^{max}) \dfrac{(1-\alpha)K_F}{4}, \\
& \leq \cI + \Big(f_D - \mu_D + \dfrac{er_F (1-\alpha)^2K_F m}{4} \beta\Big) S^{max} + \left( \mu_D - f_D   +  \dfrac{(1-\alpha)r_F}{4}\right)e(1-\alpha)K_F, \\
& \leq \cI - \Big(\mu_D -f_D - \dfrac{er_F (1-\alpha)^2K_F m}{4} \beta\Big) S^{max} + \left( \mu_D - f_D   +  \dfrac{(1-\alpha)r_F}{4}\right)e(1-\alpha)K_F.
\end{align*}

Using the definition of $S^{max}$, we finally obtain $(\nabla G_1 \cdot g)|_{S = S^{max}} \leq 0$. 

The two others inequalities are straightforward to obtain. We have:
\begin{align*}
(\nabla G_2 \cdot g)|_{F_W = F_W^{max}} &= r_F  \left(1 - \dfrac{K_F (1-\alpha)}{K_F (1-\alpha)}\right)K_F (1-\alpha)  - \lfw H_W K_F (1-\alpha), \\
(\nabla G_2 \cdot g)|_{F_W = F_W^{max}} & = - \lfw H_W K_F (1-\alpha), \\
(\nabla G_2 \cdot g)|_{F_W = F_W^{max}} & \leq 0.
\end{align*}

The computations for $(\nabla G_3 \cdot g)|_{H_W = H_W^{max}}$ give:

\begin{align*}
(\nabla G_3 \cdot g)|_{H_W = H_W^{max}} &= m_D (S - eF_W) - m_W H_W^{max}, \\
(\nabla G_3 \cdot g)|_{H_W = H_W^{max}} &= m_D (S - S^{max} -  eF_W), \\
(\nabla G_3 \cdot g)|_{H_W = H_W^{max}} & \leq 0.
\end{align*}

We have shown that $(\nabla G_1 \cdot g)|_{S = S^{max}} \leq 0$, $(\nabla G_2 \cdot g)|_{F_W = F_W^{max}} \leq 0$ and $(\nabla G_3 \cdot g)|_{H_W = H_W^{max}} \leq 0$ in  $\Omega_S$.  According to \cite{smoller_shock_1994}, this proves that $\Omega_S$ is an invariant region for system \eqref{equation:SFWHW}.

This also shows that the set  $$\Big\{\Big(H_D, F_W, H_W \Big) \in \mathbb{R}^3  \Big|H_D + eF_W \leq S^{max}, F_W \leq F_W^{max}, H_W \leq H_W^{max} \Big\}$$ is invariant for system \eqref{equation:HDFWHW}. 


Moreover, for any point $y \in \partial (\mathbb{R}_+^3)$, the vector field defined by $f(y)$ is either tangent or directed inward. Then, $\Omega$ is an invariant region for equations \eqref{equation:HDFWHW}. 

\end{proof}

The next proposition shows that system \eqref{equation:HDFWHW} defines a dynamical system on $\Omega$.

\begin{prop}
System \eqref{equation:HDFWHW} defines a dynamical system on $\Omega$, that is, for any initial condition $(t_0, y_0)$ with $t_0 \in \mathbb{R}$ and $y_0 \in \Omega$, it exists a unique solution of equations \eqref{equation:HDFWHW}, and this solution is defined for all $t \geq t_0$.
\end{prop}

\begin{proof}
We proved that system \eqref{equation:HDFWHW} admits, at least locally, a unique solution for every initial condition. Moreover, since $\Omega$ is an invariant and compact region, the solutions with initial condition on $\Omega$ are bounded. Based on uniform boundedness, we deduce that solutions of system \eqref{equation:HDFWHW} with initial condition on $\Omega$ exist globally, for all $t\geq t_0$. Therefore, \eqref{equation:HDFWHW} defines a forward dynamical system on $\Omega$.
\end{proof}

\section{\vdeux{Quasi-Steady State Approach}} \label{sec:qssa}
\subsection{Derivation of the reduced system}

Considering the characteristic times of humans and wildlife, the time scale used all along the study will be year. However, the human displacements between the domestic and wild area are daily processes, and are faster than the demographic processes. In consequence, we write $$m_W = \dfrac{\mW}{\epsilon}, \quad m_D = \dfrac{\mD}{\epsilon}$$ where $\epsilon$ is the conversion parameter between year and day. 

Taking into consideration this difference of temporality of demographic and migration processes, one could used a quasi steady state approach to analyze the system \eqref{equation:HDFWHW}. In this purpose, we rewrite the system under the form:

\begin{equation}
\def\arraystretch{2}
\left\{ 
\begin{array}{l}
\dfrac{dH_D}{dt}= \cI + e\lfw F_WH_W + (f_D - \mu_D) H_D + \dfrac{1}{\epsilon}(\mW H_W - \mD H_D), \\
\dfrac{dF_W}{dt} = r_F(1- \alpha) (1+ \beta H_W) \left(1 - \dfrac{F_W}{K_F(1-\alpha)} \right) F_W - \lfw F_WH_W, \\
\dfrac{dH_W}{dt}= - \dfrac{1}{\epsilon}(\mW H_W - \mD H_D).
\end{array} \right.
\label{equation:HDFWHW, fast}
\end{equation}

\begin{prop}
For any initial condition $(H_{D,0}, F_{W, 0})$, the solution $(H_D, F_W)$  of the following system
\begin{equation}
\def\arraystretch{2}
\left\lbrace \begin{array}{l}
\dfrac{dH_D}{dt} = \dfrac{\cI}{1+ \dfrac{\mD}{\mW}} + \dfrac{f_D - \mu_D}{1+\dfrac{\mD}{\mW}} H_D + \dfrac{e}{1+\dfrac{\mD}{\mW}} \dfrac{\mD}{\mW} \lfw F_WH_D, \\
\dfrac{dF_W}{dt} = (1-\alpha) (1+\beta \dfrac{\mD}{\mW} H_D) r_F \left(1 - \dfrac{F_W}{(1-\alpha)K_F} \right) F_W - \dfrac{\mD}{\mW} \lfw F_W H_D, \\
H_W(t) = \dfrac{\mD}{\mW} H_D(t).
\end{array} \right.
\label{equation:HDFW}
\end{equation} is such that for any time $T > 0$, 

\begin{equation*}
\def\arraystretch{2}
\left\lbrace \begin{array}{l}
\lim\limits_{ \epsilon \rightarrow 0}{H_{D, \epsilon}(t)} = H_D(t), \quad t \in [0,T], \\
\lim\limits_{ \epsilon \rightarrow 0} F_{W,  \epsilon}(t) = F_W(t),\quad t \in [0,T], \\
 \lim\limits_{ \epsilon \rightarrow 0} H_{W,  \epsilon}(t) = \dfrac{\mD}{\mW} H_D(t), \quad  t\in (0,T] \\
\end{array} \right.
\end{equation*}
where $\Big(H_{D, \epsilon}, F_{W,  \epsilon}, H_{W,  \epsilon} \Big)$ is the solution of the system \eqref{equation:HDFWHW, fast}.

\end{prop}

\begin{proof}
We will apply the Tikhonov theorem (see theorem \ref{theorem:Tikhonov}, page \pageref{theorem:Tikhonov}) on system \eqref{equation:HDFWHW, fast}. Following \cite{banasiak_methods_2014} and to avoid division by zero, we rewrite the system with the variable $H = H_D + H_W$. It gives:

\begin{equation*} 
\def\arraystretch{2}
\left\{ 
\begin{array}{l}
\dfrac{dH}{dt}= \cI + e\lfw F_WH_W + (f_D - \mu_D)(H - H_W), \\
\dfrac{dF_W}{dt} = r_F(1- \alpha) (1+ \beta H_W) \left(1 - \dfrac{F_W}{K_F(1-\alpha)} \right) F_W - \lfw F_W H_W, \\
\epsilon \dfrac{dH_W}{dt}= m_D H - (m_D + m_W) H_W.
\end{array} \right.
\label{equation:HFWHW} 
\end{equation*}

This system 
is under the appropriate form to apply the Tikhonov's theorem with $x = (H, F_W)$, $y = H_W$,  

$$f(t,x,y,\epsilon) = \begin{bmatrix}
cI + e\lfw F_W H_W + (f_D - \mu_D) (H - H_W), \\
r_F(1- \alpha) (1+ \beta H_W) \left(1 - \dfrac{F_W}{K_F(1-\alpha)} \right) F_W - \lfw F_W H_W
\end{bmatrix}  $$
and $g(t,x,y,\epsilon) = \mW \Big(m H - (1 + m)H_W \Big) $.

The functions $f(t, \cdot, \cdot, \epsilon)$ and $g(t, \cdot, \cdot, \epsilon)$ are continuously differentiable in arbitrary $\mathcal{\bar{U}}$ and $\mathcal{V}$ intervals. Therefore, \vdeux{the first assumption of the Tikhonov's theorem} is satisfied. The equation $ g(t,x,y,0) =  0$ admits for solution $H_W = \dfrac{m}{1+m}H$. It is clearly continuous as a function of $t, H$ and isolated. \vdeux{The second assumption of Tikhonov's theorem} is also satisfied.

The auxiliary equation is given by
\begin{equation*}
\dfrac{d \tilde{H_W}}{d \tau} = \mW \Big(m H(t) - (1 + m)\tilde{H_W} \Big)
\end{equation*}

for which the fixed point $\tilde{H_W} = \dfrac{m}{1+m}H(t)$ is globally uniformly asymptotically stable with respect to $H$ and $t$. Therefore, \vdeux{the third and fifth assumptions of Tikhonov's theorem} hold true.

The reduced equation writes:
\begin{equation} \label{equationHFW}
\def\arraystretch{2}
\left\lbrace \begin{array}{l}
\dfrac{dH}{dt}= \cI + e\lfw F_W\dfrac{m}{1+m}H + (f_D - \mu_D)\dfrac{H}{1+m}, \\
\dfrac{dF_W}{dt} = r_F(1- \alpha) (1+ \beta \dfrac{m}{1+m}H ) \left(1 - \dfrac{F_W}{K_F(1-\alpha)} \right) F_W - \lfw F_W \dfrac{m}{1+m}H.  \\
\end{array} \right.
\end{equation}

Using the derivability of the right hand side on $\Omega$, and the fact that those derivatives are bounded on $\Omega$, it is straightforward to show that the fourth assumption of Tikhonov's theorem is satisfied.

%\sout{
%\marc{We see that the assumption (4) of theorem \ref{theorem:Tikhonov} is satisfied.}} \sout{ \YD{"We see"? On voit? Non, en Maths, on ne voit pas, on démontre.... Dans l'hypothèse 4, on demande une condition de Lipschitz que tu n'as pas montré dans le théorème d'existence. Or là, il me semble que si pour $\theta>0$ on devrait être capable de prouver cette condition de Lipschitz, ce n'est pas sur que ce soit possible pour $\theta=0$. Merci de revoir cela très précisemment.... }}
%\marc{La fonction est lipschitzienne sur $\Omega$ par rapport à chacune de ses variables $F_W$ et $H$ (car dérivable et de dérivée bornée sur l'intervalle considéré), elle est donc aussi lipschitzienne par rapport au produit $F_W \times H$.}


Consequently, we can claim that the solutions 
$\Big(H_{ \epsilon}, F_{W,  \epsilon}, H_{W,  \epsilon} \Big)$  of system \eqref{degenerateSystem} satisfy:
\begin{equation*}
\def\arraystretch{2}
\left\lbrace \begin{array}{l}
\lim\limits_{ \epsilon \rightarrow 0}{H_{ \epsilon}(t)} = H(t), \quad t \in [0,T], \\
\lim\limits_{ \epsilon \rightarrow 0} F_{W,  \epsilon}(t) = F_W(t),\quad t \in [0,T], \\
 \lim\limits_{ \epsilon \rightarrow 0} H_{W,  \epsilon}(t) = \dfrac{m}{1+m}H(t), \quad  t\in (0,T], \\
\end{array} \right.
\end{equation*}
for any $T > 0$, where $(H, F_W)$ are the solutions to \eqref{equationHFW}. 

\medskip
Finally, we come back to the original variables that are $H_D, F_W$ and $H_W$. Using $H = H_D + H_W$ and $H_W = \dfrac{m}{1+m}H$, we obtain $H_W = m H_D$, and $H_D = \dfrac{1}{1+m}	H$. The system becomes:

\begin{equation*}
\def\arraystretch{2}
\left\lbrace \begin{array}{l}
\dfrac{dH_D}{dt} = \dfrac{\cI}{1+m} + \dfrac{f_D - \mu_D}{1+m} H_D + \dfrac{e}{1+m} m \lfw F_W H_D, \\
\dfrac{dF_W}{dt} = (1-\alpha) (1+\beta m H_D) r_F \left(1 - \dfrac{F_W}{(1-\alpha)K_F} \right) F_W - m \lfw F_WH_D, \\
H_W(t) = m H_D(t).
\end{array} \right.
\end{equation*}

\end{proof}

%On the following, in order to clarify the computations, we redefine $\cI = \dfrac{\cI}{1 + m}$, $f_D = \dfrac{f_D}{1+ m}$, $\mu_D = \dfrac{\mu_D}{1+ m}$ and $e = \dfrac{e}{1+m}$. The system under study is therefore:
%
%\begin{equation}
%\def\arraystretch{2}
%\left\lbrace \begin{array}{l}
%\dfrac{dH_D}{dt} = \cI + (f_D - \mu_D) H_D + e  m \lfw F_W H_D \\
%\dfrac{dF_W}{dt} = (1-\alpha) (1+\beta m H_D) r_F \left(1 - \dfrac{F_W}{(1-\alpha)K_F} \right) F_W - m \lfw F_W H_D 
%\end{array} \right.
%\label{equation:HDFW}
%\end{equation}

The following parts are dedicated to the study of the long term dynamics of system \eqref{equation:HDFW}. The study is separated in two cases: a first one when there is no immigration ($\cI = 0$), and a second one when $\cI > 0$.

\subsection{\vdeux{Local} analysis without immigration \YD{migration?}}

On the following, we study the existence and stability of equilibrium of model \eqref{equation:HDFW}, when $\cI = 0$.

\begin{prop}
\label{prop:equilibre, cI=0}
When $\cI = 0$, the following results hold true.
\begin{itemize}
\item The system \eqref{equation:HDFW} admits a trivial equilibrium $TE = \Big(0,0\Big)$ and a fauna-only equilibrium $EE^{F_W} = \Big(0, (1-\alpha)K_F \Big)$ that always exist.

\item When
$$
\mathcal{N}_{\cI = 0} := \dfrac{m e \lfw (1-\alpha)K_F}{\mu_D - f_D} >1,
$$ 
then the system \eqref{equation:HDFW} admits a unique coexistence equilibrium $EE^{HF_W}_{\cI = 0} = \Big(H^*_{D, \cI = 0}, F^*_{W, \cI = 0}\Big)$ \\ 
where 


$$F^*_{W, \cI = 0} = \dfrac{\mu_D - f_D}{\lfw m e},
\quad 
H^*_{D, \cI = 0} = \dfrac{(1-\alpha)r_F\Big(1 - \dfrac{F^*_{W, \cI = 0}}{K_F(1-\alpha)} \Big)}{m\left(\lfw - \beta (1-\alpha) r_F + \beta r_F  \dfrac{F^*_{W, \cI = 0}}{K_F}\right)}.
$$
\end{itemize}
\end{prop}

\begin{remark}\label{remark:existence1}
Straightforward observations show that the equilibrium of the non reduced system \eqref{equation:HDFWHW} are the same than the ones of system \eqref{equation:HDFW}, with the same condition and with $H_W^* = m H_D^*$.
\end{remark}

\begin{proof}
To derive the equilibrium of system \eqref{equation:HDFW}, we equate the rates of change of the variables to 0. Therefore, an equilibrium satisfies the system of equations:
\begin{equation}\label{equation:system-equilibre, cI=0}
\def\arraystretch{2}
\left\lbrace \begin{array}{cll}
 e \lfw m F_W^* + f_D - \mu_D = 0& \mbox{or} & H_D^* = 0,\\
m H_D^*\Big(\lfw - (1-\alpha)r_F \beta + \dfrac{r_F \beta}{K_F}F_W^* \Big) + r_F \dfrac{F_W^*}{K_F} - (1-\alpha)r_F= 0& \mbox{or} & F^*_W = 0.
\end{array} \right.
\end{equation}
When $H_D^*=0$ and $F_W^*=0$, we recover the trivial equilibrium $TE = \Big(0,0\Big)$. When $H_D^*=0$ and $F_W^*\neq0$, we obtain the fauna-only equilibrium $EE^{F_W} = \Big(0, K_F(1-\alpha)\Big)$. Finally, when $H_D^*\neq0$ and $F_W^*\neq0$, direct computations lead to a unique set of values given by:
$$F^*_{W} = \dfrac{\mu_D - f_D}{\lfw m e},
\quad 
H^*_{D} = \dfrac{(1-\alpha)r_F\Big(1 - \dfrac{F^*_{W}}{K_F(1-\alpha)} \Big)}{m\left(\lfw - \beta (1-\alpha) r_F + \beta r_F  \dfrac{F^*_{W}}{K_F}\right)} ,
\quad 
H^*_{W} = m H^*_{D}.$$

Those values are biologically meaningful if $F_W^* \leq (1-\alpha) K_F$ and if $H_D^*$ is positive. The first inequality gives the constraint $\dfrac{\mu_D - f_D}{\lfw m e} \leq (1-\alpha)K_F$. From now, we assume it. The numerator of $H^*_{D}$ is non negative, and positive when $\dfrac{\mu_D - f_D}{\lfw m e} < (1-\alpha)K_F$. We need to check the sign of its denominator, which has to be positive. We have:

\begin{align*}
\lfw - \beta (1-\alpha) r_F + \beta r_F  \dfrac{F^*_{W}}{K_F} &= \lfw\Big(1 - \dfrac{\beta (1-\alpha) r_F}{\lfw} + \beta r_F  \dfrac{\mu_D - f_D}{\lfw^2 m e K_F} \Big), \\
&= \lfw\left(1 - \dfrac{\beta (1-\alpha) r_F}{\lfw}\Big(1 -\dfrac{\mu_D - f_D}{\lfw m e K_F(1-\alpha)} \Big) \right), \\
&\geq 0,
\end{align*}

thanks to proposition \ref{propBeta}. Therefore, the equilibrium of coexistence is biologically meaningful if $\dfrac{\mu_D - f_D}{\lfw m e} < (1-\alpha)K_F \Leftrightarrow 1 < \dfrac{\lfw (1-\alpha)K_F m e}{\mu_D - f_D}$.

\end{proof}


The following proposition assesses the local asymptotic stability of the equilibrium.

\begin{prop}\label{prop:stab 2D, cI=0}
When $\cI =0$, the following results are valid.
\begin{itemize}
\item The trivial equilibrium $TE$ is unstable.
\item The fauna equilibrium $EE^{HF_W}_{\cI = 0}$ is LAS if $\mathcal{N}_{\cI = 0} < 1$.
\item When $\mathcal{N}_{\cI = 0} > 1$, the coexistence equilibrium $EE^{HF_W}$ exists. It is LAS without condition.
\end{itemize}
\end{prop}

\begin{proof}
To investigate the local stability of the equilibrium of the system \eqref{equation:HDFW}, we first compute the Jacobian matrix of the system. It is given by:

\begin{multline}
\mathcal{J}_{QSSA}(H_D, F_W) = \\ \begin{bmatrix}
- \dfrac{\mu_D - f_D}{1+m} + \dfrac{e \lfw m}{1+m}  F_W & \dfrac{e \lfw m}{1+m}  H_D \\
m\left(-\lfw + \beta (1-\alpha) r_F \Big(1- \dfrac{F_W}{(1-\alpha)K_F} \Big) \right) F_W & (1-\alpha) (1+\beta m H_D) r_F \left(1 - \dfrac{2F_W}{K_F} \right) - \lfw m H_D
\end{bmatrix}.
\label{equation:Jqssa}
\end{multline}

\begin{itemize}
\item At equilibrium $TE$, the Jacobian is given by:
\begin{equation*}
\mathcal{J}_{QSSA}(TE) = \begin{bmatrix}
- \dfrac{\mu_D - f_D}{1+m} &0 \\
0 & (1-\alpha)  r_F 
\end{bmatrix}.
\end{equation*}
$(1-\alpha) r_F > 0$ is a positive eigenvalue, an therefore $TE$ is unstable.

\item At equilibrium $EE^{F_W}$, the Jacobian is given by: 
\begin{equation*}
\mathcal{J}_{QSSA}(EE^{F_W}) = \begin{bmatrix}
- \dfrac{\mu_D - f_D}{1+m} + \dfrac{e}{1+m}\lfw m K_F(1-\alpha) &0 \\
- m \lfw (1-\alpha)K_F & -(1-\alpha)  r_F 
\end{bmatrix}.
\end{equation*}
The eigenvalues are $-(1-\alpha)  r_F$ and $- \dfrac{\mu_D - f_D}{1+m} + \dfrac{e}{1+m}\lfw m K_F(1-\alpha)$. They are both negative if $-(\mu_D - f_D) + e\lfw m K_F(1-\alpha) <0 \Leftrightarrow \mathcal{N}_{\cI = 0} < 1$.

\item At equilibrium $EE^{HF_W}_{\cI = 0}$, the Jacobian is given by:

\begin{multline*}
\mathcal{J}_{QSSA}(EE^{HF_W}_{\cI = 0}) = \\
% \\&\begin{bmatrix}
%- \dfrac{\mu_D - f_D}{1+m}  + \dfrac{e}{1+m} \lfw m F^*_{W, \cI = 0} & \dfrac{e}{1+m}\lfw m H^*_{D, \cI = 0} \\
%m\left(-\lfw + \beta (1-\alpha) r_F \Big(1- \dfrac{F^*_{W, \cI = 0}}{(1-\alpha)K_F} \Big) \right) F^*_{W, \cI = 0} & (1-\alpha) (1+\beta m H_D) r_F \left(1 - \dfrac{2F^*_{W, \cI = 0}}{K_F} \right) - \lfw m H^*_{D, \cI = 0}
%\end{bmatrix} \\
\begin{bmatrix}
0 & \dfrac{e}{1+m} \lfw m H^*_{D, \cI = 0} \\
m\left(-\lfw + \beta (1-\alpha) r_F \Big(1- \dfrac{F^*_{W, \cI = 0}}{(1-\alpha)K_F} \Big) \right) F^*_{W, \cI = 0} & -(1+\beta m H^*_{D, \cI = 0}) r_F \dfrac{F^*_{W, \cI = 0}}{K_F} 
\end{bmatrix}.
\end{multline*}

using equilibrium conditions. According to \vdeux{the Routh-Hurwitz criterion} (see theorem \ref{theorem:Routh-Hurwitz}, page \pageref{theorem:Routh-Hurwitz} ),  $EE^{HF_W}_{\cI = 0}$ is LAS if the trace of $\mathcal{J}_{QSSA}(EE^{HF_W}_{\cI = 0}) $ is negative and its determinant positive. We have:

\begin{equation*}
\Tr(\mathcal{J}_{QSSA}(EE^{HF_W}_{\cI = 0})) = -(1+\beta m H_D) r_F \dfrac{F_W}{K_F}
\end{equation*}
that is $\Tr(\mathcal{J}_{QSSA}(EE^{HF_W}_{\cI = 0})) < 0$.

The determinant is given by:

\begin{align*}
\det(\mathcal{J}_{QSSA}(EE^{HF_W}_{\cI = 0})) &=  \dfrac{- m^2 e}{1+m} \lfw \left(-\lfw + \beta (1-\alpha) r_F \Big(1- \dfrac{F^*_{W, \cI = 0}}{(1-\alpha)K_F} \Big) \right) F^*_{W, \cI = 0} H^*_{D, \cI = 0}, \\
&= \dfrac{m^2 e}{1+m} \lfw^2 \left(1 + \dfrac{\beta (1-\alpha) r_F}{\lfw} \Big(1- \dfrac{\mu_D - f_D}{me \lfw(1-\alpha)K_F} \Big) \right) F^*_{W, \cI = 0} H^*_{D, \cI = 0},
\end{align*}

using the equilibrium value. Using the proposition \ref{propBeta}, page \pageref{propBeta}, we can show that this last expression is positive. Consequently $\det(\mathcal{J}_{QSSA}(EE^{HF_W}_{\cI = 0})) > 0$ and $EE^{HF_W}_{\cI = 0}$ is LAS without condition.
\end{itemize}
\end{proof}

\subsection{\vdeux{Local} analysis with immigration}
On the following, we study the existence and local stability of equilibrium of model \eqref{equation:HDFW}, when $\cI > 0$.

\begin{prop}\label{prop:eq, cI>0}
When $\cI > 0$, the following results hold true.
\begin{itemize}
\item The system \eqref{equation:HDFW} admits a human equilibrium $EE^{H} =\Big(\dfrac{\cI}{\mu_D - f_D}, 0 \Big)$ that always exists.
\item When 
$$ \mathcal{N}_{\cI >0} :=  \dfrac{r_F(1-\alpha)\Big({\dfrac{\mu_D - f_D}{m\cI}+\beta\Big)}}{\lfw}  > 1,$$
system \eqref{equation:HDFW} has a unique coexistence equilibrium $EE^{HF_W}_{\cI > 0} = \Big(H^*_{D, \cI > 0}, F^*_{W, \cI > 0}\Big)$
where
$$F^*_{W, \cI > 0} = \dfrac{(1-\alpha)K_F}{2}\left(1 - \dfrac{\sqrt{\Delta_F}}{e(1-\alpha)r_F}\right) + \dfrac{\mu_D - f_D + \cI \beta m}{2\lfw m e},$$
$$
H^*_{D, \cI > 0} = \dfrac{(1-\alpha)r_F\Big(1 - \dfrac{F^*_{W, \cI > 0}}{(1-\alpha)K_F} \Big)}{m\left(\lfw - \beta (1-\alpha) r_F + \beta r_F  \dfrac{F^*_{W, \cI > 0}}{K_F}\right)}
$$
and
\begin{multline*}
\Delta_F = \left(e(1-\alpha)r_F - \dfrac{(\mu_D - f_D) r_F}{\lfw m K_F}\right)^2 + \dfrac{\cI \beta r_F}{\lfw K_F} \left(\dfrac{\cI \beta r_F}{\lfw K_F} + 2\dfrac{(\mu_D - f_D) r_F}{\lfw m K_F} + 2e(1-\alpha)r_F \right) + \\ 4\dfrac{er_F}{K_F}  \cI\Big(1 - \dfrac{(1-\alpha)\beta r_F}{\lfw} \Big)
\end{multline*}
\end{itemize} 
\end{prop}
%\begin{remark} \label{remark:existence2}
%Straightforward observations show that the equilibrium of the non reduced system \eqref{equation:HDFWHW} are the same than the ones of system \eqref{equation:HDFW}, with the same condition and with $H_W^* = m H_D^*$.
%\end{remark}

\begin{proof}
An equilibrium of system \eqref{equation:HDFWHW} satisfies the system of equations:
\begin{equation}\label{systemEquilibre}
\left\lbrace \begin{array}{cll}
\cI + e \lfw m F_W^* H_D^* + (f_D - \mu_D) H_D^* = 0,&&\\
F_W^* - \dfrac{(1-\alpha)K_F}{1 + \beta m H_D^*} \Big(1 - \dfrac{m(\lfw - (1-\alpha)\beta r_F) H^*_D}{(1-\alpha)r_F} \Big) = 0& \mbox{or} & F^*_W = 0.
\end{array} \right.
\end{equation}

The solution of system \eqref{systemEquilibre} when $F_W^* = 0$ is the Human-only equilibrium $EE^{H} = \Big(\dfrac{\cI}{\mu_D - f_D}, 0\Big)$.
In the sequel, we assume that $F_W^* > 0$. In this case, $F^*_W$ is solution of the quadratic equation
\begin{equation}
P_F(X) := X^2 \left(\dfrac{er_F}{K_F} \right) - X \left(e(1-\alpha)r_F + \dfrac{(\mu_D - f_D) r_F}{\lfw m K_F} + \dfrac{\cI \beta r_F}{\lfw K_F} \right) + \left(\dfrac{(\mu_D - f_D)(1-\alpha) r_F}{\lfw m} - \cI\Big(1 - \dfrac{(1-\alpha)\beta r_F}{\lfw} \Big) \right) = 0.
\end{equation}

The polynomial $P_F$ is studied in appendix \ref{section:study of PF}. In particular, we show that $P_F$ admits two real roots $F_1^* \leq F_2^*$, with $F_2^* > K_F(1- \alpha) > F_1^*$.

To define an equilibrium, $F^*_W$ must be biologically meaningful, that is be positive and lower than $(1-\alpha) K_F$. Therefore, $F_2^*$ is not biologically meaningful and $F_1^*$ is an equilibrium only if it is positive, \textit{ie} only if  $\dfrac{(\mu_D - f_D) r_F}{\lfw m } > \cI\Big(1 - \dfrac{(1-\alpha)\beta r_F}{\lfw} \Big)$ (see proposition \ref{prop:study of PF}).  $F_1^*$ is given by:
$$F^*_1 = \dfrac{(1-\alpha)K_F}{2}\left(1 - \dfrac{\sqrt{\Delta_F}}{e(1-\alpha)r_F}\right) + \dfrac{\mu_D - f_D + \cI \beta m}{2\lfw m e}.$$

According to the first equation of system \eqref{systemEquilibre}, the value of $H_D^*$ at equilibrium is given by:

$$
H_D^* = \dfrac{\cI}{\mu_D - f_D - e \lfw m F_1^*}.
$$

It is biologically meaningful if it is positive. Since $F_1^* < \dfrac{\mu_D - f_D}{e \lfw m}$ (proposition \ref{prop:study of PF}), it is the case. Finally, the coexistence equilibrium exists if $\dfrac{(\mu_D - f_D) r_F}{\lfw m } > \cI\Big(1 - \dfrac{(1-\alpha)\beta r_F}{\lfw} \Big)$.
\end{proof}

The following proposition assesses the local asymptotic stability of the equilibrium.

\begin{prop} \label{prop:stab 2D, cI>0}
When $\cI > 0$, the following results are valid.
\begin{itemize}
\item The human equilibrium $EE^{H}$ is LAS if $\mathcal{N}_{\cI > 0} < 1$.
\item When $\mathcal{N}_{\cI > 0} > 1$, the coexistence equilibrium $EE^{HF_W}_{\cI > 0}$ exists. It is LAS without condition.
\end{itemize}
\end{prop}


\begin{proof}
To investigate the local stability of the equilibrium of the system \eqref{equation:HDFW}, we first compute the Jacobian matrix of the system, computed at equation \eqref{equation:Jqssa}.

\begin{itemize}
\item At equilibrium $EE^{H}$, the Jacobian is given by:
\begin{equation*}
\mathcal{J}_{QSSA}(EE^{H}) = \begin{bmatrix}
-\dfrac{\mu_D - f_D}{1 + m} &  \dfrac{e \lfw m \cI}{(1+m)\mu_D - f_D} \\
0 & (1-\alpha)\Big(1+ \dfrac{m \beta \cI}{\mu_D - f_D}\Big)  r_F -  \dfrac{\lfw m  \cI}{\mu_D - f_D}.
\end{bmatrix}
\end{equation*}
The eigenvalues are $-\dfrac{\mu_D - f_D}{1 + m} $ and $(1-\alpha)\Big(1+ \dfrac{\cI}{\mu_D - f_D}\Big)  r_F -  \dfrac{\lfw m  \cI}{\mu_D - f_D}$. They are both negative if $(1-\alpha)\Big(1+ \dfrac{m \beta \cI}{\mu_D - f_D}\Big)  r_F -  \dfrac{\lfw m  \cI}{\mu_D - f_D} <0 \Leftrightarrow \mathcal{N}_{\cI > 0} < 1$.


\item At equilibrium $EE^{HF_W}_{\cI > 0}$, the Jacobian is given by:

\begin{multline*}
\mathcal{J}_{QSSA}(EE^{HF_W}_{\cI > 0}) \\= \begin{bmatrix}
-\dfrac{\cI}{H^*_{D, \cI > 0}} & e \lfw m H^*_{D, \cI > 0} \\
m\left(-\lfw + \beta (1-\alpha) r_F \Big(1- \dfrac{F^*_{W, \cI > 0}}{(1-\alpha)K_F} \Big) \right) F^*_{W, \cI > 0} & -(1+\beta m H_D) r_F \dfrac{F^*_{W, \cI > 0}}{K_F} 
\end{bmatrix}
\end{multline*}

using equilibrium conditions.  According to \vdeux{the Routh-Hurwitz criterion (theorem \ref{theorem:Routh-Hurwitz}, page \pageref{theorem:Routh-Hurwitz})},  $EE^{HF_W}_{\cI > 0}$ is LAS if the trace of $\mathcal{J}_{QSSA}(EE^{HF_W}_{\cI > 0}) $ is negative and its determinant positive. We have:

\begin{equation*}
\Tr(\mathcal{J}_{QSSA}(EE^{HF_W}_{\cI > 0})) = -\dfrac{\cI}{H^*_{D, \cI > 0}} -(1+\beta m H^*_{D, \cI > 0}) r_F \dfrac{F^*_{W, \cI > 0}}{K_F}, 
\end{equation*}
that is $\Tr(\mathcal{J}_{QSSA}(EE^{HF_W}_{\cI > 0})) < 0$.
The determinant is:

\begin{align*}
\det(\mathcal{J}_{QSSA}(EE^{HF_W}_{\cI > 0})) &= \dfrac{\cI}{H_D^*} \dfrac{1 + \beta m H_D}{K_F} r_F F_W^* - m^2 e \lfw \left(-\lfw + \beta(1-\alpha)r_F \Big(1- \dfrac{F_W^*}{(1-\alpha) K_F} \Big) \right) F_W^* H_D^*, \\
&= \dfrac{\cI}{H_D^*} \dfrac{1 + \beta m H_D^*}{K_F} r_F F_W^* + m^2 e \lfw \left(\lfw - \beta(1-\alpha)r_F + r_F \beta\dfrac{F_W^*}{ K_F} \Big) \right) F_W^* H_D^*.
\end{align*}

According to the proposition \ref{prop:study of PF} page \pageref{prop:study of PF}, we know that $(1-\alpha)K_F - \dfrac{\lfw K_F}{\beta r_F} < F_W^*$. Consequently $\det(\mathcal{J}_{QSSA}(EE^{HF_W}_{\cI > 0})) > 0$ and $EE^{HF_W}_{\cI > 0}$ is LAS without condition.
\end{itemize}
\end{proof}

\subsection{Global stability results}
This subsection is valid for $\cI \geq 0$. On it, we show that the equilibrium which are LAS are GAS under the same conditions, and that the system \eqref{equation:HDFW} does not admit any limit cycle. 

\begin{prop} \label{prop:no limit cycle, 2D}
System \eqref{equation:HDFW} admits no limit cycle on $\Omega$.
\end{prop}

\begin{proof}
We will use the Bendixson-Dulac criterion (see theorem \ref{theorem:Dulac}, page \pageref{theorem:Dulac}) with the function $\phi(H_D, F_W) = \dfrac{1}{H_D F_W}$ to show the result. We note $f$ the right hand side of equations \eqref{equation:HDFW}. We have:

\begin{equation*}
(\phi \times f_1)(H_D, F_W) = \dfrac{\cI}{(1+m)H_D F_W} -\dfrac{\mu_D - f_D}{F_W} - \dfrac{e}{1+m}\lfw m
\end{equation*} and therefore

\begin{equation*}
\dfrac{\partial (\phi f_1)}{\partial H_D}(H_D, F_W) = - \dfrac{\cI}{(1+m)F_W H_D^2} \leq 0.
\end{equation*}

On the other hand, we also have:
\begin{equation*}
(\phi \times f_2)(H_D, F_W) = - m \lfw + \dfrac{(1-\alpha) (1+ \beta m H_D) r_F}{H_D} - \dfrac{(1+\beta m H_D) r_F}{H_D} \dfrac{F_W}{K_F}
\end{equation*} and therefore

\begin{equation*}
\dfrac{\partial (\phi f_2)}{\partial F_W}(H_D, F_W) = - \dfrac{(1+\beta m H_D) r_F}{H_D K_F} <0.
\end{equation*}

Consequently, $\dfrac{\partial (\phi f_1)}{\partial H_D} + \dfrac{\partial (\phi f_2)}{\partial F_W} < 0$ and according to the Bendixson-Dulac criterion, the system \eqref{equation:HDFW} does not admit any limit cycle.

\end{proof}


\begin{prop} \label{prop:GAS, 2D}
Under the conditions of propositions \ref{prop:stab 2D, cI=0} and \ref{prop:stab 2D, cI>0}, the equilibrium which are LAS are GAS in $\Omega$.
\end{prop}


\begin{proof}
According to proposition \ref{prop:no limit cycle, 2D}, the system \eqref{equation:HDFW} admits no limit cycle in $\Omega$. Therefore, according to \vdeux{the Poincaré-Bendixon's theorem (see theorem \ref{theorem:Poincaré-Bendixson}, page \pageref{theorem:Poincaré-Bendixson})}, the only possible dynamic for the system in $\Omega$, is to converge towards a stable fixed point. 
\end{proof}

\subsection{Numerical Comparison between the Non-Reduced and the Reduced Systems}
\YD{a ré-écrire... on sent la traduction à plein nez.... Utilise des outils comme DeepL qui permette d'améliorer les phrases "franglaises" voire même proposer de très bonnes traductions. La qualité et la clarté de l'écriture sont fondamentales dans l'appréciation d'un article. Il faut donc y faire attention et y passer le temps suffisant.}
\marc{Version originale commentée, version réécrite dessous}

%In this section, we search to evaluate the quality of the quasi-steady state approximation we did to obtain the reduced system \eqref{equation:HDFW} from the original system \eqref{equation:HDFWHW}. In this purpose, we provide the following numerical simulations. The value of the parameters are indicated in legend, and in this section, were chosen for figure's readability.

%First, we can note that the parameters' values used on these figures give $\mathcal{N}_{I=0} = 287.17 > 1$. Therefore, according to proposition \ref{prop:GAS, 2D}, $EE^{HF_W}$ is GAS for system \eqref{equation:HDFW}. We indeed see that the system orbit, represented in dashed orange converges towards it. 
%
%When we look for the impact of $\epsilon$, which is the conversion parameter between year (demographic time scale) and day (displacement forest-village time scale), we can notice that when $\epsilon$ is small enough (for example $= 10^{-3}$), the original system also converges towards $EE^{HF_W}$, and that the two orbits are close to each others.
%However, for highest value of $\epsilon$ ($\epsilon = 10^{-1}, \dfrac{1}{365}$), the original system converges towards a limit cycle around $EE^{HF_W}$.
%
%This difference of behavior does not contradict proposition \ref{prop: equivalentSystem} (which is valid for $\epsilon \rightarrow 0$) but raises questions about the pertinence of the approximation, since $\epsilon = \dfrac{1}{365}$ is precisely the value we should use. 
%
%\vdeux{Therefore, we propose in the next section an analysis of the 3D system, using an equivalent competitive system.}

\vdeux{In this section, we seek to evaluate the quality of the quasi-steady state approximation we did. To do this, we compare numerical simulations of the original system \eqref{equation:HDFWHW} and of its approximation, the system \eqref{equation:HDFW}.}

\begin{figure}[!ht]
\centering
\begin{subfigure}{0.45\textwidth}
\centering
\includegraphics[width=1\textwidth]{PhasePlane2Dv3.png}
\caption{}
\end{subfigure}
\begin{subfigure}{0.45\textwidth}
\centering
\includegraphics[width=1\textwidth]{PhasePlane1e1v4.png}
\caption{}
\end{subfigure}
\begin{subfigure}{0.45\textwidth}
\centering
\includegraphics[width=1\textwidth]{PhasePlane365v4.png}
\caption{}
\end{subfigure}
\begin{subfigure}{0.45\textwidth}
\centering
\includegraphics[width=1\textwidth]{PhasePlane1e3v4.png}
\caption{}
\end{subfigure}
\caption{Orbits of system \eqref{equation:HDFW} and \eqref{equation:HDFWHW} for $\epsilon = 0.1$, $\epsilon = 1/365$ and $\epsilon = 1e^{-3}$ in the $H_D - F_W$ plane. In the two first cases, the 3D system \eqref{equation:HDFWHW} converges towards a limit cycle while for $\epsilon = 1e^{-3}$ it converges towards $EE^{HF_W}$, as the reduced system \eqref{equation:HDFWHW}. \\
Parameters values: $\cI = 0$, $\beta = 0$, $r_F = 3.35$, $K_F = 22725$, $\alpha = 0.4$, $\lfw = 0.04$, $e = 3$, $\mu_D = 0.2$, $f_D = 0.0164$, $\mD = 0.021$, $\mW = 0.067$. \\
\vdeux{Correction des légendes: $1e^{-1}$ remplacé par $10^{-1}$}}
\label{fig:comparison 2D-3D}
\end{figure}


\vdeux{
The values of the parameters, indicated in legend, were chosen for figure's readability. They give $\mathcal{N}_{I=0} = 287.17 > 1$. Therefore, according to proposition \ref{prop:GAS, 2D}, the equilibrium $EE^{HF_W}$ is GAS for system \eqref{equation:HDFW}. That is what we see in the figures \ref{fig:comparison 2D-3D}, where the solution of the system \eqref{equation:HDFW}, represented by the dashed orange line, converges towards $EE^{HF_W}$. }

\vdeux{
The figures also illustrate the importance of the value of $\epsilon$, which is the conversion parameter between the demographic time scale (year) and the forest-village travel time scale (day). When $\epsilon$ is small enough (for example $= 10^{-3}$), the original system also converges towards $EE^{HF_W}$, and the two orbits are close to each others.
However, for highest value of $\epsilon$ ($\epsilon = 10^{-1}, \dfrac{1}{365}$), the solution of the original system converges towards a limit cycle around $EE^{HF_W}$.
}

This difference of behavior does not contradict proposition \ref{prop: equivalentSystem} (which is valid for $\epsilon \rightarrow 0$) but raises questions about the pertinence of the approximation, since $\epsilon = \dfrac{1}{365}$ is precisely the value we should use. 

\vdeux{Therefore, we propose in the next section an analysis of the 3D system, using an equivalent competitive system.}

\section{\vdeux{Analysis of an equivalent competitive system}} \label{sec:competitive}
The quasi-steady state approach was a tentative to provide a full and simple analysis of our model. However, as seen above, it does not allow to recover the complete dynamic of the original system \eqref{equation:HDFWHW}. In consequence, we provide in this section an analysis of the original system \eqref{equation:HDFWHW}, using the theory of monotone system (see definition \ref{def:monotone}, page \pageref{def:monotone}). To this end, we work on a equivalent system, which is competitive. This characteristic will allow us to apply the theorem \ref{theorem:Zhu}, page \pageref{theorem:Zhu} \YD{(see \cite{zhu_stable_1994})}.

\subsection{\vdeux{Derivation of the equivalent competitive system}}
On this section we derive an equivalent system to system \eqref{equation:HDFWHW} by a change of variable, and state some of its properties.  
\begin{prop} \label{prop: equivalentSystem}
The system \eqref{equation:HDFWHW} is equivalent to:
\begin{equation}
\def\arraystretch{2}
\left\{ \begin{array}{l}
\dfrac{dh_D}{dt}= \cI + e\lfw h_W f_W + (f_D - \mu_D) h_D - m_D h_D - m_W h_W, \\
\dfrac{df_W}{dt} = (1-\alpha)(1 - \beta h_W) r_F \left(1 + \dfrac{f_W}{K_F(1-\alpha)} \right) f_W + \lfw f_W h_W, \\
\dfrac{dh_W}{dt}= -m_D h_D - m_W h_W. 
\end{array} \right.
\label{equation:hdfwhw}
\end{equation}
We note $y_{eq} = (h_D, f_W, h_W)$ the equivalent variable and $f_{eq}(y_{eq})$ the right hand side of \eqref{equation:hdfwhw} and \vdeux{
$$
\Omega_{eq} = \Big\{\Big(h_D, f_W, h_W \Big)  \Big|h_D - ef_W \leq S^{max}, 0 \leq h_D,  -F_W^{max} \leq f_W \leq 0, -H_W^{max} \leq H_W \leq 0 \Big\}
$$ the domain of interest. }
This equivalent system is irreducible and dissipative on $\Omega_{eq}$, and
\begin{itemize}
\item if $\lfw - (1-\alpha)\beta r_F \geq 0$, it is competitive on $\mathcal{D} = \R_+ \times \R_-^2$,
\item if $\lfw - (1-\alpha)\beta r_F < 0$, it is competitive on $$\Big\{(h_D, f_W, h_W) | 0 \leq h_D, f_W \leq K_F\big(\dfrac{\lfw}{\beta r_F}-(1-\alpha)\big), h_W \leq 0 \Big\}.$$ 
\end{itemize}

\end{prop}

\begin{proof}
Following \cite{wang_predator-prey_1997}, we do the following change of variable: 

\begin{equation}\label{equation:change of variable}
h_D =  H_D, \quad f_W = -F_W, \quad h_W = -H_W.
\end{equation} The system \eqref{equation:HDFWHW} is transformed into:

\begin{equation}
\def\arraystretch{2}
\left\{ \begin{array}{l}
\dfrac{dh_D}{dt}= \cI + e\lfw h_W f_W + (f_D - \mu_D) h_D - m_D h_D - m_W h_W, \\
\dfrac{df_W}{dt} = (1-\alpha)(1 - \beta h_W) r_F \left(1 + \dfrac{f_W}{K_F(1-\alpha)} \right) f_W + \lfw f_W h_W, \\
\dfrac{dh_W}{dt}= -m_D h_D - m_W h_W 
\end{array} \right.
\end{equation}
It is clear that $\mathcal{D}$ is a $p$-convex set, in which $f_{eq}$ is analytic. According to the proposition \ref{prop:invariantRegion}, the region $\Omega_{eq}$ is an invariant region for system \eqref{equation:hdfwhw}, and a compact subset of $\mathcal{D}$. This shows that the system \eqref{equation:hdfwhw} is dissipative for initial condition in  $\Omega_{eq}$.

The Jacobian of $f_{eq}$ is given by:

{\footnotesize
\begin{multline}
\mathcal{J}_{f_{eq}}(y_{eq}) = \\ \begin{bmatrix}
f_D -\mu_D - m_D & e \lfw h_W & e \lfw f_W - m_W \\
0 & r_F (1-\alpha)(1-\beta h_W) \Big(1 + \dfrac{2 f_W}{K_F(1-\alpha)}\Big) + \lfw  h_W & \left(\lfw- (1-\alpha)\beta r_F - \beta r_F \dfrac{f_W}{K_F} \right)f_W\\
-m_D & 0 & -m_W
\end{bmatrix}.
\label{equation:jacobianMatrix compet}
\end{multline}}
Therefore, it is clear that system \eqref{equation:hdfwhw} is irreducible. The system is competitive if the non-diagonal term $\lfw f_W - (1-\alpha)\beta r_F f_W - \beta r_F \dfrac{f_W^2}{K_F}$ is non-positive. For $f_w \in (-\infty, 0]$, we have:

\begin{align*}
&\lfw f_W - (1-\alpha)\beta r_F f_W - \beta r_F \dfrac{f_W^2}{K_F} \leq 0, \\
%&\Leftrightarrow \lfw - (1-\alpha)\beta r_F - \beta r_F \dfrac{f_W}{K_F} \geq 0 \\
&\Leftrightarrow \lfw - (1-\alpha)\beta r_F \geq \beta r_F \dfrac{f_W}{K_F}.
\end{align*}

Therefore, the system is competitive on $(-\infty, 0]$ if $\lfw - (1-\alpha)\beta r_F \geq 0$. When $\lfw - (1-\alpha)\beta r_F<0$, the previous computations show that the system is competitive only for $f_W \in \Big(-\infty, -K_F(1-\alpha) + \dfrac{K_F \lfw}{\beta r_F}\Big]$. 
\end{proof}

As stated by the previous proposition, when $\lfw - (1-\alpha)\beta r_F < 0$, the equivalent system is not competitive on the whole domain of interest, $\Omega_{eq}$, but only on a subdomain, noted $\Omega_{eq, 1}$. However, this is not so important since $\Omega_{eq, 1}$ is an invariant and absorbing set. That is the subject of the next proposition.


\begin{prop} \label{prop:absorbing set}
\vdeux{Reformulation de la proposition}

\vdeux{
When $\dfrac{r_F(1-\alpha) \beta}{\lfw} > 1$, we define 
$$
f_W^{compet} := -K_F(1-\alpha) + \dfrac{K_F \lfw}{\beta r_F} < 0
$$
and
$$
\Omega_{eq, 1} = \Big\{\Big(h_D, f_W, h_W \Big)  \Big|h_D - ef_W \leq S^{max}, 0 \leq h_D,  -F_W^{max} \leq f_W \leq f_w^{compet}, -H_W^{max}\leq  H_W \leq 0 \Big\}.
$$
The set $\Omega_{eq, 1}$ is an invariant region for system \eqref{equation:hdfwhw}. Moreover, any solution of system \eqref{equation:hdfwhw} with initial condition in $\Omega \setminus \Omega_{eq, 1}$ will enter in $\Omega_{eq, 1}$.}
\end{prop}

\begin{proof}
We start by showing that $\Omega_{eq, 1}$ is an invariant region. In fact, since we already prove that $\Omega_{eq}$ is an invariant region, we only need to show that 
$\nabla (G \cdot f_{eq})_{|f_W = f_W^{compet}}(y_{eq}) < 0$, for $y_{eq} \in \Omega_{eq, 1}$ where $G = f_W - f_W^{compet}$. We have:

\begin{align*}
(\nabla G \cdot y_{eq})_{|f_W = f_W^{compet}} &= r_F(1-\alpha)(1-\beta h_W) \left(1 + \dfrac{f_W^{compet}}{(1-\alpha) K_F} \right)f_W^{compet} + \lfw h_W f^{compet}_W, \\
&= \left(r_F(1-\alpha)(1-\beta h_W) \left(1 + \dfrac{-(1-\alpha) K_F + \dfrac{K_F \lfw}{r_F \beta}}{(1-\alpha) K_F}\right) + \lfw h_W \right) f^{compet}_W \\
&= \left((1-\beta h_W) \left( \dfrac{\lfw}{\beta}\right) + \lfw h_W \right) f^{compet}_W, \\
&= \dfrac{\lfw}{\beta} f_W^{compet}, \\
&< 0.
\end{align*}

Therefore, $\Omega_{eq, 1}$ is an invariant region. Now, we show that any solution with initial condition in $\Omega_{eq, 2}$ enter in $\Omega_{eq, 1}$. We consider $f_W \in (f_W^{compet}, 0)$, and using the same computations than before, we obtain:

\begin{align*}
\dfrac{df_W}{dt} = &r_F(1-\alpha)(1-\beta h_W) \left(1 + \dfrac{f_W}{(1-\alpha) K_F}\right)f_W + \lfw h_W  f_W, \\
& \leq \left(r_F(1-\alpha)(1-\beta h_W) \left(1 + \dfrac{f_W^{compet}}{(1-\alpha) K_F}\right) + \lfw h_W  \right) f_W, \\
& \leq \dfrac{\lfw}{\beta} f_W,\\
&< 0
\end{align*}
This means that any solution with initial condition in $\Omega_{eq} \setminus \Omega_{eq, 1}$ will enter in $\Omega_{eq, 1}$.
\end{proof}

\begin{remark} \label{remark:competitivity}
This proposition allow us to use the competitiveness of the system on the whole domain $\Omega_{eq}$ without consideration for the sign of $\lfw - (1-\alpha)\beta r_F$. Moreover, direct computations show that $EE^{f_W}$ and $EE^{hf_W}_{\cI \geq 0}$ belong to $\Omega_{eq, 1}$.
\end{remark}

\subsection{Equilibrium Existence}
\vdeux{réécrit avec les variables compétitives ; plus de mention de "fauna/human/cexistence equilibrium", cf remarque Yves par mail}

As state in remark \ref{remark:existence1}, the equilibrium of the system \eqref{equation:HDFWHW} are the same than the ones of system \eqref{equation:HDFW}. Therefore, the equilibrium of system \eqref{equation:hdfwhw} are obtained after applying the change of variable described by equations \eqref{equation:change of variable}.

\begin{prop}
When $\cI = 0$, the following results hold true.
\begin{itemize}
\item The system \eqref{equation:hdfwhw} admits a trivial equilibrium $TE = \Big(0,0, 0\Big)$ and an equilibrium $EE^{f_W} = \Big(0, -(1-\alpha)K_F,0 \Big)$ that always exist.

\item When
$$
\mathcal{N}_{\cI = 0} = \dfrac{m e \lfw (1-\alpha)K_F}{\mu_D - f_D} > 1,
$$ 
then the system \eqref{equation:hdfwhw} admits an other equilibrium $EE^{hf_W}_{\cI = 0} = \Big(h^*_{D, \cI = 0}, f^*_{W, \cI = 0}, h^*_{W, \cI = 0}\Big)$ \\ 
where $h^*_{D, \cI = 0} = H^*_{D, \cI = 0}$ and $f^*_{W, \cI = 0} = -F^*_{W, \cI = 0}$ are given in proposition \ref{prop:equilibre, cI=0}, and
$$ 
h^*_{W, \cI = 0} = -m h^*_{D, \cI = 0}.
$$
\end{itemize}

When $\cI > 0$, the following results hold true.
\begin{itemize}
\item The system \eqref{equation:hdfwhw} admits an equilibrium $EE^{h} = \Big(\dfrac{\cI}{\mu_D - f_D}, 0, -m\dfrac{\cI}{\mu_D - f_D} \Big)$ that always exists.
\item When 
$$ \mathcal{N}_{\cI >0} =\dfrac{r_F(1-\alpha)\Big({\dfrac{\mu_D - f_D}{m\cI}+\beta\Big)}}{\lfw}  > 1,$$
the system \eqref{equation:hdfwhw} admits an other equilibrium $EE^{hf_W}_{\cI > 0} = \Big(h^*_{D, \cI > 0}, f^*_{W, \cI > 0}, h^*_{W, \cI > 0}\Big)$
where $h^*_{D, \cI > 0} = H^*_{D, \cI > 0}$ and $f_{W, \cI > 0} = - F^*_{W, \cI > 0}$ are given in proposition \ref{prop:eq, cI>0}, and
$$ 
h^*_{W, \cI > 0} = -m h^*_{D, \cI > 0}.
$$
\end{itemize} 
\end{prop}  

We continue by studying the local stability of the equilibrium.

\subsection{Local Stability of the Equilibrium}

\subsubsection{Stability analysis without immigration}
\begin{prop}\label{prop:stab, cI=0}
\vdeux{réécrit avec les variables compétitives}

 When $\cI = 0$, the following results are valid.
\begin{itemize}
\item The trivial equilibrium $TE$ is unstable.
\item When $\mathcal{N}_{\cI = 0} < 1$, the equilibrium $EE^{f_W}$ is LAS.
\item When $\mathcal{N}_{\cI = 0} > 1$, the equilibrium $EE^{hf_W}_{\cI =0}$ exists. It is LAS if $\Delta_{Stab} > 0$ where 
\begin{multline*}
\Delta_{Stab, \cI =0} = \Big(\mu_D - f_D + m_D - r_F (1-\beta h_W^*) \dfrac{f_W^*}{K_F} + m_W\Big) \times \\ \big( \mu_D  -f_D + m_D + m_W \big) r_F(1 - \beta h_W^*) \dfrac{-f^*_W}{K_F} + 
m_D e \lfw (1- \alpha) r_F \left(1 + \dfrac{f_W^*}{(1- \alpha)K_F}\right) f_W^*,
\end{multline*}
and unstable when $\Delta_{Stab, \cI =0} < 0.$
\end{itemize}
\end{prop}

\begin{proof}
To prove this proposition, we look at the Jacobian of system \eqref{equation:hdfwhw}, given by equation \eqref{equation:jacobianMatrix compet}.


\begin{itemize}
\item At equilibrium $TE$, we have:
\begin{equation*}
\mathcal{J}(TE) = \begin{bmatrix}
f_D-\mu_D - m_D & 0 &  -m_W \\
0 & r_F(1-\alpha)  &  0\\
-m_D & 0 & -m_W
\end{bmatrix}.
\end{equation*}
and $r_F > 0$ is a positive eigenvalue of $\mathcal{J}(TE)$. So, $TE$ is unstable.
\item At equilibrium $EE^{f_W}$, we have
\begin{equation*}
\mathcal{J}(EE^{f_W}) = \begin{bmatrix}
f_D-\mu_D - m_D & 0 & -e\lfw K_F(1-\alpha) - m_W \\
0 & -(1-\alpha)r_F  & -\lfw(1-\alpha)K_F  \\
-m_D & 0 & -m_W
\end{bmatrix}.
\end{equation*}

The characteristic polynomial of $\mathcal{J}(EE^{F_W})$ is given by:
\begin{equation*}
\chi(X) = \big(X +(1-\alpha)r_F\big) \times \left(X^2 - X\Big(f_D - \mu_D - m_D - m_W \Big) + m_W(\mu_D - f_D) - m_D e \lfw K_F(1-\alpha) \right).
\end{equation*}
The equilibrium is LAS if the real part of the roots of $\chi(X)$ are negative. One of the root is $-(1-\alpha)r_F < 0$, and therefore we only need to determine the sign of the real part of the second factor's roots. As the coefficient in $X$ is positive, the sign of the real part of the root is determined by the sign of the constant coefficient.
They have a negative real part if the constant coefficient is positive \textit{i.e.} if $\dfrac{m e \lfw K_F(1-\alpha)}{\mu_D - f_D} < 1 $, and a positive real part if $\dfrac{m e \lfw K_F(1-\alpha)}{\mu_D - f_D} > 1 $. The local stability of $EE^{f_W}$ follows.

\item Now, we look for the asymptotic stability of the equilibrium of coexistence $EE^{hf_W}_{\cI=0}$. The first part of the computations are common with the ones for proving the LAS of equilibrium $EE^{hf_W}_{\cI >0}$.
%To keep some generality, we use notation $EE^{HF_W}$ for both $EE^{HF_W}_{\cI =0}$ and $EE^{HF_W}_{\cI >0}$.
We have


\begin{equation*}
\mathcal{J}(EE^{hf_W}_{\cI \geq 0}) = \begin{bmatrix}
f_D -\mu_D - m_D & e \lfw h_W^* & e \lfw f^*_W - m_W \\
0 & r_F(1 - \beta h_W^*)\dfrac{f_W^*}{K_F} & \big( \lfw - (1-\alpha)\beta r_F\big) f_W^* -  \dfrac{r_F\beta}{K_F} (f_W^*)^2 \\
-m_D & 0 & -m_W
\end{bmatrix}.
\end{equation*} 

%The characteristic polynomial of $\mathcal{J}(EE^{H F_W}_{\cI \geq 0})$  is given by: $\chi = X^3 + a_2 X^2 + a_1 X + a_0$. In particular, we know that $a_2 = - \Tr(\mathcal{J}(EE^{H F_W}))$ and $a_0 = - \det (\mathcal{J}(EE^{H F_W}))$.

We will use the Routh-Hurwitz criterion (see theorem \ref{theorem:Routh-Hurwitz}, page \pageref{theorem:Routh-Hurwitz}) to determine the local stability of $EE^{hf_W}$. First, it requires to compute the coefficients $a_i, i =1,2,3$, which have to be positive.
We have:

\begin{subequations}
\begin{align}
a_2 &= - \Tr\Big(\mathcal{J}(EE^{hf_W}_{\cI \geq 0})\Big), \\
 &= -\Big(f_D - \mu_D - m_D + r_F (1 - \beta h_W^*) \dfrac{h_W^*}{K_F} - m_W\Big), \\
 &= \mu_D - f_D + m_D - r_F (1 - \beta h_W^*) \dfrac{f_W^*}{K_F} + m_W
 \label{equation:coefficient a2}
\end{align}
\end{subequations}
that is $a_2>0$. Coefficient $a_0$ is given by:

\begin{subequations}
\begin{align}
a_0 &= -\det\Big(\mathcal{J}(EE^{hf_W}_{\cI \geq 0})\Big), \\
a_0 &= -\Big(\mu_D + m_D -f_D \Big) m_W r_F (1 - \beta h_W^*) \dfrac{f^*_W}{K_F}  - m_D r_F (1 - \beta h_W^*) \dfrac{f_W^*}{K_F}(e\lfw f_W^* - m_W) + \\
\nonumber
&  m_D e \lfw  \left(\lfw - (1-\alpha)\beta r_F  - \dfrac{r_F\beta}{K_F} f_W^* \right)h_W^* f_W^*, \\
a_0 &= -\Big(\mu_D -f_D \Big) r_F m_W (1 - \beta h_W^*) \dfrac{f^*_W}{K_F}  - m_D e\lfw (1 - \beta h_W^*) r_F \dfrac{(f_W^*)^2}{K_F} + \\
\nonumber
&  m_D e \lfw \left(\lfw - (1-\alpha)\beta r_F  - \dfrac{r_F\beta}{K_F} f_W^* \right)h_W^*f_W^*, \\
a_0 &= -\Big(\mu_D -f_D \Big) r_F m_W (1 - \beta h_W^*) \dfrac{f^*_W}{K_F}  - m_D e\lfw (1 - \beta h_W^*) r_F \dfrac{(f_W^*)^2}{K_F} - \\
\nonumber
&  m_D e \lfw (1- \alpha) r_F \left(1 + \dfrac{f_W^*}{(1- \alpha)K_F}\right) f_W^*, \\
a_0 &= -m_D \lfw e r_F (1 - \beta h_W^*) \left(\dfrac{\mu_D -f_D }{e \lfw m} + f_W^*\right) \dfrac{f_W^*}{K_F} - \\
\nonumber
& m_D e \lfw (1- \alpha) r_F \left(1 + \dfrac{f_W^*}{(1- \alpha)K_F}\right) f_W^*,  \\
a_0 &= -m_D \lfw e r_F \left(\dfrac{\mu_D -f_D }{e \lfw m K_F} + 2\dfrac{f_W^*}{K_F} + (1-\alpha) - \dfrac{\beta h_W^*}{K_F} \left(\dfrac{\mu_D -f_D }{e \lfw m} + f_W^*\right) \right) f_W^*.   \label{equation:coefficient a0}
\end{align}
\end{subequations}

When $\cI = 0$, we have:

\begin{equation*}
f_W^* = - \dfrac{\mu_D - f_D}{\lfw m e}.
\end{equation*} 

Using this expression in equation \eqref{equation:coefficient a0}, we obtain:

\begin{equation*}
a_{0, \cI=0} = -m_D \lfw e r_F  (1- \alpha) \left(1 + \dfrac{f_W^*}{(1 - \alpha)K_F}\right) f_W^* 
\end{equation*}
that is $a_{0, \cI=0}>0$. The coefficient $a_1$ is given by:
\begin{subequations}
\begin{align}
a_1 &= - \big( \mu_D  -f_D + m_D) r_F(1 - \beta h_W^*) \dfrac{f^*_W}{K_F} + (\mu_D -f_D + m_D) m_W - r_F(1 - \beta h_W^*) \dfrac{f_W^*}{K_F} m_W + \\ \nonumber &m_D (e\lfw f^*_W - m_W), \\
a_1 &= - \big( \mu_D-f_D + m_D + m_W) r_F(1 - \beta h_W^*) \dfrac{f^*_W}{K_F} + (\mu_D -f_D) m_W  + m_D e\lfw f^*_W, \\
a_1 &= -\big( \mu_D  -f_D + m_D + m_W) r_F(1 - \beta h_W^*) \dfrac{f^*_W}{K_F} + \left(\dfrac{\mu_D -f_D}{e\lfw m} + f_W^*\right) e \lfw m_D . \label{equation:coefficient a1}
\end{align}
\end{subequations}

Again, using the expression of $F^*_W$ in the case where $\cI = 0$, we have:

\begin{equation*}
a_{1, \cI =0} = -\big( \mu_D  -f_D + m_D + m_W) r_F(1 - \beta h_W^*) \dfrac{f^*_W}{K_F}
\end{equation*}
and we do have $a_{1, \cI =0} > 0$.

The first assumption of the Rough-Hurwitz criterion is verified, $a_{i, \cI =0} > 0$ for $i=1,2,3$. Therefore, the asymptotic stability of $EE^{HF_W}_{\cI =0}$ only depends on the sign of $\Delta_{Stab}= a_2 a_1 - a_0$, which has to be positive. 
\end{itemize}
\end{proof}


\begin{prop} \label{prop:stab, cI=beta=0}
\vdeux{Comme discuté, j'ai déplacé cette proposition ici}

When $\beta = 0$, the following holds true:
$$ \Delta_{Stab, \cI = \beta = 0} > 0 \Leftrightarrow \lfw < \lfw^*$$
where 
\begin{multline*}
\lfw^* := \left[m_{W}(\mu_{D}-f_{D})+\big(\mu_{D}-f_{D}+m_{D}+m_{W})^{2}\right] \times \\
 \dfrac{\left(1+\sqrt{1+4\dfrac{(1-\alpha)m_{W}r_{F}\left(\mu_{D}-f_{D}\right)\big(\mu_{D}-f_{D}+m_{D}+m_{W})}{\left[m_{W}\dfrac{\mu_{D}-f_{D}}{e}+\big(\mu_{D}-f_{D}+m_{D}+m_{W})^{2}\right]^{2}}}\right)}{2em_D (1-\alpha) K_F }.
\end{multline*}
\end{prop}

\begin{proof}
The proof is calculatory and left in appendix \ref{sec:stab, cI = beta = 0}, page \pageref{sec:stab, cI = beta = 0}.
\end{proof}


\subsubsection{Stability analysis with immigration}
Now, we look for the local stability of the equilibrium.
\begin{prop}\label{prop:stab, cI>0} 
\vdeux{réécrit avec les variables compétitives}

When $\cI > 0$, the following results are valid.
\begin{itemize}
\item When $\mathcal{N}_{\cI > 0} < 1$, the equilibrium $EE^{h}$ is LAS.
\item When $\mathcal{N}_{\cI > 0} > 1$, the equilibrium $EE^{hf_W}$  exists. It is LAS if 
$$\Delta_{Stab, \cI > 0} > 0,$$  where 

\begin{multline*}
\Delta_{Stab, \cI > 0} = \left(\mu_D -f_D + m_D - (1 - \beta h_W^*)r_F \dfrac{f_W^*}{K_F} + m_W  \right) \times \\ \left(- \big( \mu_D  -f_D + m_D + m_W) r_F(1 - \beta h_W^*) \dfrac{f^*_W}{K_F} + \left(\dfrac{\mu_D -f_D}{e\lfw m} + f_W^*\right) e \lfw m_D \right) + \\
m_D \lfw e r_F \left(\dfrac{\sqrt{\Delta_F}}{er_F} - \dfrac{\cI \beta}{\lfw K_F e} - \dfrac{\beta h_W^*}{K_F} \left(\dfrac{\mu_D -f_D }{e \lfw m} + f_W^*\right)\right)  f^*_{W}.
\end{multline*}
\end{itemize}
\end{prop}

\begin{proof}
To assess the local stability or instability of the equilibrium, we look at the Jacobian matrix. The Jacobian of the system \eqref{equation:hdfwhw}, noted $\mathcal{J}$, was computed in equation \eqref{equation:jacobianMatrix compet}.

\begin{itemize}
\item At equilibrium $EE^{h}$, we have
\begin{equation*}
\mathcal{J}(EE^{h}) = \begin{bmatrix}
f_D-\mu_D - m_D & -e \lfw \dfrac{m \cI}{\mu_D - f_D} & -m_W \\
0 & r_F(1-\alpha)\left(1-\beta\dfrac{m\cI}{\mu_D - f_D}\right) - \lfw\dfrac{m\cI}{\mu_D - f_D} & 0 \\
-m_D & 0 & -m_W
\end{bmatrix}.
\end{equation*}


The characteristic polynomial of $\mathcal{J}(EE^{h})$ is given by:
\begin{multline*}
\chi(X) = \left(X - r_F(1-\alpha)\left(1+\beta\dfrac{m\cI}{\mu_D - f_D}\right) + \lfw\dfrac{m\cI}{\mu_D - f_D} \right) \times \\ \left(X^2 - X\Big(f_D - \mu_D - m_D - m_W \Big) + m_W(\mu_D - f_D)\right).
\end{multline*}

The constant coefficient of the second factor and its coefficient in $X$ are positive. So, the roots of the second factor have a negative real part. Therefore, only the sign of $r_F(1-\alpha)\left(1+\beta\dfrac{m\cI}{\mu_D - f_D}\right) - \lfw\dfrac{m\cI}{\mu_D - f_D}$ determines the stability of $EE^{h}$. If it is negative, $EE^{h}$ is LAS; if it is positive it is unstable. When it is equal to 0, the equilibrium is degenerated.

\item In order to determine the asymptotic stability of the equilibrium of coexistence $EE^{hf_W}_{\cI > 0}$, we will use the Routh-Hurwitz criterion (see theorem \ref{theorem:Routh-Hurwitz}, page \pageref{theorem:Routh-Hurwitz}). General expressions of the Jacobian matrix and of coefficients $a_i, i = 1,2,3$ were computed in equations \eqref{equation:jacobianMatrix compet}, \eqref{equation:coefficient a2}, \eqref{equation:coefficient a0} and \eqref{equation:coefficient a1}. 

According to \eqref{equation:coefficient a2}, we have:
\begin{equation*}
a_2 =  \mu_D - f_D + m_D - r_F (1 - \beta h_W^*) \dfrac{f_W^*}{K_F} + m_W
\end{equation*}
and we still have $a_2>0$. According to \eqref{equation:coefficient a0}, we have:
\begin{equation*}
a_0 = -m_D \lfw e r_F \left(\dfrac{\mu_D -f_D }{e \lfw m K_F} + 2\dfrac{f_W^*}{K_F} + (1-\alpha) - \dfrac{\beta h_W^*}{K_F} \left(\dfrac{\mu_D -f_D }{e \lfw m} + f_W^*\right) \right) f_W^*.
\end{equation*}

Using
\begin{equation*}
f_W^* = -\dfrac{(1-\alpha)K_F}{2}\left(1 - \dfrac{\sqrt{\Delta_F}}{e(1-\alpha)r_F}\right) - \dfrac{\mu_D - f_D + \cI \beta m}{2\lfw m e},
\end{equation*}
we obtain
\begin{equation*}
a_0 = - m_D \lfw e r_F \left(\dfrac{\sqrt{\Delta_F}}{er_F} - \dfrac{\cI \beta}{\lfw K_F e} -  \dfrac{\beta h_W^*}{K_F} \left(\dfrac{\mu_D -f_D }{e \lfw m} + f_W^*\right)\right)  f^*_{W}.
\end{equation*}

Using the proposition \ref{prop:study of PF}, we know that $\dfrac{\mu_D -f_D }{e \lfw m} + f_W^* > 0$. Let show that  $\dfrac{\sqrt{\Delta_F}}{er_F} - \dfrac{\cI \beta}{\lfw K_F e}$ is also positive. According to the proposition \ref{prop:eq, cI>0}, we have:

\begin{multline*}
\Delta_F = \left(e(1-\alpha)r_F - \dfrac{(\mu_D - f_D) r_F}{\lfw m K_F}\right)^2 + \dfrac{\cI \beta r_F}{\lfw K_F} \left(\dfrac{\cI \beta r_F}{\lfw K_F} + 2\dfrac{(\mu_D - f_D) r_F}{\lfw m K_F} + 2e(1-\alpha)r_F \right) + \\ 4\dfrac{er_F}{K_F}  \cI\Big(1 - \dfrac{(1-\alpha)\beta r_F}{\lfw} \Big)
\end{multline*}
 which gives $\Delta_F > \left(\dfrac{\cI \beta r_F}{\lfw K_F}\right)^2$ and

\begin{equation*}
\dfrac{\sqrt{\Delta_F}}{er_F} - \dfrac{\cI \beta}{\lfw K_F e} > 0.
\end{equation*}

Therefore, we obtain $a_0 > 0$.

According to \eqref{equation:coefficient a1}, coefficient $a_1$ is given by:
\begin{equation*}
a_1 = -\big( \mu_D  -f_D + m_D + m_W) r_F(1 - \beta h_W^*) \dfrac{f^*_W}{K_F} + \left(\dfrac{\mu_D -f_D}{e\lfw m} + f_W^*\right) e \lfw m_D .
\end{equation*}

which is positive, since  $\dfrac{\mu_D - f_D}{e \lfw m} +  f^*_{W} > 0$.

The first assumption of the Rough-Hurwitz criteria is verified, $a_i > 0$ for $i=1,2,3$. Therefore, the local asymptotic stability of $EE^{HF_W}$ only depends on the sign of $\Delta_{Stab}= a_2 a_1 - a_0$, which has to be positive.
\end{itemize}
\end{proof}

\subsection{Global Stability of the Equilibrium and Existence of Limit Cycle}
\vdeux{
In this section, we are interested in the global stability of the equilibrium. The proof of propositions \ref{prop:limitCycle, cI=0} and \ref{prop:limitCycle, cI>0} concerning the equilibrium $EE^{hf_W}_{\cI \geq 0}$ are done using the theorem \ref{theorem:Zhu}, page \pageref{theorem:Zhu}. To proof the global stability of equilibrium $EE^{f_W}$ and $EE^{h}$ (propositions \ref{prop:EEFGAS} and \ref{prop:EEHGAS}) we use the theorem \ref{theorem: monotone GAS}, page \pageref{theorem: monotone GAS}, which involves inequalities between vectors of $\R^3$. These inequalities are considered in the following sense:}
\begin{itemize}
\item $\mathbf{a} \leq \mathbf{b} \Leftrightarrow \left\lbrace \begin{array}{l} 
a_1 \leq b_1 \\ -a_2 \leq - b_2 \\ -a_3 \leq - b_3
\end{array} \right.$
\item $\mathbf{a} < \mathbf{b} \Leftrightarrow \mathbf{a} \leq \mathbf{b}$ and $\mathbf{a}\neq \mathbf{b}$
\end{itemize}
We also note $[\mathbf{a},\mathbf{b}] = \lbrace x \in \mathbb{R}^3 | \mathbf{a} \leq x \leq \mathbf{b} \rbrace$.



\subsubsection{Without Immigration}
We first investigate the global stability of the system in the absence of immigration, \textit{i.e.} when $\cI = 0$.
\begin{prop}\label{prop:EEFGAS}If $\mathcal{N}_{I =0} \leq 1
$, the equilibrium $EE^{f_W}$ is globally asymptotically stable (GAS) on $\Omega_{eq}$ for system \eqref{equation:hdfwhw}.
\end{prop}

\begin{proof}
\vdeux{
We will use the theorem \ref{theorem: monotone GAS}, page \pageref{theorem: monotone GAS} to proof this proposition.
\begin{itemize}
\item First, let assume that $\lfw -(1-\alpha) \beta r_F \geq 0$. In this case, according to the proposition \ref{prop: equivalentSystem}, the system \eqref{equation:hdfwhw} is competitive on $\mathcal{D} = \R_+\times \R_-^2$. Let $\mathbf{a} = \mathbf{0}_{\R^3}$ and $\mathbf{b} = \Big(S^{max}, -K_F(1-\alpha), -mS^{max}\Big)$. We have $\mathbf{a} < \mathbf{b}$ and $[\mathbf{a}, \mathbf{b}] \subset \mathcal{D}$. 
It is immediate to check that $EE^{f_W} \in [\mathbf{a}, \mathbf{b}]$, and since $\N_{\cI = 0} \leq 1$, it is the only equilibrium of the system. Moreover, $f_{eq}(\mathbf{a}) = \mathbf{0} \geq \mathbf{0}$. Noting $f_{eq}(\mathbf{b}) = \Big(f_1, f_2, f_3\Big)$, we have:
\begin{align*}
f_1 &= \Big(e\lfw K_F(1-\alpha) + m_W\Big)mS^{max} - \Big(\mu_D - f_D + m_D\Big) S^{max}, \\
&= \Big(m e\lfw K_F(1-\alpha) - (\mu_D - f_D) \Big)S^{max}, \\
& \leq 0
\end{align*}
using $N_{I= 0} = \dfrac{m e \lfw (1-\alpha)K_F}{\mu_D - f_D} \leq 1$. 
Moreover $f_2 = \lfw m (1-\alpha) K_F S^{max} > 0$ and $f_3 = 0$. All these prove that $f_{eq}(\mathbf{b}) \leq 0$. Therefore, according to the theorem \ref{theorem: monotone GAS}, $EE^{f_W}$ is GAS on $[\mathbf{a}, \mathbf{b}]$ and since $\Omega_{eq} \subset [\mathbf{a}, \mathbf{b}]$, it is GAS on $\Omega_{eq}$.
\item Now, let assume that $\lfw -(1-\alpha) \beta r_F < 0$. In this case, according to the proposition \ref{prop: equivalentSystem}, the system \eqref{equation:hdfwhw} is competitive only on $\Big\{(h_D, f_W, h_W) | 0 \leq h_D, f_W \leq K_F\big(\dfrac{\lfw}{\beta r_F}-(1-\alpha)\big), h_W \leq 0 \Big\}$, which does not contain $\mathbf{0}_{\R^3}$. Therefore, instead of using $\mathbf{a}= \mathbf{0}_{\R^3}$, we use $\mathbf{a} = \Big(0, \dfrac{\lfw}{\beta r_F}K_F -	(1-\alpha)K_F, 0 \Big)$  to show that $EE^{f_W}$ is GAS on $\Omega_{eq}^1 \subset[\mathbf{a}, \mathbf{b}]$. Since $\Omega_{eq}\setminus\Omega_{eq}^1$ is absorbed by $\Omega_{eq}^1$ (see proposition \ref{prop:absorbing set}), $EE^{f_W}$ is GAS on $\Omega_{eq}$.
\end{itemize}}
\end{proof}

%\begin{proof} \marc{est ce que \cite{anguelov_monotone_2010} est une bonne ref ? est ce aussi simple ? Sinon autre preuve dessous}
%When $\mathcal{N}_{\cI = 0} \leq 1$, it follows from the previous propositions \ref{prop:equilibre, cI=0} and \ref{prop:stab, cI=0} that $EE^{F_W}$ is the only existing equilibrium and it is LAS. Since the equivalent system \eqref{equation:HDFWHW} is competitive on $\Omega_{eq}$, it follows from theorem \ref{theorem: monotone GAS} that $EE^{F_W}$ is GAS on $\Omega$.
%\end{proof}

%\begin{proof}
%In the following, we assume $ \mathcal{N}_{I =0} < 1$. We consider a solution $(H_D^s, F_W^s, H_W^s)$ of equations \eqref{equation:HDFWHW} with initial conditions in $\Omega$. Using the fact that $\Omega$ is an invariant region, we have:
%
%\begin{equation}
%\def\arraystretch{2}
%\left\{ \begin{array}{l}
%\dfrac{dH^s_D}{dt} \leq e\lfw H^s_W K_F(1-\alpha) + (f_D - \mu_D) H^s_D - m_D H^s_D + m_W H^s_W , \\
%\dfrac{dF^s_W}{dt} = (1-\alpha)(1 + \beta H_W^s) r_F \left(1 - \dfrac{F^s_W}{K_F(1-\alpha)} \right) F^s_W - \lfw F^s_W H^s_W, \\
%\dfrac{dH^s_W}{dt}= m_D H^s_D - m_W H^s_W.
%\end{array} \right.
%\end{equation}
%
%We consider the auxiliary system, given by:
%\begin{equation}
%\def\arraystretch{2}
%\left\{ \begin{array}{l}
%\dfrac{dH_D}{dt} = \Big(e\lfw K_F(1-\alpha) + m_W\Big)H_W + (f_D - \mu_D - m_D) H_D, \\
%\dfrac{dF_W}{dt} =(1-\alpha)(1 + \beta H_W) r_F \left(1 - \dfrac{F_W}{K_F(1-\alpha)} \right) F_W - \lfw F_W H_W, \\
%\dfrac{dH_W}{dt}= m_D H_D - m_W H_W 
%\end{array} \right.
%\label{equation:limitSystem}
%\end{equation}
%
%We will apply theorem \ref{theorem:Vidyasagar} on this system, with $x = (H_D, H_W)$, $y = F_W$, $x^* = (0,0)$ and $y^* = K_F(1- \alpha)$.
%We have that $x^*$ is GAS for system $\dfrac{dx}{dt} = f_{[1,3]}(x)$ on the compact $\Omega_{[1,3]}$. Indeed, $x^*$ is the unique equilibrium of this system, and it is LAS since $\dfrac{\mu_D - f_D}{\lfw m e K_F(1-\alpha)} >1$. By applying the Bendixson-Dulac theorem \ref{theorem:Dulac}, we show that $\dfrac{dx}{dt} = f_{[1,3]}(x)$ does not admit any limit cycle. Therefore, the Poincaré-Bendixson theorem \ref{theorem:Poincaré-Bendixson} shows that $(0, 0)$ is GAS for $\dfrac{dx}{dt} = f_{[1,3]}(x)$.
%
%It is quite immediate to show that $y^*$ is GAS for system $\dfrac{dy}{dt} = f_{[2]}(x^*, y)$. 
%
%Moreover, the trajectories of the solution of auxiliary system \eqref{equation:limitSystem} with initial condition in $\Omega$ are bounded (proposition \ref{prop:invariantRegion}). So, we can apply theorem \ref{theorem:Vidyasagar}, and we obtain that equilibrium $\Big(0, K_F(1-\alpha), 0 \Big)$ is GAS on $\Omega$ for the auxiliary system, and therefore for the original system \eqref{equation:HDFWHW}.
%\end{proof}

We now consider the case $\mathcal{N}_{I = 0} > 1$ and use the Zhu's  theorem, recalled in theorem \ref{theorem:Zhu} page \pageref{theorem:Zhu} to complete the characterization of the system.

\begin{prop}
\vdeux{réécrit avec les variables compétitives}
\label{prop:limitCycle, cI=0}
If $\mathcal{N}_{I =0} > 1$, and if:
\begin{itemize}
\item $\Delta_{stab, \cI =0} > 0$, the equilibrium $EE^{hf_W}$ is GAS on $\Omega_{eq}$ for system \eqref{equation:hdfwhw}.
\item $\Delta_{stab, \cI =0} < 0$, the system \eqref{equation:hdfwhw} admits an orbitally asymptotically stable periodic solution.
\end{itemize}
\end{prop}

\begin{remark}
When $\beta = 0$, using proposition \ref{prop:stab, cI=beta=0}, we equivalently have:
\begin{itemize}
\item if $\lfw <  \lfw^*$, then $EE^{hf_W}$ is GAS on $\Omega_{eq}$ for system \eqref{equation:hdfwhw}.
\item if $\lfw  > \lfw^*$, the system \eqref{equation:hdfwhw} admits an orbitally asymptotically stable periodic solution.
\end{itemize}
\end{remark}

\begin{proof}
When $\mathcal{N}_{I =0} > 1$, the system \eqref{equation:hdfwhw} admits a unique positive and asymptotically stable equilibrium, $EE^{hf_W}$. By applying theorem \ref{theorem:Zhu} and remark \ref{remark:competitivity} to this system, we know that either $EE^{hf_W}$ is GAS, or it exists an asymptotically stable periodic solution. According to the proposition \ref{prop:stab, cI=0}, the condition for stability is precisely $0 < \Delta_{stab, \cI =0}$. 
\end{proof}

The results we obtained are summarized in the following table:
%
%\begin{table}[!ht]
%\centering
%\def\arraystretch{2}
%\begin{tabular}{c|c|c|c|c}
%$\cI$ &$\beta$ & $\mathcal{N}_{I =0}$ &  $\dfrac{\lfw}{  \lfw ^*}$ & \\
%\hline
%\multirow{4}{*}{$=0$}&\multirow{4}{*}{$=0$} & $ < 1$ & &$EE^{F_W}$ exists and is GAS.  \\
%\cline{3-5}
% & & \multirow{3}{*}{$> 1$} & $ <1$ &$EE^{HF_W}_{\cI=0}$ exists and is GAS.\\
% \cline{4-5}
% & & &\multirow{2}{*}{$ > 1$} & $EE^{HF_W}_{\cI=0}$ exists and is unstable ; there is an asymptotically \\
%& & & &  stable periodic solution.
%\end{tabular}
%\caption{\centering Conditions of existence and asymptotic stability of equilibrium for system \eqref{equation:HDFWHW}, when $\beta = 0$}
%\end{table}


%\begin{table}[!ht]
%\centering
%\def\arraystretch{2}
%\begin{tabular}{c|c|c|c|c}
%$\cI$ &$\beta$ & $\mathcal{N}_{I =0}$ &  $\Delta_{Stab, \cI =0}$ & \\
%\hline
%\multirow{4}{*}{$=0$}&\multirow{4}{*}{$>0$} & $ < 1$ & &$EE^{F_W}$ exists and is GAS.  \\
%\cline{3-5}
% & & \multirow{3}{*}{$> 1$} & $ >0$ &$EE^{HF_W}_{\cI=0}$ exists and is GAS.\\
% \cline{4-5}
% & & &\multirow{2}{*}{$ <0 $} & $EE^{HF_W}_{\cI=0}$ exists and is unstable ; there is an asymptotically \\
%& & & &  stable periodic solution.
%\end{tabular}
%\caption{\centering Conditions of existence and asymptotic stability of equilibrium for system \eqref{equation:HDFWHW}}
%\label{table:long term dynamic, I = 0}
%\end{table}

\begin{table}[!ht]
\centering
\def\arraystretch{2}
\begin{tabular}{c|c|c|c}
$\cI$  & $\mathcal{N}_{I =0}$ &  $\Delta_{Stab, \cI =0}$ & \\
\hline
\multirow{4}{*}{$=0$}& $ \leq 1$ & &$EE^{F_W}$ exists and is GAS.  \\
\cline{2-4}
 &  \multirow{3}{*}{$> 1$} & $ >0$ &$EE^{HF_W}_{\cI=0}$ exists and is GAS.\\
 \cline{3-4}
 &  &\multirow{2}{*}{$ <0 $} & $EE^{HF_W}_{\cI=0}$ exists and is unstable ; there is an asymptotically \\
&  & &  stable periodic solution.
\end{tabular}
\caption{\centering Conditions of existence and asymptotic stability of equilibrium for system \eqref{equation:HDFWHW}}
\label{table:long term dynamic, I = 0}
\end{table}

\begin{figure}[!ht]
\begin{subfigure}{0.45\textwidth}
\centering
\includegraphics[width=1\textwidth]{LCI0.png}
\caption{\centering Orbits in the $H_D + H_W ; F_W$ plane converging towards the LC.}
\label{fig:LCI0, 1}
\end{subfigure}
\begin{subfigure}{0.45\textwidth}
\centering
\includegraphics[width=1\textwidth]{LCI0HF.png}
\caption{\centering Values of $H_D + H_W$ and $F_W$ as function of time on the limit cycle.}
\label{fig:LCI0, 2}
\end{subfigure}
\caption{Illustration of the system's convergence toward a stable Limit Cycle (LC). \\
Parameters' values: $\cI = 0$, $\beta = 0$, $r_F = 3.35$, $K_F = 22725$, $\alpha = 0.4$, $\lfw = 0.04$, $e = 3$, $\mu_D = 0.2$, $f_D = 0.0164$, $m_D = 7.83$, $m_W = 24.3$.}
\end{figure}


\subsubsection{With Immigration}

As in the case of $\cI = 0$, we can supplement the local results obtained when $\cI > 0$ with information about the global behaviour of the system. 

\begin{prop}
The condition $\mathcal{N}_{I > 0} < 1$ implies $(1-\alpha) \beta r_F < \lfw$. Therefore, when $\mathcal{N}_{I > 0} < 1$, the equivalent system \eqref{equation:hdfwhw} is competitive on the whole domain $\Omega_{eq}$.
\end{prop}

\begin{proof}
Since $\mathcal{N}_{\cI > 0} = \dfrac{r_F(1-\alpha)}{\lfw}\dfrac{\mu_D - f_D}{m \cI} + \dfrac{r_F(1-\alpha) \beta}{\lfw}$, we have $\dfrac{r_F(1-\alpha) \beta}{\lfw} < \mathcal{N}_{\cI > 0}$. Therefore, $\mathcal{N}_{\cI > 0} < 1$ implies $\dfrac{r_F(1-\alpha) \beta}{\lfw} < 1$, which implies, according to proposition \ref{prop: equivalentSystem}, that the equivalent system \eqref{equation:hdfwhw} is competitive on $\Omega_{eq}$.
\end{proof}

%\begin{prop}
%Let $f_W^{\max} = \max \Big(-K_F(1-\alpha) , -\dfrac{\mu_D-f_D}{e \lfw}\Big)$, and assume $\N_{I > 0} \leq 1$. The set 
%$$
%\Omega_{eq, 2} = \left\{(h_D, f_W, h_W) \Big| \dfrac{\cI}{\mu_D - f_D} \leq h_D \leq S^{max} + e f_W, f_W^{\max} \leq f_W  \leq 0, -mS^{max} \leq h_W \leq -m\dfrac{\cI}{\mu_D - f_D} \right\}
%$$
%is an invariant set for system \eqref{equation:hdfwhw}. Moreover, any solutions of \eqref{equation:hdfwhw} with initial condition in $\Omega_{eq}$ will enter in $\Omega_{eq, 2}$.
%\end{prop}
%
%\begin{proof}
%We first prove that $\Omega_{eq, 2}$ is an invariant set. Since we already prove the proposition \ref{prop:invariantRegion} about the invariance of $\Omega_{eq}$, it is sufficient to show that $(\nabla G_1 \cdot g)|_{h_D^*} \geq 0$, $(\nabla G_2 \cdot g)|_{f_W^{\max}} \geq 0$ and $(\nabla G_3 \cdot g)|_{h_W^*} \leq 0$ in $\Omega_{eq, 2}$, where
%
%$$
%G_1(z) = h_D - h_D^*,
%\quad
%G_2(z) = f_W - f_W^{max},
%\quad
%G_3(z) = h_W - h_W^*,
%\quad 
%h_D^* = \dfrac{\cI}{\mu_D - f_D}, \quad
%h_W^* = -m \dfrac{\cI}{\mu_D - f_D}.
%$$
%
%We have:
%
%\begin{align*}
%(\nabla G_1 \cdot g)|_{h_D^*} &= \cI + e\lfw h_W f_W + (f_D - \mu_D) \dfrac{\cI}{\mu_D - f_D} - m_D \dfrac{\cI}{\mu_D - f_D} - m_W h_W, \\
%& \geq  - m_D \dfrac{\cI}{\mu_D - f_D} - m_W h_W, \\
%& \geq 0
%\end{align*}
%since $h_W \leq -m\dfrac{\cI}{\mu_D - f_D}$. It is straightforward to show that $(\nabla G_3 \cdot g)|_{h_W^*} \leq 0$. We compute $(\nabla G_2 \cdot g)_{f_W^{\max}}$:
%
%\begin{align*}
%(\nabla G_2 \cdot g)_{f_W^{\max}} &= \left(r_F(1-\alpha)(1-\beta h_W) \left(1 + \dfrac{f_W^{\max}}{K_F(1-\alpha)} \right) + \lfw h_W\right) f_W^{\max}, \\
%&= \left(r_F(1-\alpha)(1-\beta h_W) + \lfw h_W + \dfrac{f_W^{\max}}{K_F(1-\alpha)} \right) f_W^{\max}.
%\end{align*}
%Since $h_W \leq -m\dfrac{\cI}{\mu_D - f_D}$ and $\N_{\cI > 0}\leq 1$, we have $r_F(1-\alpha)(1-\beta h_W) + \lfw h_W \leq 0$. Moreover $ f_W^{\max} < 0$, and therefore $(\nabla G_2 \cdot g)_{f_W^{\max}} \geq 0$.
%
%Now, we show that any solutions of \eqref{equation:hdfwhw} with initial condition in $\Omega_{eq}$ will enter in $\Omega_{eq, 2}$. Assume that $\Big(h_D, f_W, h_W \Big) \in \Omega_{eq} \setminus \Omega_{eq, 2}$. Using the same computations than before, we have
%\begin{align*}
%\dfrac{dh_D}{dt} &= \cI + e\lfw h_W f_W + (f_D - \mu_D)h_D- m_D h_D - m_W h_W \\
%& > \cI + (f_D - \mu_D)\dfrac{\cI}{\mu_D - f_D} -m_D\dfrac{\cI}{\mu_D - f_D} - m_W h_W
%\end{align*}



%\end{proof}





\begin{prop}\label{prop:EEHGAS}

If $\mathcal{N}_{\cI > 0} \leq 1$, the equilibrium $EE^{h}$ is GAS on $\Omega_{eq}$ for system \eqref{equation:hdfwhw}.
\end{prop}

\vdeux{
\begin{proof}
Assume $\N_{I> 0} \leq 1$.
We note $f_W^{\min} = \max\Big(-\dfrac{\mu_D - f_D}{e m \lfw}, -K_F(1-\alpha)\Big)$ and $\Omega_{GAS}=\Big\{\Big(h_D, f_W, h_W \Big)  \Big|h_D - ef_W \leq S^{max}, 0 \leq h_D,  f_w^{max} \leq f_W \leq 0, -H_W^{max}\leq  H_W \leq 0 \Big\}$. We first show that $EE^{f_W}$ is GAS on $\Omega_{GAS}$ and then we show that any solution with initial condition in $\Omega_{eq}$ enters in $ \Omega_{GAS}$. These two steps show that $EE^{f_W}$ is GAS on $\Omega_{eq}$.
\begin{itemize}
\item We use theorem \ref{theorem: monotone GAS}, page \pageref{theorem: monotone GAS} to show that $EE^{f_W}$ is GAS on $\Omega_{GAS}$. Let $\mathbf{a} = \mathbf{0}_{\R^3}$ and $\mathbf{b} = \Big(h_D^{\max}, f_W^{\min}, -mh_D^{\max} \Big)$, where $h_D^{max} = \max\Big(S^{max}, \dfrac{\cI}{\mu_D - f_D - e m \lfw f^{\min}_W}\Big)$. We have $\mathbf{a} < \mathbf{b}$, $\mathbf{0}_{\R^3} \leq f_{eq}(\mathbf{a})$ and $EE^{f_W}$ is the only equilibrium in $[\mathbf{a}, \mathbf{b}]$. We note $f(\mathbf{b}) = (f_1, f_2, f_3)$. We have:
\begin{align*}
f_1 &= I + \Big(-e\lfw f_W^{min} + m_W\Big)mh_D^{\max} - \Big(\mu_D - f_D + m_D\Big) h_D^{\max}, \\
&= I + \Big(-e m \lfw f_W^{min} - (\mu_D - f_D) \Big)h_D^{\max}, \\
& \leq 0
\end{align*}
since $h_D^{\max} \geq \dfrac{\cI}{\mu_D - f_D - e m \lfw f^{\min}_W}$. We immediately have $f_3 = 0$, and since $\N_{I>0} \leq 1$ and $-m h_D^{\max} \leq -mS^{max} \leq -m\dfrac{I}{\mu_D -f_D}$, it is straightforward to obtain $-f_2 \leq 0$. All this prove that $f_{eq}(\mathbf{b}) \leq 0$, and according to theorem \ref{theorem: monotone GAS}, $EE^{f_W}$ is GAS on $ \Omega_{GAS} \subset [\mathbf{a}, \mathbf{b}]$ for system \eqref{equation:hdfwhw}.
\item Now we show that any solution of \eqref{equation:hdfwhw} with initial condition in $\Omega_{eq}$ enters in $\Omega_{GAS}$. 
Let $\Big(h_D^f, f_W^f, h_W^f \Big)$ be a solution of system \eqref{equation:hdfwhw}. We introduce the following system:
\begin{equation}
\def\arraystretch{2}
\left\{ \begin{array}{l}
\dfrac{dh_D}{dt}= \cI + (f_D - \mu_D) h_D - m_D h_D - m_W h_W, \\
\dfrac{df_W}{dt} = (1-\alpha)(1 - \beta h_W) r_F \left(1 + \dfrac{f_W}{K_F(1-\alpha)} \right) f_W + \lfw f_W h_W, \\
\dfrac{dh_W}{dt}= -m_D h_D - m_W h_W. 
\end{array} \right.
\label{equation:model H GAS}
\end{equation}
We note $g(y)$ its right hand side. For all $y \in \Omega_{eq}$, we have $g(y) \leq f_{eq}(y)$. Since $f_{eq}$ is competitive, we have the following inequality between the solutions of the two systems, with the same initial condition:
$$
(h_D^g,f_W^g, h_W^g) \leq (h_D^f,f_W^f, h_W^f).
$$
The system \eqref{equation:model H GAS} can be written like the system \eqref{equation: eqVidyasagar} in the Vidyasagar's theorem (see theorem \ref{theorem:Vidyasagar}, page \pageref{theorem:Vidyasagar}), with $y_1 = (h_D, h_W)$ and $y_2 = (f_W)$. Using this theorem, and since $\N_{I>0} \leq 1$, we obtain that $EE^{h}$ is GAS on $\Omega_{eq}$ for system \eqref{equation:model H GAS}. In particular this means that $h_W^g$ converges, when $t\rightarrow +\infty$, towards $-m\dfrac{\mu_D - f_D}{\cI}$. 
Therefore, since $\N_{\cI > 0} = r_F(1-\alpha)\dfrac{1 + \beta m\dfrac{\mu_D - f_D}{\cI}}{\lfw m\dfrac{\mu_D - f_D}{\cI}} \leq 1$, it exists $T>0$ such that for all $t\geq T$, $r_F(1-\alpha) \dfrac{1-\beta h_W^g}{-h_W^g} \leq 1$. Now, using $-h_W^g \leq -h_W^f$, we obtain that for all $t\geq T$, $r_F(1-\alpha) \dfrac{1-\beta h_W^f}{-h_W^f} \leq 1$. This gives that for all $t\geq T$, $\dfrac{df_W^f}{dt} \leq 0$, and $f_W^f$ converges towards 0. Therefore, it exists $T_1 > 0$, such that $0\leq f_W > f_W^{max}$, and this means that  $(h_D^f,f_W^f, h_W^f)(T_1) \in \Omega_{GAS}$.
\end{itemize}
\end{proof}}


\begin{prop} \label{prop:limitCycle, cI>0}
If $\mathcal{N}_{\cI > 0} > 1$ and if 

\begin{itemize}
\item $\Delta_{Stab} > 0$, the equilibrium $EE^{hf_w}_{\cI >0}$ is GAS on $\Omega_{eq}$ for system \eqref{equation:hdfwhw}.
\item $\Delta_{Stab} < 0$, the system \eqref{equation:hdfwhw} admits an orbitally asymptotically stable periodic solution on $\Omega_{eq}$.
\end{itemize}
\end{prop}

\begin{proof}
When $\mathcal{N}_{I =0} > 1$, system \eqref{equation:hdfwhw} admits a unique positive and AS equilibrium, $EE^{hf_W}$. By applying theorem \ref{theorem:Zhu} to this system, we know that either $EE^{hf_W}$ is GAS, or it exists a asymptotically stable periodic solution. Since the condition for stability is precisely $0 < \Delta_{stab, \cI > 0}$, the proposition is proven. 
\end{proof}


The long term dynamic of the system is summarized in the following table:
\begin{table}[!ht]
\def\arraystretch{2}
\centering
\begin{tabular}{c|c|c|c}
$\cI$ & $\mathcal{N}_{\cI > 0} $ & $\Delta_{Stab, \cI > 0}$ & \\
\hline
\multirow{3}{*}{$>0$} & $\leq1$ & &$EE^{H}$ exists and is GAS on $\Omega$ \\
\cline{2-4}
 & \multirow{3}{*}{$> 1$}  & $>0$ &$EE^{HF_W}_{\cI>0}$ exists and is GAS on $\Omega$ \\
 \cline{3-4}
 & & \multirow{2}{*}{$ < 0$} & $EE^{HF_W}_{\cI>0}$ exists and is unstable ; there is an asymptotically \\
 & & &  stable periodic solution. \\
\end{tabular}
\caption{\centering Conditions of existence and asymptotic stability of equilibrium for system \eqref{equation:HDFWHW}}
\label{table:long term dynamic, I > 0}
\end{table}

\begin{figure}[!ht]
\begin{subfigure}{0.45\textwidth}
\centering
\includegraphics[width=1\textwidth]{LCI.png}
\caption{\centering Orbits in the $H_D + H_W ; F_W$ plane converging towards the LC.}
\label{fig:LCI, 1}
\end{subfigure}
\begin{subfigure}{0.45\textwidth}
\centering
\includegraphics[width=1\textwidth]{LCIHF.png}
\caption{\centering Values of $H_D + H_W$ and $F_W$ as function of time on the limit cycle.}
\label{fig:LCI, 2}
\end{subfigure}
\caption{ Illustration of the system's convergence toward a stable Limit Cycle (LC). \\
Parameters' values: $\cI = 10$, $\beta = 0$, $r_F = 3.35$, $K_F = 22725$, $\alpha = 0.4$, $\lfw = 0.04$, $e = 3$, $\mu_D = 0.2$, $f_D = 0.0164$, $m_D = 7.83$, $m_W = 24.3$.}
\end{figure}

\subsubsection{Long Term Dynamic when $\epsilon \rightarrow 0$}
In this section, we look for what happen for system \eqref{equation:HDFWHW} when $\epsilon \rightarrow 0$. It is not obvious that the system has the same long term behavior than its approximation \eqref{equation:HDFW} since the proposition \ref{prop: equivalentSystem} is only valid for finite time interval (see section 4.4.3 in \cite{banasiak_methods_2014} for a counter example). However, in the case of this model, the dynamics are the same, as stated by the following proposition.

\begin{prop}
The following holds true:
$$
\lim\limits_{ \epsilon \rightarrow 0}{\Delta_{Stab, \cI \geq 0}} > 0.
$$
This means that for $\epsilon$ small enough the dynamics of system \eqref{equation:HDFWHW} only depends on $\mathcal{N}_{\cI \geq 0}$ and corresponds to the one of the system \eqref{equation:HDFWHW}.
\end{prop}
\begin{proof}
First we can note that for $\cI \geq 0$, the value at equilibrium $EE^{HF_W}$ does not depend on $\epsilon$. However, according to \eqref{equation:coefficient a2}, \eqref{equation:coefficient a0} and \eqref{equation:coefficient a1} the Routh-Hurwitz coefficient are (the expression are valid for $\cI \geq 0$):

\begin{align*}
a_2  &= \mu_D - f_D + \dfrac{\tilde m_D}{\epsilon} + (1+\beta H_W^*)r_F \dfrac{F_W^*}{K_F} + \dfrac{\tilde m_W}{\epsilon} \\
a_0 &= e \lfw \dfrac{\tilde m_D}{\epsilon} r_F \left(\dfrac{\mu_D -f_D }{e \lfw m K_F} - 2\dfrac{F_W^*}{K_F} + (1-\alpha) + \dfrac{\beta H_W^*}{K_F} \left(\dfrac{\mu_D -f_D }{e \lfw m} - F_W^*\right) \right) F_W^* \\
a_1 &= \Big( \mu_D  -f_D + \dfrac{\tilde m_D}{\epsilon} + \dfrac{\tilde m_W}{\epsilon} \Big) r_F(1+ \beta H_W^*) \dfrac{F^*_W}{K_F} + \left(\dfrac{\mu_D -f_D}{e\lfw m} - F_W^*\right) e \lfw \dfrac{\tilde m_D}{\epsilon}.
\end{align*}

Therefore, direct computations lead to:

\begin{equation*}
\Delta_{Stab} := a_2 a_1 - a_0 = \dfrac{C_2}{\epsilon^2} + \dfrac{C_1}{\epsilon} + C_0
\end{equation*}
where $C_2 > 0$.


Therefore, when $\epsilon \rightarrow 0$, $\Delta_{Stab} > 0$. The rest of the proposition comes from tables \ref{table:long term dynamic, I = 0} and \ref{table:long term dynamic, I > 0}.
\end{proof}


\section{Ecological interpretation of the analytical results} \label{sec:ecological}
In this section, we give an ecological interpretation of the different results we obtained. We are specially interested in the role played by the parameters of hunting rate $\lfw$ and the level of anthropisation, $\alpha$, which traduce the possibility of over-hunting and the deforestation. 

\subsection{Existence of hunting rate thresholds}
\vdeux{
The mathematical analysis show that the situation is totally different with or without immigration. We continue to make the distinction on the following.}


\begin{itemize}
\item \vdeux{When no immigration is considered, the system always admits a fauna only equilibrium, $EE^{F_W}$, and may admit a coexistence equilibrium between wild fauna and human $EE^{HF_W}$. This equilibrium exists only if $1 < \N_{\cI = 0}$, which can be rewritten as a condition on the hunting rate; the coexistence equilibrium exists only if 
$$
\lambda_{F, \cI = 0}^{\min} = \dfrac{\mu_D - f_D}{(1-\alpha) e m K_F} < \lfw,
$$ that is if the hunting rate is sufficiently high. 
This is because the human population relies on bush meat for subsistence. We can notice that the minimum hunting rate, $\lambda_{F, \cI = 0}^{\min}$ increases with the anthropization level. The more anthropised the area is, the fewer wild animals there are, meaning that humans have to hunt more.
}

\vdeux{
When $\lambda_{F, \cI = 0}^{\min} < \lfw $, the system converges towards the coexistence equilibrium, or towards a periodic cycle around this equilibrium. When no feedback from human population to wild fauna growth is considered (case $\beta = 0$), we characterize this difference in behavior using a second hunting rate threshold. If 
$$
\lambda_{F, \cI=\beta=0}^{\max} = \lfw^* < \lfw,
$$ (where $\lfw^*$ is defined in proposition \ref{prop:stab, cI=beta=0}) then both human and wild fauna population oscillate over time. This means that a too large hunting rate can reduce the wild fauna below a level sufficient to feed the human population. Consequently, the human population decreases before increasing again, once the wild fauna population has grown sufficiently. In this case, it is important to assess the period and amplitude of the cycles, in order to determine the extent and duration of  the populations variations.
}

\item \vdeux{When we consider immigration, it is the existence of wild fauna which is no longer clear. Indeed, if an human only equilibrium $EE^{H}$ always exist, the wild fauna exists only if the coexistence equilibrium $EE^{HF_W}$ exists, that is only if $1 < \N_{I>0}$. We can rewrite this condition as:
$$
\lfw < \lambda_{F, \cI > 0}^{max} = (1-\alpha) r_F\left(\dfrac{\mu_D - f_D}{m \cI} + \beta \right).
$$
Therefore, the wild fauna exists only if the hunting rate is not too large. Moreover, $\lambda_{F, \cI > 0}^{max}$ is a decreasing function of both immigration and anthropization. This indicates that the wild fauna is endangered by these two parameters, which must be controlled if a coexistence between the two populations is desired.
}

\vdeux{
We also prove the possibility of periodic solutions around the coexistence of equilibrium. Due to the complexity of the expressions, this possibility was not described with a hunting rate threshold; instead, we investigate it numerically.
}
\end{itemize}



\subsection{Numerical Simulations}
On the following table, we provide an estimations of the other parameters based on an ecological and anthropological review concerning the South-Cameroon.

\begin{table}[ht]
\centering
\begin{tabular}{|c|c|c|c|}
\hline 
Parameter & Unit & Value & Reference \\ 
\hline 
$e$ & - &\sout{Unit Order} \YD{]0,1[??} & Assumed\\
$f_D$ & Year$^-1$ &0.0164 & \cite{koppert_consommation_1996}\\
$\mu_D$ & Year$^-1$  & $1/50$ & \cite{ins_demographie}\\
$m_D$ & Year$^-1$  &$7.83$ & \cite{avila_interpreting_2019}\\
$m_W$ &Year$^-1$  &24.3 & \cite{avila_interpreting_2019}\\
$r_F$ & Year$^-1$ & $3.35$ & \cite{robinson_intrinsic_1986}\\
$K_F$ & Ind& 22725 & \cite{janson_ecological_1990} \\
$\alpha$ &-&  $[0, 1)$ & Parameter of interest; varies \\
$\beta$ & Ind$^-1$ & $\in [0, \beta^*)$ &  \\
$\lfw$ & Ind Year$^-1$ & - & Parameter of interest; varies \\
$\mathcal{I}$ &  Ind Year$^-1$ & & Varies \\
\hline
\end{tabular}
\caption{Parameters values}
\label{table:param values}
\end{table}

These values are used to draw bifurcation diagrams (figures \ref{fig:bifurcation, I=0}, \ref{fig:bifurcation, I=0.5} and \ref{fig:bifurcation, I=1}) and solution (figure \ref{fig:1d trajectory}) of the system.

\vdeux{
The first one, presented in figure \ref{fig:bifurcation, I=0} is drawn for $\cI = 0$. On it, we see that the range of values of $\lfw$ and $\alpha$ for which the wild fauna persists without human is very low. That is due to the fact that the human population is almost self-sufficient ($f_D - \mu_D << 1$), and needs only few food coming from hunt to subsist. Therefore, even if the hunting rate is low or anthropization high, the human population will find enough resources. 
}

\begin{figure}[!ht]
\centering
\includegraphics[width=1\textwidth]{DiagBifI0+Zoom.png}
\caption{Bifurcation diagram when $\cI = 0$. Other parameters values are indicated in table \ref{table:param values}.}
\label{fig:bifurcation, I=0}
\end{figure}



\vdeux{
The second and third bifurcation diagram (figures \ref{fig:bifurcation, I=0.5} and \ref{fig:bifurcation, I=1}) are respectively drawn for $\cI = 0.1$ and $\cI = 1$. As mentioned above, if the hunting rate or the anthropization is too high, only the human population subsists. Moreover, the range of values for which coexistence is possible decreases as $\cI$ increases, as does the possibility of limit cycles. This last point is important. Indeed, the value of $F_W^*$ at equilibrium rapidly decreases when the hunting rate, the anthropisation or the immigration increases. Therefore, in order to ensure the presence of a significant wild fauna population, the system must be in a limit cycle configuration. Even in this case, and as shown in figure \ref{fig:1d trajectory}, the population of wild fauna never reaches high levels, and even then, only for a short time.
}

\vdeux{All these observations show that, according to our model and our parametrization, the situation in South Cameroon is such that wild fauna is in great danger. It also underlines that immigration of people complicates the possibility of coexistence between human and wild fauna populations, which becomes possible only at a low level of anthropisation and hunting.}

\begin{figure}[!ht]
\centering
\begin{subfigure}{0.49\textwidth}
\includegraphics[width=1\textwidth]{DiagBifI01e1.png}
\caption{Bifurcation diagram when $\cI = 0.1$. Other parameters values are indicated in table \ref{table:param values}.}
\label{fig:bifurcation, I=0.5}
\end{subfigure}
\begin{subfigure}{0.49\textwidth}
\centering
\includegraphics[width=1\textwidth]{DiagBifI1e1.png}
\caption{Bifurcation diagram when $\cI = 1$. Other parameters values are indicated in table \ref{table:param values}.}
\label{fig:bifurcation, I=1}
\end{subfigure}
\caption{}
\end{figure}

\begin{figure}[!ht]
\centering
\begin{subfigure}{0.49\textwidth}
\centering
\includegraphics[width=1\textwidth]{HofTimee1.png}
\end{subfigure}
\begin{subfigure}{0.49\textwidth}
\centering
\includegraphics[width=0.95\textwidth]{FofTimee1+Zoom.png}
\end{subfigure}
\caption{Solution of system \eqref{equation:HDFWHW} as function of time. For this simulation, $\cI = 0.1$, $\alpha=0.1$ and $\lfw = 0.015$. The system converges towards a stable limit cycle.}
\label{fig:1d trajectory}
\end{figure}

\section{Conclusion} \label{sec:conclusion}

\vtrois{
With the aim to better understand the risk caused by over hunting and habitat anthropisation on the coexistence between a forest based human population and wild fauna, we developed a model based on ordinary differential equations. We investigated the model's long term behavior;  first using a quasi steady state approach. However, numerical simulations show that this approach gave incomplete results. To make it up, we provided a second analysis using the theory of monotone systems. Depending on the value of different thresholds, the system can converge towards the extinction of one population, the coexistence of both populations, or a limit cycle. Ecological interpretation of the thresholds are given.
}

\vtrois{
Our work take place within the I-CARE project (Impact of the Anthropisation on the Risk of Emergence of Zoonotic Arboviroses in Central Africa). This project, which involves field experts, focuses on South Cameroon. That is why we provide numerical illustrations with parameter values that correspond to this region. We propose bifurcation diagrams in the $\lfw-\alpha$ plane. They underline the importance of maintaining low level of hunting and habitat anthropisation in order to preserve coexistence, especially in areas subject to human immigration. Numerical simulations shows that even in the case of limit cycle, the wild fauna population vanishes for long time. This gives great ecological concern and raises modeling issues.
}
\marc{à continuer}

 

\marc{faire un recap de ce qui a été fait, parler d'ICARE ?, limites du modèle; ouvrir sur HT2 et ou épidémio}

\begin{itemize}
\item population petite; ode appropriée ?
\item immigration : impulsion plutôt que continu ?
\item pb numerique : atto fox, article de Lobry et Sari
\item borne sur beta (trop) petite. Trouver mieux ?
\item HT1 : problème d'action de masse. Peut faire mieux avec du HT2
\item Compléter avec de l'épidémio
\end{itemize}


\newpage
\section*{Appendix}
\begin{appendix}
\section{Useful theorems coming from the literature} \label{sec:litterature theorems}
\vdeux{
On the following, we consider the system of differential equation
\begin{equation}
\dfrac{dy}{dt} = f(y), \quad y(0) = y_0,
\label{equation:generic system}
\end{equation}
defined on an open subset $\mathcal{D}$ of $\mathbb{R}^n$. We assume that $f \in C^1(\mathcal{D})$, and we note $\mathcal{J}(y)$ the Jacobian of $f$ at $y$.}

\begin{theorem}  \label{theorem:Poincaré-Bendixson} [Poincaré-Bendixson's Theorem  \cite{wiggins_introduction_2003}]

The following theorem holds true when $\mathcal{D}$ is an open subset of $\mathbb R ^2$.

Let $\mathcal{M}$ be a positively invariant region for $f$, containing a finite number of fixed points. Let $y \in \mathcal{M}$ and consider its $\omega$-limit set, $\omega(y)$. Then one of the following possibilities holds:
\begin{enumerate}
\item $\omega(y)$ is a fixed point;
\item $\omega(y)$ is a closed orbit;
\item $\omega(y)$ consists of a finite number of fixed points $y_1, \ldots, y_n$ and regular heteroclinic or homoclinic orbits joining them.
\end{enumerate}
\end{theorem}

\begin{theorem} \label{theorem:Dulac} [Dulac's Criteria \cite{perko_differential_1996}]

The following theorem holds true when $\mathcal{D}$ is a simply connected region in $\mathbb R ^2$. 

If there exists a function $B \in \mathbb{C}^1(\mathcal{D})$ such that $\nabla \cdot (Bf)$ is not identically zero and does not change sign in E, then the system has no closed orbit lying entirely in $\mathcal{D}$.
\end{theorem}

\begin{theorem} \label{theorem:Routh-Hurwitz} [Routh-Hurwitz's Criterion \cite{wiggins_introduction_2003}]

Let $y^*$ be a critical point of the system. We note $J^*$ the Jacobian matrix at this point, and $\chi_{J^*}$ its characteristic polynomial.

\begin{itemize}
\item When $\mathcal{D}\subset \mathbb{R}^2$, we have $\chi_{J^*} = X^2 - \Tr(J^*) X + \det(J^*)$ and $y^*$ is Locally Asymptotically Stable (LAS) if $\Tr(J^*) < 0$ and $\det(J^*) > 0$.
\item When $\mathcal{D}\subset \mathbb{R}^3$, we have $\chi_{J^*} = X^3 - \Tr(J^*) X^2 + a_1 X - \det(J^*)$, where $$a_1 = J^*_{11}J^*_{22} + J^*_{11} J^*_{33} + J^*_{22}J^*_{33} - J^*_{31}J^*_{13} - J^*_{32}J^*_{23} - J^*_{12}J^*_{21}.$$  $y^*$ is LAS if $\Tr(J^*) < 0$, $a_1 > 0$, $\det(J^*) < 0$ and $-\Tr(J^*) a_1 + \det(J^*) > 0$.
\end{itemize}
\end{theorem}

\begin{theorem} \label{theorem:Vidyasagar} [Vidyasagar's Theorem  \cite{vidyasagar_decomposition_1980, dumont_mathematical_2012}]

Assume that the system can written as
\begin{equation}
\def\arraystretch{2}
\left\{ \begin{array}{l}
\dfrac{dy_1}{dt} = f_1(y_1), \\
\dfrac{dy_2}{dt} = f_2(y_1, y_2) 
\end{array} \right.
\label{equation: eqVidyasagar}
\end{equation}

with $(y_1, y_2) \in \mathbb{R}^{n_1} \times\mathbb{R}^{n_2}$. Let $(y_1^*, y^*_2)$ be an equilibrium point.
If $y^*_1$ is Globally Asymptotically Stable (GAS) in $\mathbb{R}^{n_1}$ for the system $\dfrac{dy_1}{dt} = f_1(y_1)$, and if $y^*_2$ is GAS in $\mathbb{R}^{n_2}$ for the system $\dfrac{dy_2}{dt} = g(y_1^*, y)$, then $(y_1^*, y_2^*)$ is (locally) asymptotically stable for system \eqref{equation: eqVidyasagar}. Moreover, if all trajectories of \eqref{equation: eqVidyasagar} are forward bounded, then $(y_1^*, y_2^*)$ is GAS for \eqref{equation: eqVidyasagar}.
\end{theorem}

\begin{theorem} \label{theorem:Tikhonov} [Thikonov's Theorem \cite{banasiak_methods_2014}]

We consider the following systems of ODEs,
\begin{equation}\label{orginalProbelm}
\def\arraystretch{2}
\left\lbrace \begin{array}{l}
\dfrac{d x}{dt} = f(t,x,y,\epsilon), \quad x(0) = x_0, \\
\epsilon \dfrac{d y}{dt} = g(t,x,y,\epsilon), \quad y(0) = y_0, \\
\end{array} \right.
\end{equation}and the following assumptions:
\begin{enumerate}
\item \textbf{Assumption 1:} Assume that the functions $f, g$:
$$
f : [0, T]\times \mathcal{\bar{U}} \times \mathcal{V} \times [0, \epsilon_0] \rightarrow \mathbb{R}^n,
$$
$$
g : [0, T]\times \mathcal{\bar{U}} \times \mathcal{V} \times [0, \epsilon_0] \rightarrow \mathbb{R}^m
$$
are continuous and satisfy the Lipschitz condition with respect to the variables $x$ and $y$ in $[0, T]\times \mathcal{\bar{U}} \times \mathcal{V}$, where $\mathcal{\bar{U}}$ is a compact set in $\mathbb{R}^n$, $\mathcal{V}$ is a bounded open set in $\mathbb{R}^m$ and $T, \epsilon_0 > 0$.
\item \textbf{Assumption 2:} The corresponding degenerate system reads
\begin{equation} \label{degenerateSystem}
\def\arraystretch{1.2}
\left\lbrace \begin{array}{l}
\dfrac{d x}{dt} = f(t,x,y,0), \quad x(0) = x_0, \\
0 =  g(t,x,y,0).
\end{array} \right.
\end{equation}

Assume that there exists a solution $y = \phi(t, x) \in \mathcal{V}$ of the second equation of \eqref{degenerateSystem}, for $(t,x) \in [0, T]\times \mathcal{\bar{U}}$. The solution is such that
$$
\phi \in \mathcal{C}^0([0, T]\times \mathcal{\bar{U}} ; \mathcal{V})
$$
and is isolated in $[0, T]\times \mathcal{\bar{U}}$.
\item \textbf{Assumption 3:} Consider the following auxiliary equation:
\begin{equation}\label{auxiliaryEquation}
\dfrac{d \tilde{y}}{d \tau} =  g(t,x,\tilde{y},0),
\end{equation}
where $t$ and $x$ are treated as parameters.

Assume that the solution $\tilde{y}_0 := \phi(t, x)$ of equation  \eqref{auxiliaryEquation} is asymptocially stable, uniformly with respect to $(t,x) \in [0, T]\times \mathcal{\bar{U}}$.

\item \textbf{Assumption 4:} 
Consider the reduced equation:
\begin{equation}\label{reducedEquation}
\dfrac{d\bar{x}}{dt} = f(t,\bar{x},\phi(t,\bar{x}), 0), \quad \bar{x}(0) = x_0.
\end{equation}
Assume that the function $(t,x) \mapsto f(t,x,\phi(t,x), 0)$ satisfies the Lipschitz condition with respect to $x$ in $[0, T]\times \mathcal{\bar{U}}$. Assume moreover that there exists a unique solution $\bar{x}$ of equation \eqref{reducedEquation} such that $$\bar{x}(t) \in Int \: \mathcal{\bar{U}}, \quad \forall t \in (0,T).$$

\item \textbf{Assumption 5:} We consider the equation \eqref{auxiliaryEquation} in the particular case $t=0$ and $x = x_0$:
\begin{equation}\label{auxiliaryEquation, 0}
\dfrac{d \tilde{y}}{d \tau} =  g(0, x_0,\tilde{y},0), \quad \tilde{y}(0) = y_0.
\end{equation}
Assume that $y_0$ belongs to the region of attraction of the solution $y = \phi(0, x_0)$ of equation $g(0, x_0,\tilde{y},0) = 0$.
\end{enumerate}

Let assumptions 1, 2, 3, 4, 5 be satisfied. There exists $\epsilon_0$ such that for any $\epsilon \in (0, \epsilon_0]$ there exists a unique solution $(x_\epsilon(t), y_\epsilon(t))$ of problem \eqref{orginalProbelm} on $[0,T]$ and
\begin{equation}
\def\arraystretch{1.2}
\left\lbrace \begin{array}{l}
\lim\limits_{\epsilon \rightarrow 0}{x_\epsilon(t)} = \bar{x}(t) \quad t \in [0,T], \\
\lim\limits_{\epsilon \rightarrow 0} y_\epsilon(t) = \phi(t, \bar{x}(t)) := \bar{y}(t) \quad t \in (0,T] \\
\end{array} \right.
\end{equation}
where $\bar{x}(t)$ is the solution of problem \eqref{reducedEquation}.
\end{theorem}

We continue by defining the notions used in theorem \ref{theorem:Zhu}
\begin{definition}
An open set $\mathcal{D} \in \mathbb{R}^n$ is said to be p-convex provided that for every $x, y \in \mathcal{D}$, with $x \leq y$, the line segment joining $x$ and $y$ belongs to $\mathcal{D}$.
\end{definition}

\begin{definition}\cite{kaszkurewicz_matrix_2012}
A square matrix $A \in M_n (\mathbb{R})$ is said reducible if 
%there exists a permutation matrix $P$ such that $P^ T AP$ is block triangular. Otherwise, $A$ is called irreducible.
for each nonempty proper subset $I$ of $N = \{1, ..., n\}$, there exists $i \in I$ and $j \in N\backslash I$ such that $A_{i,j} \neq 0$
\end{definition}


\begin{definition} \label{def:monotone}\cite{smith_monotone_1995} The system is said to be 
\begin{itemize}
\item irreducible if the Jacobian matrix $\mathcal{J}(y)$ is irreducible.
\item competitive if $\mathcal{J}(y)$ has non positive off-diagonal elements:
$$ \dfrac{\partial f_i}{\partial y_j}(y) \leq 0, \quad i \neq j.
$$
\end{itemize}
\end{definition}

\begin{theorem} \label{theorem: monotone GAS} [Theorem 3 of \cite{anguelov_monotone_2010}]

Let the system \eqref{equation:generic system} be monotone (and in particular competitive) and let $a,b \in \mathcal{D}$ such that $a <b$, $[a, b] \subset \mathcal{D}$ and $f(b) \leq \mathbf{0} \leq f(a)$. Then the system defines a positive dynamical system on $[a, b] $. Moreover, if $[a, b] $ contains a unique equilibrium then it is GAS on $[a, b] $.
\end{theorem}

\begin{theorem}\label{theorem:Zhu} [Zhu and Smith's Theorem \cite{zhu_stable_1994}]
The following theorem holds true when $\mathcal{D} \subset \R^3$ and verifies the following assumption:
\begin{itemize}
\item $\mathcal{D}$ is an open, $p$-convex subset of $\mathbb{R}^3$,
\item $\mathcal{D}$ contains a unique equilibrium point $y^*$ and $\det(\mathcal{J}(y^*)) < 0$,
\item $f$ is analytic in $\mathcal{D}$,
\item the system is competitive and irreducible in $\mathcal{D}$,
\item the system is dissipative: For each $y_0 \in \mathcal{D}$, the positive semi-orbit through $y_0$, $\phi^+(y_0)$ has a compact closure in $\mathcal{D}$. Moreover, there exists a compact subset $\mathcal{B}$ of $\mathcal{D}$ with the property that for each $y_0 \in \mathcal{D}$, there exists $T(y_0) > 0$ such that $y(t, y_0) \in \mathcal{B}$ for $t \geq T(y_0)$.
\end{itemize}

then either $y^*$ is stable, or there exists at least one non-trivial orbitally asymptotically stable  periodic orbit in $\mathcal{D}$.
\end{theorem}

\section{Preliminary Results}
\begin{prop} \label{propBeta}
Assuming $\beta < \beta^*$, the following inequality holds
$$
\dfrac{\beta (1-\alpha) r_F}{\lfw} \Big(1 - \dfrac{\mu_D - f_D}{\lfw m e (1-\alpha) K_F}\Big) < 1
$$
\end{prop}

\begin{proof}
We have
\begin{equation*}
\dfrac{\beta (1-\alpha) r_F}{\lfw} \Big(1 - \dfrac{\mu_D - f_D}{\lfw m e (1-\alpha) K_F}\Big) < \dfrac{4 \mu_D - f_D}{\lfw m e (1-\alpha) K_F} \Big(1 - \dfrac{\mu_D - f_D}{\lfw m e (1-\alpha) K_F}\Big).\\
\end{equation*}

It is straightforward to show that $x(1 - x) \leq \dfrac{1}{4}$ for $x \in \mathbb{R}$. Therefore,
\begin{equation*}
\dfrac{\beta (1-\alpha) r_F}{\lfw} \Big(1 - \dfrac{\mu_D - f_D}{\lfw m e (1-\alpha) K_F}\Big) <  1
\end{equation*}
\end{proof}
\subsection{Study of $P_F$} \label{section:study of PF}
We define the following polynomial, which is used in sections \marc{ref}:
\begin{multline}
P_F(X) := X^2 \left(\dfrac{er_F}{K_F} \right) - X \left(e(1-\alpha)r_F + \dfrac{(\mu_D - f_D) r_F}{\lfw m K_F} + \dfrac{\cI \beta r_F}{\lfw K_F} \right) + \\ \left(\dfrac{(\mu_D - f_D)(1-\alpha) r_F}{\lfw m} - \cI\Big(1 - \dfrac{(1-\alpha)\beta r_F}{\lfw} \Big) \right).
\label{polynome-Feq}
\end{multline} 

\begin{prop}\label{prop:study of PF}
When $\beta < \beta^*$ and $\cI > 0$, the following results are true:
\begin{itemize}
\item $P_F\Big((1-\alpha)K_F\Big) = -\cI < 0$
\item $P_F\Big(\dfrac{\mu_D - f_D}{\lfw m e}\Big) < 0$
\item $P_F\Big((1-\alpha)K_F - \dfrac{K_F \lfw}{\beta r_F}\Big) > 0$
\item $P_F$ admits two real roots, $F_1^* \leq F_2^*$. 
\item If $\dfrac{(\mu_D - f_D) r_F}{\lfw m } \leq \cI\Big(1 - \dfrac{(1-\alpha)\beta r_F}{\lfw} \Big)$,  $F_2^*$ is positive and $F_1^*$ is non positive. If $\dfrac{(\mu_D - f_D) r_F}{\lfw m } > \cI\Big(1 - \dfrac{(1-\alpha)\beta r_F}{\lfw} \Big)$, $F^*_1$ and $F^*_2$ are positive. Moreover, $P_F$ is positive on $(-\infty, F_1^*)$, negative on $(F^*_1, F^*_2)$ and positive on $(F^*_2, +\infty)$.
\item From the precedent points, it follows that $$(1-\alpha)K_F - \dfrac{K_F \lfw}{\beta r_F} \leq F^*_1 \leq (1-\alpha)K_F, \dfrac{\mu_D - f_D}{\lfw m e} \leq F_2^* $$
\end{itemize}

\end{prop}

\begin{proof}
We have:
\begin{align*}
P_F((1-\alpha) K_F) &= \Big((1-\alpha) K_F \Big)^2 \left(\dfrac{er_F}{K_F} \right) - (1-\alpha) K_F \left(e(1-\alpha)r_F + \dfrac{(\mu_D - f_D) r_F}{\lfw m K_F} + \dfrac{\cI \beta r_F}{\lfw K_F} \right) + \\ &\left(\dfrac{(\mu_D - f_D)(1-\alpha) r_F}{\lfw m} - \cI\Big(1 - \dfrac{(1-\alpha)\beta r_F}{\lfw} \Big) \right), \\
&=(1-\alpha)^2 e K_F r_F - e(1-\alpha)^2 K_F r_F - \dfrac{(\mu_D - f_D) (1-\alpha) r_F}{\lfw m} - \dfrac{\cI \beta (1-\alpha)r_F}{\lfw}  + \\ &\dfrac{(\mu_D - f_D)(1-\alpha) r_F}{\lfw m} - \cI +\cI \dfrac{(1-\alpha)\beta r_F}{\lfw}, \\
&= -\cI< 0.
\end{align*}
Then, 

\begin{align*}
P_F\Big(\dfrac{\mu_D - f_D}{\lfw m e}\Big) &= \left(\dfrac{\mu_D - f_D}{\lfw m e}\right)^2 \left(\dfrac{er_F}{K_F} \right) - \dfrac{\mu_D - f_D}{\lfw m e} \left(e(1-\alpha)r_F + \dfrac{(\mu_D - f_D) r_F}{\lfw m K_F} + \dfrac{\cI \beta r_F}{\lfw K_F} \right) + \\ & \left(\dfrac{(\mu_D - f_D)(1-\alpha) r_F}{\lfw m} - \cI\Big(1 - \dfrac{(1-\alpha)\beta r_F}{\lfw} \Big) \right), \\
&= - \dfrac{(\mu_D - f_D) \cI \beta r_F}{e \lfw ^2 K_F} - \cI + \dfrac{\cI (1-\alpha)r_F \beta}{\lfw}, \\
&= -\cI \left( 1 - \dfrac{(1-\alpha)r_F \beta }{\lfw} + \dfrac{(\mu_D - f_D)  \beta r_F}{e \lfw ^2 K_F} \right), \\
&= -\cI \left( 1 - \dfrac{\beta(1-\alpha)r_F  }{\lfw}\Big(1 - \dfrac{(\mu_D - f_D) }{ m e \lfw (1-\alpha) K_F}\Big) \right), \\
& < 0,
\end{align*}
thanks to proposition \ref{propBeta}. We have:

\begin{align*}
P_F\Big((1-\alpha)K_F - \dfrac{K_F \lfw}{\beta r_F}\Big) &= P_F\Big((1-\alpha)K_F\Big) + \Big(\dfrac{K_F \lfw}{\beta r_F}\Big)^2 \dfrac{er_F}{K_F} - 2(1-\alpha)K_F \dfrac{K_F \lfw}{\beta r_F}\dfrac{er_F}{K_F} + \\ &\left(e(1-\alpha)r_F + \dfrac{(\mu_D - f_D) r_F}{\lfw m K_F} + \dfrac{\cI \beta r_F}{\lfw K_F} \right) \dfrac{K_F \lfw}{\beta r_F}, \\
&= -\cI + \dfrac{K_F \lfw^2}{\beta^2 r_F} - 2 \dfrac{(1-\alpha)K_F \lfw e}{\beta} +\dfrac{(1-\alpha)K_F \lfw e}{\beta} + \dfrac{\mu_D - f_D}{\beta m} + \cI, \\
&= \dfrac{K_F \lfw^2}{\beta^2 r_F} -  \dfrac{(1-\alpha)K_F \lfw e}{\beta} + \dfrac{\mu_D - f_D}{\beta m}, \\
&= \dfrac{K_F \lfw^2}{\beta^2 r_F} \left(1 - \dfrac{\beta (1-\alpha) r_F}{\lfw} \Big(1 - \dfrac{\mu_D - f_D}{m e \lfw K_F(1-\alpha)} \Big) \right), \\
&> 0,
\end{align*}
using proposition \ref{propBeta}.

To show the last point of the proposition, we start by computing the discriminant of $P_F$, $\Delta_F$. We have:
\begin{align*}
\Delta_F &= \left(e(1-\alpha)r_F + \dfrac{(\mu_D - f_D) r_F}{\lfw m K_F} + \dfrac{\cI \beta r_F}{\lfw K_F} \right)^2 - 4\dfrac{er_F}{K_F}  \left(\dfrac{(\mu_D - f_D)(1-\alpha) r_F}{\lfw m} - \cI\Big(1 - \dfrac{(1-\alpha)\beta r_F}{\lfw} \Big) \right), \\
\Delta_F &= \left(e(1-\alpha)r_F - \dfrac{(\mu_D - f_D) r_F}{\lfw m K_F}\right)^2 + \dfrac{\cI \beta r_F}{\lfw K_F} \left(\dfrac{\cI \beta r_F}{\lfw K_F} + 2\dfrac{(\mu_D - f_D) r_F}{\lfw m K_F} + 2e(1-\alpha)r_F \right) + 4\dfrac{er_F}{K_F}  \cI\Big(1 - \dfrac{(1-\alpha)\beta r_F}{\lfw} \Big), \\
\Delta_F & > 0.
\end{align*}

Therefore, $P_F$ admits two real roots. Their sign depends on the sign of the constant coefficient. $P_F$ admits:
\begin{itemize}
\item One non positive root $F^*_1$ and one positive root $F^*_2$ if $$\dfrac{(\mu_D - f_D)(1-\alpha) r_F}{\lfw m} - \cI\Big(1 - \dfrac{(1-\alpha)\beta r_F}{\lfw} \Big) \leq 0 \Leftrightarrow \dfrac{(\mu_D - f_D) r_F}{\lfw m } \leq \cI\Big(1 - \dfrac{(1-\alpha)\beta r_F}{\lfw} \Big).$$
\item Two positive roots $F^*_1\leq  F^*_2$ if $\dfrac{(\mu_D - f_D) r_F}{\lfw m } > \cI\Big(1 - \dfrac{(1-\alpha)\beta r_F}{\lfw} \Big)$.
\end{itemize}
They are given by:

\begin{equation*}
F_i^* = \dfrac{K_F(1-\alpha)}{2}\left(1 \pm \dfrac{\sqrt{\Delta_F}}{e(1-\alpha)r_F}\right) + \dfrac{\mu_D - f_D}{2\lfw m e} + \dfrac{\cI \beta}{2\lfw e}, \quad i=1,2.
\end{equation*}
\end{proof}

\subsection{Proof of proposition \ref{prop:stab, cI=beta=0}} \label{sec:stab, cI = beta = 0}

\begin{proof}
When $\beta = 0$, we have
\begin{multline} \label{DeltaStab, generalCase}
\Delta_{Stab, \cI =\beta = 0} =  \left(\mu_D - f_D + m_D + m_W + r_F\dfrac{F_W^*}{K_F} \right) \\ \times   \left( \mu_D -f_D + m_D + m_W \right) - m_D e \lfw \Big(K_F(1-\alpha) - F_W^* \Big),
\end{multline}

with $F_W^* = \dfrac{\mu_D - f_D}{\lfw m e}$. Therefore,

\begin{multline*}
\Delta_{Stab, \cI=\beta = 0} > 0 \\
\Leftrightarrow \left(\mu_D - f_D + m_D + m_W + r_F \dfrac{\mu_D - f_D}{\lfw K_F m e} \right) \times   \left( \mu_D -f_D + m_D + m_W \right) > \\ m_D e \lfw \left(K_F(1-\alpha) - \dfrac{\mu_D - f_D}{\lfw m e} \right), \\
\Leftrightarrow (\mu_D - f_D + m_D + m_W)^2 + r_F \dfrac{\mu_D - f_D}{\lfw K_F m e}  \times   \left( \mu_D -f_D + m_D + m_W \right) > \\ m_D e \lfw K_F(1-\alpha) - (\mu_D - f_D)m_W , \\
\Leftrightarrow \lfw (\mu_D - f_D + m_D + m_W)^2 + r_F \dfrac{\mu_D - f_D}{K_F m e}  \times   \left( \mu_D -f_D + m_D + m_W \right) > \\ m_D e \lfw^2 K_F(1-\alpha) - \lfw (\mu_D - f_D)m_W , \\
\Leftrightarrow 0 > \lfw^2 (1-\alpha) K_F  m_D e - \lfw \Big((\mu_D - f_D + m_D + m_W)^2 +(\mu_D - f_D)m_W \Big) - \\ \dfrac{r_F (\mu_D - f_D) }{K_F m e}  \big( \mu_D -f_D + m_D + m_W \big).\\
\end{multline*}

We define 
\begin{multline*}
P_{\Delta_{Stab, \cI= \beta = 0}}(X) := X^2 (1-\alpha) K_F  m_D e - X \Big((\mu_D - f_D + m_D + m_W)^2 +(\mu_D - f_D)m_W \Big) - \\ \dfrac{r_F (\mu_D - f_D) m_W}{K_F m_D e}  \big( \mu_D -f_D + m_D + m_W \big),
\end{multline*} 

such that we have 
\begin{equation}
\Delta_{Stab, \cI= \beta = 0} > 0 \Leftrightarrow P_{\Delta_{Stab, \cI= \beta = 0}}(\lfw) < 0.
\label{equation: equivalence DeltaStab}
\end{equation}

$P_{\Delta_{Stab, \cI = \beta = 0}}$ has a positive dominant coefficient, and its other coefficients are negative. So,  $P_{\Delta_{Stab, \cI= \beta = 0}}$ admits a unique positive root, noted $\lfw^*$, given by:
\begin{multline}
\lfw^* = \\
 \dfrac{\left[m_{W}(\mu_{D}-f_{D})+\big(\mu_{D}-f_{D}+m_{D}+m_{W})^{2}\right]\left(1+\sqrt{1+4\dfrac{(1-\alpha)m_{W}r_{F}\left(\mu_{D}-f_{D}\right)\big(\mu_{D}-f_{D}+m_{D}+m_{W})}{\left[m_{W}(\mu_{D}-f_{D})+\big(\mu_{D}-f_{D}+m_{D}+m_{W})^{2}\right]^{2}}}\right)}{2em_D (1-\alpha) K_F }
\end{multline}

Moreover, $P_{\Delta_{Stab, \cI = \beta = 0}}$ is negative on $\left[0, \lfw^* \right)$ and positive on $\left(\lfw ^*, +\infty \right)$. Using \eqref{equation: equivalence DeltaStab}, we obtain that $EE^{HF_W}_{\cI = \beta = 0}$ is asymptotically stable if $\lfw  < \lfw ^*$.
\end{proof}

\end{appendix}



\newpage

\bibliographystyle{plain}
\bibliography{Biblio/Math, Biblio/Context, Biblio/interactionsHumanEnvironmentModel}
\end{document}
